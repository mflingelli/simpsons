\HTMLOutput{

}{
	\begin{savequote}[55mm]
	Auf den Alkohol -- die Ursache und die Lösung aller Probleme!
	\qauthor{Homer J. Simpson}
	\end{savequote}
}
\chapter{Die Charaktere}\label{Charaktere}

Die Simpsons sind die am längsten laufenden Zeichentrickserie im Fernsehen und die am längsten laufende Sitcom in der Geschichte des amerikanischen Fernsehens. Die Serie wurde von Matt Groening\index{Groening!Matt} entwickelt. Das Fernsehdebüt gaben die Simpsons am 17. Dezember 1989 auf FOX. Die Serie ist ein Spinoff der \glqq Tracy Ullman Show\grqq . Derzeit gibt es \staffelAnzahl\ in das Deutsche synchronisierte Staffeln der Simpsons mit insgesamt \episodenAnzahl\ Folgen.

Mit dem Abschluss der Staffel \staffelAnzahl\ werden \episodenAnzahl\ Folgen ausgestrahlt worden sein. Den Rekord für die Serie mit den meisten Folgen hält bislang \glqq Rauchende Colts\grqq , eine Western-Serie mit 635 Episoden (siehe \cite{Serienrekord}). Diesen Rekord übertrafen die Simpsons mit ihrer 636. Episode \glqq Rezeptfrei\grqq\ (siehe \ref{XABF09}).

Die Serie zeichnet sich durch sarkastischen Humor aus und übt frech Kritik am -- ach so schönen -- \glqq American Way of Life\grqq . Die Serie hat diverse Auszeichnungen (u.a. \emmyAnzahl\ Emmy Awards, Stand Ende 2014) gewonnen und wird in über 90 Ländern in rund 20 verschiedenen Sprachen ausgestrahlt.

Die Simpsons sind alles andere als eine Zeichentrickserie für Kinder. So sagt der Kunsthistoriker Henry Keazor\index{Keazor!Henry} in einem Interview von 2008 gegenüber dem Westdeutschen Rundfunk (WDR):
\begin{quotation}
\scshape
\glqq Man kann die Simpsons auch anschauen, ohne die ganzen Be\-zü\-ge und Verweise zu entdecken -- dann ist es nur eine lustige und originelle Cartoonserie für Kinder. Aber die meisten Episoden sind auf verschiedenen Niveaus angesiedelt. Je tiefer man einsteigt, desto mehr kann man finden, und es ist sehr spannend, den verschiedenen Verweisen und Zitaten nachzugehen. Die können mal sehr offensichtlich und bekannt sein, mal sind sie nur für Insider zu entschlüsseln \cite{HenryKeazor}\grqq .
\end{quotation}

Die Simpsons werden auch in der Politik wahrgenommen. So merkte der ehemalige bayerische Ministerpräsident Dr. Edmund Stoiber\index{Stoiber!Edmund} an, im deutschen Fernsehen gebe es \glqq nur noch kaputte Familien. Außer den Simpsons gibt es keine normale Familie mehr im TV\grqq\ \cite{Stoiber}. In anderen Staaten wie beispielsweise in Venezuela wird das anders gesehen. Dort mussten die Simpsons aus dem Vormittagsprogramm des Senders \glqq Televen\grqq\ genommen werden. Ersetzt wurden die Simpsons übrigens durch \glqq Baywatch\grqq\ mit David Hasselhoff\index{Hasselhoff!David} \cite{Venzuela}. Auch der ehemalige Präsident der Vereinigten Staaten von Amerika George Bush sen. meinte, dass die Simpsons ein schlechtes Beispiel für Familien seien und dass sie sich eher an den Waltons orientieren sollten. Seine Frau Barbara Bush kritisierte die erste Staffel. Sie nannte sie \glqq das Dümmste, was ich je gesehen habe\grqq\ (siehe \cite{Reiss19}).

In den Episoden kamen einige ehemalige und aktuelle Präsidenten der Vereinigten Staaten vor. Keiner von ihnen synchronisierte seine Rolle selbst. Der damalige britische Premierminister Tony Blair hingegen synchronisierte seinen Auftritt in der Episode \glqq Die Queen ist nicht erfreut!\grqq\ (siehe \ref{EABF22}) selbst.

Im Iran wurde 2012 Homer zur Persona non grata erklärt. Figuren von ihm und seiner Familien dürfen nicht mehr verkauft werden. Das staatliche Institut für internationale Entwicklung von Kindern hat etwas dagegen \cite{IranMullahs}.

Die Universität von Glasgow in Schottland bietet ein Seminar mit dem Titel \glqq D'Oh! The Simpsons Introduce Philosophy\grqq\ an (siehe \cite{UoG}). In diesem Seminar sollen die Weisheiten von Homer mit denen der großen Philosophen Aristoteles, Voltaire oder Plato verglichen werden. Dr. John Donaldson, Tutor an der Universität und Erfinder des Kurses, erklärte gegenüber britischen Medien: \glqq Matt Groening, der Macher der Simpsons, hat selbst Philosophie studiert und das merkt man in jeder Simpsons-Folge.\grqq\ Und weiter: \glqq Die Simpsons sind Kult und stecken voller Philosophie.\grqq

Mittlerweile wurde in Myrtle Beach (im US-Bundesstaat South Carolina) ein Kwik-E-Mart originalgetreu nachgebaut. Ende 2018 soll des Weiteren noch das \glqq Aztec Theatre\grqq\ eröffnen (\cite{SVZ18}).
% Kritik an Simpsons Produktionsbedingungen \cite{SimpsonsBanksy}, \cite{SimpsonsBanksyVideoSperren}

Die Beschreibung der Charaktere wurde im Wesentlichen dem Springfield Shopper \cite{SpringfieldShopper} und Wikipedia \cite{Wikipedia} entnommen. Ergänzungen sind entsprechend gekennzeichnet.

\section{Familie Simpson}
Die Familie Simpson wohnt in 742 Evergreen Terrace in Springfield USA wie auf einen an Homer adressierten Brief in Folge \glqq Und der Mörder ist\dots\grqq\ (siehe \ref{EABF01}) zu lesen ist. Allerdings gibt es zahlreiche Episoden, in denen die Adresse der Simpsons anders lautet. Beispielsweise lautet in der Episode \glqq Die Kontaktanzeige\grqq\ (siehe \ref{8F16}) die Adresse der Simpsons 94 Evergreen Terrace.


\subsection{Homer Jay}\index{Simpson!Homer}\label{HomerSimpson}
\kommentar{
\begin{floatingfigure}[h]{0.3\textwidth}
  \centering
  \includegraphics[width=0.25\textwidth]{./Bilder/homer_simpson.png}
  \caption{Homer J. Simpson}
  \label{fig:HomerSimpson}
\end{floatingfigure}
}
Homer Simpson, geboren am 10. Mai 1955, ist der Vater der Familie. Sein zweiter Vorname ist Jay, den kannte er nur unter J. und er erfuhr in der Folge \glqq Homer ist ein toller Hippie\grqq\ (siehe \ref{AABF02}), dass er ausgeschrieben Jay lautet. Er hat drei schwarze Haare und wiegt 108 bis 117 Kilo. Er macht sich das Leben einfach und gibt dies auch offen zu.

Homer verdient sein Geld überwiegend als Sicherheitsinspektor in Sektor 7G im Springfielder Atomkraftwerk. Er kam durch eine Quotenregelung für benachteiligte Personen zu dieser Stelle, besitzt aber keine entsprechende Ausbildung für diesen Job. Für seinen Job ist eigentlich ein Hochschulstudium der Kernphysik notwendig. In der Folge \glqq Homer an der Uni\grqq\ (siehe \ref{1F02}) fiel dies auch der Atomaufsichtsbehörde auf und diese verlangte von Homers Chef, dass Homer ein entsprechendes Studium nachholen muss. Das Studium bestand er nur, da Kommilitonen mit Computerbetrug seine Noten aufbesserten. Seine Arbeit besteht im Wesentlichen aus dem Überwachen der Instrumente und der Steuerkonsole, mit der er im Ernstfall eingreifen und so eine nukleare Katastrophe verhindern könnte. Leider ist Homer höchst unqualifiziert und kann nicht einmal immer den Süßigkeitenautomaten im Kernkraftwerk ohne Probleme bedienen. In der Folge \glqq Die rebellischen Weiber\grqq\ (siehe \ref{1F03}) bleiben Homers Arme in einem Getränke- und einem Süßigkeitenautomaten stecken. Dies spiegelt sich auch in etlichen Kernschmelzen wieder, die allesamt auf sein Konto gehen. Hinzu kommt seine absolute Faulheit. Einen Großteil des Tages verbringt er schlafend in seinem Stuhl versunken oder in der Kantine bei seinen heiß geliebten Jelly Donuts. Dass er noch nicht längst endgültig entlassen wurde, ist zum einen der Senilität seines Bosses, dem Millionär Charles Montgomery Burns und zum anderen vielen glücklichen Zufällen zu verdanken, in denen sich alles zum Guten gewendet hat und Homer zum Teil sogar noch als Held da stand.

Auch als Ehemann und Vater glänzt Homer nur selten. Seine Frau Marge, die er auf der Highschool kennengelernt hat, muss in der Ehe viel einstecken. Sei es durch Homers immer wieder vorkommende absolut verrückte Ideen oder wenn er sich spät abends sternhagelvoll blicken lässt, wenn er wieder mal auf ein Bier mit seinen Freunden Barney, Lenny und Carl bei Moes war. Auch gemeinsame Interessen sind kaum vorhanden. Marge interessiert sich für Kultur, wie Theater, Malerei und Ähnliches, während Homer mehr auf Crash Derbys oder Chili-Wettessen mit Kollegen steht. Homer hat auch zu seinen Kindern Bart, Lisa und Maggie ein nicht allzu gutes Verhältnis.

Die Kleinste, Maggie, wird häufig vernachlässigt. Die Babyvideos, die Marge aufgenommen hat, sind von Homer mit Footballspielen überspielt worden. Oft kommt es auch vor, dass er Maggie gar nicht als eines seiner Kinder erwähnt oder ihren Namen vergessen hat. Am schlechtesten ist das Verhältnis wohl zu Lisa, der älteren Tochter. Sie ist ein hoch intelligentes Mädchen, die nur Bestnoten nach Hause bringt und sehr gut Saxophon spielt. Homer hat kein Verständnis für den \glqq Krach\grqq , den sie jeweils in ihrem Zimmer veranstaltet. Und meistens gibt sie klein bei. Uneinig sind sich die beiden auch in Grundsatzfragen, wenn es um Ökologie oder Fleischkonsum geht (siehe \glqq Lisa als Vegetarierin\grqq , \ref{3F03}).

Zu Bart, dem kleinen Teufel, hat Homer ein gemischtes Verhältnis. Wenn es um illegale Sachen wie Schnapsbrennen oder einfach um die wilderen Seiten des Lebens geht, sind die beiden ein Herz und eine Seele. Vielfach geraten sie aber auch in einen Streit, welcher meistens von Bart provoziert wird. Charakteristisch für die Beziehung der beiden ist die typische Würgeszene, die aus den Episoden kaum mehr wegzudenken ist. Wenn Bart etwas verbockt hat oder es Homer nur vermutet, springt er Bart an die Gurgel.

Mit seinem Nachbarn Ned Flanders kommt Homer nicht gerade gut aus. Flanders ist ein gläu\-bi\-ger Mensch und glaubt an das Gute im Menschen, was Homer gnadenlos ausnutzt. Flanders versucht immer wieder, ein gutes Verhältnis zu Homer aufzubauen, doch dieser macht keine Anstalten in diese Richtung, sondern zeigt ihm klar, was er von ihm hält -- nichts! Ned lernt jedoch nie etwas aus solchen Situationen und leiht Homer regelmäßig Dinge wie Rasenmäher, Videokamera, Werkzeuge und so weiter, welche er sehr selten zurückbekommt. Als es einmal in Springfield einen Rekordhitze-Sommer gab, hat Homer bei den Flanders die Klimaanlage abmontiert. Darauf kommt Flanders zu den Simpsons rüber und fragt höflich, ob er sie denn wieder zurück haben könne, seine Frau habe gerade einen Hitzschlag erlitten.

Im Grunde ist er ein sympathischer liebenswerter Kerl, der das Leben auf seine Weise lebt (und genießt). Als die NASA bei den Fernsehübertragungen der Spaceshuttlestarts fast keine Zuschauer mehr vor die Fernseher locken konnte, entschlossen sie sich, einen Durchschnittsbürger mit auf eine Mission zu schicken. Homer musste daraufhin ein Ausscheidungstraining gegen seinen besten Freund Barney absolvieren und obwohl er eigentlich nur zweiter Sieger war, wurde er zusammen mit zwei Astronauten auf die Mission geschickt.

Wenn ihm etwas misslingt ist ein lautes \glqq Nein!!!\grqq\ zu hören (im Original ein lautes \glqq D'oh!\grqq ). Das Oxford English Dictionary führt seit 2001 Homers Ausspruch und Markenzeichen \glqq D'oh!\grqq\ \cite{GelbRegiertDieWelt}. Er wurde von Mutter Natur nicht gerade mit Intelligenz verwöhnt, was ihn vom Regen in die Traufe bringt und dann ins nächste Fettnäpfchen. So hat es zumindest den Anschein, aber das ist nicht ganz richtig. In der Folge \glqq Der berüchtigte Kleinhirn-Malstift\grqq\ (siehe \ref{BABF22}) wurde festgestellt, dass ein Wachsmalstift, den Homer sich als Kind tief in die Nase schob, seine Gehirnfähigkeiten um ein vielfaches beeinflusste. Als der Stift entfernt wurde, war Homer so hoch intelligent, dass er fast mit keinem seiner eigentlichen Freunde mehr reden konnte (die Gespräche waren ihm zu anspruchslos). Jedoch wurde die Beziehung zu seiner Tochter Lisa enorm gefestigt, da sie geistig auf einer Wellenlänge waren. Aber das reichte ihm nicht, er wollte wieder so \glqq dumm\grqq\ wie alle anderen sein und ließ sich von Moe medizinisch wieder einen Wachsmalstift injizieren, wodurch seine Gehirnaktivitäten wieder eingeschränkt waren. Homer schrieb jedoch vorher noch einen Brief an Lisa, in dem er sich für seine Rückumwandlung entschuldigte, wovon der alte Homer aber nicht mehr viel wusste. Er übernimmt sich regelmäßig und denkt so gut wie nie darüber nach, was sein Handeln für Konsequenzen mit sich bringt. Er mag es vor allem bequem und mit einem kühlen Duff Bier, dann ist seine Welt in Ordnung.

Interessant ist die Beziehung zu seinem Vater, Abraham Simpson. Homer wuchs bei seinem Vater auf, da seine -- mittlerweile verstorbene -- Mutter die Familie verlassen hat, als Homer noch klein war. Sie war stark in der Hippiekultur verwurzelt und ist seit einer gewaltsamen Demonstration in den 68er Jahren auf der Flucht vor dem FBI. Diese Zeit hatte auch einen Einfluss auf Homer, was sich aber wieder legte. Homers Vater war Unteroffizier bei den US Marines und dementsprechend wurde Homer auch hart und ohne Mutter erzogen. Sein Vater schärfte ihm immer wieder ein, dass er es nie zu etwas bringen würde. Homer war schon zu Highschool-Zeiten faul und hatte nur Rockmusik im Kopf. Am liebsten hätte er immer nur Partys gefeiert und den Ernst des Lebens beiseite gelassen. Dies sieht man am besten in der Szene, wo Homers Vater die Mondlandung im Fernseher ansieht und Homer auf dem Sofa nebenan Musik hört und geistig woanders ist. Die beiden standen sich nie sehr nahe, jedoch versucht Abe, wie sein Vater immer genannt wird, im Alter immer mehr den Kontakt zu seinem Sohn wiederherzustellen. Abe wurde in das Springfield Retirement Castle (das Altenheim) abgeschoben, wo er ein unglückliches Dasein fristet, obwohl er Homer Geld geliehen hatte, damit er sich ein Haus kaufen konnte. Damals sagte Homer, es sei eine Ehre für ihn, wenn Abe bei ihm wohnen würde. Homer hält seinen Vater für senil und liegt dabei gar nicht so falsch. Dabei versucht er ihm oft aus dem Weg zu gehen, um nicht seinen Anekdoten aus früheren Zeiten zuhören zu müssen.


Homer hatte neben seiner Arbeit als Sicherheitsinspektor im Sektor 7G im Atomkraftwerk viele weitere Jobs. So war er beispielsweise der Manager der Country Sängern Lurleen Lumpkin\index{Lumpkin!Lurleen} in \glqq Homer auf Abwegen\grqq\ (siehe \ref{8F19}), das Mas\-kott\-chen der Spring\-field Isotopes in \glqq Das Maskottchen\grqq\ (siehe \ref{7F05}), Ein-Schienenbahnlokführer in \glqq Homer kommt in Fahrt\grqq\ (siehe \ref{9F10}), der Leib\-wäch\-ter des Bürgermeisters Joe Quimby in \glqq Der unerschrockene Leibwächter\grqq\ (siehe \ref{AABF05}), ein Restaurantkritiker in \glqq Homer als Restaurantkritiker\grqq\ (siehe \ref{AABF21}) und Filmschauspieler in \glqq Everyman begins\grqq\ (siehe \ref{LABF13}).

Homer ist sehr erfolgreich, was seine Musikkarriere angeht. Als Marge auf der Universität studierte, finanzierte er ihr das Studium durch das Geld, welches er mit der Band \glqq Sadgasm\index{Sadgasm}\grqq\ verdiente. Die Band erhielt mehrere goldene Schallplatten. Zu den erfolgreichen Titel gehören \glqq Politically Incorrect\grqq , \glqq Shave Me\grqq\ und \glqq Margarine\grqq . Jahre nach seinen Erfolgen mit Sadgasm war er Mitglied der \glqq Überspitzen\grqq . Mit den Überspitzen konnte er sogar einen Grammy gewinnen. In der Band Covercraft war er Bassist. Er war nicht nur als Sänger erfolgreich, sondern auch als Komponist und Songschreiber. Als Lisa an einem Gesangswettbewerb teilnahm, schrieb er ihre Lieder (siehe \glqq Lisa Simpson: Superstar\grqq , \ref{GABF13}). Lisa gewann schließlich den Wettbewerb.

Homer steht auf der Liste der 100 größten fiktiven Charaktere der Zeitschrift \glqq Entertainment Weekly\grqq\ auf dem ersten Platz. Auf dem zweiten Platz folgt übrigens Harry Potter \cite{GreatestCharacters}. 

Das Gen RGS14\index{RGS14} bei Mäusen wird von den Forschern auch scherzhaft \glqq Homer-Simpson-Gen\index{Homer-Simpson-Gen}\grqq\ genannt, denn dieses Gen wirft die Mäuse geistig offenbar zurück. Wird dieses Gen ausgeschaltet, sind die Tiere schlauer und nicht mehr so vergesslich \cite{HomerSimpsonGen}.

\subsection{Marge}\index{Simpson!Marge}\label{MargeSimpson}
Marge, Homers Frau und die Mutter von Bart, Lisa und Maggie, heißt mit Mädchennamen Marjorie Bouvier und hat zwei ältere Zwillingsschwestern, Patty und Selma. Sie wurde in Capitol City geboren. Ihr Vater hieß Clancy Bouvier, ihre Mutter Jacqueline (Jacky) Bouvier. Auffällig sind ihre zu einer Turmfrisur gesteckten blauen (gefärbten) Haare. Die 35-jährige Marge verbringt die meiste Zeit zu Hause und führt den Haushalt. Im Gegensatz zu Homer ist sie eine recht moralische Person und bringt ihre Standpunkte immer wieder in Familiendiskussionen ein. Wenn Homer wieder einmal eine seiner verrückten Ideen hat, ist sie diejenige, die auch mal an mögliche Konsequenzen denkt und Homer bittet, sein Vorhaben aufzugeben. Doch meistens gelingt es ihr nicht, ihn davon abzubringen. Marge verbringt längst nicht ein so aufregendes Leben wie ihr Ehemann, doch hat sie mit der Familie genug zu tun -- insbesondere mit Sohn Bart, der keine Gelegenheit auslässt, Unsinn zu treiben. Marge hält die Fäden in der Familie zusammen und ist eine der ersten Ansprechpersonen, wenn Probleme auftauchen.

Marge war auf der selben Highschool wie Homer. Damals hatte sie noch lange glatte Haare und war eine überzeugte Feministin. Sie arbeitete als Lokalreporterin für den Springfield Shopper (siehe \glqq Doppeltes Einkommen, kinderlos\grqq, \ref{XABF06}) werden. Sie interessierte sich auch schon damals für Kunst und Malerei. Sie malte auch das Bild mit dem Boot, das über der Couch im Wohnzimmer hängt (siehe \glqq Die Trillion-Dollar-Note\grqq , \ref{5F14}) und sie malte Mr. Burns nackt. Sie war das einzige Mädchen, das Homer etwas abgewinnen konnte. Trotzdem war Bart nicht geplant. Ihre Schwestern rieten ihr immer ab, Homer zu heiraten, er sei nicht der richtige für sie. Trotz gelegentlicher Streitereien liebt Marge Homer immer noch wie am ersten Tag.

Marge ist eine typische Hausfrau, die keine Anstalten unternimmt, aus dieser Rolle auszubrechen. Doch ab und zu hat sie Phasen, in denen sie aus dem Alltag ausbricht und etwas aus der Sicht der anderen völlig Irrationales tut. Als Beispiel sei die Episode \glqq Die Springfield Connection\grqq\ (siehe \ref{2F21}) genannt, in der sie einen Posten bei der Polizei als Kriminalbeamtin annimmt und eine Mafiabande stellt. Außerdem war sie u.a. noch Brezelverkäuferin und Immobilienmaklerin.

Ihre größte Schwäche ist das Glücksspiel. In der Episode \glqq Vom Teufel besessen\grqq\ (siehe \ref{1F08}) verbringt sie soviel Zeit im Casino, dass sie sich nicht mehr um die Familie kümmern kann und trotz Versprechen Lisa nicht bei ihrem Schulprojekt helfen kann.

Am 16. Oktober 2009 erschien in den USA die Kiosk-Version des Playboys\index{Playboy} mit Marge als Covergirl. Im Heftinneren posierte sie nackt und gab ein Interview, auch der zweiseitige Ausfalter in der Mitte der Zeitschrift fehlte nicht \cite{MargePlayboyCover}.


\subsection{Bart}\index{Simpson!Bart}\label{BartSimpson}
Bart Simpson, mit vollem Namen Bartholomew J. Simpson, alias \glqq El Barto\grqq\index{El Barto}, das älteste Kind der Simpsons, ist ein regelrechter Teufelsbraten. Mit seinen zehn Jahren hat er einiges auf dem Kasten. Zu seinem Geburtsdatum gibt es widersprüchliche Angaben. Zum einen wird behauptet, der 1. April sei sein Geburtstag. In der Folge \glqq Simpsorama\grqq\ (siehe \ref{SABF16}) gibt Bart den 23. Februar als seinen Geburtstag an.

Er verfügt über ein unerschöpfliches Repertoire von Streichen und Wegen der Rache, mit einem Spektrum von Lausbubenstreichen bis hin zu internationalen Trickbetrügereien. Viele Episoden beginnen damit, dass Bart zur Strafe einen Spruch an die Wandtafel in der Schule schreiben muss. Dies lässt einen erahnen, was sich der Simpson-Sprössling geleistet hat.

Hier einige Beispiele (die vollständige Liste kann im Anhang nachgeschlagen werden, siehe \ref{Tafelanschriebe}):
\begin{itemize}
	\item Ich darf nicht eigenmächtig Probealarm geben.
	\item Sprengstoff gehört nicht in die Schule.
	\item Eine Feuerübung erfordert kein Feuer.
	\item Nicht alles, was ich wissen muss, habe ich im Kindergarten gelernt.
	\item Ich werde nicht in der Klasse rülpsen.
	\item Ich darf die Kletterstange nicht einfetten.
	\item Bart-Dollar ist keine legale Währung.
\end{itemize}

Bart steht auf Kriegsfuß mit der Schule. Seine Klassenlehrerin Mrs. Krabappel treibt er regelmäßig mit seinen Schandtaten im Unterricht in den Wahnsinn und ist deshalb ein notorischer Nachsitzer. Er ist nie um einen Streich verlegen und kann die Lehrerin aus dem Konzept bringen, in dem er die Aufmerksamkeit der ganzen Klasse auf sich lenkt. Auch dem Rektor Seymour Skinner der \glqq Springfield Elementary\grqq\ (Grundschule von Springfield) ist Bart wohlbekannt. Regelmäßig werden Homer und Marge in das Rektorzimmer zitiert, wo sie eine Moralpredigt über sich ergehen lassen müssen. Skinner hält an alten, zum Teil prüden Sitten fest und ist gerade deshalb Barts favorisiertes Ziel. So werden Hauswände besprüht, Schulaufführungen sabotiert oder Hausmeister Willie geärgert. Respekt hat er einzig und allein vor den vier Schlägertypen Nelson, Jimbo, Dolph und Kearney, die ihm regelmäßig das Milchgeld herausprügeln.

Bart hat allerdings auch eine weiche Seite. Besonders wenn es um seine kleine Schwester geht, bricht er öfters aus seiner Rolle aus und hilft ihr. Ebenso ist verwunderlich, dass er Briefmarken sammelt, wie in Folge \glqq Die Springfield Bürgerwehr\grqq\ (siehe \ref{1F09}) zu erfahren ist. Bart ist mit Milhouse Van Houten befreundet und verbringt viel Zeit mit ihm, obwohl dieser der typische Loser ist. Bart ist ein großer Fan von Krusty, dem Clown. Er ist Mitglied in Krustys Fanclub und hat die Mitgliedsnummer 16302, was aus der Episode \glqq Der Vater eines Clowns\grqq\ (siehe \ref{8F05}) hervorgeht.

Bart ist auch für seine Spaßanrufe in Moes Taverne bekannt, die er auch teilweise in Gegenwart von Lisa durchgeführt hat. Einer der bekanntesten dürfte der folgende sein, der in der Episode \glqq Die 24-Stunden-Frist\grqq\ (siehe \ref{7F11}) vorkommt:

\begin{description}
	\item [Moe:] \glqq Hallo? Moes Taverne. Geburtsstätte von der Bloody Mary.\grqq
	\item [Bart:] \glqq Hier ist Fred Rumsch. Ich möchte meinen Vater, Herrn Rumsch.\grqq
	\item [Moe:] \glqq Öhm, Moment Bitte. [ruft laut] Hey, ist hier einer der Rumsch heißt? Keiner der \glq Rumscheisst\grq ? Hey, jetzt hört doch mal! Rumsch! Ich suche jemand, der Rumsch heißt.\grqq
\end{description}

Obwohl er einen kleinen Bauch hat, betreibt er u.a. folgende Sportarten: Eishockey (\glqq Lisa auf dem Eise\grqq , \ref{2F05}), Fußball (\glqq Großer Bruder -- Kleiner Bruder\grqq , \ref{9F12}), Baseball (\glqq Mr. Burns wird entlassen\grqq , \ref{EABF10}) und sogar Ballett (\glqq Homer gegen Patty und Selma\grqq , \ref{2F14}).

\subsection{Lisa}\index{Simpson!Lisa}\label{LisaSimpson}
Lisa Marie Simpson ist das zweitälteste Kind von Homer und Marge. Ihr Geburtstag ist der 9. Mai (siehe \glqq Lisa legt los\grqq, \ref{XABF01}). Mit ihren acht Jahren ist sie schon sehr reif für ihr Alter. Sie ist blond, intelligent und bringt nur Bestnoten nach Hause, was sie vor allem dem pausenlosen Lernen zu verdanken hat. Ihr IQ liegt bei 156, was Homer in der Episode \glqq Homer hatte einen Feind\grqq\ (siehe \ref{4F19}) angibt. In der Folge \glqq Klug \& Klüger\grqq\ (siehe \ref{FABF09}) sagt Lisa allerdings selbst, dass ihr IQ bei 159 liege. In der Episode \glqq Die Stadt der primitiven Langweiler\grqq\ (siehe \ref{AABF18}) wird sie wegen ihres Intellekts in den Verein MENSA\index{MENSA}\footnote{Neben Lisa gehören MENSA noch Prof. Dr. John Frink, Dr. Julius Hibbert, Seymour Skinner, Lindsey Naegle und Jeff Albertson an. Ihre Treffen finden in der Euclid Street 13 statt.} aufgenommen, dem die intelligentesten Bewohner Springfields angehören. In der Folge \glqq Ich will nicht wissen, warum der gefangene Vogel singt\grqq\ (siehe \ref{JABF19}) wird sie als Schülerin des Jahrtausends ausgezeichnet. Sie ist auch eine recht moralische Person und durchlebt zudem die ganze Melancholie des Lebens. Sie fühlt sich stets missverstanden und fragt sich oft, ob sie nicht als Baby im Krankenhaus vertauscht worden sei. Diese Frage wird in der Episode \glqq Wer ist Mona Simpson?\grqq\ (siehe \ref{3F06}) beantwortet: Ihre Begabung hat sie von Homers Mutter Mona geerbt.

Sie ist zwar intelligent, aber dafür schlecht im Sportunterricht. Ihr droht dort sogar die Note ungenügend, wie in der Episode \glqq Lisa auf dem Eise\grqq\ (siehe \ref{2F05}) zu erfahren ist. Sie tritt schließlich einem Eishockeyverein bei und entpuppt sich dabei als Torwarttalent. Sie nimmt in der Folge \glqq L.S. -- Meisterin des Doppellebens\grqq\ (siehe \ref{DABF15}) Turnunterricht beim strengen Sportlehrer Lugash\index{Lugash}. Außerdem spielt sie noch Fußball (\glqq Marge online\grqq , \ref{JABF10}) und nimmt Ballettunterricht (\glqq Schall und Rauch\grqq , \ref{KABF08}).

Die meisten Frauen im Simpsons-Clan haben es im Leben weit gebracht, wäh\-rend die Männer relativ dumm sind. Als Grund wird das Simpsons-Gen angegeben, welches nur auf Männer vererbt werden kann und einen trüben Geist bewirkt. Homer hat es, Bart auch. Marge ist keine echte Simpson und scheidet damit aus. Lisa hat keine echte Verbindung zu ihrem Vater, dennoch bemüht sie sich immer wieder, seine Aufmerksamkeit zu erregen und Zuneigung zu gewinnen:

\begin{description}
	\item[Lisa:] \glqq Schau mal, Dad! Ich hatte beim großen Buchstabiertest null Fehler!\grqq
	\item[Homer:] \glqq Lisa, siehst du nicht dass Daddy fern sieht?\grqq
	\item[Lisa:] \glqq Ach ja, der Donut-Fresswettbewerb, den hatte ich ganz vergessen!\grqq
\end{description}

Bart und Lisa schwingen auch nicht auf derselben Welle. Im Gegensatz zu Bart liebt Lisa die Schule über alles. Sie kann mit Barts Streichen in der Regel wenig anfangen (bis auf die Telefonstreiche gegen Moe) und lässt ihn das auch spüren. Lisa fühlt sich ihrem Bruder intellektuell überlegen, geht aber trotzdem immer wieder auf seine Argumente ein. Es kann aber auch vorkommen, dass die beiden sich sehr gut vertragen, wenn es um etwas geht, was beide interessiert, wie zum Beispiel die Zeichentrickserie \glqq Itchy \& Scratchy\grqq , die sie sich immer gemeinsam ansehen.

Die einzige Person in ihrer Familie, die ihr immer wieder Mut macht, ist Marge. Doch Lisa kann mit ihrer Hausfrauenrolle nicht viel anfangen. Marge kommt mit Lisas Begabung manchmal nicht zurecht und behandelt sie wie ein kleines Kind. Beispiel: Homer und Bart gehen zu einem Crash-Derby (Schrottautorennen). Lisa und Marge bleiben zu Hause und haben nichts Schlaues zu tun. Marge schlägt Lisa vor, einen Frauennachmittag zu veranstalten und bringt einige Vorschläge wie Puzzles zusammensetzen, Ostereier anmalen usw. Lisa aber will nichts mit ihrer Mutter unternehmen und verschwindet in ihrem Zimmer. Dort beginnt sie, Saxophon zu spielen. Lisa spielt im Schulorchester Saxophon. Außerdem hat sie eine kraftvolle Gesangsstimme, mithilfe derer sie einen Gesangswettbewerb gewinnen kann (siehe \glqq Lisa Simpson: Superstar\grqq , \ref{GABF13}). Neben ihrer Leidenschaft für Musik ist sie eine begeisterte Leserin, so liest sie u.a. die Angelica Button Buchreihe.

Obwohl sie erst acht Jahre alt ist, engagiert sich Lisa schon stark politisch und ökologisch. In der Schule bringt sie immer ihre kritischen Ansichten zur Gesellschaft zur Sprache. Da es für ein so junges Mädchen eigentlich nicht üblich ist, sich über solche Dinge Gedanken zu machen, wird sie meistens mit einer unbefriedigenden Antwort abgespeist. Dies könnte auch eine Anspielung von Matt Groening auf die angebliche Ignoranz der amerikanischen Bevölkerung sein. Lisa ist den anderen in ihrem Alter weit voraus. Sie macht sich Gedanken über Dinge, von denen andere keine Ahnung haben, zum Beispiel in der Folge \glqq Lisa kontra Malibu Stacy\grqq\ (siehe \ref{1F12}), die sich satirisch mit Barbie-Puppen beschäftigt:

\begin{description}
	\item[Lisa:] \glqq Merkt ihr denn nicht, dass diese Puppe ein sexistisches Weltbild vermittelt?\grqq
	\item[Sherry:] \glqq Hihi, Lisa hat ein schmutziges Wort gesagt!\grqq\
\end{description}

Lisa wendet sich in der Folge \glqq Allein ihr fehlt der Glaube\grqq\ (siehe \ref{DABF02}) ihrem bisherigen Glauben ab und wendet sich dem Buddhismus zu. Sie ist somit Springfields jüngste Buddhistin (\glqq Der Tortenmann schlägt zurück\grqq , \ref{FABF15}). Neben ihr sind noch Lenny und Carl Buddhisten.

Lisa ist seit der Folge \glqq Lisa als Vegetarierin\grqq\ (siehe \ref{3F03}) Vegetarierin.

Lisa ist in der Episode \glqq Alles über Lisa\grqq\ (siehe \ref{KABF13}) kurzzeitig Krustys Assistentin und erhält anschließend Krustys Show. Sie wird dafür mit der Auszeichnung \glqq Entertainer des Jahres\grqq\ ausgezeichnet. Nachdem ihr Sideshow Mel über seinen Absturz nach seiner Auszeichnung berichtet hatte, beendet Lisa ihren Ausflug in das Showbusiness.


\subsection{Maggie}\index{Simpson!Maggie}\label{MaggieSimpson}
Maggie Simpson ist ein Jahr alt und somit das jüngste Mitglied der Simpsons. Ihr eigentlicher Name ist Margaret (\glqq Bei Simpsons stimmt was nicht!\grqq , \ref{3F01}). Ihr zweiter Vorname lautet Lenny (\glqq Es ist ein Todd entsprungen\grqq, \ref{QABF09}). Außer dass sie ständig schnullert, sehr oft hinfällt und mit ihrem Augenzwinkern jeden auf ihre Seite zieht, ist von ihr nicht viel zu hören. Ihr erstes und einziges echtes Wort war \glqq Daddy\grqq (\glqq Am Anfang war das Wort\grqq , \ref{9F08}), obwohl sie auch mehrere Male in der Fantasie von anderen oder in Halloween-Folgen gesprochen hat (Beispiel aus Barts Fantasie: \glqq Du bist Schuld, dass ich nicht sprechen kann.\grqq ). Ihre Schweigsamkeit war lange Zeit ein feststehender Fakt der Serie. Selbst in Folgen, die in der Zukunft spielten, war von der Teenager-Maggie kein Wort zu hören. Sie hat einen Erzfeind, Gerald, ein Baby mit einer durchgehenden Augenbraue. Zu ihren beiden Geschwistern hat sie kein besonderes Verhältnis. So liebt sie zum Beispiel den Fernseher mehr als die beiden.

Maggie hat eine gewisse Affinität zu Waffen. Sie schoss nicht nur mit einem Revolver auf Mr. Burns (in \glqq Wer erschoss Mr. Burns? Teil 2\grqq , \ref{2F20}), sondern rettete durch schnelles Schießen (im Stile der Westernhelden) mit einem Gewehr auch ihren Vater Homer, der sich als Boss der Sicherheitsunternehmen \glqq SpringShield\index{SpringShield}\grqq\ mit Fat Tony anlegte, sodass dieser Homer beseitigen wollte (\glqq Sicherheitsdienst Springshield\grqq , \ref{DABF17}).

Maggie scheint wie alle weiblichen Simpsons sehr intelligent zu sein. Sie weiß zum Beispiel schon, was eine Biedermeier-Anrichte ist, kann mit ihren Spielzeugklötzchen die Formel $E = mc^2$ legen und ihren eigenen Namen auf ihrem \glqq Etch-A-Sketch\grqq\ schreiben. Außerdem kann sie auf ihrem Baby-Xylophon Tschaikowski spielen.

Im Vorspann vieler Episoden ist zu sehen, wie Maggie über den Scanner an der Supermarktkasse gezogen wird, daraufhin erscheint auf dem Kassendisplay \$ 847,63, ein Betrag, der einmal als der Geldbedarf angegeben wurde, den ein Baby in den USA in einem Monat an Unterhalt kostet. Ab der ersten in HDTV ausgestrahlten Episode \glqq Quatsch mit Soße\grqq\ (siehe \ref{LABF01}) beträgt der Betrag im neuen Vorspann nur noch \$ 243,26. In der Folge \glqq Die 138. Episode, eine Sondervorstellung\grqq\ (siehe \ref{3F31}) wird scherzhaft behauptet, die Kasse zeige \glqq NRA4EVER\grqq\ (NRA für immer) an. Die NRA ist die \glqq National Rifle Association\grqq , eine US-amerikanische Organisation, die sich dem Sportschießen, dem Training zum sicheren und geschickten Umgang mit Schusswaffen sowie besonders dem Eintreten für Waffenbesitz und die Rechte der Schusswaffenbesitzer verschrieben hat \cite{KostenBaby}.


\section{Haustiere}

\subsection{Knecht Ruprecht}\index{Knecht Ruprecht}\index{Santa's Little Helper}
Knecht Ruprecht (im Original Santa's Little Helper) ist der Hund der Familie Simpson. Er wurde in der Episode \glqq Es weihnachtet schwer\grqq\ (siehe \ref{7G08}) von Homer und Bart nach einem Windhundrennen \glqq aufgelesen\grqq . Knecht Ruprecht wurde von seinem vorhergehenden Besitzer verstoßen, da er bei den Rennen immer letzter wurde.

Bart liebt Knecht Ruprecht über alles, allerdings tauscht er ihn in der Episode \glqq Der tollste Hund der Welt\grqq\ (siehe \ref{4F16}) gegen Laddie\index{Laddie} ein. Er bekommt aber dermaßen starke Sehnsucht nach ihm, dass er Knecht Ruprecht zurückholt.


\subsection{Schneeball I}\index{Schneeball I}
Schneeball I war die erste Katze der Simpsons. Sie wurde 1988 geboren und 1990 von Clogus, dem biersaufenden Bruder des Bürgermeisters überfahren.

\subsection{Schneeball II}\index{Schneeball II}
Nachdem Schneeball I starb, kam die zweite Katze, Schneeball II. Schneeball II wurde in der Folge \glqq Häuptling Knock-A-Homer\grqq\ (siehe \ref{FABF04}) überfahren. Die nächste Katze, Schneeball III, ertrank im Aquarium. Coltrane (Schneeball IV) starb durch einen Sprung aus Lisas Zimmer. Schneeball V wurde wieder Schneeball II genannt. Lisa erhielt Schneeball V von der Katzenlady.


\subsection{Stampfi}\index{Stampfi}
In der Episode \glqq Bart gewinnt Elefant!\grqq\ (siehe \ref{1F15}) hat Bart Simpson einen Elefanten als Haustier. Aufgrund der sehr hohen Futterkosten musste er Stampfi einem Safaripark geben. Homer wollte ihn eigentlich einem Elfenbeinhändler verkaufen. Nachdem Stampfi ihm aber das Leben gerettet hatte, entschied er sich -- sehr zur Freude der Familie -- den Elefanten kostenlos dem Safaripark zu überlassen. Ein zweites Mal ist er in der Episode \glqq Marge -- oben ohne\grqq\ (siehe \ref{DABF18}) zu sehen.

\subsection{Schnuffi}\index{Schnuffi}\label{Schnuffi}
Schnuffi, der Hamster von Lisa, taucht in zwei Episoden auf. Lisa Simpson verglich ihn in der Folge \glqq Die Geburtstagsüberraschung\grqq\ (siehe \ref{7F24}) dabei für ein Schulexperiment die Intelligenz ihres Hamsters Schnuffi mit der ihres Bruders Bart. Ihr Bruder schnitt bei diesem Test schlecht ab. Seinen zweiten Auftritt hat er in der Folge \glqq Der eingebildete Dachdecker\grqq\ (siehe \ref{GABF10}), in welcher er die von Homer errichtete Wasserleitung hinunterrutscht und durch den Briefschlitz auf die Straße hinausgespült wird.

\subsection{Prinzessin}\index{Prinzessin}
Prinzessin tritt in der Episode \glqq Lisas Pony\grqq\ (siehe \ref{8F06}) auf. Lisa bekam es von Homer, weil er glaubte, dass sie ihn nicht mehr lieben würde. Sie wird am Ende der Folge aufgrund der sehr hohen Kosten wieder verkauft.

\subsection{Mojo}\index{Mojo}
Mojo war als Affe ein kurzzeitiges Haustier der Simpsons in der Episode \glqq Die neusten Kindernachrichten\grqq\ (siehe \ref{5F15}). Eigentlich sollte der Affe für Homer Arbeit verrichten, wie Donut klauen. Aber der Affe saß je länger er Umgang mit Homer hatte genau so faul auf dem Sofa, trank Bier und sah fern. Auf Marges Aufforderung hin, hat ihn Homer wieder zurückgebracht. 

\subsection{Zwickie}\index{Zwickie}
Zwickie war der in der Folge \glqq Die große Betrügerin\grqq\ (siehe \ref{AABF03}) der Haushummer der Simpsons, bis Homer ihn zu heiß badete und anschließend verspeiste.



\section{Verwandtschaft}

\subsection{Abraham Jebediah Simpson}\label{AbeSimpson}\index{Simpson!Abraham}\index{Grampa}
Abraham und dessen Eltern wuchsen in einem Land auf, an dessen Namen er sich leider nicht mehr erinnern kann (vermutlich in Europa). Jedenfalls sind sie später nach Amerika ausgewandert und sind dort in die Freiheitsstatue eingezogen. Sie mussten jedoch wieder ausziehen, als der gesamte Kopf mit Müll voll war (\glqq Volksabstimmung in Springfield\grqq , \ref{3F20}). 

Abraham Simpson ist der Vater von Homer Simpson. Er wohnt im Altersheim von Springfield und ist 86 Jahre alt (\glqq Fidel Grampa\grqq, \ref{VABF19}). In der Folge \glqq Grampa ist ganz Ohr\grqq\ (siehe \ref{WABF19}) feiert er seinen 87. Geburtstag. Er ist unter anderem Mitglied der Loge der Steinmetze (eine Parodie auf die Freimaurer). Er ist ziemlich senil. Er kämpfte im Ersten und Zweiten Weltkrieg. Im zweiten Weltkrieg war er Feldwebel bei der Einheit Fighting Hellfish\index{Fighting Hellfish}\footnote{Neben Abraham Simpson und C. M. Burns gehörten noch Sheldon Skinner\index{Skinner!Sheldon}, Arnie Gumble\index{Gumble!Arnie}, Asa Phelps\index{Phelps!Asa}, Iggy Wiggum\index{Wiggum!Iggy}, Milton \glqq Ox\grqq\ Haas\index{Haas!Milton}, Etch Westgrin\index{Westgrin!Etch} und Griff McDonald\index{McDonald!Griff} den Fighting Hellfish an.} (kämpfende Höllenfische) und hätte sogar fast Adolf Hitler getötet. Er erzählt immer gerne -- zumeist frei erfundene -- Anekdoten, die niemand hören will. 

Neben dem ehelichen Homer hat er noch zwei uneheliche Kinder und zwar Herbert Powell\index{Powell!Herbert} und Abbie\index{Abbie}. Herbert ging aus einer Affäre von ihm mit einer Schaustellerin hervor. Zu dieser Zeit war er bereits mit seiner späteren Frau Mona zusammen (\glqq Ein Bruder für Homer\grqq , \ref{7F16}). Abbie zeugte er während eines kurzen Aufenthalts als Soldat in England 1944.

Trotz seines Alters ist Abe immer noch an Frauenbekanntschaften interessiert, so war er in \glqq Die Erbschaft\grqq\ (siehe \ref{7F17}) in die mittlerweile verstorbene Beatrice Simmons und in \glqq Liebhaber der Lady B.\grqq\ (siehe \ref{1F21}) in Marges Mutter Jacqueline Bouvier verliebt. In der Folge \glqq Ein unmögliches Paar\grqq\ (siehe \ref{JABF08}) heiratet er Marges Schwester Selma. Auch wenn seine Neigungen dem weiblichen Geschlecht gelten, ist der der Präsident der Schwulen- und Lesbenvereinigung in Springfield (\glqq Homer, der Auserwählte\grqq , \ref{2F09}). In der Episode \glqq Auf der Flucht\grqq\ (siehe \ref{FABF14}) wird er von dem homosexuellen Raoul\index{Raoul} umgarnt.

Er ist genauso wie sein Freund Jasper ein großer Matlock-Fan\index{Matlock}. Obwohl im moderne Technik und das metrische System Angst bereiten, hat er eine eigne Internetseite: (\glqq Der Dicke und der Bär\grqq, \ref{EABF19}).

In späteren Folgen wird er meist Grampa\index{Grampa} genannt.


\subsection{Herbert Powell}\index{Powell!Herbert}
Herbert Powell ist der Halbbruder von Homer Simpson und Sohn von Abraham Simpson. Nachdem Abraham bei einem Volksfest am Schießstand ein junges Mädchen kennen gelernt hatte, führte dies zu einer kleinen und kurzfristigen Affäre. Ein Jahr später zeigte das Mädchen Abraham ihr Kind Herbert. Sie gaben das Kind im Waisenhaus von Shelbyville\index{Shelbyville} ab. Herbert wurde danach von einer Familie mit Namen Powell adoptiert. Als Homer die Nachricht bekam, dass er einen Halbbruder hat, begann er, ihn zu suchen. Bald fand Homer die Adresse heraus und besuchte ihn sofort in Detroit. Herbert war Chef einer Automobilfirma, doch als er entschied, Homer einen \glqq Durchschnittsbürger-Wagen\grqq\ bauen zu lassen (in das er sein ganzes Vermögen hineinsetzte) und dieses Projekt scheiterte, wurde Herbert arm und obdachlos.

In einer späteren Episode kam er jedoch zurück: Als Homer 2.000 Dollar gewann, stand Herbert in der Tür und schlug vor, ein neues Projekt zu starten, für das er Homers Geld benötigte. Da Homers Familie ihm noch etwas schuldig war, gaben sie ihm das Geld. Er erfand einen \glqq Baby-Sprache-Übersetzer\grqq , der ihm soviel Geld einbrachte, dass er wieder zum Millionär wurde. Homer und Herbert sehen identisch aus, außer dass Herbert noch seine braunen Haare hat und schlanker ist. Im amerikanischen Original leiht der bekannte Schauspieler Danny DeVito\index{DeVito!Danny} Herbert seine Stimme.



\subsection{Abbie}\index{Abbie}
Abbie ist die Halbschwester von Homer Simpson und die Tochter von Abraham Simpson. Während seiner Zeit in der Armee nutzte Abraham die Wirkung seiner Uniform bei Frauen, und hatte eine kurze Affäre mit einem Mädchen namens Edwina. 58 Jahre später bei einem Besuch in England trifft er sie bei seiner Abreise wieder, und lernt seine Tochter kennen. Sie sieht aus wie Homer und scheint auch seine Vorlieben bei der Ernährung zu teilen (\glqq Die Queen ist nicht erfreut!\grqq , \ref{EABF22}).


\subsection{Mona J. Simpson}\index{Simpson!Mona}\label{MonaSimpson}
Mona Simpson ist Homers Mutter und die Ehefrau von Abraham Simpson. In den sechziger Jahren bekommt sie Kontakt zur Hippie-Bewegung und muss nach ihrer Beteiligung an einem Anschlag auf Mr. Burns Labor für biologische Waffen untertauchen; Homer ist da noch sehr klein und konnte sich deshalb kaum an seine Mutter erinnern und dachte sogar, sie wäre tot, da ihm das sein Vater gesagt hatte. Nachdem Mona nach Springfield kam, weil sie dachte, dass Homer gestorben sei und dass eine Beerdigung für ihn stattfinde, traf sie Homer wieder. Denn Homer war nicht verstorben, er hatte lediglich seinen Tod vorgetäuscht, um nicht Samstag Nachmittag für Mr. Burns arbeiten zu müssen (\glqq Wer ist Mona Simpson?\grqq , \ref{3F06}). Da Homers Tod ein Irrtum war, konnten sie sich besser kennen lernen. Allerdings tauchte Mona bald wieder unter, da sie von der Polizei gesucht wurde. Sie tritt aber in der späteren Episode \glqq Wiedersehen nach Jahren\grqq\ (siehe \ref{EABF18}) wieder auf. Auf der Flucht vor der Polizei hatte sie die Decknamen Mona Stevens\index{Stevens!Mona}, Martha Stewart\index{Stewart!Martha}, Penelope Olsen\index{Olsen!Penelope} und Muddy Mae Suggins\index{Suggins!Muddy Mae} \cite{Wikipedia}. In der Episode \glqq Lebwohl, Mona\grqq\ (siehe \ref{KABF12}) stirbt sie. In ihrem Testament verfügt sie, dass ihre Asche auf dem höchsten Punkt im Springfielder Nationalpark verstreut werden soll. Wie sich herausstellt, befindet sich dort eine Raketenabschussrampe, die Mr. Burns dazu nutzt, seinen radioaktiven Giftmüll in den Amazonasregenwald zu schießen. Ihre Asche setzt das Raketenlenksystem außer Betrieb und Homer kann die Abschussanlage zerstören.

Abe sagt in der Episode \glqq Air Force Grampa\grqq\ (siehe \ref{TABF13}), er habe Mona in einer Air-Force-Bar kennengelernt. Dort habe sie als Kellnerin gearbeitet.

\subsection{Patty Bouvier}\index{Bouvier!Patty}
Patty Bouvier ist eine der beiden älteren Schwestern von Marge Simpson und 41 Jahre alt. Sie arbeitet in der Führerschein-Zulassungsstelle von Springfield. Außerdem ist sie Kettenraucherin. Sie liebt, genauso wie ihre Zwillingsschwester, MacGyver über alles. Niemand darf etwas Negatives über ihn sagen. Sie hatte mal ein Liebes\-ver\-hält\-nis mit Rektor Skinner, lehnte seinen Heiratsantrag aber ab.

In der Episode \glqq Drum prüfe, wer sich ewig bindet\grqq\ (siehe \ref{GABF04}) stellt sich heraus, dass Patty lesbisch ist. Diese Folge der 16ten Staffel ist eine Anspielung auf den Kleinkrieg um die Legalisierung gleichgeschlechtlicher Ehen unter anderem in San Francisco und löste in den USA bei der Religiösen Rechten (\glqq Religious Right\grqq ) große Empörung aus, da sie, wie alle Folgen, im Vorabendprogramm für Kinder lief. In der Episode \glqq Frauentausch\grqq (siehe \ref{HABF08}) hat sie eine Affäre mit Verity Heathbar, einer angesehen Professorin und sorgt dafür, dass diese ihren Mann verlässt.


\subsection{Selma Bouvier}\index{Bouvier!Selma}\label{SelmaBouvier}
Selma Bouvier ist die Zwillingsschwester von Patty und arbeitet ebenfalls in der Führerschein-Zulassungsstelle und ist ebenso Kettenraucherin. Sie ist um zwei Minuten älter als Patty (\glqq Der Heiratskandidat\grqq , \ref{7F15}). Im Gegensatz zu Patty hatte Selma schon viele Affären und war mit Troy McClure, Sideshow Bob, Lionel Hutz, Disco Stu, Abraham Simpson und Fat Tony verheiratet. Laut Marges Aussage konnte sie eigentlich nur Disco Stu als einen ihrer Ehemänner leiden.

Seit einem Unfall mit einem Böller in ihrer Kindheit hat Selma keinen Geruch- und Geschmackssinn mehr, was ihr bei ihrer Kurzehe mit Sideshow Bob fast das Leben gekostet hätte, da sie das Gas nicht riechen konnte, mit welchem Sideshow Bob sie umbringen wollte (\glqq Bis dass der Tod euch scheidet\grqq , \ref{8F20}).

Da sie schon zu alt ist, um ein eigenes Kind kriegen zu können, adoptiert sie mit Homers Hilfe die kleine Ling\index{Ling} aus China (\glqq Der lächelnde Buddha\grqq , \ref{GABF06}).


\subsection{Jacqueline Bouvier}\index{Bouvier!Jacqueline}\label{JacquelineBouvier}
Jacqueline ist die Mutter von Patty, Selma und Marge. Sie war mit Clancy Bouvier\index{Bouvier!Clancy}, einem männlichen Flugbegleiter, verheiratet, der bei einem Achterbahnunglück verstarb. Sie hatte eine Schwester Gladies\index{Gladies}, die den Simpsons ihr gesamtes Hab und Gut vererbte.

In der Folge \glqq Liebhaber der Lady B.\grqq\ (siehe \ref{1F21}) hätte sie beinahe Charles M. Burns geheiratet, was allerdings Abe Simpson in letzter Sekunde verhindern konnte.

Ihr vermutlich peinlichster Moment war, als sie im Jahre 1923 wegen \glqq Erregung öffentlichen Ärgernisses\grqq\ verhaftet wurde \cite{SpringfieldAt}.

In der Episode \glqq Der weinende Dritte\grqq\ (siehe \ref{MABF13}) feiert sie bei Marge und Homer ihren 80. Geburtstag.

\section{Im Atomkraftwerk}

\subsection{Charles Montgomery Burns}\index{Burns!Charles Montgomery}\label{CMBurns}
Charles Montgomery Burns ist der reichste Mann in Springfield, Besitzer des örtlichen Atomkraftwerkes und somit Homers Arbeitgeber. Sein voller Name lautet: Charles Montgomery Plantagenet Schicklgruber Burns. Laut eigenen Angaben hat er von seinen Eltern 100 Milliarden Dollar geerbt (siehe \glqq Süßer die Glocken nie tingeln\grqq, \ref{ZABF0}). Er hat in Yale studiert und gehörte dort dem Ringerteam an. Er hat einen Lehrstuhl an der Springfielder Universität (siehe \glqq Homer an der Uni\grqq, \ref{1F02}). 

Er vergisst immer wieder Homers Namen.
\begin{description}
	\item[Burns:] \glqq Wer ist dieser Mann, Smithers?\grqq
	\item[Smithers:] \glqq Das ist Homer Simpson, Sir, Sicherheitsbeauftragter aus Sektor 7G.\grqq
	\item[Burns:] \glqq Simpson, hä?\grqq
\end{description}
Er ist unglaublich dünn, unglaublich böse und unglaublich alt, wobei es über sein Alter widersprüchliche Aussagen gibt. In den Episoden \glqq Wer erschoss Mr. Burns? -- Teil 1\grqq\ (siehe \ref{2F16}) und \glqq Gloria, die wahre Liebe\grqq\ (siehe \ref{CABF18}) ist er 104 Jahre alt und in der Episode \glqq Karriere mit Köpfchen\grqq\ (siehe \ref{7F02}) ist er 81 Jahre alt. Er ist körperlich sehr schwach und ohne seinen Assistenten Waylon Smithers nicht überlebensfähig. Er wurde einmal von Maggie angeschossen und überlebte. In der Folge \glqq Wenn ich einmal reich wär!\grqq\ (siehe \ref{BABF08}) war er beim Arzt und hatte, so wie es scheint, jede erdenkliche Krankheit (sogar bis dato noch unbekannte Krankheiten), die sich aber gegenseitig aufhoben und daher nicht ausbrechen konnten.

Mr. Burns muss jeden Freitag eine bestimmte Untersuchung machen, um dem Tod eine weitere Woche zu entgehen. Danach fühlt er sich immer sehr verwirrt und wurde versehentlich für einen Außerirdischen gehalten. Das rief die imaginären FBI-Agenten Scully und Mulder aus Akte X\index{Akte X} auf den Plan, da Mr. Burns zusätzlich im Dunkeln grün leuchtet. Er selbst erklärt es damit, dass man \glqq ein gesundes grünes Leuchten\grqq\ bekommt, wenn man sein Leben lang in einem Atomkraftwerk arbeitet.

Mr. Burns hatte seiner Mutter ihre Affäre mit Präsident Theodore Roosevelt nie verziehen. Außerdem hat er einen unehelichen Sohn, Larry Burns\index{Burns!Larry}. Obwohl Burns nie mit der Mutter seines Sohnes verheiratet war, bleibt unklar, wieso Larry mit Nachnamen Burns heißt.

Während einer Geburtstagsparty in der Folge \glqq Kampf um Bobo\grqq\ (siehe \ref{1F01}) für ihn, spielte die Band Ramones\index{Ramones} \cite{OffizielleHomepage}. In dieser Episode ist weiterhin zu erfahren, dass er einen Bruder mit Namen George hat und seine leiblichen Eltern schon früh verlassen hat, um zu einem verschrobenen alten Milliardär zu ziehen.

Mr. Burns ist der Gründer und Intendant der Oper von Springfield (\glqq Homerotti\grqq , \ref{JABF18}). Als er erfährt, dass Homer eine klassische Singstimme hat, wenn er auf dem Rücken liegt, engagiert er ihn.

Mr. Burns ist nicht nur Mitglied der Steinmetze sondern auch der Freimaurer. Das Auge und die Pyramide, die auf dem 1-Dollar-Schein abgebildet sind, gehören ihm, behauptet er in der Episode \glqq Auf der Jagd nach dem Juwel von Springfield\grqq\ (siehe \ref{LABF04}).

Er war Mitglied der Schutzstaffel (SS\index{SS}) der NSDAP (\glqq Jailhouse Blues\grqq , \ref{MABF08}).

Er wurde vom People-Magazin einmal zum Sexiest Man Alive gewählt (\glqq Silly Simpsony\grqq , \ref{SABF02}).

Gegenüber Waylon Smithers behauptet er, der Besitzer des Atomkraftwerks in Tschernobyl zu sein (\glqq Mr. und Mrs. Smithers\grqq , \ref{WABF03}). Er droht sogar Waylon, in dahin zu versetzen, falls er es nicht schafft, dass Homer eine Klage gegen ihn zurückzieht.

\subsection{Waylon Smithers}\index{Smithers!Waylon}\label{WaylonSmithers}
Waylon Smithers ist der persönliche Assistent des Atomkraft-Magnaten Charles Montgomery Burns und erfüllt seine Aufgabe mit großem Eifer. Er hat an Weihnachten Geburtstag (siehe \glqq Süßer die Glocken nie tingeln\grqq, \ref{ZABF0}). Dass der von ihm so verehrte Chef seine aufopferungsvollen Dienste als selbstverständlich betrachtet, schränkt seine Loyalität nicht im Mindesten ein. Smithers erscheint als extrem unterwürfiger Mensch, seine masochistische Veranlagung wird kaum verhohlen. Schon Smithers Vater arbeitete im Atomkraftwerk für Burns, jedoch lies Smithers sen. sein Leben, um die Bewohner Springfields vor einer Atomexplosion zu retten. Burns vertuschte den Tod und kümmerte sich um Smithers junior (\glqq Aus dunklen Zeiten\grqq , \ref{CABF21}). Anfangs versteckt, später jedoch immer öfter und deutlicher, tauchen in der Serie Anspielungen auf seine Homosexualität und Liebe zu seinem Chef auf (beispielsweise in \glqq Der Tag der Abrechnung\grqq , \ref{5F05}). Allerdings war er einmal verheiratet, wie in der Episode \glqq Ehegeheimnisse\grqq\ (siehe \ref{1F20}) zu erfahren ist. Er war bei der Marine, bis er wegen homophober Handlungen unehrenhaft entlassen wurde. Er hatte von Kameraden unsittliche Bilder angefertigt (\glqq Seid nett zu alten Leuten\grqq , \ref{AABF16}). Er sammelt Malibu-Stacy-Puppen (das Springfield-Gegenstück zu Barbie). In der Folge \glqq Der Versager\grqq\ (siehe \ref{7G03}) tritt Smithers als Schwarzer auf. Er wurde versehentlich mit der falschen Farbe animiert.

Waylon hat eine Laktoseintoleranz (siehe \glqq Silly Simpsony\grqq\ , \ref{SABF02}) und er reagiert auf Bienenstiche allergisch (siehe \glqq 22 Kurzfilme über Springfield\grqq , \ref{3F18}).

Mike Reiss schreibt in seinem Buch \glqq Springfield Confidential\grqq\ (\cite{Reiss19}), dass der Produzent Sam Simon in der zweiten Staffel zu den Autoren gesagt hat \glqq Smithers ist schwul! Sagt das nicht direkt, macht keine Witze darüber, aber behaltet es bei seinen Auftritten im Kopf.


\subsection{Lenny Leonard}\index{Leonard!Lenny}\label{LennyLeonard}
Lenford Leonard -- meistens Lenny genannt -- ist ein guter Freund von Homer, der auch im Atomkraftwerk arbeitet und abends in Moes Bar rumhängt. Ursprünglich stammt er aus Chicago (\glqq Und der Mörder ist\dots\grqq , \ref{EABF01}). Sein Vater war der Herausgeber des Life-Magazins (siehe \glqq Daddicus Finch\grqq, \ref{YABF01}). Seine leibliche Mutter will mit ihm nichts mehr zu tun haben (siehe \glqq Mission Simpossible\grqq, \ref{YABF08}).

Er hat einen Magister in Atomphysik. Seine (stark ausgeprägte) freundschaftliche Beziehung zu Carl wird häufig erwähnt. Zudem scheint er viele Fans in der Bevölkerung Springfields zu haben (\glqq Alle mögen Lenny.\grqq ). Eine schlechte Angewohnheit von ihm ist es, andere Personen ständig zu unterbrechen. Lenny ist oder war verheiratet, da Carl in der Episode \glqq Auf dem Kriegspfad\grqq\ (siehe \ref{EABF16}) sagt, er habe auf seiner Hochzeit gesungen. Er hat eine Schwester. Außerdem ist er Buddhist. Als Mr. Burns in der Folge \glqq Der alte Mann und Lisa\grqq\ (siehe \ref{4F17}) vorübergehend pleite war, wurde er von der Bank mit der Leitung des Atomkraftwerks beauftragt \cite{SpringfieldAt}.

Lenny scheint über eine künstlerische Ader zu verfügen. Er war Gitarrist in der Grunge-Band \glqq Sadgasm\index{Sadgasm}\grqq . Neben ihm spielten noch Carl, Homer und Lou, der Polizist, in der Band, die den Musikstil Grunge erfand (siehe \glqq Die wilden 90er\grqq , \ref{KABF04}). Später ist er als Schauspieler mit einem kleinen Auftritt in dem Horrorfilm \glqq The Re-Deaden\-ing\footnote{Die Wiedertotmachung}\grqq\ (\glqq Rat mal, wer zum Essen kommt\grqq , \ref{FABF08}) zu sehen. In diesem Film ist eine Puppe namens \glqq Baby Button Eyes\grqq\ vom Geist eines psychopathischen Killers besessen. Im Film spielt Lenny einen Gärtner, der von der Puppe ermordet wird \cite{WikiEnLenny}. Er wirkt außerdem als Statist in der Everyman-Verfilmung mit (siehe \glqq Everyman begins\grqq, \ref{LABF13}). Er und Carl wirken in der Folge \glqq Es lebe die Seekuh!\grqq\ (siehe \ref{GABF18}) in dem Pornofilm mit, der in Homers Haus gedreht wird. Er scheint ein Faible für Horrorgeschichten zu haben, denn in der Folge \glqq Ballverlust\grqq\ (siehe \ref{JABF11}) gibt er an, dass er mehrere Bestsellerthriller, u.a. \glqq The Murderer Did It\grqq\ (Der Mörder war es), geschrieben hat. Neben seiner Schauspielerei liest er auch Homers Buch \glqq America: Love it or I'll punch you!\grqq\ als Hörbuch ein (siehe \glqq Im Rausch der Macht\grqq , \ref{PABF03}).

Lenny gehört der Geheimorganisation der Steinmetze an. Er trägt dort die Nummer 12 (\glqq Homer, der Auserwählte\grqq, \ref{2F09}).

Lenny ist wohl der republikanischen Party zugewandt, da er in der Episode \glqq Geächtet\grqq\ (siehe \ref{FABF17}) auf seiner Brust eine Tätowierung zeigt, die offenbart, dass er 1996 den republikanischen Kandidaten zur Präsidentschaftswahl, Bob Dole, unterstützt hat.

Lenny war drei Jahre im Gefängnis, weil er einen von Moes Ratschlägen befolgt hat, wie er in der Folge \glqq Große, kleine Liebe\grqq\ (siehe \ref{LABF06}) erzählt. In der Episode \glqq Ein Schweinchen namens Propper\grqq\ (siehe \ref{WABF06}) gibt er an, dass er ein Jahr im Gefängnis gesessen habe.


\subsection{Carl Carlson}\index{Carlson!Carl}\label{CarlCarlson}
Carl Carlson ist ebenfalls ein guter Freund von Homer, der mit ihm im Atomkraftwerk arbeitet und abends in Moes Bar rumhängt. Nachdem der Supervisor Ted in den Ruhestand geht, wird er dessen Nachfolger und somit Vorgesetzter von Homer und Lenny. Oft taucht er zusammen mit Lenny auf; die beiden sind gute Freunde. Beide schneiden sich gegenseitig die Haare, das ein Geheimnis bleiben soll. Er spielte wie Lenny in der Grunge-Band \glqq Sadgasm\grqq . Er war deren Bassist. Carl hat ebenfalls einen Magister in Atomphysik und ist -- genau wie Lenny -- Buddhist. Er hat einen IQ von 214 (siehe \glqq Enter the Matrix\grqq , \ref{SABF06}). Carl ist verheiratet, was er in Episode \glqq Nur für Spieler und Prominente\grqq\ (siehe \ref{AABF08}) verrät. Aufgewachsen ist er in Island, was er in der Episode \glqq Nacht über Springfield\grqq\ (siehe \ref{EABF11}) andeutet. In der Folge \glqq Die Legende von Carl\grqq\ (\ref{RABF14}) fliegt er nach Island zu seinen Adoptiveltern. Er hat eine Schwester. Ab der Folge \glqq Die Stadt der primitiven Langweiler\grqq\ (siehe \ref{AABF18}) hat er Diabetes. Carl gehört der Geheimorganisation der Steinmetze an. Er trägt dort die Nummer 14 (\glqq Homer, der Auserwählte\grqq, \ref{2F09}). Als Homer in der Episode \glqq Sicherheitsdienst Springshield\grqq\ (siehe \ref{DABF17}) einen privaten Sicherheitsdienst gründet, arbeiten er und Lenny in Homers Firma mit.

In der Folge \glqq Homersche Eröffnung\grqq\ (siehe \ref{WABF08}) behauptet er, dass der norwegische Schachweltmeister Magnus Carlsen\index{Carlsen!Magnus} sein Cousin sei.

\subsection{Charlie}\index{Charlie}
Charlie ist der Inspektor für gefährliche Strahlung im Atomkraftwerk. Er hat eine Schwester, welche ein Holzbein hat und das er sich zum Softballspielen in der Episode \glqq Der Wunderschläger\grqq\ (siehe \ref{8F13}) ausleiht.
In der Folge \glqq Die Trillion-Dollar-Note\grqq\ (siehe \ref{5F14}) wurde er verhaftet, weil er drohte, die örtlichen Verantwortlichen wegen des langsamen Fortschritts von HDTV zu verprügeln.

In der Episode \glqq Homer liebt Mindy\grqq\ (siehe \ref{1F07}) wurde er von Mr. Burns durch eine Röhre in ein orientalisches Land, vermutlich Indien, geschickt, wo er für feixende Männer mit Fesen und Turbanen tanzen musste.
 
 
\subsection{Frank Grimes}\index{Grimes!Frank}
Ein Pechvogel, der sich im Leben stets alles hart erkämpfen musste und schließlich eine Anstellung im Springfielder Atomkraftwerk findet. Hier lernt er Homer kennen, der trotz seiner Inkompetenz ein besseres Leben führt als Frank. Schließlich dreht er durch und hält sich für Homer Simpson. Dabei kommt er bei einer Begegnung mit einem Stromkasten ums Leben:
\begin{description}
	\item[Frank:] [Vor einem Stromkasten mit Warnschild] \glqq Äußerste Hochspannung? Ach was, ich brauch' keine Schutzhandschuhe, denn ich bin Homer Simp\dots \grqq 
\end{description}
Er hinterließ einen Sohn namens Frank Grimes jun., der aus einer Affäre mit einer Prostituierten hervorging. Dieser versuchte später (in der Folge \glqq Und der Mörder ist\dots\grqq , \ref{EABF01}), den Tod seines Vaters zu rächen, indem er mehrere (erfolglose) Anschläge auf Homer Simpson verübte.

\subsection{Karl}\index{Karl}
Karl, der Assistent von Homer in der Episode \glqq Karriere mit Köpfchen\grqq\ (siehe \ref{7F02}), ist ein sehr guter Motivator. Mit seiner Hilfe gelang es Homer, Selbstvertrauen für seinen gehobenen Posten im Atomkraftwerk zu erlangen. Auch nahm Karl die Schuld auf sich, als herauskam, dass Homer ein Haarwuchsmittel (Dimoxinil\index{Dimoxinil}) von der Firmenkrankenkasse bezahlen hatte lassen; aufgrund dessen wurde Karl von Smithers entlassen.


\subsection{Zutroy}\index{Zutroy}
Zutroy ist wahrscheinlich ein illegaler Angestellter im Atomkraftwerk. Er spricht kein Wort englisch und ist vermutlich Türke oder Kurde. Er ist in der Folge \glqq Homer liebt Mindy\grqq\ (siehe \ref{1F07}) zu sehen.
\begin{description}
	\item[Burns:] Arbeite fleißig mein lieber Zutroy, dann bekommst du jeden Tag eine blinkende Münze.
	\item[Zutroy:] Ich nix verstehen Mr. Bruns! 
\end{description}



\section{Flanders}

\subsection{Ned}\index{Flanders!Ned}\label{NedFlanders}
Nedward \glqq Ned\grqq\ Flanders ist der Ehemann von Maude Flanders, Vater von Rod und Todd Flanders und Nachbar von Homer Simpson. Er wurde als hyperaktives Kind von seinen Eltern (die Beatniks waren) in eine Klinik gesteckt. Da er zum Teil auch aggressiv war, beschloss sein Therapeut, Dr. Foster\index{Foster!Dr.}, ihm 18 Monate ununterbrochen auf den Po zu \glqq klapsen\grqq\ (\glqq Der total verrückte Ned\grqq , \ref{4F07}). Ned lebt als unglaublich frommer Mensch in Springfield, obwohl er atheistisch aufgewachsen ist (siehe \glqq Himmlische Geschichten\grqq, \ref{XABF17}). Homer nutzt seine Freundlichkeit gnadenlos aus. Ned spendet oft für wohltätige Zwecke und ist Mitglied der Kirchengemeinde. Da er ständig beim Priester anruft und Fragen stellt (z.B. was Gott meint, wenn er eine Nadel verschluckt), geht er sogar ihm auf die Nerven. Ned ist Inhaber des Leftoriums\index{Leftorium}, einem Laden für Linkshänder-Artikel. Obwohl Ned eigentlich keinen Alkohol trinkt, hat er nach einer durchzechten Nacht mit Homer Simpson in Las Vegas erneut geheiratet, ohne sich jedoch zuvor von Maude scheiden zu lassen (\glqq Wir fahr'n nach\dots Vegas\grqq , \ref{AABF06}). Ned lebt also de facto in Bigamie bis Maudes Tod. 

Seine Telefonnummer ist übrigens 555-8904. Ned ist ein extremer Beatles-Fan (\glqq die sind besser als Jesus!\grqq ) und besitzt einen Raum voller Fan-Artikel, zum Beispiel die Anzüge, die sie bei der Ed-Sullivan-Show getragen haben.
Ned Flanders ist zum einen als extremer Kontrast zu Homer eine wichtige Figur in der Serie, außerdem ist er als Parodie auf christliche Fundamentalisten in den USA angelegt.

Nach dem Tod seiner Frau hatte er in der Episode \glqq Ein Stern wird neu geboren\grqq\ (siehe \ref{EABF08}) eine Affäre mit der bekannten Schauspieler Sara Sloane. In der Episode \glqq Schlaflos mit Nedna\grqq\ (siehe \ref{PABF15}) heiratet er heimlich die Grundschullehrerin Edna Krabappel.

Nach eigener Aussage spricht er fließend Aramäisch (siehe \glqq Links liegen gelassen\grqq, \ref{XABF12}).

Als junger Mann arbeitete er als Trampolin-Verkäufer (siehe \glqq Himmlische Geschichten\grqq, \ref{XABF17}).


\subsection{Maude}\index{Flanders!Maude}
Maude ist die brünette Ehefrau von Ned Flanders, die Mutter von Rod und Todd und lebt in Springfield. Sie ist, wie ihr Ehemann, sehr religiös, jedoch übertreibt sie es nicht so wie ihr Mann. Sie unterstützt ihren \glqq Neddie\grqq\ aber tatkräftig bei der Etablierung seines Ladens für Linkshänder. Maude Flanders kommt durch einen Unfall an einer Rennstrecke ums Leben: Sie fällt von der letzten Reihe einer Zuschauertribüne, weil sie von einem T-Shirt, dass aus einer Kanone geschossen wurde, getroffen worden ist. Dadurch stürzt sie über das Geländer und verletzt sich tödlich.

In der späteren Episode \glqq Wunder gibt es immer wieder\grqq\ (siehe \ref{CABF15}) errichtete Ned ihr zu Ehren sogar einen Vergnügungspark namens \glqq Praiseland\grqq , der allerdings wegen einer undichten Gasleitung, von der alle Besucher Halluzinationen bekommen hatten und es für ein Wunder hielten, geschlossen wurde.

Hintergrund für Ihren Serientod sind übrigens Streitigkeiten um das Gehalt (je Folge) für die amerikanische Original-Sprecherin.

\subsection{Todd}\index{Flanders!Todd}\label{ToddFlanders}
Eines der Kinder der Flanders. In den meisten Folgen ist er der größere der beiden, doch da sind die Produzenten im Laufe der Zeit selbst durcheinander gekommen. In der Folge \glqq Der Wettkampf\grqq\ (siehe \ref{7F08}) ist er beispielsweise der Kleinere. Und obwohl er der Kleinere ist, sagt der Moderator des Minigolftuniers, dass er zehn Jahre alt sei. Gemäß der Episode \glqq Es ist ein Todd entsprungen\grqq\ (siehe \ref{QABF09}) ist er sechs Jahre alt und das jüngere Kind. Sein zweiter Vorname lautet Homer, da Homer Maude bei der Geburt unterstützt hat. Bei beiden Kindern wird einige Male auf eine homosexuelle Veranlagung angespielt. In der Episode \glqq Schau heimwärts, Flanders\grqq\ (siehe \ref{GABF15}) behauptet Todd, er habe in Humbleton eine Freundin gefunden. Die beiden Kinder lieben gewaltlose Bibelspiele und lassen sich als Gutenachtgeschichte von Ned Passagen aus der Bibel erzählen.

\subsection{Rod}\index{Flanders!Rod}
Eines der Kinder der Flanders, meistens der kleinere. Er leidet an Diabetes wie in der Episode \glqq Nach Kanada der Pillen wegen\grqq\ (siehe \ref{FABF16}) zu erfahren ist. Wie sein Bruder ist er sehr \glqq kirchentreu\grqq . In der Folge \glqq König der Berge\grqq\ (siehe \ref{5F16}) ist beispielsweise Rod der größere der Flanders Söhne.

\section{Schule}

\subsection{Seymour Skinner}\index{Skinner!Seymour}\index{Tamzerian!Armin}\label{SeymourSkinner}
Seymour Skinner ist der Rektor der Springfielder Grundschule. Er ist sehr streng und ordentlich. Er lebt immer noch bei seiner Mutter Agnes, die ihn sehr bevormundet. Seymour hatte eine Affäre mit mit der Lehrerin Edna Krabappel, die er in der Episode \glqq Hochzeit auf Klingonisch\grqq\ (siehe \ref{FABF12}) heiraten wollte. Früher hatte er ein kurzes Verhältnis mit Patty Bouvier. Sein richtiger Name ist Armin Tamzerian. In der Folge \glqq Alles Schwindel\grqq\ (siehe \ref{4F23}) taucht auf einmal der richtige Seymour Skinner auf, der auf einer Kundschafter-Mission im Vietnamkrieg ums Leben gekommen sein soll, aber nur gefangengenommen wurde. Armin Tamzerian hatte den Auftrag, der Mutter von Seymour Skinner die Nachricht des Todes ihres Sohns zu überbringen. Doch Armin brachte es nicht übers Herz und gab sich stattdessen als ihr Sohn aus. Auch wenn Agnes wusste, dass er nicht ihr leiblicher Sohn ist, akzeptierte sie ihn. Als der richtige Seymour Skinner auftaucht, wünschen sich alle, dass Armin wieder Anges Sohn sei und so überschreibt Richter Snyder letztendlich Seymour Skinners Namen, seine Vergangenheit, Gegenwart und Zukunft an Armin Tamzerian.

Er scheint auch eine künstlerische Ader zu haben, denn er gehörte neben Homer, Barney und Apu den \glqq Überspitzen\grqq\ an und gewann mit diesen einen Grammy. Seine geheime Leidenschaft ist das Kino. Er schrieb schon mehrere Drehbücher, die alle von den Studios abgelehnt worden sind. Außerdem produzierte mit seiner Produktionsfirma \glqq chalmskin\grqq , die er gemeinsam mit Oberschulrat Chalmers betreibt, Filme von Lisa Simpson und Nelson Muntz. Beide Filme wurden erfolgreich auf dem Sundance Filmfestival gezeigt.

\subsection{Edna Krabappel}\index{Krabappel!Edna}\label{Krabappel}
Edna Krabappel ist Bart Simpsons Grundschullehrerin in der vierten Klasse. Häufig ist sie von ihm und seinen Streichen frustriert und sie scheint doch gewisse freundschaftliche Sympathien für ihn zu hegen. Sie scheint von ihrem Beruf frustriert zu sein, zumindest sieht man sie häufig seufzend und stöhnend, obwohl sie in der Episode \glqq Der Eignungstest\grqq\ (siehe \ref{8F15}) angibt, in Harvard studiert zu haben. Andererseits hat sie es mit einer Klasse, in der Bart Simpson, Milhouse van Houten und Nelson Muntz sitzen, sicherlich nicht leicht.

Seit Edna von ihrem Mann verlassen wurde, ist sie ein einsamer Single und sehnt sich nach Liebe. Als sie nach Springfield gezogen ist, hatte sie eine kurze Affäre mit Moe Szyslak (\glqq Die scheinbar unendliche Geschichte\grqq , \ref{HABF06}). Im Laufe der Serie hat sich aber eine Beziehung zu Schuldirektor Seymour Skinner entwickelt. Diese verläuft nicht immer harmonisch, vor allem da sie sich von dessen fürsorglichen und despotischer Mutter gestört fühlt. In der Folge \glqq Lehrerin des Jahres\grqq\ (siehe \ref{EABF02}) macht er ihr jedoch einen Heiratsantrag. Weitere Affären hatte sie u.a. mit Hausmeister Willie und Oberschulrat Chalmers (siehe Episode \glqq Eine Taube macht noch keinen Sommer\grqq , \ref{NABF02}). In der Episode \glqq Schlaflos mit Nedna\grqq\ (siehe \ref{PABF15}) heiratet sie heimlich Ned Flanders.

Edna hat in Harvard studiert (siehe \glqq Der Eignungstest\grqq, \ref{8F15}).


\subsection{Elizabeth Hoover}\index{Hoover!Elizabeth}
Miss Hoover unterrichtet in der Grundschule von Springfield die 2. Klasse. Somit ist sie unter anderem die Lehrerin von Lisa und Ralph Wiggum, dem Sohn des Polizeichefs. Sie hat ein Nervenproblem (Hypochondrie) und scheint von ihrem Job frustriert, oft fordert sie die Schüler auf, die letzten Minuten des Unterrichts ruhig dazu sitzen und konzentriert ins Leere zu starren, anstatt sie zu unterrichten. Viel mehr ist über sie jedoch nicht bekannt. Sie hat braunes, in älteren Folgen blaues Haar und trägt eine Brille. Angeblich ist sie Alkoholikerin.

\subsection{Dewey Largo}\index{Largo!Dewey}\label{DeweyLargo}
Der 53-jährige Dewey Largo ist Musiklehrer und Leiter der Schulband. Er ist im Vorspann zu sehen, taucht in den Episoden aber nur sehr selten auf. Sein musikalisches Talent scheint eher gering zu sein, da die Band fast immer nur Stars and Stripes forever probt, wenn man sie sieht. Sobald Mr. Largo den Raum verlässt, spielt die Band klassische Musik, was sie die \glqq verbotene Musik\grqq\ nennen.

In der Folge \glqq Ein perfekter Gentleman\grqq\ (siehe \ref{HABF05}) ist er der Schulhausmeister, während Willie als Empfangschef im Restaurant \glqq Gilded Truffle\grqq\ arbeitet. In der Episode \glqq Milhouse lebt hier nicht mehr\grqq\ (siehe \ref{FABF07}) arbeitet er bei einer Umzugsfirma.

\subsection{Gary Chalmers}\index{Chalmers}\label{GaryChalmers}
Oberschulrat Gary Chalmers taucht immer wieder zu Inspektionen der Springfield Elementary auf. Er ist für 14 Schulen zuständig. Die penible Ordnung von Rektor Skinner endet jedoch regelmäßig im Chaos, meist durch Bart Simpson verursacht, sodass Skinner mit verlegenen Ausreden die Lage zu retten versucht. Einmal sperrt Skinner alle Rowdys unter einem Vorwand in einen Kellerraum, als er von einem Besuch Chalmers erfährt.

In der Episode \glqq Pranksta Rap\grqq\ (siehe \ref{GABF03}) sagt er, dass er verheiratet ist. Mittlerweile ist er allerdings Witwer (siehe \glqq Bart ist nicht tot\grqq, \ref{XABF19}). Er hat eine Tochter namens Shauna (siehe \glqq Ihr Kinderlein kommet\grqq , \ref{SABF14}). Trotzdem sieht man ihn öfters mit Agnes Skinner ausgehen, wie beispielsweise in \glqq Allgemeine Ausgangssperre\grqq\ (siehe \ref{AABF07}). Ferner hatte er eine Affäre mit der Lehrerin Edna Krabappel. Er hat außerdem eine Tochter, die Unterwäsche verkauft (\glqq Beim Testen nichts Neues\grqq , \ref{LABF02}).
In der Folge \glqq Down by Lisa\grqq\ (siehe \ref{KABF11}) produziert er mit Rektor Skinner die Filme von Lisa und Nelson, die auf dem Sundance Filmfestival mit großem Erfolg gezeigt werden.

Garys Vater war Psychologe (siehe \glqq Das Schweigen der Rowdys\grqq , \ref{TABF15}).

\subsection{Leopold}\index{Leopold}
Leopold ist ein Mitarbeiter von Oberschulrat Chalmers und scheint, Kinder zu hassen: \glqq Sperrt die Lauscher auf, ich werde das nur einmal sagen\dots !\grqq\ Leopold kommt jedoch nur in sehr wenigen Episoden vor, unter anderem als Rektor Skinner seinen Job verliert und als Mitglied der \glqq Steinmetze\grqq\ in \glqq Homer der Auserwählte\grqq\ (siehe \ref{2F09}). Er wird immer mit einem sehr grimmigen Gesichtsausdruck dargestellt und spricht stets mit zusammengepressten Zähnen. Leopold hat in keiner Episode eine tragende Rolle und soll nur eine Parodie auf die Kinderfeindlichkeit der Schulbürokratie darstellen.

\subsection{Mr. Bergstrom}\index{Bergstrom}
Mr. Bergstrom war in der Episode \glqq Der Aushilfslehrer\grqq\ (siehe \ref{7F19}) die Vertretung von Miss Hoover, als diese scheinbar an einer sehr schlimmen Krankheit litt. Seine Lehrmethoden unterscheiden sich radikal von denen anderer Lehrer. Den Klassen erzählt er u.a. Geschichten des Wilden Westens. Er schämt sich auch nicht öffentlich zu weinen, was ihm den Spott von Homer und Bart einbringt. Lisa hingegen ist ganz begeistert von ihm und als Mr. Bergstrom aufbricht, um armen Menschen in Capital City zu helfen, ist sie dementsprechend deprimiert.

\subsection{Dr. J. Loren Pryor}\index{Pryor!Dr. J. Loren}
Dr. J. Loren Pryor ist der Schulpsychologe der Grundschule von Springfield. Homer und Marge Simpson müssen ihn konsultieren, als Bart wieder einmal etwas angestellt hat (Homer: \glqq Hey, Dr. J!\grqq ). Er schickte Bart in der Episode \glqq Bart wird ein Genie\grqq\ (siehe \ref{7G02}) auf eine Hochbegabtenschule, weil Bart bei einem IQ-Test seine Arbeit mit der seines Klassenkameraden Martin Prince vertauscht hatte und deshalb das beste Ergebnis in der Klasse bekam.

In der Episode \glqq Die Saxophon-Geschichte\grqq\ (siehe \ref{3G02}), die in einer Zeit spielt, als Bart gerade eingeschult wird, trifft er erstmals auf die Simpsons. Bei seinem ersten Auftritt in der Folge \glqq Bart wird ein Genie\grqq , die natürlich viel früher produziert wurde, lernt er die Simpsons erst kennen. Das ist eine typische Kontinuitätsunstimmigkeit in der Serie, die in dieser Form recht häufig auftritt.


\subsection{Otto Man}\index{Man!Otto}\label{OttoMan}
Otto Man wurde gemäß seinem Führerschein am 18. Januar 1963 geboren und ist der Schulbusfahrer, der ständig Kopfhörer trägt, Gitarre spielt und kifft. Seine Leidenschaft zur Heavy Metal Musik ist ihm wichtiger als seine Braut. Er ist Schulbusfahrer, obwohl er keinen gültigen Führerschein besitzt. Die Bouvier-Zwillinge, die in der Führerscheinstelle arbeiten, ließen ihn bei der Nachprüfung dennoch durchkommen. Ottos Vater ist Admiral, wie in der Folge \glqq Wird Marge verrückt gemacht?\grqq\ (siehe \ref{BABF18}) zu sehen ist. Er hat einen Bruder, der seine Ex-Freundin geheiratet hat (\glqq Der Ernstfall\grqq , \ref{8F04}). In späteren Folgen ist er auch in anderen Berufen zu finden. So ist er der Episode \glqq Homer, die Ratte\grqq\ (siehe \ref{GABF08}) Wärter in dem von Mr. Burns betriebenen Gefängnis. Er wurde trotz seines Drogenkonsums eingestellt, weil er beim Drogentest die Probe mit der von Homer vertauschte.

Otto scheint neben seiner Tätigkeit als Schulbusfahrer noch weiteren Tätigkeiten nachzugehen. So ist er in der Folge \glqq Lisa und das liebe Vieh\grqq\ (siehe \ref{VABF08}) als Bademeister in einem Erlebnisbad zu sehen.

 
\subsection{Willie}\index{Willie}\label{Willie}
Willie ist der Hausmeister der Grundschule. Er stammt aus North Kilt-Town in Schottland (\glqq Ein jeder kriegt sein Fett\grqq , \ref{5F20}) und trägt zu besonderen Anlässen einen Kilt. Obwohl seine Eltern in Schottland hingerichtet wurden, als Willie noch ein Kind war, tauchen sie lebendig auf, als er, Homer, Prof. Frink und Mr. Burns in der Folge \glqq Burns möchte geliebt werden\grqq\ (siehe \ref{AABF17}) das Monster von Loch Ness suchen und finden. In der Episode \glqq Ein perfekter Gentleman\grqq\ (siehe \ref{HABF05}) arbeitet er kurzzeitig als Empfangschef im Restaurant \glqq Gilded Truffle\grqq . In dieser Episode ist ebenso zu erfahren, dass Willie keinen Nachnamen hat. In der Folge \glqq Die Chroniken von Equalia\grqq\ (siehe \ref{KABF22}) sagt er allerdings, dass er bis zu seiner Ankunft auf Ellis Island Dr. William MacDougal\index{MacDougal!Dr. William} hieß. Der Beamte der Einwanderungsbehörde sagte, von nun an heiße er nur noch Hausmeister Willie. 

Er war mit dem Kindermädchen Shary Bobbins verlobt (\glqq Das magische Kindermädchen\grqq , \ref{3G03}). Willie arbeitet an den Wochenenden und im Sommer als Greenkeeper im Springfielder Country-Club (siehe \glqq Jäger des verlorenen Handys\grqq , \ref{KABF15}). Er arbeitet außerdem im Sommer noch im Wasserpark Aquatraz\index{Aquatraz} (siehe \glqq Gorilla Ahoi\grqq, \ref{YABF20}).

Willie ist auch der Trainer der Rugby-Mannschaft der Springfielder Grundschule (siehe \glqq Orange Is the New Yellow\grqq , \ref{VABF15}).


\subsection{Doris Peterson}\index{Peterson!Doris}\label{KuechenhilfeDoris}
Doris arbeitet nicht nur als Küchenhilfe in der Schulküche, sondern zeitweise auch als Schulkrankenschwester (allerdings ohne Qualifikation). Sie ist groß, massig und raucht viel. Als Küchenhilfe bereitet sie Speisen aus alten Sportmatten oder Tierinnereien (\glqq Mehr Hoden bedeutet mehr Eisen\dots \grqq ) zu und teilt diese an die Kinder aus. Demzufolge wird sie von der Vegetarierin Lisa auch oft genervt.

Anscheinend ist sie außerdem die Mutter des Teenagers mit Akne, der in zahlreichen Simpsons-Episoden auftaucht. Als er in der Folge \glqq Homers Bowling-Mannschaft\grqq\ (siehe \ref{3F10}) im \glqq Bowlarama\index{Bowlarama}\grqq\ arbeitet und zu Homer sagt, dass es so voll sei, dass er nicht mal seiner eigenen Mutter eine Bahn geben könne, kommt sie gerade vorbei und verkündet: \glqq Ich habe keinen Sohn.\grqq 

In der Folge \glqq Das Bart Ultimatum\grqq\ (siehe \ref{RABF03}) wird sie des Giftmordes an ihrem Ehemann verdächtigt. Seine Leiche wurde allerdings nie gefunden.

\subsection{Martin Prince}\label{MartinPrince}\index{Prince!Martin}
Martin ist der beste Schüler in Barts Klasse, wahrscheinlich sogar der ganzen Schule. Er hat einen extrem hohen IQ von 216, der einmal Bart zugeschrieben wurde, als dieser die Namen auf den Prüfungen vertauschte. Obwohl Martin einen dermaßen hohen IQ hat, ist er nicht Mitglied der Gruppe MENSA. Außerdem wird er zum Klassensprecher gewählt. Er kann Bart übertrumpfen, da nur zwei Schüler zur Wahl gehen und der Rest bei Barts Wahlparty schon dessen Sieg feiert. Aufgrund seiner Leistungen und seines Verhaltens ist er bei den Lehrern sehr beliebt, während seine Mitschüler ihn als Streber ansehen und oft genug drangsalieren. Er beschäftigt sich in seiner Freizeit u.a. mit Querflötenspielen und dem Züchten von Schmetterlingen. Er war der Schulsprecher der Schule, bis bei einem Casinoabend in der Folge \glqq Die Perlen-Präsidentin\grqq\ (siehe \ref{EABF20}) ein Chaos ausbrach. Er wurde daraufhin von Lisa beerbt.

Als zielstrebiger, ehrgeiziger und engagierter Schüler stellt er das krasse Gegenteil zu Bart Simpson dar. Er bezeichnet sich daher selbst als \glqq Barts natürlichen Feind\grqq , ist mit ihm aber nur selten aktiv in Konflikte verwickelt. In einer Zukunftsvision wird erwähnt, dass er bei einer Explosion auf einer Forschungsmesse ums Leben kommt, in Wirklichkeit aber, entstellt, als \glqq Phantom der Schule\grqq\ im Keller der Schule Orgel spielt. Er könnte homosexuell sein, was sich zeigte, als er von sich als \glqq die Königin des Sommers\grqq\ sprach, als er einen Pool baute, um die Simpsons auszubooten.

 
\subsection{Nelson Muntz}\index{Muntz!Nelson}\label{NelsonMuntz}
Nelson Mandela Muntz ist der Schläger in der Schule, der sich ständig durch ein spött\-isch\-es \glqq Ha-ha\grqq\ bemerkbar macht. Er hat sogar einen Papagei, dem er diesen Spruch beigebracht hat. Sein Großvater hat als Richter schon 47 Menschen zum Tode verurteilt, worauf er sehr stolz ist.

Seine Mutter interessiert sich nicht besonders für Nelson, da sie, wenn sie nicht gerade im Gefängnis sitzt, mit ihrem Alkoholmissbrauch beschäftigt ist. In der Episode \glqq Lisa will lieben\grqq\ (siehe \ref{4F01}) war Lisa in ihn verliebt. Nelson leidet sehr darunter, dass sein Vater die Familie verlassen hat, obwohl er eigentlich nur kurz Zigaretten kaufen wollte. Trotzdem glaubt Nelson zu wissen, dass er vorsätzlich Autounfälle baut. In der Folge \glqq Der Feind in meinem Bett\grqq\ (siehe \ref{FABF19}) kehrt sein Vater wieder zurück. Es stellt sich heraus, dass ein Zirkus ihn als Attraktion vorführte, nachdem er durch seine Erdnuss-Allergie entstellt und verrückt gewirkt hatte, was auf den Film \glqq Der Elefantenmensch\grqq\ hinweist.

Wie die meisten Schläger in Barts Schule, hat auch Nelson immense emotionale Probleme, die er durch seine brutale Art weitestgehend kaschieren kann, die jedoch vereinzelt immer wieder hervorbrechen.

In der Zukunftsfolge \glqq Future-Drama\grqq\ (siehe \ref{GABF12}) bekommen Sherri und Terri gleichzeitig jeweils Zwillinge von ihm, woraufhin er Verständnis für seinen Vater äußert und ebenfalls unter dem Vorwand, nur kurz Zigaretten holen zu wollen, verschwindet.

Nelson steht in einem zweigeteilten Verhältnis zu Bart. Meistens bekommt auch Bart sein Hobby als Schläger zu spüren, wenn er mit seinen Kumpels Barts Taschengeld stiehlt. In verschiedenen Folgen ist Nelson aber auch Bart Simpsons Freund, besonders dann, wenn Bart selbst ein \glqq kleines\grqq\ Rowdieleben führt und Streiche spielt. Dennoch lässt er es sich auch dann nicht nehmen, ihm für \glqq ungebührliche\grqq\ Aussagen eine Ohrfeige oder ähnliches zu verpassen.

Nelson ist ein großer Fan von Andy Williams\footnote{Andy Williams (* 3. Dezember 1927 als Howard Andrew Williams in Wall Lake, Iowa) ist ein US-amerikanischer Popsänger und TV-Entertainer. Sein Stellenwert in den USA ist vergleichbar mit Dean Martin und Frank Sinatra.}. In der Folge \glqq Die Reise nach Knoxville\grqq\ (siehe \ref{3F17}) sieht man ihn etwa zum Hit \glqq Moon River\grqq\ Tränen vergießen.


\subsection{Jimbo Jones}\index{Jones!Jimbo}\label{JimboJones}
Jimbo Jones ist einer der drei Schlägertypen. Er ist an dem Totenkopf-T-Shirt und der violetten Strickmütze zu erkennen. Sein voller Name lautet Corky James Jones, was in der Episode \glqq 24 Minuten\grqq\ (siehe \ref{JABF14}) zu erfahren ist. Seine Eltern sind geschieden (\glqq Kunst am Stiel\grqq , \ref{HABF22}) und er hat eine Schwester. Kurzzeitig war Laura\index{Powers!Laura}, die Tochter von Ruth Powers, seine Freundin (\glqq Laura, die neue Nachbarin\grqq , \ref{9F06}).

Er, Kearney und Dolph bilden die Band \glqq Ear Poison\grqq . Jimbo spielt in der Band Gitarre, Kearney Schlagzeug und Dolph spielt ebenfalls Gitarre und übernimmt den Gesang (siehe \glqq Ziemlich beste Freundin\grqq , \ref{SABF15}).


\subsection{Kearney Zzyzwicz}\index{Kearney}\index{Zzyzwicz!Kearney}\label{KearneyZzyzwicz}
Kearney Zzyzwicz ist der untersetzte Junge mit dem rasierten Kopf. Er hat bereits einen Sohn, ist geschieden, wie in der Episode \glqq Scheide sich, wer kann\grqq\ (siehe \ref{4F04}) zu erfahren ist. Er ist außerdem alt genug, um Autofahren zu dürfen. In der Episode \glqq Lisa will lieben\grqq\ (siehe \ref{4F01}) fährt er einen Hundai und in der Folge \glqq Marge Simpson im Anmarsch\grqq\ (siehe \ref{AABF10}) fährt er einen VW-Käfer. In der Episode \glqq Future-Drama\grqq\ (siehe \ref{GABF12}) wird er Co-Rektor der Springfielder Grundschule, während Skinner immer noch Rektor ist und in der Folge \glqq Barts Blick in die Zukunft\grqq\ (siehe \ref{BABF13}) ist er Leibwächter von Lisa, welche die erste Präsidentin der Vereinigten Staaten von Amerika ist.

Er, Jimbo und Dolph spielen in der Band \glqq Ear Poison\grqq .

\subsection{Dolph Starbeam}\index{Starbeam!Dolph}\label{DolphStarbeam}
Dolphin, genannt \glqq Dolph\grqq\ ist ein jüdischer Freund von Jimbo und einer der Schlägertypen in der Schule (\glqq Der Sicherheitssalamander\grqq , \ref{GABF21}). Er hat lange rote Haare, die sein rechtes Auge bedecken.

Er, Jimbo und Kearney spielen in der Band \glqq Ear Poison\grqq .

\subsection{Ralph Wiggum}\index{Wiggum!Ralph}
Ralph Wiggum ist der Sohn von Polizeichef Clancy Wiggum. Er ist geistig stark zurückgeblieben, hat Probleme einfachste Dinge zu verstehen oder mit ihnen umzugehen. Trotz allem hat er es bis in die zweite Klasse geschafft. Er war einmal in Lisa verliebt, als diese ihm aus Mitleid eine Valentinskarte überreichte. Ralphs große Stunde schlägt, als er beim Diorama-Wettbewerb der Schule überraschend gegen die Favoriten Lisa Simpson und Allison Taylor gewinnt, obwohl er lediglich eine Box mit Star Wars-Figuren in der Originalverpackung präsentiert, wodurch er Rektor Skinner auf seiner Seite hat. Zudem zeigt er ein starkes Verlangen, alle möglichen Dinge in seine Nase oder seinen Mund zu stecken und zu essen. Er hat durch die Nachlässigkeit seines Vaters in der Folge \glqq Der merkwürdige Schlüssel\grqq\ (siehe \ref{5F13}) Zugriff auf dessen Waffenschrank. Zudem lässt sich Ralph immer wieder von einem kleinen, grünen Kobold sagen, dass er anderer Leute Eigentum in Brand setzen soll. Ralph erwidert dem immer nur ein zustimmendes Lächeln. Wenn man einer Fernsehnachricht in einer Zukunftsblende in \glqq Lisas Hochzeit\grqq\ (siehe \ref{2F15}) glauben schenken darf, wird Ralph später zu Sideshow Ralph Wiggum, einem Sidekick von Krusty dem Clown, der, ähnlich wie Sideshow Bob, kriminell geworden ist. Die Band Bloodhound Gang\index{Bloodhound Gang} hat Ralph Wiggum das Lied \emph{Ralph} auf dem Album \glqq Hefty Fine\grqq\ gewidmet, das ausschließlich aus Zitaten von ihm besteht.

\subsection{Uter}\index{Uter}
Uter ist ein Austauschschüler aus der Schweiz, isst sehr gerne Schokolade und andere Süßigkeiten. Demzufolge ist er sehr dick. Sein Vater arbeitet als Vorarbeiter bei den Basler Kaugummiwerken. Er wird von anderen Schülern oft aufgrund seiner Fettleibigkeit oder seiner anderen Nationalität gehänselt. Unter anderem wird er von Homer Simpson in der Folge \glqq Lisa auf dem Eise\grqq\ (siehe \ref{2F05}) durch die Umkleide mit einem Handtuch verfolgt, worauf Uter sagt: \glqq Jag' mich nicht, ich will nicht laufen! Ich bin voll mit Schoggi!\grqq

Im US-Original der Serie ist Uter Deutscher und heißt Üter. Die Zeichner benannten ihn so, da sie dachten, \glqq Üter\grqq\ wäre ein geläufiger deutscher Name. Vermutlich wurde \glqq Üter\grqq\ aber mit \glqq Günther\grqq\ verwechselt.

Gekleidet ist er allerdings wie ein Tiroler und entspricht dem Klischee des \glqq Lederhosen-Bayern\grqq\ der US-Amerikaner. In der elften Staffel wurde bekannt, dass Uter auf dem letzten Schulausflug verschwand -- er taucht jedoch in späteren Staffeln wieder auf, z.B. in der Folge \glqq Die böse Hexe des Westens\grqq\ (siehe \ref{GABF05}). Seine eindeutige Nationalität ist deswegen nicht bekannt, weil er nur in der deutschen Fassung als Schweizer bezeichnet wird. In der Originalfassung arbeitet sein Vater als Vorarbeiter in den Düsseldorfer Kaugummiwerken (Düsseldorf Gumworks), aber er soll wahrscheinlich eine Anspielung auf alle deutschsprachigen Mitteleuropäer sein.

\subsection{Wendell Queasly}\index{Queasly!Wendell}\label{WendellQueasly}
Wendell besucht Barts Klasse, er trägt ein hellblaues T-Shirt und hat weiße, lockige Haare. Er ist sehr blass und da er ständig krank ist, hat er die meisten Fehlstunden an der Springfielder Grundschule. Bei jeder Fahrt mit dem Schulbus wird es Wendell übel und auch sonst hat er nicht gerade den stärksten Magen, was der Küchenhilfe Doris gerade recht kommt. Sein Lieblingsort, um sich zu erbrechen, ist der Sandkasten auf dem Schulhof, den anschließend Hausmeister Willie sehr verärgert reinigen muss.

\section{Showbusiness}

\subsection{Kent Brockman}\index{Brockman!Kent}\label{KentBrockman}
Kent Brockman ist der Nachrichtensprecher des Senders Channel 6 in Springfield, Emmy-Gewinner und Moderator der Sendungen \glqq Smartline\index{Smartline}\grqq, \glqq Eye on Springfield\grqq, \glqq My Two Cents\grqq\ und \glqq Bite Back\grqq . In der deutschen Fassung heißt \glqq Eye on Springfield\grqq\ in frühen Episoden \glqq Ein Auge auf Springfield\grqq , dann \glqq Blickpunkt Springfield\grqq . In seinen Reportagen scheut er sich nicht davor, für seine Seite deutlich Farbe zu bekennen. So propagiert er beim vermeintlichen Angriff der \glqq Killerameisen\grqq\ bereits deren Herrschaft. 

Kent Brockman wurde durch einen Lottogewinn in der Episode \glqq Auf den Hund gekommen\grqq\ (siehe \ref{8F17}) sehr reich. Außerdem hat er bereits einen Emmy für seine Arbeit gewonnen, wie in der Episode \glqq Das Fernsehen ist an allem schuld\grqq\ (siehe \ref{7F09}) zu erfahren ist.

Kent Brockman ist bzw. war mit der Wetterfee von Channel 6, Stefanie, verheiratet (\glqq Die Erbschaft\grqq , \ref{7F17}) und er hat eine schulpflichtige und eine erwachsene Tochter (\glqq Fake-News\grqq, \ref{VABF21}). Außerdem hat er eine Schwester, die ebenfalls Journalistin ist und in Washington bei CNN arbeitet, wie er in der Folge \glqq Die neusten Kindernachrichten\grqq\ (siehe \ref{5F15}) angibt.


\subsection{Krusty, der Clown}\index{Krusty}\index{Krustofski!Herschel}\label{HerschelKrustofski}
Krusty (bürgerlicher Name: Herschel Krustofski) arbeitet als Fernseh-Clown, hat eine eigene Show und stellt in dieser Rolle eine übertriebene Albernheit zur Schau. Er ist starker Raucher und hat Starallüren. Sein Publikum besteht aus Kindern, die ihn als Superstar verehren. Weiterhin ist Krusty Besitzer der Fast-Food-Kette Krusty-Burger. In mehreren Folgen spielt auch Krustys Judentum eine Rolle. Er hat einen Herzschrittmacher, nachdem er 1986 während einer Sendung eine Herzattacke erlitten hatte und eine dritte Brustwarze. Sein Erzfeind ist Sideshow Bob (Tingeltangel-Bob), ein ehemaliger Assistent bei seiner Show, der ihm einen Überfall auf den Kwik-E-Mart anhängen wollte. Krustys Vater wollte lange nichts von seinen Sohn wissen, bis Bart und Lisa sie wieder zusammengebracht haben. Denn Krustys Vater ist Rabbi, wie in der Folge \glqq Der Vater eines Clowns\grqq\ (siehe \ref{8F05}) zu erfahren ist. Krusty ist Analphabet, aber in einigen Folgen scheint er doch lesen (\glqq Am Anfang war das Wort\grqq , \ref{9F08}) oder schreiben (\glqq Barts Komet\grqq , \ref{2F11}) zu können. Krusty hat noch eine Schwester, mit der wohl Sideshow Mel eine Affäre hatte. Außerdem hat er zwei uneheliche Kinder, eine Tochter namens Sophie, deren Mutter eine ehemalige Soldatin ist und einen Sohn mit einer anderen Frau. Für seine Tochter Sophie hat er das Sorgerecht für eine Woche pro Jahr (\glqq Krustliche Weihnachten\grqq, \ref{WABF02}). In der Episode \glqq Grundschul-Musical\grqq\ (siehe \ref{MABF21}) gibt er an, verheiratet zu sein.

Einer seiner größten Fans ist Bart Simpson. Bart hat ihm schon öfter aus miss\-lichen Situationen geholfen, doch Krusty kann sich nur selten an ihn erinnern.

Er ist für das Feuer auf dem Reifenhaufen in Springfield verantwortlich. Er hat ihn aus Versehen angezündet (\glqq Nichts bereuen\grqq , \ref{RABF18}).

Er ist außerdem das Teammaskottchen der Toronto Raptors (siehe \glqq Krusty macht ernst\grqq, \ref{XABF08}).

\subsection{Sideshow Bob}\index{Sideshow!Bob}\index{Terwilliger!Robert Underdunk}\label{SideshowBob}
Sideshow Bob behält in der ersten Staffel der deutschen Fassung, für die damals noch der Fernsehsender ZDF verantwortlich war, zunächst seinen Originalnamen; den Namen Tingeltangel-Bob trägt er in der deutschen Fassung seit der Folge \glqq Tingeltangel-Bob\grqq\ (siehe \ref{2F02}).

Sideshow Bob (bürgerlicher Name: Robert Underdunk Terwilliger) ist der ehemalige Assistent von Krusty, dem Clown in dessen Fernsehshow. Sein Vater ist der Arzt Dr. Robert Terwilliger und seine Mutter ist die Schauspielerin Judith Underdunk. Er hat in Yale studiert (\glqq Tingeltangel-Bob\grqq , \ref{2F02}). Bob war auch Artist im Cirque du Soleil (siehe \glqq Gone Boy\grqq, \ref{XABF02}). Er wurde Krusty Assistent, weil er seinen Bruder Cecil zum Vorsprechen für die Rolle gefahren hatte (\glqq Die beiden hinterhältigen Brüder\grqq , \ref{4F14}). Seine Arbeit ist ihm verhasst, weil Krustys Oberflächlichkeit ihn anwidert. Er hängt Krusty einen Überfall des Kwik-E-Mart an, übernimmt seine Show und versucht, sie in eine hoch intellektuelle Sendung umzuwandeln (unter anderem mit Gesprächen über Stoizismus und einer Titelmelodie von Wolfgang Amadeus Mozart). Dabei scheitert er und zeigt daraufhin seine wahre Natur als Psychopath, der fortan Krusty und vor allem dessen ergebenem Anhänger Bart Simpson nach dem Leben trachtet. In mehreren Folgen wird sein Größenwahn thematisiert: So versucht er unter anderem, als Bürgermeisterkandidat der Republikaner die Bürgermeisterwahlen von Springfield zu manipulieren und bedroht die Stadt mit einer Atombombe. Er versucht auch, Barts Tante Selma zu ermorden, mit der er verheiratet war. Alle diese Verbrechen werden aber von Bart und Lisa vereitelt. Zeitweise ist ihm das Fernsehen dermaßen zuwider, dass er droht, Springfield mit einer Atombombe in die Luft zu jagen, wenn nicht innerhalb einer bestimmten Frist alle Fernsehsender den Sendebetrieb einstellen (\glqq Tingeltangel-Bobs Rache\grqq , \ref{3F08}). Dieser Plan scheitert ebenfalls und er wird wieder ins Gefängnis gesteckt. Er rehabilitiert sich, indem er verhindert, dass sein Bruder Cecil den Staudamm von Springfield sprengt. Leider glaubt ihm niemand und er wird mitsamt einem Bruder wieder ins Gefängnis gesteckt. Zu seinem Nachfolger in Krustys Fernsehshow wird der Brite Sideshow Mel (Tingeltangel-Mel).

Während seiner Haftstrafe ist er als Schauspieler tätig und gewinnt sogar einen Emmy für seine Arbeit, wie in der Episode \glqq Bis dass der Tod euch scheidet\grqq\ (siehe \ref{8F20}) zu erfahren ist. Er kann schließlich aus dem Gefängnis entkommen und begibt sich nach Italien in das kleine Dorf Salsiccia\index{Salsiccia} in der Toskana. Dort wird er sogar zum Bürgermeister gewählt. Er hat dort die Braut Francesca, mit der er einen Sohn namens Gino hat. Als die Simpsons dort auftauchen und sein Geheimnis, dass er ein ehemaliger Häftling ist, offenbaren, muss er als Bürgermeister zurücktreten und schwört den Simpsons \glqq Vendetta\footnote{Blutrache}\grqq\ (\glqq Der italienische Bob\grqq , \ref{HABF02}).

\subsection{Sideshow Mel}\index{Sideshow!Mel}\index{Van Horne!Melvin}\label{MelvinVanHorne}
Sideshow Mel (bürgerlicher Name: Melvin Van Horne) ist der Assistent von Krusty, dem Clown in dessen Fernsehshow, seit Sideshow Bob sein Psychopatentum offenbart hat und wegen verschiedener Vergehen im Gefängnis sitzt. In seinem aufgetürmten, grünen Haar trägt Mel einen Knochen; auch seine übrigen Kennzeichen (nackter Oberkörper, Bambusröckchen und Kette mit Tierzähnen) entsprechen denen einer Karikatur des \glqq primitiven Wilden\grqq . In Kontrast dazu steht seine distinguierte, im Original oxford-englische Aussprache. Er studierte zusammen mit dem Schriftsteller Thomas Pynchon\index{Pynchon!Thomas} an der Cornell University (\glqq Die geheime Zutat\grqq , \ref{FABF20}). Bevor er für Krusty arbeitete, war er Theaterschauspieler u.a. in London und Los Angeles. Er spielte den Biff Loman im Stück \glqq Tod eines Handlungsreisenden\grqq\ von Arthur Miller (\glqq Alles über Lisa\grqq , \ref{KABF13}). Er wurde dafür mit dem Preis \glqq Entertainer des Jahres\grqq\ ausgezeichnet. In Krustys Show tritt er vor allem durch sein Spiel auf der Schiebeflöte hervor. In der deutschen Fassung trägt er den Namen Tingeltangel-Mel.

In der Episode \glqq Homer mobil\grqq\ (siehe \ref{GABF07}) erwähnt Marge, dass sie auch Sideshow Mel heiraten hätte können. Mittlerweile ist er verheiratet und seine Frau erwartet ein Kind (\glqq Alles über Lisa\grqq , \ref{KABF13}). Laut Folge \glqq Der Uhr-Großvater\grqq\ (siehe \ref{WABF11}) hat er mehrere Kinder, die bei seiner Mutter in Nebraska leben. Krusty hatte mit seiner Frau über elf Jahre lang eine Affäre (\glqq Marge macht mobil\grqq , \ref{LABF16}). Er hat einen unehelichen Sohn, der auf die Grundschule in Springfield geht (siehe \glqq Bitte lächeln!\grqq, \ref{ZABF05}). Mel scheint religiös zu sein, da er dem Springfielder Kirchenrat angehört (\glqq Gott gegen Lisa Simpson\grqq, \ref{HABF14}).

\subsection{Troy McClure}\index{McClure!Troy}
Troy McClure ist ein mäßig erfolgreicher Darsteller in B-Movies. Er taucht meistens in seltsamen TV-Shows, Werbesendungen oder billig produzierten Dokumentarfilmen als Moderator auf. Jeder seiner Auftritte beginnt mit einem \glqq Hallo, ich bin Troy McClure, sie kennen mich vielleicht noch aus Filmen, wie\dots \grqq . In der Episode \glqq Selma heiratet Hollywoodstar\grqq\ (siehe \ref{3F15}) war er mit Selma Bouvier verheiratet, obwohl er eigentlich nicht auf Frauen steht. Er heiratete sie nur der Karriere wegen. Seine Figur geht zurück auf Doug McClure\index{McClure!Doug}, den Westernschauspieler der 1960er und 70er Jahre, der in den späten 80ern in der TV-Serie \glqq Mein Vater ist ein Außerirdischer\grqq\ einen abgehalfterten Schauspieler verkörperte. Er starb 1995, doch bei den Simpsons taucht er weiterhin regelmäßig auf, bis der Originalsprecher Phil Hartman verstarb. Da dieser mit Groening gut befreundet war, wurde kein neuer Sprecher für McClure besetzt, sondern die Figur verschwand.

\subsection{Pédro Chespirito}\index{Chespiritio!Pédro}\index{Hummelmann}\index{Bienenmann}\label{Hummelmann}
Der Hummelmann ist mexikanischer Abstammung und dreht eine Comedy-Serie fürs Fernsehen (auf Kanal Ocho), die von Homer gerne gesehen wird. Er trägt für seine Serie das Kostüm einer Hummel. In seiner Serie, die jeweils als kurzer Ausschnitt zu sehen ist, sagt er immer den Satz \glqq Hayayay, no me gusta, hayayay\grqq\ (sinngemäß: \glqq ich mag das nicht\grqq ). Obwohl er im englischen ganz klar \glqq Bumblebee-Man\grqq\ (also eigentlich soviel wie Hummel-Mann) genannt wird, heißt er im Deutschen \glqq der Bienenmann\grqq . Er hat einen Chihuahua mit überdimensional großem Kopf als Haustier, in Anspielung auf eine ehemalige Werbefigur der amerikanischen Fast-Food-Kette \glqq Taco Bell\grqq .

In der Episode \glqq 22 Kurzfilme über Springfield\grqq\ (siehe \ref{3F18}) lässt sich seine Frau von ihm scheiden, nachdem er das Haus zerstört hatte.

In der Folge \glqq Mein Freund, der Wunderbaum\grqq\ (siehe \ref{PABF22}) wechselt er von Kanal Ocho zu Kanal 6.

Seine Eltern kamen bei einem Paintball-Unfalls ums Leben (siehe \glqq Nichts bereuen\grqq , \ref{RABF18}).

In der deutschen Synchronisation wird er auch oft als Bienenmann bezeichnet. Im Englischen heißt er \glqq Bumblebee Man\grqq\ und \glqq Bumblebee\grqq\ ist das Englische Wort für Hummel\footnote{Die Hummeln sind eine zu den Echten Bienen gehörende Gattung staatenbildender Insekten.}.

\subsection{Rainier Wolfcastle}\index{Wolfcastle!Rainier}\label{RainierWolfcastle}
Rainier Luftwaffe Wolfcastle ist ein erfolgreicher Hollywoodschauspieler, welcher vornehmlich in vor Gewalt berstenden Actionfilmen auftritt. Sein erfolgreichstes Werk ist die McBain-Reihe. Er spielte im Film \glqq Radioactive Man\grqq\ den Titelhelden. An seiner Seite spielte Milhouse den Fallout Boy (\glqq Filmstar wider Willen\grqq , \ref{2F17}). Ferner ist er der Moderator der Late-Night-Show \glqq Up Late with McBain\grqq .

Durch sein Auftreten und seinen österreichischen Akzent\footnote{im englischen Original}, wird nahegelegt, dass es sich um das Springfield-Pendant von Arnold Schwarzenegger handelt. Sein richtiger Name ist Rainer Wolfgangsee\index{Wolfgangsee!Rainer}. Er hat eine Tochter namens Greta\index{Greta}, die in der Episode \glqq Nach Kanada der Liebe wegen\grqq\ (siehe \ref{DABF06}) erst Barts und später Milhouses Freundin war. In der Folge \glqq Homerazzi\grqq\ (siehe \ref{JABF06}) heiratet er Maria Shriver Kennedy Quimby. Rainier Wolfcastle gehört der Republikanischen Partei in Springfield an.


\subsection{Arnold \glqq Arnie\grqq\ Pye}\index{Pye!Arnold}\label{ArniePye}
Arnold Pye ist bekannt als Arnie \glqq in the sky\grqq\ Pye. Er ist der Reporter, der regelmäßig aus dem Verkehrsbeobachtungshubschrauber berichtet. In der Episode \glqq Das böse Wort\grqq\ (siehe \ref{JABF15}) übernimmt er den Posten des Nachrichtensprechers, als Kent Brockman zum Wetterfrosch degradiert wird. Er ist Mitglied der Steinmetze.


\subsection{Chester J. Lampwick}\index{Lampwick!Chester J.}
Er ist der Erfinder der beliebten Zeichentrickfigur Itchy. Als Lisa und Bart ihn kennenlernen, lebt er als Obdachloser auf der Straße. Bart holt sich von Homer daraufhin 1000 Dollar, um einen Anwalt zu engagieren, der Chesters Rechte als Erfinder von Itchy geltend machen soll. Dazu wurde jedoch ein Beweisstück benötigt. Das erste war ein Schwarz-Weiß-Film von Chester, der aber bei der Privatvorführung verbrannte. Das zweite war ein Bild von Itchy im Comicbuchladen, das Bart für 750 Dollar während der Gerichtsverhandlung kaufte. Chester erhielt Recht und wurde reich, hatte aber kein Interesse daran, sich weiterhin mit der Zeichentrickserie Itchy und Scratchy zu beschäftigen.

\subsection{Booberella}\label{Booberella}\index{Booberella}
Booberella ist eine vampirähnliche Fernsehpersönlichkeit, die spät Nachts Grusel- und Horrorshows moderiert. Markantes Merkmal von ihr sind ihre großen Brüste, die sie oft und gern zur Schau stellt. Sie wurden auch schon \glqq die Zwillinge\grqq\ genannt. Mr. Burns erwähnt, dass sie die Tochter von Bill aus der Buchhaltung sei. In der Episode \glqq Es war einmal in Springfield\grqq\ (siehe \ref{LABF20}) sitzt sie in der Jury zu \emph{America's Next Krusty}.


\section{Verwaltung, Polizei und Gesetz}

\subsection{Joe Quimby}\index{Quimby!Joe}
Joe Quimby ist der korrupte Bürgermeister von Springfield. Sein vollständiger Name lautet Joseph \glqq Joe\grqq\ Fitzpatrick Fitzgerald Fitzhenry Quimby. Sein Siegel ent\-hält die Worte \glqq Mayor of Springfield -- corruptus in extremis\grqq\ (zu sehen auf seinem Rednerpult). Er hat zahlreiche Affären mit Models und ist auch sonst nicht gerade integer. Seine Fragwürdigkeit wird dabei immer offensichtlich dargestellt, jedoch scheinen sich die Springfielder Bürger daran selten zu stören. Er lässt sich zum Beispiel von der Springfielder Mafia bestechen. Er tritt stets mit seiner Bürgermeisterschärpe auf. Da Quimby kein Republikaner und damit Demokrat ist, sind Parallelen zu Bill Clinton nicht zu übersehen \cite{Wikipedia}.

\subsection{Chief Wiggum}\index{Wiggum!Clancy}\label{ChiefWiggum}
Clancy Wiggum ist der Polizeichef der Stadt Springfield. Er ist äußerst dick und nicht gerade kompetent. In seiner Jugend litt er an Asthma. Er ähnelt in seinem Äußeren einem Schwein, dies ist eine Anspielung auf das in den USA besonders während der Bürgerrechtsbewegung gebräuchliche Schimpfwort \glqq pig\grqq\ für Polizisten. Er hat einen geistig behinderten Sohn -- Ralph Wiggum -- und ist verheiratet mit Sarah Wiggum. Es gibt zwei Versionen, wie er Sarah kennengelernt hat. Die erste Variante ist, dass er ihr Drogen zugesteckt hat, um sie kennenzulernen (\glqq Ein Stern wird neu geboren\grqq, \ref{EABF08}). In der zweiten Variante war sie Mitglied einer Diebesbande und sollte ihn ursprünglich vom Diebstahl ablenken (\glqq Die weiblichen Verdächtigen\grqq, \ref{QABF10}).
Er geht nie wählen, ist Internet abhängig und hat Angst vor Spinnen im Bad (\glqq Homer mit den Fingerhänden\grqq , \ref{NABF13}).
In der Episode \glqq Pranksta Rap\grqq\ (siehe \ref{GABF03}) wird er sogar zeitweilig zum Commissioner befördert.


\subsection{Lou}\index{Lou}\label{Lou}
Lou ist Polizeibeamter. Der Name seiner Ex-Frau ist Amy. Er ist Vater eines Sohnes, der in Baltimore lebt (\glqq Affenhilfe\grqq, \ref{WABF01}). Er hat eine Schwester, die ein Windspielgeschäft in Springfield betreibt (\glqq La Pura Vida\grqq, \ref{ZABF03}). Lou verbessert oft Wiggum, wenn der Fehler macht. Er ist in der Episode \glqq Pranksta Rap\grqq\ (siehe \ref{GABF03}) kurzzeitig Polizeichef, als Wiggum zum Commissioner befördert wird. Eigentlich wollte er Jura studieren, konnte es sich aber nicht leisten, wie in der Episode \glqq Angst essen Seele auf\grqq\ (siehe \ref{GABF16}) zu erfahren ist. Lou war Schlagzeuger der Grunge-Band \glqq Sadgasm\grqq . In der Episode \glqq Der Tortenmann schlägt zurück\grqq\ (siehe \ref{FABF15}) wird Homer als Tortenmann verkleidet von ihm angeschossen.

\subsection{Eddie}\index{Eddie}\label{Eddie}
Eddie ist auch Polizeibeamter. Er spricht sehr selten. Genau wie Lou hat er keinen Nachnamen. In der Folge \glqq Am Anfang war die Schreiraupe\grqq\ (siehe \ref{DABF16}) wird gesagt, sie hätten beide keinen Nachnamen genauso wie Cher.

Aufgrund von Sparmaßnahmen wird er aus dem Polizeidienst entlassen (siehe \glqq Manacek\grqq, \ref{XABF05}).

\subsection{Richter Roy Snyder}\label{RichterSynder}\index{Synder!Roy}
Roy Snyder ist der Richter in Springfield. Wenn Bart mit dem Gesetz in Konflikt kommt, kann er immer auf ihn zählen, denn er verurteilt keine Jungen, \glqq da Jungs nun mal Jungs sind\grqq . In einer Folge überschreibt er Armin Tamzarian die Identität von Rektor Skinner (\glqq Alles Schwindel\grqq , \ref{4F23}). Einmal hat Richter Snyder Urlaub genommen und hatte als Vertretung Richterin Constance Harm, die das krasse Gegenteil von ihm darstellte \cite{Wikipedia}. Richter Snyder hat mindestens einen Sohn, wie in Episode \glqq Marge wird verhaftet\grqq\ (siehe \ref{9F20}) zu erfahren ist. Er ist geschieden (siehe \glqq Der Prozess\grqq, \ref{YABF07}).

In der Episode \glqq Bart kommt unter die Räder\grqq\ (siehe \ref{7F10}) wird er noch Richter Moulton\index{Moulton} genannt. Die beiden Show Runner Bill Oakley und Josh Weinstein wussten dies nicht und nannten ihn später Richter Synder.


\subsection{Richterin Constance Harm}\index{Harm!Constance}
Richterin Constance Harm hat in einer Episode Richter Roy Snyder vertreten, da er im Urlaub war. In dieser Zeit, hat sie Bart und Homer Simpson bestraft, indem sie die beiden aneinander kettete. Als Marge das Seil zwischen den beiden durchgeschnitten hatte, wollte sie, dass Homer und Marge zugeben, dass sie schlechte Eltern sind. Nachdem Richterin Harm die beiden bestrafte, weil Marge sich weigerte, rächten sie sich und ließen das Hausboot der Richterin untergehen (\glqq Ich bin bei dir, mein Sohn\grqq , \ref{CABF22}). Constance Harm hat auch in den neuesten Simpsons-Folgen mehrere Auftritte, die Bart sogar einmal besonders hart treffen \cite{Wikipedia}.

\section{Ärzte}
\subsection{Dr. Julius Hibbert}\index{Hibbert!Dr. Julius}\label{JuliusHibbert}
Er ist der seriöse Arzt in Springfield und er ist mit Bernice verheiratet und hat drei Kinder. Sein Zwillingsbruder ist Direktor des Waisenhauses in Shelbyville. Sein älterer Bruder ist Zahnfleischbluter Murphy, an den er sich nur noch dunkel erinnert.

Obwohl über Dr. Hibberts Vergangenheit außer der Tatsache, dass er eigentlich aus Alabama stammt, wenig bekannt ist, so kann man doch davon ausgehen, dass das Verhältnis zu seinen Geschwistern schon immer äußerst gestört war: Von seinem Zwillingsbruder wurde er schon früh getrennt, es ist zu vermuten, dass er nicht einmal weiß, dass er existiert und das, obwohl er nur wenig von Springfield weg in der Nachbarstadt Shelbyville lebt. Den Kontakt zu seinem älteren Bruder verlor er ebenfalls früh, er weiß lediglich, dass er Jazz-Musiker ist, genaueres ist ihm unbekannt. Er studierte schließlich Medizin an der John Hopkins Medical School (siehe \glqq Bart kommt unter die Räder\grqq, \ref{7F10}) und wurde so der angesehenste Arzt in Springfield. Während seines Studiums jobbte er als Tänzer und war unter dem Künstlernamen \glqq Malcolm Sex\grqq\ bekannt (\glqq Mr. Burns wird entlassen\grqq , \ref{EABF10}). Er arbeitete außerdem während seines Studiums noch als Barkeeper (\glqq Moe Szyslak und das Königreich des Kristallschädels\grqq, \ref{QABF15}). Er fand auch sein privates Glück und gründete eine Familie. Dennoch liegt weiterhin ein kleiner Schatten auf den Hibberts: Seine Bernice ist eine Alkoholikerin und ihn selbst drohte der Stress umzubringen, bis er ein Ventil fand, den Stress, der ihn plagte, abzulassen: Grundloses Kichern, auch zu unpassendsten Gelegenheiten.

Er sieht aus wie Dr. Huxtable\index{Huxtable!Dr.} (Bill Cosby\index{Cosby!Bill}) aus der \glqq Cosby Show\grqq . Auch Hibberts Familie und sein charakteristisches Lachen ist dem von Huxtable nachempfunden \cite{Hibbert}.

Er scheint eine musikalische Ader zu haben, da er in der Band Covercraft das Schlagzeug spielte.


\subsection{Dr. Marvin Monroe}\index{Monroe!Dr. Marvin}
Dr. Marvin Monroe stellt einen Psychologen dar, der neben einer Familien- und einer Hunde\-the\-ra\-pie-Einrichtung auch eine psychologische Hotline unterhält. Er rät Marge Simpson regelmäßig dazu, ihren Mann Homer zu verlassen. Der Leitspruch seiner Hundetherapie ist: \glqq Nicht ihr Hund ist das Problem, sondern Sie.\grqq\ 
Seine Figur basiert auf den Radiotherapeuten David Liscott\index{Liscott!David}, der, wie Monroe in der Originalversion, eine sehr nervige Stimme hat.

\begin{itemize}
	\item In der Folge \glqq Der Babysitter ist los\grqq\ (siehe \ref{7G01}) war Monroe der Moderator einer Radiosendung, in der Leute mit Problemen anrufen können.
	\item In der Folge \glqq Vorsicht, wilder Homer\grqq\ (siehe \ref{7G09}) ist er Teil des Teams von Wissenschaftlern, die herausfinden wollen, ob es sich bei Homer um Bigfoot handelt.
	\item In der Episode \glqq Alpträume\grqq\ (siehe \ref{8F02}) berät er in Barts Alptraum Homer, dass er mit Bart mehr Zeit verbringen soll.
	\item Er ist Mitglied des \glqq We're sending our love down the well\grqq -Chors in der Folge \glqq Wer anderen einen Brunnen gräbt\grqq\ (siehe \ref{8F11}).
\end{itemize}

Dies war sein vermeintlich letzter Auftritt in der Serie. Es scheint so, als sei Marvin Monroe verstorben, was jedoch nie explizit erwähnt wurde. In der Folge \glqq Ned Flanders: Wieder allein\grqq\ (siehe \ref{BABF10}) ist auf dem Springfielder Friedhof ein Grabstein mit seinem Namen zu sehen. Ein weiterer Hinweis ist, dass das örtliche Krankenhaus in das Marvin Monroe Memorial Hospital umbenannt wurde. In der fünfzehnten Staffel taucht er in der Folge \glqq Fantasien einer durchgeknallten Hausfrau\grqq\ (siehe \ref{FABF05}) jedoch wieder auf (\glqq ich bin sehr krank gewesen\grqq ).

Er hat einen Bruder mit dem Namen Mervin\index{Monroe!Mervin}. Dieser ist Tätowierer.


\subsection{Dr. Nick Riviera}\index{Riviera!Dr. Nick}
Dr. Nick Riviera ist der unseriöse Arzt in Springfield, der keine Ausbildung hat und stets durch Pfusch Prozesse am Hals hat. Er moderierte auch eine Fernsehsendung, die mit der Begrüßung \glqq Hallo zusammen!\grqq\ -- \glqq Hallo, Dr. Nick!\grqq\ begann. Er bietet Operationen jeder Art zum Dumpingpreis von \$ 129,95 an, was Homer in der Folge \glqq Oh Schmerz, das Herz!\grqq\ (siehe \ref{9F09}) das Leben rettet.


\section{Sonstige Bürger Springfields}

\subsection{Agnes Skinner}\label{AgnesSkinner}\index{Skinner!Agnes}
Agnes Skinner ist die dominante Mutter von Schuldirektor Seymour Skinner. Sie ist nicht seine leibliche Mutter. Ihr Sohn geriet im Vietnam-Krieg in Kriegsgefangenschaft und wurde von einem seiner Untergegebenen, Armin Tamzerian\index{Tamzerian!Armin}, fälschlicherweise für tot gehalten. Er sollte Agnes die Todesnachricht ihres Sohnes überbringen. Er brachte dies nicht über sein Herz und gab sich als ihr Sohn aus. Obwohl sie merkte, dass er nicht ihr Sohn ist, ist sie mit dem Schwindel einverstanden (\glqq Alles Schwindel\grqq , \ref{4F23}). Sie kontrolliert die meisten Aspekte seines Lebens. Regelmäßig veranstalten sie einen Silhouetten-Abend zu Zweit, bei dem Seymours Kontur mit einer Lampe an die Wand projiziert wird und dann von Agnes ausgeschnitten wird. Sie hatte mal eine Affäre mit dem Comicbuchverkäufer. Ab und zu stritten sich die beiden über ein aufblasbares Badekissen, O-Ton: \glqq Das wir beide sehr gern haben\grqq . Sie war dreimal mit Abschleppfahrern verheiratet (siehe \glqq Abgeschleppt!\grqq , \ref{JABF21}). Die Skinners wohnen wohl neben dem Haus der Familie Prince. Sie sammelt seit 1941 Kuchenbilder, die sie aus Zeitungen ausgeschnitten hat, obwohl sie keine Kuchen mag, da sie ihr zu süß sind (siehe \glqq Wenn der Rektor mit der Lehrerin\dots\grqq , \ref{4F09}). Sie war auf dem Playdude-Cover als Miss Kalter Krieg zu sehen (siehe \glqq Gone Boy\grqq, \ref{XABF02}).

Sie hat 1952 an den olympischen Sommerspielen im Stab-Hochsprung teilgenommen, obwohl sie damals mit Seymour schwanger war. Beim Überqueren der Hochsprunglatte riss sie diese, da sich ihr ungeborenes Kind im Mutterleib bewegt hat. Dies ist auch der Grund dafür, dass Sie Seymour immer streng behandelt. Der Vater von Seymour ist ein Soldat, der im Koreakrieg gekämpft hat (siehe \glqq Curling Queen\grqq , \ref{MABF05}).

\subsection{Familie Nahasapeemapetilon}

\subsubsection{Apu Nahasapeemapetilon}\index{Nahasapeemapetilon!Apu}\label{ApuNahasapeemapetilon}
Apu ist am 9. Januar 1962 geboren und der indische Geschäftsinhaber des Kwik-E-Marts, der Haupteinkaufsquelle der Simpsons. Er ist mit Manjula verheiratet, die ihm bereits im Kindesalter zur Hochzeit versprochen worden ist. Beide sind Eltern von Achtlingen (Sechs Söhne: Poonam, Uma, Anoop, Sandeep, Nabendu, Gheet; zwei Töchter: Sashi, Pria). Einer der Söhne kann schon mit Gewehren umgehen und manchmal den Kwik-E-Mart bewachen. Apu hat einen Bruder namens Sanjay. Dieser tritt jedoch nur selten auf -- meistens als Vertretung, wenn Apu den Kwik-E-Mart kurzfristig verlässt. Außerdem hat er noch einen Neffen namens Jamshed, eine Nichte namens Pamchuseta und viele weitere Verwandte in Indien.

Apu ist als der Klügste von sieben Millionen Indern in die Vereinigte Staaten gereist, um im Springfield Heights Institute of Technology (abgekürzt: SHIT) zu studieren. Nach neun Jahren machte er seinen Doktor in Computerwissenschaften (\glqq Hochzeit auf indisch\grqq , \ref{5F04}) und wollte nicht sofort wieder nach Indien zurück, sondern noch eine Weile im Kwik-E-Markt arbeiten, um sein Studentendarlehen zurückzuzahlen. Außerdem scheint er einen Doktortitel in Philosophie zu besitzen (\glqq Volksabstimmung in Springfield\grqq , \ref{3F20}). Nachdem er das Geld zurückgezahlt hatte, blieb er aber trotzdem noch in Springfield, weil er dort zu viele Freunde gefunden hatte. Er studierte auch am Massachusetts Institute of Technology (MIT), von welchem er ohne Abschluss flog (\glqq Sky-Polizei\grqq, \ref{TABF09}).

Es ist nicht sehr empfehlenswert im Kwik-E-Mart einzukaufen, da Apu das Haltbarkeitsdatum bei Milchprodukten ändert, auf den Boden gefallene Hot Dogs verkauft und die Preise von abgelaufenem Fleisch ändert, anstatt es wegzuwerfen, was ihm kurzfristig den Job kostete.

Er ist überzeugter Vegetarier und isst keine Produkte, welche von Tieren stammen, wie beispielsweise Käse oder Eier.

Apu war ebenfalls ein Mitglied des Quartetts \glqq Die Überspitzen\grqq\ und gewann mit ihnen einen Grammy. In der Band Covercraft war er als Sänger aktiv.

Im Spielfilm \glqq Der Tempel des Todes\grqq\ von Steven Spielberg hat er laut eigener Aussage mitgespielt (siehe \glqq Apucalypse Now\grqq , \ref{VABF05}).


\subsubsection{Manjula Nahasapeemapetilon}\index{Nahasapeemapetilon!Manjula}
Manjula ist die Ehefrau von Apu. Sie wurde ihm bereits als Jugendlicher zur Ehefrau versprochen. Apu und Manjula haben Achtlinge. Die Namen der Kinder lauten:
Poonam, Uma, Anoop, Sandeep, Nabendu, Gheet (Söhne), Sashi und Pria (Töchter) \cite{SpringfieldAt}.

\subsubsection{Sanjay Nahasapeemapetilon}\index{Nahasapeemapetilon!Sanjay}\label{NahasapeemapetilonSanjay}
Sanjay ist der jüngere Bruder von Apu. Er ist der Vater von Jamshed und Pahusacheta. Sanjay arbeitet wie Apu ebenfalls im Kwik-E-Mart.


\subsection{Barney Gumble}\index{Gumble!Barney}
Barney ist Stammgast in Moes Taverne. Er ist Alkoholiker und gehört zu Homers Freunden. Barney ist 40 Jahre alt (\glqq Springfield Film-Festival\grqq , \ref{2F31}) und arbeitslos. Er hatte aber schon zahlreiche Jobs wie, z.B. den eines selbständigen Schneepflugfahrers oder als Testperson für wissenschaftliche Experimente. Bevor er arbeitslos wurde, war er bei der Army, wie in der Episode \glqq Der Heiratskandidat\grqq\ (siehe \ref{7F15}) zu erfahren ist. Barney studierte fünf Jahre Ausdruckstanz und sechs Jahre Stepptanz, was er Kent Brockman in der Folge \glqq Vom Teufel besessen\grqq\ (siehe \ref{1F08}) erzählt. Außerdem hat Barney eine ausgebildete Gesangsstimme, was ihm einen Platz in Homers Barbershop-Gruppe \glqq Die Überspitzen\grqq\ \index{"Uberspitzen} und den Gewinn des Grammy sicherte. Eine Karriere als Astronaut hat Barney dadurch vergeben, dass er vom alkoholfreien Sekt, mit dem NASA-Vertreter mit ihm auf seine Einstellung als Raumfahrer angestoßen haben, abhängig wurde. In der Folge \glqq Barneys Hubschrauber-Flugstunde\grqq\ (siehe \ref{BABF14}) wird Barney zum Anti-Alkoholiker, was sein Leben ziemlich ändert. In einer Folge wird enthüllt, das Barney in seiner Jugend ein abstinenter Musterschüler war, bis er von Homer erstmals zum Trinken animiert wurde (siehe \glqq Einmal als Schneekönig\grqq , \ref{9F07}). In der Episode \glqq Buchstabe für Buchstabe\grqq\ (siehe \ref{EABF07}) wird er allerdings wieder rückfällig.


\subsection{Capital-City-Knalltüte}
Die Captial-City-Knalltüte ist das Maskottchen der Capital-City Baseballmannschaft. In der Folge \glqq Das Maskottchen\grqq\ (siehe \ref{7F05}) erhält es Unterstützung durch Homer.


\subsection{Horatio McCallister}\index{McCallister!Horatio}\label{HoratioMcCallister}
Horatio McCallister ist der stadtbekannte glasäugige Seemann. Neben dem Glasauge hat er auch noch mindestens ein Holzbein. Ein Bein hat er bei Dreharbeiten in Mexiko verloren, als er Mitarbeiter einer Film-Crew war (siehe \glqq Projekt Weltraumsand\grqq, \ref{YABF06}). Er ist Besitzer des Spezialitätenrestaurants \glqq Rusty Barnacle\grqq , das später zum \glqq The Frying Dutchman\grqq\ umbenannt wird. Wenn er etwas sagt, begleitet er es meist mit dem Zusatz \glqq Harr!\grqq . Die Geburt seines Sohnes bezeichnete er als \glqq Fang des Tages\grqq . Sein Sohn ist genauso alt wie Maggie Simpson (\glqq Und Maggie macht drei\grqq , \ref{2F10}). In der Episode \glqq Mensch gegen Maschine\grqq\ (siehe \ref{ZABF06}) behauptet er allerdings, verheiratet und kinderlos zu sein.


\subsection{Jeff Albertson}\index{Albertson!Jeff}\index{Comicbuchverkäufer}\label{JeffAlbertson}
Der Verkäufer im Comic-Ladens taucht in etlichen Episoden auf, bekam aber erst in der achten Folge der 16. Staffel (\glqq Homer und die Halbzeit-Show\grqq , \ref{GABF02}) einen Namen: Jeff Albertson. Er spiegelt das Klischee eines Nerds und über\-stei\-ger\-ten Science-Fiction-Fans wieder. Diese Parodie eines Comic-Fans verwenden Macher der Simpsons auch oft, um sich über ihre eigenen Fans lustig zu machen. In der Episode \glqq Drei Freunde und ein Comic-Heft\grqq\ (siehe \ref{7F21}) gibt er an, dass er Ägyptologie und Byzantismus studiert hat. Außerdem hat er einen Abschluss in Chemie (siehe \glqq Manga Love Story\grqq , \ref{SABF03}).

Obwohl er nicht sehr sportlich scheint, gibt er im Freizeitzentrum Unterricht in der Jahrtausenden alten Kunst des Shaolin Kung Fu (\glqq Die Chroniken von Equalia\grqq , siehe \ref{KABF22}).

Er hatte in der Folge \glqq Die schlechteste Episode überhaupt\grqq\ (siehe \ref{CABF08}) ein bizarres und kurzlebiges Verhältnis zur betagten Agnes Skinner. In der Episode \glqq Hochzeit auf Klingonisch\grqq\ (siehe \ref{FABF12}) hatte er eine Affäre mit Edna Krabappel.

In seiner Freizeit zeichnet er eine eigene Comicbuchreihe: Everyman. Das Comic wird schließlich so erfolgreich, dass das Comic verfilmt wird. Er kann sich gegen die Produzenten des Films durchsetzen und darf den Filmhelden besetzen. Er entscheidet sich für Homer (siehe \glqq Everyman begins\grqq , \ref{LABF13}).

Laut Matt Groening\index{Groening!Matt} ist der \glqq Comic Book Guy\grqq\ eine Parodie auf ihn selber, zudem gefällt es ihm, sich in verschiedene Rollen hineinzuversetzen, wie zum Beispiel Batman und anderen Actionhelden. Er ist Besucher vieler Science-Fiction- und Fantasy-Conventions. Das Heck seines Autos ist von vielen Aufklebern verziert, wie zum Beispiel \glqq My other car is a Millenium Falcon\grqq\ oder \glqq I brake for Tribbles\grqq\ \cite{Wikipedia}.


\subsection{Disco Stu}\index{Disco Stu}\index{Disco Stu}
Disco Stu tritt immer als Nebenrolle auf, die einen Bezug auf die Disco-Kultur der 70er Jahre hat. So sagt er in einer Folge, in der Marge Zucker in Springfield verbieten will (\glqq Die süßsaure Marge\grqq , \ref{DABF03}): \glqq Es war in den 70er Jahren, als Disco Stu von diesem weißen Zeug abhängig wurde\grqq\ (eine Anspielung auf Kokain). Ansonsten spricht Disco Stu meist in Reimen und von sich selbst in der dritten Person. Der Name soll wohl suggerieren, dass ihm zum Disco Stud (Disco-Hengst) noch ein wenig fehlt. Doch eigentlich hasst er Discos, was er in der Folge \glqq It's only Rock'n'Roll\grqq\ (siehe \ref{DABF22}) gesteht.

In der Episode \glqq Drum prüfe, wer sich ewig bindet\grqq\ (siehe \ref{GABF04}) ist zu erfahren, dass Selma kurzfristig mit ihm verheiratet war.

\subsection{Dr. Colossus}\index{Colossus!Dr.}
Dr. Colossus ist ein Verbrecher mit Superkräften, der in Springfield lebt. Er war mit der Schöpferin der Malibu-Stacy-Puppe, Stacy Lovell verheiratet. Zu seinen Geheimwaffen zählen seine Colosso-Stiefel, die es ihm erlauben, bis zu drei Meter groß zu werden (\glqq Wer erschoss Mr. Burns? Teil 2\grqq , \ref{2F20}). Trotzdem ist er der erfolgloseste Schurke, den die Welt je gesehen hat und es ist äußerst selten, dass ihm seine Vorhaben auch gelingen. Die Figur ist eine Parodie auf Dr. Oktopus aus den Spider-Man-Comics.


\subsection{Fat Tony}\index{Fat Tony}\index{D'Amico!Anthony}\index{Williams!Anthony}\label{FatTony}
Fat Tony (in der Episode \glqq Auf in den Kampf\grqq , siehe \ref{4F03}, wird sein Name mit Anthony D'Amico angegeben, in der Folge \glqq {Verbrechen lohnt sich nicht\grqq\ heißt er Anthony Williams, siehe \ref{8F03} und in der Episode \glqq 22 für 30\grqq, siehe \ref{WABF10}, heißt er Marion D'Amico) ist der Boss der Springfielder Mafia. Er war mit Anna Maria verheiratet. Seine Frau wurde ermordet. Aus dieser Ehe ging sein Sohn Michael hervor. 

Durch sein Engagement bekam die Springfielder Grundschule behindertengerechte Eingangsrampen, die jedoch aufgrund der mangelhaften Qualität des Betons direkt nach Fertigstellung zusammenbrachen. Weiterhin hat die Mafia die Mensa der Grundschule mit Rattenmilch beliefert. Regelmäßige Zahlungen der Mafia an Bürgermeister Quimby und Polizeichef Wiggum sorgen dafür, dass Tony nahezu ungehindert seinen Geschäften nachgehen kann.

Auch die Simpsons hatten unfreiwillige Kontakte zur Mafia: Bart arbeitete kurzzeitig in der Folge \glqq Verbrechen lohnt sich nicht\grqq\ (siehe \ref{8F03}) in der Bar der Mafiosi. Diese Folge ist eine Anspielung auf den Robert De Niro-Film \glqq In den Straßen der Bronx\grqq . Außerdem manipuliert Bart im Sinne Fat Tonys die Spielergebnisse beim Basketball (\glqq 22 für 30\grqq, \ref{WABF10}). Als Fat Tony in der Episode \glqq Der Koch, der Mafioso, die Frau und ihr Homer\grqq\ (siehe \ref{HABF15}) angeschossen wird, unterstützen Homer und Bart Fat Tonys Sohn Michael bei dessen Geschäften.

Marge Simpson hatte über Homer Kontakt zur Mafia: Marges Brezelverkauf (\glqq Marge und das Brezelbacken\grqq , \ref{4F08}) verlief sehr schleppend. Auf Homers Bitte hin half Fat Tony aus und sorgte für \glqq Insolvenzen\grqq\ diverser anderer Snack-Anbieter, sodass Marge ein Monopol aufbauen konnte. Sie war jedoch nicht bereit, an die Mafia Schutzgelder zu zahlen. Diese Weigerung endete darin, dass die anderen Anbieter eine Gruppe der Yakuza anheuerten, die eine Prügelei mit der Mafia in dem Garten der Simpsons anzettelten.

\subsection{Gil Gunderson}\index{Gil}\index{Gunderson!Gil}\label{GilGunderson}
Gil ist der typische Verlierertyp. Er scheitert in seinem Beruf genauso wie in seinem Privatleben. Er ist zwar verheiratet, doch seine Frau betrügt ihn (\glqq Marge Simpson im Anmarsch\grqq, \ref{AABF10}). Er hat eine Tochter (\glqq Dogtown\grqq, \ref{WABF15}). Er ist mittlerweile Großvater und hat eine Enkelin (siehe \glqq Weihnachten in Florida\grqq, \ref{YABF02}). In seiner Wohnung hängt ein Poster mit der Aufschrift \glqq You're a winner\grqq\ (\glqq Du bist ein Gewinner(-typ)\grqq ), das beispielhaft für seinen unerschütterlichen, verzweifelten Optimismus ist, der immer wieder auf das Neue enttäuscht wird. Die Figur ist eine Anspielung auf die Hauptfigur Willy Loman\index{Loman!Willy} aus Arthur Millers Bühnenstück \glqq Tod eines Handlungsreisenden\grqq\ \cite{Wikipedia}. In der Episode \glqq Kill Gil: Vol. 1 \& 2\grqq\ (siehe \ref{JABF01}) lebte er fast ein Jahr bei den Simpsons. Er nimmt in der Episode \glqq Ich will nicht wissen, warum der gefangene Vogel singt\grqq\ (siehe \ref{JABF19}) einen Job als Wachmann an und wird bei einem Banküberfall erschossen.

Hier eine Auswahl seiner Jobs:
\begin{itemize}
 	\item Schuhverkäufer (\glqq Krustys letzte Versuchung\grqq , \ref{5F10})
	\item Autoverkäufer (\glqq Die Gefahr, erwischt zu werden\grqq , \ref{5F18})
	\item Computerverkäufer (\glqq Die große Betrügerin\grqq , \ref{AABF03})
	\item Türklingelverkäufer (\glqq Das Geheimnis der Lastwagenfahrer\grqq , \ref{AABF13})
	\item Kwik-E-Mart-Verkäufer (\glqq Schon mal an Kinder gedacht?\grqq , \ref{BABF03})
	\item Anwalt (u.a. in \glqq Am Anfang war die Schreiraupe\grqq , \ref{DABF16})
	\item Fahrlehrer (\glqq Geächtet\grqq , \ref{FABF17})
	\item Weihnachtsmann (\glqq Kill Gil: Vol. 1 \& 2\grqq, \ref{JABF01})
	\item Immobilienmakler (u.a. in \glqq Kill Gil: Vol. 1 \& 2\grqq, \ref{JABF01})
	\item Wachmann (\glqq Ich will nicht wissen, warum der gefangene Vogel singt\grqq , \ref{JABF19})
	\item Versicherungsmakler (\glqq Dänisches Krankenlager\grqq, \ref{XABF13})
	\item Handelsvertreter (\glqq Himmlische Geschichten\grqq, \ref{XABF17})
	\item Manager des Fast-Food-Restaurants Razzle Dazzle's (\glqq Die Rückkehr der Pizza-Bots\grqq, \ref{QABF08})
\end{itemize}


\subsection{Hans Maulwurf}\index{Maulwurf!Hans}\index{Melish!Ralph}\label{HansMaulwurf}
Ralph Melish ist unter seinem Künstlernamen Hans Moleman\index{Moleman!Hans} bzw. Hans Maulwurf bekannt. Er sieht deutlich älter aus als er eigentlich ist. Sein Aussehen, wird damit erklärt, dass er Alkoholiker ist (\glqq Burns Erbe\grqq , \ref{1F16}). Er hat nur seltene und kurze Gastauftritte. Laut Führerschein, der in der Folge \glqq Selma will ein Baby\grqq\ (siehe \ref{9F11}) zu sehen ist, ist er 31 Jahre alt. In der Episode \glqq Gorilla Ahoi!\grqq\ (siehe \ref{YABF20}) behaupt er, 88 Jahre alt zu sein. Er ist relativ klein und hat einen braunen Kopf. Weil er stark kurzsichtig ist, trägt er eine Brille mit dicken Gläsern. In der Folge \glqq Die Jazz-Krise\grqq\ (siehe \ref{VABF11}) stellt er allerdings fest, dass er eigentlich keine Brille brauche. In einer Folge soll er als Bart-Ersatz dienen (Homer: \glqq Küss ihn mal, Marge. Er schmeckt nach alten Kartoffeln\grqq ), in einer anderen sitzt er im Gefängnis und wartet auf seine Hinrichtung oder wird sogar lebendig begraben. Dr. Hibbert hat ihn einmal in einem Röntgengerät vergessen. Er ist der absolute Pechvogel, der immer dann hinhalten muss, wenn die anderen Charaktere zu schade sind. Als er beispielsweise in der Episode \glqq Ich will nicht wissen, warum der gefangene Vogel singt\grqq\ (siehe \ref{JABF19}) in der First Bank of Springfield arbeitet, wird diese überfallen. Sein Beitrag beim \glqq Springfield Film-Festival\grqq\ (siehe \ref{2F31}) war \glqq Von einem Fußball getroffen\grqq , was Homers Favorit war. Seine Hobbys sind Kochen, Basketball spielen und ins Kino gehen. Er hört gerne Musik von Metallica\index{Metallica} (\cite{ET742}).

Er arbeitet unter anderem als Zeitungsreporter beim Springfield Shopper (siehe \glqq Feigheit kommt vor dem Fall\grqq , \ref{SABF18}).

In der Episode \glqq Fett ist fabelhaft\grqq\ (siehe \ref{TABF06}) wird behauptet, er sei viermal Bürgermeister von Springfield gewesen. Während seiner Amtszeit hatte er achtmal einen ausgeglichenen Haushalt.

\subsection{Familie Lovejoy}

\subsubsection{Reverend Timothy Lovejoy}\index{Lovejoy!Timothy}\label{TimothyLovejoy}
Reverend Timothy Lovejoy ist der Pfarrer der \glqq First Church of Springfield\grqq . Der Einfachheit halber scheinen fast alle Bürger Springfields dieser protestantischen Kirche anzugehören. Er studierte an der Christlichen Universität in Texas.

Lovejoy ist ein ruhiger und toleranter, manchmal gleichgültiger Zeitgenosse und tritt außerhalb seiner Kirche mit gelegentlichen sarkastischen Anmerkungen in Erscheinung. Lovejoy ist passionierter Modelleisenbahner. Vom tief gläubigen Gemeindemitglied Ned Flanders wird Lovejoy regelmäßig mit religiösen Fragen genervt, während er mit seiner Eisenbahn spielt. Dabei kommt es häufiger vor, dass seine Lok Feuer fängt. Seine Frau heißt Helen und ist im Gegensatz zu ihrem Mann überaus geschwätzig. Die beiden haben eine hübsche Tochter namens Jessica, die etwa im Alter von Bart Simpson ist, jedoch nicht die örtliche Schule sondern ein Internat besucht.

Obwohl Lovejoy Reverend ist, hat er keine eigene Bibel, sondern leiht sie sich jede Woche von Neuem von der Stadtbücherei aus. Sein Wissen in theologischen Fragen und seine Fähigkeiten zur Exegese sind beschränkt. Als einer der vielen Vertreter des Christentums außerhalb des Katholizismus in Amerika ist er mit den \glqq Kleiderträgern\grqq\ auf Kriegsfuß (\glqq Der Vater, der Sohn und der heilige Gaststar\grqq, \ref{GABF09}), zudem hilft er in \glqq In den Fängen einer Sekte\grqq\ (siehe \ref{5F23}) Marge dabei, ihre Familie von der Sekte der \glqq Fortschrittarier\grqq\ zu befreien. Zu Geistlichen anderen Konfessionen wie der römisch-katholischen Kirche und der jüdischen Gemeinde hat er ein distanziert-freundschaftliches Verhältnis.

Die Darstellung des Reverends hat sich im Laufe der Zeit gewandelt. Während er in den ersten Simpsons-Staffeln liberal-inhaltslose Seelsorger und theologisch nur mäßig gebildete Pfarrer parodierte, tritt er nun häufig als sittenstrenger und mit Hölle drohender Fundamentalist auf, voller Doppelmoral und falscher Frömmigkeit. Dies dürfte mit der Zunahme von christlichem Fundamentalismus Ende der 90er und Anfang des neuen Jahrtausends in den USA zu tun haben.

In der Coverband Covercraft war er als Gitarrist tätig.

\subsubsection{Helen Lovejoy}\index{Lovejoy!Helen}
Helen, die Ehefrau von Reverend Lovejoy, ist sehr geschwätzig und arbeitet als lokale Telefonseelsorgerin. Aufgrund von Ned Flanders Daueranrufen gibt sie diese Stelle kurzzeitig an Marge Simpson ab.

\subsubsection{Jessica Lovejoy}\index{Lovejoy!Jessica}
Jessica ist die Tochter von Timothy und Helen. Bart war in der Episode \glqq Barts Freundin\grqq\ (siehe \ref{2F04}) so sehr in sie verliebt, dass er Alles für sie getan hat. Barts Beschreibung für Jessica: Sie ist wie ein Fliegenpilz, lieblich von außen und dabei sehr, sehr giftig \cite{SpringfieldAt}.
 
\subsection{Familie Van Houten}
 
\subsubsection{Kirk Van Houten}\index{Van Houten!Kirk}\label{KirkVanHouten}
Kirk Van Houten ist der Vater von Milhouse. Seine Mutter lebt in Sizilien (\glqq Marge und der Frauen-Club\grqq , \ref{GABF22}). Inzwischen ist er ehemaliger Manager der ortsansässigen Keksfabrik \glqq Southern Cracker\index{Southern Cracker}\grqq . Nach der Scheidung von seiner Frau Luann, deren Ursprung ein Streit bei einer Party bei den Simpsons war (\glqq Scheide sich, wer kann\grqq , \ref{4F04}), lebt er in einem Männer\-wohn\-heim.

Während seiner College-Zeit war er der Stürmerstar des Lacrosseteams, den \glqq Gaudgers\grqq. So fungiert er als Homers Co-Trainer des Kinder-Lacrosseteams (\glqq Mein peinlicher Freund\grqq , \ref{VABF22}).

Kirk spielt in der Coverband Covercraft Keyboard und Tamburin. Nach der Scheidung versucht er, eine Gesangskarriere einzuschlagen, was ihm allerdings misslingt. Ferner versucht er sich als DJ (siehe \glqq Keine Frau für Moe\grqq, \ref{XABF20}).

\subsubsection{Luann Van Houten}\index{Van Houten!Luaan}\label{LuannVanHouten}
Luann ist die Mutter von Milhouse. Sie wurde in Shelbyville geboren. Ihre Mutter lebt vermutlich in Omaha (\glqq Die geheime Zutat\grqq , \ref{FABF20}). Sie war mit Kirk Van Houten verheiratet. Auf einer Dinnerparty bei den Simpsons fällte sie den Entschluss, sich scheiden zu lassen. Sie hatte einige Dates mit verschiedenen Männern, u.a. mit Pyro\index{Pyro} von den American Gladiators \cite{SpringfieldAt}. Bei der Windpockenparty bei den Simpsons kommen sie und ihr Ex-Mann sich wieder näher. Es scheint so, als wollten sie einen Neuanfang wagen (\glqq Milhouse aus Sand und Nebel\grqq , \ref{GABF19}). In der Episode \glqq Kleiner Waise Milhouse\grqq\ (siehe \ref{JABF22}) heiraten sie wieder. Auf der anschließenden Kreuzfahrt während der Flitterwochen gehen sie über Bord und stranden auf einer einsamen Insel.

Laut Aussage von Milhouse in der Folge \glqq Air Force Grampa\grqq\ (siehe \ref{TABF13}) ist Luann eine Cousine von Kirk.

\subsubsection{Milhouse Van Houten}\index{Van Houten!Milhouse}\label{MilhouseVanHouten}
Milhouse Mussolini Van Houten ist Barts bester Freund und unglücklich in Lisa verliebt. In der Episode \glqq Marge und der Frauen-Club\grqq\ (siehe \ref{GABF22}) bringt er Lisa Italienisch bei. Er spricht außerdem noch Spanisch (\glqq Moe Szyslak und das Königreich des Kristallschädels\grqq, \ref{QABF15}). Lisa küsst ihn sogar. Aber er macht mit seinem Machogehabe die Romanze zunichte. Er hat blaue Haare und trägt eine Brille. Milhouse gehört nicht gerade zu den beliebtesten Kindern in der Schule und so genießt er in Barts Gegenwart wenigstens bei einigen ein bisschen Anerkennung. Milhouse ist von sehr schmächtiger Statur und träumt davon, eines Tages die Welt als Superheld zu retten. Homer kann ihn nicht leiden und vergisst immer seinen Namen. Milhouses Eltern sind geschieden und er lebt bei seiner Mutter. Er sollte den \glqq Fallout Boy\grqq\ für den neuen \glqq Radioactive Man\grqq -Film spielen, verliert aber während der Dreharbeiten die Lust. Milhouse ist zudem Bettnässer und vermutlich homosexuell veranlagt. Er hatte aber bereits Greta, die Tochter von Rainier Wolfcastle (\glqq Nach Kanada der Liebe wegen\grqq, \ref{DABF06}) und Samantha Stanky (\glqq Liebe und Intrige\grqq, \ref{8F22}) als Freundinnen. 


\subsection{Larry Burns}\index{Burns!Larry}
Larry Burns ist der uneheliche Sohn von Charles Montgomery Burns. Er arbeitet an einem Souvenirstand an den Bahngleisen in der Nähe von Capital City, wo er einmal einen Zeppelin gesehen hat. Dort trifft er auf seinen Vater und folgt diesem nach Springfield. Nach ein paar Tagen voller peinlicher Ereignisse und einer angeblichen \glqq Entführung\grqq\ von ihm und Homer inszeniert, streitet er sich mit seinem Vater, verlässt Springfield und kehrt nach einer Woche schließlich wieder zu seiner eigenen Familie zurück.

 
\subsection{Moe Szyslak}\index{Szyslak!Moe}\label{MoeSzyslak}
Moe (eigentlich Morris) Szyslak ist der Barkeeper von Homers Lieblingsbar (Moe's Tavern), auch wenn die Bar von der Gesundheitsbehörde als bedenklich eingestuft wurde (siehe \glqq Ein Herz und eine Krone\grqq , \ref{TABF08}). Hier gibt es Homers Lieblingsgetränk: Duff Bier. Er betrieb in seiner Bar außerdem mehrere illegale Unternehmen, beispielsweise ein Casino und eine illegale Chirurgie. Moe hat Homer als Boxchampion gemanagt und gehört der freiwilligen Feuerwehr an. In den ersten Folgen wurde er regelmäßig von Barts Scherzanrufen geplagt, die in den späteren Episoden aber eingestellt wurden.

Zu den Stammgästen gehören neben Homer auch sein Freund Barney Gumble (der zeitweise über Moes Kneipe in einer Wohnung lebt), zwei Arbeitslose (Larry und Sam) sowie in regelmäßigen Abständen Homers Atomkraftwerkskollegen Lenny Leonard und Carl Carlson. Andere Gäste sind selten, vor allem Frauen lassen sich selten blicken, weswegen Moe hinter der Tür \glqq Frauentoilette\grqq\ ein kleines Büro hat. Ein Schild in der Bar verrät, dass Dienstag Frauenabend ist. Aber auch dies führt nicht zu erhöhtem Frauenbesuch in seiner Kneipe.

Moe ist in der Regel schlecht gelaunt, mürrisch, zynisch und ungehobelt. Aufgrund seines geringen Selbstbewusstseins ist er häufig depressiv, sehr misstrauisch und schnell gereizt. Moe ist Single und leidet sehr darunter, dass er alles andere als ein Frauenschwarm ist. Er scheint eine Schwäche für Marge Simpson zu haben, der er sich mehrmals zu nähern versucht. Allerdings nennt er sie des Öfteren Mitch.

Moes tragikomisches Verhältnis zu Frauen ist Thema mehrerer Episoden. So trifft er in der Folge \glqq Eine Frau für Moe\grqq\ (siehe \ref{5F12}) auf die hübsche Renée (im Original gesprochen von der Hollywood-Schauspielerin Helen Hunt), die halb aus Mitleid halb aus Interesse eine Beziehung mit ihm anfängt. Moe überschüttet Renée mit Geschenken. Als sein Geld knapp wird, begeht er mit Homers Hilfe einen Versicherungsbetrug. Homer wird geschnappt, ohne dass Moe ihm zu Hilfe kommt. Als Renée das erfährt, macht sie Schluss. In der Episode \glqq Große, kleine Liebe\grqq\ (siehe \ref{LABF06}) lernt er die zwergenwüchsige Maya über einen Chatroom im Internet kennen. Sie treffen sich und verlieben sich. Moe macht ihr einen Heiratsantrag und macht sich dabei über ihre Größe lustig. Daraufhin beendet sie die Beziehung.

In der Folge \glqq Das Erfolgsrezept\grqq\ (siehe \ref{8F08}) kommt Moes Kneipe aufgrund eines von Homer geklauten Rezepts für einen Cocktail (\glqq Flaming Moe\grqq ) groß raus. Zum ersten Mal in der Geschichte der Kneipe kann Moe sich eine Kellnerin leisten, mit der er eine Affäre beginnt. Diese überzeugt Moe, den Erfolg mit Homer zu teilen, doch Homer ist so verbittert, dass er allen das Rezept verrät und Moe und Homer am Ende beide leer ausgehen. Die Kellnerin verlässt Moe.

Moe hat zur Not immer eine Waffe zur Hand und ist Mitglied der National Rifle Association, sowie der \glqq Steinmetze\grqq\ (Anspielung auf die Freimaurerloge).

Eines seiner Geheimnisse ist, dass er immer Mittwochabends seine Bar schließt, um Obdachlosen Bücher vorzulesen. Eine Zeit lang war er ein sehr erfolgreicher Boxer, doch nach und nach ging seine Karriere den Bach runter. Seine Spitznamen als Boxer in chronologischer Reihenfolge waren Kid Grandios, Kid Annehmbar, Kid Grausig und Kid Moe.

Als kleiner Junge war er Schauspieler und spielte bei den kleinen Strolchen mit, wie in der Folge \glqq Filmstar wider Willen\grqq\ (siehe \ref{2F17}) zu erfahren ist. Nachdem sich Moe in der Episode \glqq Moe mit den zwei Gesichtern\grqq\ (siehe \ref{BABF12}) einer Schönheitsoperation unterzieht, erhält er eine Hauptrolle in einer Soap. Für diese Rolle hatte er früher schon einmal vorgesprochen und wurde abgelehnt. Als er jedoch glaubt, aus der Serie rausgeschrieben zu werden, inszeniert er mit Homer einen Eklat und fliegt wieder raus. 

In der Episode \glqq Geächtet\grqq\ (siehe \ref{FABF17}) verrät er, dass er ursprünglich aus Holland stammt. Während er in der Folge \glqq Hallo, du kleiner Hypnose-Mörder\grqq\ (siehe \ref{CABF10}) behauptet, in Indiana geboren worden zu sein. Allerdings wird in der Episode \glqq Volksabstimmung in Springfield\grqq\ (siehe \ref{3F20}) enthüllt, dass er ein illegaler Immigrant ist.

Moe war Gast in der Rateshow \glqq Me Wantee\grqq , die stark an \glqq Wer wird Millionär\grqq\ erinnert. Er gewinnt dort 500.000 US Dollar. Als Telefonjoker rief er Homer an und Lisa nannte ihm die richtige Antwort (\glqq Hallo, du kleiner Hypnose-Mörder\grqq , \ref{CABF10}).

Den Vornamen Moe hat er nur angenommen, damit er für seine Bar kein neues Schild kaufen musste, wie er in der Folge \glqq Große, kleine Liebe\grqq\ erzählt.

Obwohl Moe des Öfteren in der Springfielder Kirche zu sehen ist, hat er Erfahrungen mit verschiedenen Sekten. In der Folge \glqq Der beste Missionar aller Zeiten\grqq\ (siehe \ref{BABF11}) gibt er an, aus der Kirche ausgeschlossen worden zu sein. So gibt er in der Folge \glqq Die Farbe Grau\grqq\ (siehe \ref{NABF06}) an, Satanismus ausprobiert zu haben. In der Episode \glqq In den Fängen einer Sekte\grqq\ (siehe \ref{5F23}) gehört er genauso wie Homer und Barney der Sekte der Fortschrittarier an. Gegen Ende der Episode will er Barneys Durst mit einer Voodoo-Puppe wecken. In der Folge \glqq Ein gotteslästerliches Leben\grqq\ (siehe \ref{9F01}) sagt er, er sei als Schlangenbeschwörer geboren und werde als Schlangenbeschwörer sterben.

Moe scheint eine literarische Ader zu haben. So hat er u.a. fünf mehr oder weniger erfolgreiche Kinderbücher veröffentlicht. Eines trägt den Titel \glqq There's a rainbow in my basement\grqq\ (\glqq Homers Sieben\grqq , \ref{NABF22}).

Moe hat einen Bruder Marv und eine Schwester Minnie. Seine Gewister betreiben mit seinem Vater Morty die Matratzenkette \glqq Mattress King\index{Mattress King}\grqq. Da er sich in seiner Jugend weigerte, einen Konkurenzkette zu sabotieren, zerstritt er sich mit seinem Vater (siehe \glqq Der Matratzenkönig\grqq, \ref{XABF10}).

\subsection{Roy}\index{Roy}
Roy ist ein etwa 16- bis 19-jähriger Gast bei Marge und Homer Simpson, der in der Episode \glqq Homer ist \glq Poochie der Wunderhund\grq \grqq\ (siehe \ref{4F12}) auftritt und am Ende mit zwei Frauen in eine Wohngemeinschaft zieht. Roy diente dazu, um Lisas Aussage, dass häufig versucht wird, eine ehemals beliebte Serie durch das Auftreten neuer Figuren aufzupeppen, zu bekräftigen.

\subsection{Zahnfleischbluter Murphy}\index{Zahnfleischbluter Murphy}\label{ZahnfleischbluterMurphy}
Murphy ist ein stadtbekannter, jedoch verarmter, Jazz-Saxofonist. Bekanntestes Werk von Murphy ist \glqq Sax on the beach\grqq . Er starb in der Folge \glqq Zu Ehren von Murphy\grqq\ (siehe \ref{2F23}). Er ist wahrscheinlich ein verschollener Bruder von Doktor Hibbert. Sein Vorname laut Oscar, wie einer seiner Neffen in der Episode \glqq Lisa hat den Blues\grqq\ (siehe \ref{XABF11}) erwähnt.

Im neuen Vorspann, der seit der Folge \glqq Quatsch mit Soße\grqq\ (siehe \ref{LABF01}) ausgestrahlt wird, ist im Musikraum der Springfielder Grundschule ein Foto von ihm zu sehen.


\subsection{Herman}\index{Herman}\label{HermanHermann}
Herman Hermann ist Besitzer des Ladens \glqq Herman's Military Antiques\grqq . Er verlor seinen rechten Arm, als er diesen zu tief in eine Häckselmaschine drückte, wie in der Episode \glqq Bart schlägt eine Schlacht\grqq\ (siehe \ref{7G05}) zu erfahren ist. In der Folge \glqq Wem der Bongo schlägt\grqq\ (siehe \ref{RABF01}) ist zu sehen, dass ihm sein rechter Arm beim Trampen vom jungen Clancy Wiggum abgefahren wird. Clancy ist mit dem Hundefängermobil auf der Suche nach Homers Hund Bongo\index{Bongo}. In der Episode \glqq Projekt Weltraumsand\grqq\ (siehe \ref{YABF06}) heuert er bei einer Film-Crew an. Vor den Dreharbeiten hatte er noch beide Arme und während der Dreharbeiten ist er nur noch mit dem linken Arm zu sehen.

Seine Originalstimme erinnert an die von George W. Bush (Senior). Herman ist Mitglied der Steinmetze \cite{uloc},

Er besitzt viele seltene Waffen, unter anderem eine Miniatombombe, die in den 1950ern über Beatniks abgeworfen werden sollte.

Der Charakter hat auch eine gewisse Ähnlichkeit mit dem antisemitischen Ladenbesitzer aus \glqq Falling Down -- Ein ganz normaler Tag\grqq .

\subsection{Cletus Del Roy}\index{Cletus}\index{Del Roy!Cletus}\label{CletusSpuckler}
Cletus ist gekennzeichnet durch seine wenigen Zähne und seinen primitiven Bauerndialekt. Cletus und seine Frau Brandine\index{Del Roy!Brandine}, die auch seine Schwester ist, leben auf dem Lande, wo sie neben einer fetten Cousine noch eine Unzahl Kinder haben. In der Folge \glqq Maggies erste Liebe\grqq\ (siehe \ref{YABF13}) wird behauptet, Cletus sei der Vater von 31 Kindern. Die Kinder heißen: Tiffany, Heather, Cody, Dylan, Dermott, Jordan, Taylor, Brittany, Wesley, Rumer, Scout, Cassidy, Zoe, Chloe, Max, Hunter, Kendall, Katlin, Noah, Sasha, Morgan, Kira, Ian, Lauren, Q-Bert und Phil. In der Episode \glqq Stille Wasser sind adoptiv\grqq\ (siehe \ref{RABF04}) gibt er an, 17 Kinder zu haben.

Cletus Nachname ist eigentlich Del Roy, in den neueren Staffeln heißt er auf einmal Spuckler. Cletus Cousin heißt Merl\index{Merl}. Sein verstorbener Hund hieß Geech\index{Geech}.

Er besitzt die außergewöhnliche Fähigkeit, die Zukunft mithilfe von Holzschnitzereien voraussehen zu können (\glqq Die Verurteilten\grqq , \ref{FABF11}). 


\subsection{Professor John Frink}\index{Frink!John}\label{JohnFrink}
Professor Dr. John Frink trägt Brille, Laborkittel und hat einen Hasenzahn. Er ist hoch intelligent und hat einen IQ von 199 deshalb ist er auch Mitglied der Gruppe MENSA (\glqq Die Stadt der primitiven Langweiler\grqq , \ref{AABF18}). Er hat an der Cornell University studiert (siehe \glqq Arche Monty\grqq, \ref{XABF04}). Seine beiden Eltern waren Chemiker gewesen (siehe \glqq Frinkcoin\grqq, \ref{ZABF07}). Früher war er Professor am Springfield Heights Institut für Technik und entwickelte dort die \glqq Frinkiac\grqq -Computer-Baureihe (\glqq Volksabstimmung in Springfield\grqq , \ref{3F20}). Jetzt arbeitet er als selbständiger Wissenschaftler und Erfinder. Sein allererstes Patent war der Wählautomat AT-5000\index{AT-5000}, was er in der Episode \glqq Lisa will lieben\grqq\ (siehe \ref{4F01}) angibt. Ferner hat er die Frinkskrankheit entdeckt, das Element Frinkonium erschaffen und die Pille für acht Monate danach entwickelt (\glqq Springfield wird erwachsen\grqq , \ref{JABF07}). Ebenso hat er die VR-Brille Froculus entwickelt (\glqq Die virtuelle Familie\grqq, \ref{VABF18}). Er ist ebenfalls der Entwickler der Kryptowährung Frinkcoind (siehe \glqq Frinkcoin\grqq, \ref{ZABF07}).

Neben seiner Tätigkeit als Wissenschaftler ist er auch Unternehmer. So besitzt er die Firma Frink\-labs\index{Frinklabs} (\glqq Wem der Bongo schlägt\grqq , \ref{RABF01}) und arbeitet gelegentlich als Berater im Atomkraftwerk von Mr. Burns (\glqq Liebe liegt in der N2-O2-Ar-CO2-Ne-He-CH4\grqq , \ref{VABF07}).

Als die Lehrer an der Springfielder Grundschule streikten, ließ er sich nicht zweimal bitten und stellte sein ganzes Wissen sofort den Vorschülern zur Verfügung (\glqq Der Lehrerstreik\grqq , \ref{2F19}). In der Episode \glqq Barthood\grqq\ (\ref{VABF02}) war er als Hauslehrer von Bart tätig. Er arbeitete auch für die Regierung und war in den 60er Jahren bei der Entwicklung der Napalmbombe beteiligt (\glqq Wer ist Homers Vater?\grqq, \ref{HABF03}).

Sein Verhältnis zum weiblichen Geschlecht ist sehr kompliziert -- er sehnt sich einerseits nach einer Gefährtin, so entwickelt er einen sehr weiblichen Roboter und tanzt mit \glqq ihr\grqq\ -- andererseits kann er mit echten Frauen nicht viel anfangen. Er ist das typische Abbild eines Fachidioten und kommt deswegen auch nicht sehr gut bei Frauen an. In der Folge \glqq Der vermisste Halbbruder\grqq\ (siehe \ref{8F23}) steuert Professor Frink ein Flugzeug fern, in welchem sein Sohn sitzt. Nachdem dieser durch ein Fenster fliegt, sagt Professor Frink, dass dies seiner Frau nicht gefallen wird. Nur zu Lisa hat er ein gutes Verhältnis, oft erörtern sie wissenschaftliche Probleme oder philosophieren über die Gesellschaft. In der Folge \glqq Liebe liegt in der N2-O2-Ar-CO2-Ne-He-CH4\grqq\ (siehe \ref{VABF07}) weist Homer ihn daraufhin, dass seine Stimme an seinem Single-Dasein schuld sein könnte. Wissenschaftlich will er sich diesem Problem stellen und eine Lösung finden. Überwältigt von den Reaktionen erschafft er einen Algorithmus, der perfekt die einsamen Frauen und Männer Springfields zusammenbringen soll.

In der Episode \glqq Lisa will lieben\grqq\ (sieh \ref{4F01}) erfahren wir seine Telefonnummer: 555-5782. In der Folge \glqq Das Wunder von Burns\grqq\ (siehe \ref{NABF01}) hat der den Vornamen Jonathan.


\subsection{Snake}\index{Snake}\index{Turley!Chester}\label{Snake}
Snake (bürgerlicher Name: Chester Turley) ist ein Verbrecher, der Dank der Inkompetenz der Springfielder Polizei immer wieder auf freien Fuß gelangt. Bei seinen Verbrechen beschränkt er sich auf Diebstähle, seltener sind es Geiselnahmen. Sein Auto, das er Lil' Bandit getauft hat, ist sein Ein und Alles. Er hat einen Sohn mit Namen Jeremy, der, ganz der Vater, Lisas Rad klaut (\glqq Daddy, I'm stealing, I'm stealing!\grqq\ -- \glqq Oh, that's my little dude!\grqq ). Auch freundet er sich im Knast kurzzeitig mit Tingeltangel-Bob an. Immer zu erkennen ist Snake an der unter seinem T-Shirt-Ärmel eingesteckten Zigarettenschachtel \cite{Wikipedia}. Ursprünglich war Snake ein Abenteurer (in Anlehnung an Indiana Jones) der einen Goldschatz gefunden hat, der ihm jedoch von Moe gestohlen wurde. Daraufhin erst schlug er, als Racheakt an der Gesellschaft, die Verbrecherkarriere ein.


\subsection{Luigi Risotto}\index{Risotto!Luigi}\label{LuigiRisotto}
Luigi Risotto ist Italoamerikaner und Betreiber des Restaurants Luigi's. Sein Geburtstag ist der 9. Mai (siehe \glqq Lisa legt los\grqq, \ref{XABF01}). Immer wenn er in die Küche geht, hört man, wie er mit seinen Angestellten über die Kunden lästert. In der 17. Staffel kommt heraus, dass er kein Italienisch spricht, sondern nur Englisch mit starken Akzent, weil es das ist, was seine Eltern immer gesprochen haben. Man kann annehmen, dass seine Eltern ursprünglich aus Italien stammen, er selbst jedoch nicht \cite{Wikipedia}. Er ist Mitglied der Auslandspresse in Springfield (\glqq Wütender Dad -- Der Film\grqq , \ref{NABF07}).

\subsection{Jasper Beardsley}\index{Beardsley!Jasper}\index{Jasper}\label{JasperBeardsley}
Jasper ist Rentner und wohnt im Altenheim von Springfield. Seine verstorbene Frau hieß Estelle. Er ist ein guter Freund von Abraham Simpson. Sein linkes Bein hat er verloren und trägt stattdessen eine Holzprothese. Seine populärste Phase hatte er, als er sich in einem Gefrierschrank des Kwik-E-Marts einfror (\glqq Vertrottelt Lisa?\grqq , siehe \ref{4F24}), um in der Zukunft wieder aufzuwachen. Apu verkleidete ihn nämlich als Wikinger und stellte ihn gegen Geld zur Schau. Jasper arbeitet mit den Pfadfindern zusammen, damit diese üben können, alte Leute zu waschen. Auch als Aushilfslehrer hat er sich einen Namen gemacht: \glqq Aus dem Fenster sehen gibt's eins mit dem Paddel!\grqq .

\subsection{Lindsey Naegle}\index{Naegle!Lindsey}\label{LindseyNaegle}
Lindsey Naegle ist eine Finanzplanerin (\glqq Homer und das Geschenk der Würde\grqq, \ref{CABF04}), Werbeberaterin und Mobiltelefon\-ver\-käuf\-er\-in, die in manchen Episoden auftritt. In früheren Jahren leitete sie eine weibliche Diebesbande, der auch Sarah Wiggum angehörte (\glqq Die weiblichen Verdächtigen\grqq, \ref{QABF10}). Sie ist Single und versucht dies manchmal mit sehr drastischen Mitteln zu ändern, eine Anspielung auf \glqq Sex and the City\grqq . Ihre zeitweilige Kollegin, ärgste Konkurrentin bei der Partnersuche und trotz allem wohl beste Freundin ist Cookie Kwan. Sie fährt einen grünen VW Käfer, wie in der Episode \glqq Vergiss-Marge-Nicht\grqq\ (siehe \ref{KABF02}) zu sehen ist.


\subsection{Cookie Kwan}\index{Kwan!Cookie}
Cookie Kwan, \glqq Die Nummer 1 im Westbezirk\grqq , ist eine ameroasiatische Immobilienmaklerin, die, obwohl sie schon als Kind das Buch \glqq Lose your accent in 30 years\grqq\ (Verliere Deinen Akzent in 30 Jahren) gelesen hat, immer noch mit japanischem Akzent spricht. Sie hat ein Kind mit Bürgermeister Diamond Joe Quimby und arbeitet eigentlich in der East Side.


\subsection{Lionel Hutz}\index{Hutz!Lionel}\label{LionelHutz}
Lionel Hutz war in zahlreichen Folgen der Anwalt der Simpsons. In der Episode \glqq Todesfalle zu verkaufen\grqq\ (siehe \ref{5F06}) war er als Immobilienmakler zu sehen. Hutz nahm zeitweise den Namen Miguel Sanchez an, als er Probleme mit dem Staat bekam. Als er die Identität wechselte, war er auch Babysitter für Lisa, Bart und Maggie (zu dieser Zeit war er auch unter dem Namen Dr. Nguyen Van Phuoc bekannt).

Seine Vorbereitung auf seine Prozesse besteht meist darin, dass er sich etwas zu Essen in seinem Aktenkoffer mitnimmt. Während der Verhandlung neigt er zur Flucht, wenn seine Argumentation in sich zusammenfällt. Meistens ist er beeindruckt von der Argumentation der Gegenseite (\glqq Der gewinnt den Prozess!\grqq ) und hat wenig Vertrauen in seine eigenen Fähigkeiten als Anwalt. Seine Prozesse werden oft von Prozessen gefolgt, in denen er selbst der Beklagte ist (\glqq Und nun zum nächsten Prozess: Internationale Kirchenvereinigung gegen Lionel Hutz! -- Oh ja, das war DIE Sache\dots \grqq ).

Seine Beziehung zum örtlichen Richter Snyder ist unpraktischerweise am Boden, da er Snyders Sohn mehrfach überfahren hat. Er wirbt mit den ausgefallensten Mitteln für seine Dienste als Anwalt, wie Pizzas oder rauchende Affen. 

Hin und wieder sieht man ihn in Mülltonnen graben. Fraglich ist auch, ob er überhaupt über irgendeine juristische Qualifikation verfügt, geschweige denn eine Anwaltszulassung hat: \glqq Seien Sie unbesorgt, ich habe gerade in einer Bar \glq Matlock\grq\ gesehen, zwar ohne Ton, aber ich denke, ich habe das Wichtigste verstanden\grqq . Weiterhin fällt in diesem Kontext negativ auf, dass Lionel Hutz nur über spärliche Rechtschreibkenntnisse verfügt. Als er Richter Snyder den Urteilszettel überreicht, in diesem Fall eine von Hutz beschmierte Serviette, ist auf dieser das Wort \glqq Schuldig\grqq\ falsch geschrieben.

Nachdem der Originalsprecher von Lionel Hutz, Phil Hartman, von seiner Frau 1998 ermordet wurde, trat Lionel Hutz nur noch in Massenszenen auf und hatte von da an keine Sprechrollen mehr. Seitdem werden auch die Simpsons vor Gericht entweder von dem blau haarigen Anwalt vertreten, der auch oft als einer von Mr. Burns Anwälten auftritt oder von Gil, wie zum Beispiel in der Episode \glqq Die süßsaure Marge\grqq\ (siehe \ref{DABF03}). 

\subsection{Artie Ziff}\index{Ziff!Artie}\label{ArtieZiff}
Artie ist einer von Marge Simpsons Lovern, doch er schafft es aufgrund seiner Art nicht, Marge zu beeindrucken. Er lebte kurze Zeit auf den Dachboden der Simpsons, da er wegen Veruntreuung von Aktionärsgeldern gesucht wird. Er hat auch ein Verhältnis mit Selma Bouvier und sitzt derzeit im Springfielder Gefängnis eine Strafe ab. Artie wurde in Springfield geboren.

\subsection{Akira}\index{Akira}
Akira ist ein japanischer Kellner im Happy Sumo, der auch schon Barts Karatelehrer und außerdem Möbelverkäufer war. Er tritt hin und wieder in der Serie auf.

\subsection{Mrs. Glick}\index{Glick}
Mrs. Glick ist eine alte Dame, für die Bart einmal gearbeitet hat, weil er Geld brauchte. Obwohl er sehr hart gearbeitet hatte, bekam er nicht einmal einen Dollar dafür. In einer anderen Episode hat Dr. Hibbert bei einer Operation die Schlüssel für seinen Porsche in Mrs. Glick vergessen. Ihr Bruder Asa\index{Glick!Asa} ist im Zweiten Weltkrieg gestorben.

\subsection{Drederick Tatum}\index{Tatum!Drederick}\label{Tatum}
Drederick Tatum, ist ein bekannter Boxer und Schwergewichtsweltmeister. Er wurde in Springfield geboren und er wurde wegen Raubüberfalls und Totschlags zu einer Haftstrafe verurteilt. Vermutlich ist seine Rolle eine Anspielung auf den amerikanischen Boxer Mike Tyson. Er tritt in verschiedenen Rollen in der Serie auf. Sowohl als Olympiasieger von 1984 in der Folge \glqq Am Anfang war das Wort\grqq\ (siehe \ref{9F08}), als auch in einem Boxkampf gegen Homer Simpson, aus dem Homer in letzter Sekunde von Moe gerettet wurde (\glqq Auf in den Kampf\grqq , \ref{4F03}). Manchmal fungiert er auch als Werbeträger für eher minderwertige Lifestyle-Produkte. Sein Manager ist Lucius Sweet.

Ursprünglich war er ein schüchterner, dürrer Springfielder, aber sein Leben machte eines Tages eine Rechtskurve, als ihm ein Rowdy am Strand Sand ins Gesicht kickte. Er schwor Rache und trainierte jahrelang, bis er sehr muskulös und durchtrainiert zurückkam und den Sand entfernen ließ. Sein Statussymbol ist der goldene Spuckeimer.

\subsection{Lucius Sweet}\index{Sweet!Lucius}
Lucius Sweet ist ein afroamerikanischer Boxpromoter und der Manager von Drederick Tatum. Er stellt eine ziemlich eindeutige Parodie auf Don King dar, was man vor allem an seiner Frisur, deren Geheimnis laut einer Trading Card aus zwei Teilen Gel auf ein Teil Superkleister liegt, erkennt.

Sweet trat erstmals in der Episode \glqq Auf in den Kampf\grqq\ (siehe \ref{4F03}) auf und war seitdem nicht mehr in einer größeren Rolle zu sehen. In \glqq Die Trillion-Dollar-Note\grqq\ (siehe \ref{5F14}) steht er in der Warteschlange des Finanzamtes und wischt sich mit einem 1000 Dollar-Schein den Schweiß von der Stirn.

\kommentar{in "Milhouse aus Sand und Nebel" ist er in der Springfielder Kirche für Afroamerikaner zu sehen.}

\subsection{Jacques Brunswick}\index{Brunswick!Jacques}
Als Homer Marge ein wirklich enttäuschendes Geburtstagsgeschenk macht, eine Bowlingkugel mit seinem eigenen Namen eingraviert, geht Marge aus Wut tatsächlich bowlen, ganz zum Erstaunen Homers. Auf der Bowlingbahn trifft sie auf Jacques, einen französischen Casanova, der Marge Nachhilfestunden geben will. Schnell stellt sich heraus, dass Jacques nur an Marge selbst interessiert ist. Als sich Marge nun zwischen diesem Lustmolch und Homer entscheiden muss, nimmt sie natürlich lieber Homer.

In der Episode \glqq Der schöne Jacques\grqq\ (siehe \ref{7G11}), in der er auf Marge trifft, verliert er zweimal seinen französischen Akzent. Einmal, als er nach Zwiebelringen schreit, und einmal, als er voller Vorfreude in seinem Singleapartment in der Fiesta Terrace\index{Fiesta Terrace} auf Marge wartet: \glqq Jacques, heut' legst du wieder eine flach!\grqq 

Jacques tritt auch in späteren Episoden mehrmals auf, aber eigentlich immer nur im Hintergrund und ohne Sprechrolle.

\glqq Brunswick\grqq\ ist die Marke vieler Bowling-Artikeln, weshalb der Name für Jacques sehr treffend ist.

In \glqq Betragen mangelhaft\grqq\ (siehe \ref{7F14}) erfahren wir, dass er einen Hund besitzt. Und in \glqq Das Erfolgsrezept\grqq\ (siehe \ref{8F08}) stellt sich heraus, dass er ein großer Fan von Aerosmith\index{Aerosmith} ist.

Jacques ist auch im Vorspann zu sehen, zusammen mit Helen Lovejoy, Apu, Moe, Barney, Zahnfleischbluter Murphy und Chief Wiggum, an denen Bart auf seinem Skateboard vorbeifährt.


\subsection{Eleanor Abernathy}\index{Katzenlady}\index{Abernathy!Eleanor}\label{EleanorAbernathy} 
Die als \glqq Katzenlady\grqq\ bezeichnete Frau heißt mit bürgerlichem Namen Eleanor Abernathy und ist eine ältere, verrückte Frau, die allein mit unzähligen Katzen lebt und diese aus den nichtigsten Gründen auf andere Menschen wirft. Sie ist immer verwirrt, was durch ihren irren Blick und ihre unordentlichen grauen Haare unterstrichen wird. Sie hat zudem eine Sprachstörung und ist nur sehr schwer zu verstehen. In einer Folge werden ihre Gedanken durch eine Maschine deutlich ausgesprochen: \glqq Ich habe eine seltene Krankheit, ich brauche mehr Katzen.\grqq\ In einer anderen Folge wird angespielt, dass ihre Katzen Junge von Snowball II seien, welche ihr die Simpsons unterjubelten und sie dadurch Wahnsinnig wurde. In ihrer Jugend war sie eine fleißige Schülerin mit großen Zielen, die sie durch das Erreichen eines Doktors in Medizin (Studium in Harvard) und eines Doktors in den Rechtswissenschaften (Studium in Yale) auch erreicht hat. Über ihrer Arbeit wurde sie jedoch depressiv und zur Katzenlady (\glqq Springfield wird erwachsen\grqq , \ref{JABF07}).

\subsection{Jeremy Peterson}\index{Peterson!Jeremy}\label{JeremyPeterson} 
Der Teenager im Stimmbruch hat Akne, rote Haare und ständig einen anderen jener miesen und schlecht bezahlten Jobs, wie sie typisch für amerikanische Teenager sind. Er war unter anderem schon Verkäufer bei Krusty Burgers, Autowäscher, Kinokartenverkäufer, Casinorausschmeißer, Hilfskraft in einem Baumarkt, Führer im Wachsfigurenmuseum, Einpacker im Supermarkt, Angestellter eines Arcadegeräte-Spielgeschäfts und Erschrecker in der Geisterbahn. Seine Sprechtexte sind mit einer durch den Stimmbruch bedingten Quietschstimme wiedergegeben. In der Folge \glqq Marge gegen Singles, Senioren, kinderlose Paare, Teenager und Schwule\grqq\ (siehe \ref{FABF03}) verliert er zum ersten Mal die hohe Stimme.

Darüber hinaus wird in der Folge \glqq Homers Bowling-Mannschaft\grqq\ (siehe \ref{3F10}), bekannt, dass er vermutlich der Sohn von Küchenhilfe Doris ist. Denn als er zu Homer und seinen Freunden sagt, dass sie nicht spielen können, weil am heutigen Tag nur Ligaspieler spielen dürfen, sagt er in etwa \glqq ich könnte nicht einmal meine Mutter spielen lassen\grqq\ und daraufhin sieht man Küchenhilfe Doris, wie sie zu ihm sagt \glqq ich habe keinen Sohn\grqq .

In der Folge, in der Grandpa im Krusty Burger arbeitet, erfährt man, dass der Teenager im Stimmbruch den Nachnamen Peterson trägt. Zitat von Grampa: \glqq Endlich ist der kleine Peterson weg!\grqq

In einer herausgeschnittenen Szene aus der Episode \glqq Freund oder Feind!\grqq\ (siehe \ref{1F18}), die auf der DVD der fünften Staffel zu sehen ist, spricht Skinner ihn mit Jeremy an. Daraus lässt sich schließen, dass sein vollständiger Name Jeremy Peterson lautet. In der Folge \glqq Gestrandet\grqq\ (siehe \ref{PABF11}) hingegen trägt er ein Namensschild, auf dem Steve steht.

In der Episode \glqq G.I. Homer\grqq\ (siehe \ref{HABF21}) wird er allerdings Mr. Friedman genannt.

\subsection{Reicher Texaner}\label{ReicherTexaner}
Der namenlose reiche Texaner ist eine Verkörperung und Parodie auf verschiedene Lobbies und versucht meistens irgendwelche umweltzerstörerischen Bauprojekte durchzupeitschen. Er stammt aus Conneticut (\glqq Rache ist dreimal süß\grqq , \ref{JABF05}). Geboren wurde er allerdings in New Hampshire (\glqq Fidel Grampa\grqq, }\ref{VABF19}). Er beleidigt gerne andere Personen (z.B. Lisa beim Tisch-deck-Wettbewerb) und zeigt seine Freude am liebsten durch seine zwei Revolver, mit denen er in die Luft schießt, während er von einem Bein auf das andere hüpft und \glqq jihaa\grqq\ schreit. Er trägt oft einen riesigen Cowboyhut und fährt einen luxuriösen Wagen mit den Hörnern eines Longhorns auf dem Kühler, was seine texanische Herkunft unterstreichen soll. Er hat Pogonophobie (Angst vor Bärten und Schnauzern), einen Zwang nach jedem Schießen viermal aufzutreten und er ist Republikaner. Das Schießen mit echten Patronen wurde ihm verboten. Außerdem gehören ihm einige Armenviertel in Springfield, ein Vergnügungspark, der aber geschlossen wurde, nachdem dort ein kleiner Junge seinen Kopf verloren hat, in der Folge \glqq Die unendliche Geschichte\grqq\ (siehe \ref{HABF06}) besaß er kurzzeitig das Atomkraftwerk und den höchsten Baum der Stadt. Er hat eine Tochter, Paris Texan\index{Texan!Paris}, und einen Enkel, den er trotz seiner Homosexualität immer noch liebt, was er sich jedoch anfangs nicht eingestehen wollte. Er ist Chef der Firma Omni-Pave\index{Omni-Pave} und einer Kupferrohrfabrik. In der Folge \glqq Vom Teufel besessen\grqq\ (siehe \ref{1F08}) ist er Senator.


\section{Die Herkunft der Namen}\label{HerkunftNamen}
Aber auch die Namen vieler Nebenfiguren haben einen tieferen Sinn \cite{Wikipedia}:
\begin{itemize}
  \item Homer ist der Name des Vaters und einer der Söhne von Matt Groening. Ein Homer Simpson existiert übrigens auch als Hauptfigur des satirischen Hollywood-Romans \glqq Tag der Heuschrecke\grqq\ (1939) von Nathanael West: Es ist hier ein naiver Provinz-Tölpel.
  \item Lisa war auch der Name von Matt Groenings Schwester. Ebenso könnte der Name an Lisa Marie Presley angelehnt sein.
  \item Auch bei Maggie stammte der Name wieder aus Matt Groenings Familie -- eine seiner Schwestern heißt ebenfalls Margaret, wird aber Maggie gerufen.
  \item Abraham wurde zufällig von den Simpsons-Autoren genommen. Aber auch der Name eines Großvaters von Matt Groening ist Abraham.
  \item Mona Simpson ist die Ex-Frau von Simpsons-Autor Richard Appel\index{Appel!Richard}. Er benannte Homers Mutter nach seiner Frau. Mona Simpsons ist eine bekannte US-amerikanische Romanautorin und Essayistin. Sie ist außerdem die leibliche Schwester von Apple-Mitbegründer Steve Jobs\index{Jobs!Steve} \cite{Burciu10}.
  \item Jacqueline Bouvier -- Bouvier war der Mädchenname von John F. Kennedys Ehefrau Jacqueline.
  \item Seymour Skinner -- Mike Reiss behauptet, der Simpsons-Autor Jon Vitti gab ihm den Namen nach dem Psychologen Burrhus Frederic Skinner\index{Skinner!Burrhus Frederic}, über den das Gerücht kursierte, er nehme seine eigenen Kinder als Versuchskaninchen (siehe \cite{Reiss19}).
  \item Dr. Pryor -- Der Name stammt ebenfalls vom Simpsons-Autor Jon Vitti. Der Name leitet sich aus dem englischen Verb \emph{to pry} (herumschnüffeln) ab, weil Dr. Pryor gerne im Leben seiner Kinder herumschnüffelt.
  \item Clancy Wiggum -- Der Mädchenname von Matt Groenings Mutter lautete Wiggum \cite{Vincent2013}.
  \item Miss Elizabeth Hoover -- Eine Grundschullehrerin Groenings.
  \item Apu Nahasapeemapetilon -- Apus Weg ins Leben, einer von Groenings Lieblingsfilmen. Simpsons-Autor Jeff Martin gab Apu den Nachnamen \glqq Nahasapeemapetilon\grqq , da einer seiner Schulfreunde so heiß.
  \item Maude, Rod und Todd Flanders -- reimen sich alle auf god (dt. Gott).
  \item Kang und Kodos (Aliens in den Horror-Folgen) -- Ein Klingone (Kang) bzw. ein Diktator (Kodos) aus Raumschiff Enterprise.
  \item Barney Gumble -- Barney Rubble (dt. Ge\-röll\-hei\-mer) aus der Familie Feuerstein.
  \item Troy McClure -- Die B-Movie-Schauspieler Troy Donohue und Doug McClure
  \item Dr. Julius Hibbert -- Er war ursprünglich eine Ärztin namens Julia, welche der Simpsons-Autor Jay Kogen nach seiner Freundin Julia Hibbert benannte.
  \item Dr. Nick Riviera (genannt Dr. Nick) -- Elvis Presleys Leibarzt George C. Nichopoulos, genannt Dr. Nick.
  \item Milhouse Van Houten -- höchst\-wahr\-schein\-lich nach Richard Milhous Nixon und Leslie Van Houten, einem Mitglied der \glqq Manson\grqq -Familie.
  \item Charles Montgomery Burns -- \glqq See, Montgomery burns\grqq : Anspielung auf den Ku-Klux-Klan, der Häuser von schwarzen Bürgern anzündete (Montgomery, Hauptstadt von Alabama).
  \item Artie Ziff -- Art Garfunkel (Simon \& Garfunkel), ebenfalls auch optische Ähnlichkeit.
  \item Dolph -- Ein Schulkamerad Matt Groenings hieß Dolph Timmerman. Matt Groening sagte, dieser sei zwar kein Rowdie gewesen, aber ein \glqq echt cooler Typ\grqq .
  \item Matt Groening wuchs in Portland, Oregon, auf. Dort gibt es eine \glqq Kearney Street\grqq . Nach dieser Straße könnte Kearney benannt sein.
  \item In Matt Groenings Heimatstadt Portland gab es auch eine Straße namens \glqq NE Flanders\grqq . Auf das Schild wurde des Öfteren ein \glqq D\grqq\ geschrieben.
  \item Der Name des Bürgermeisters Joe Quimby kommt von der Straße \glqq NW Quimby Street\grqq\ in Portland, Oregon.
  \item Der Vorname Waylon von Mr. Smithers wurde zu Ehren der schwulen Bauchrednerlegende Wayland Flowers\index{Flowers!Wayland} gewählt (siehe \cite{Reiss19}).
\end{itemize}