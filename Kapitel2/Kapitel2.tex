\HTMLOutput{

}{
	\begin{savequote}[55mm]
	Ich hätte nie gedacht, dass ich ein deutsches Flugzeug abschießen könnte -- doch im letzten Jahr habe ich mir das Gegenteil bewiesen.
	\qauthor{Abraham J. Simpson}
	\end{savequote}
}
\chapter{Episodenführer}\label{EpisodenFuehrer}
Die Beschreibung der Episoden ist im Wesentlichen den Seiten \cite{Epguides}, \cite{Fernsehserien} und \cite{SpringfieldShopper} entnommen.

Begonnen hat die Serie \glqq The Simpsons\grqq\ als Kurzfilmreihe innerhalb der \glqq Tracey Ullman Show\grqq\ am 19. April 1987 in den USA. Seit dem 17. Dezember 1989 laufen die Simpsons auf Fox Network in ihrer jetzigen, halbstündigen Version zur Prime-Time um 20:00 Uhr.
Die ersten drei Staffeln wurden von der Produktionsfirma Klasky-Csupo hergestellt, die auch schon an den \glqq Tracey Ullman Show\grqq -Kurzfilmen gearbeitet haben. Die späteren Staffeln wurden von \glqq Film Roman\grqq\ produziert.
Laut Produzent und Regisseur David Silverman dauerte die Produktion der Simpsons-Kurzfilme für die \glqq Tracey Ullman Show\grqq\ etwa vier Wochen. Heute benötigt die Produktion einer Staffel der Simpsons etwa sechs Monate \cite{PRO7St01}.

\section{Staffel 1}

\subsection{Der Babysitter ist los}\label{7G01}
Marge ist mit ihrem Eheleben unzufrieden. Sie vertraut sich einem Psychologen in einer Radiosendung an. Homer hört dies und wagt sich nicht mehr nach Hause. Schließlich besorgt er Blumen und organisiert einen romantischen Abend. Um ausgehen zu können, engagieren sie einen Babysitter. Der gemietete Babysitter entpuppt sich jedoch als in \glqq America's Most Armed and Dangerous\grqq\ gesuchter Verbrecher.

\notiz{
\begin{itemize}
  \item Marge gibt an, 34 Jahre alt zu sein.
  \item Musikzitat: Die Hintergrundmusik als Ms. Botz\index{Botz} (eigentlich Lucille Botzcowski\index{Botzcowski!Lucille}) sich nähert, stammt aus \glqq Der weiße Hai\grqq .
  \item Aufgrund grober Schnitzer der koreanischen Animationsfirma bedeutete diese erste Folge der Simpsons fast das frühzeitige Aus der Serie. Knapp 60 \% mussten neu animiert werden, um ein zufriedenstellendes Ergebnis zu erzielen.
\end{itemize}
}


\subsection{Bart wird ein Genie}\label{7G02}
Bei einem Intelligenztest in der Schule vertauscht Bart seinen Testbogen mit dem des Klassenprimus, Martin Prince\index{Prince!Martin} (siehe \ref{MartinPrince}). Danach wird er vom Schulpsychologen Dr. Pryor\index{Pryor!Dr. J. Loren} auf eine Hochbegabten-Schule geschickt. Bart ist von der Lernideologie dort zunächst begeistert: Jeder muss hier nur das tun, wozu er Lust hat. Doch schon bald wird Bart von seinen neuen Mitschülern gehänselt, weil er nicht mitkommt. Auch seine alten Freunde wollen mit dem \glqq genialen\grqq\ Bart nichts mehr zu tun haben.

\notiz{
\begin{itemize}
  \item Maggie baut mit ihren Spielsteinen \glqq EMCSQU\grqq , das steht für $E = mc^2$.
  \item Der IQ von Bart ist angeblich 216. In Wirklichkeit ist das der IQ von Martin Prince.
  \item Im Bücherregal der Hochbegabtenschule sind u.\,a. die Bücher Dante's Inferno, Moby Dick und Odyssee zu sehen.
  \item Fehler: Milhouse hat in dieser Episode einmal blaue Haare und später ist er mit schwarzen Haaren zu sehen.
  \item Bart legt beim Scrabble das Wort Kwyjibo\index{Kwyjibo}. Laut ihm bezeichnet es \glqq einen dicken, fetten, blöden, nordamerikanischen Affen, der die Haare verliert\dots\grqq\ und  Marge ergänzt: \glqq \dots und schnell die Geduld.\grqq  
\end{itemize}
}

\subsection{Der Versager}\label{7G03}
Ausgerechnet während Barts Schulklasse das Atomkraftwerk besucht und über Sicherheit und Nutzen desselben belehrt wird, verursacht Homer einen seiner zahlreichen Störfälle. Er wird gefeuert. Homer fühlt sich unnütz und will sich von einer Brücke stürzen. Doch zuvor rettet er noch seine Familie, die ihn vom Selbstmord abhalten will, vor einem heranbrausenden Auto. Und nun hat er eine neue Bestimmung gefunden: Er engagiert sich für mehr Sicherheit und wendet sich als Sicherheitsaktivist gegen seinen ehemaligen Arbeitgeber.

\notiz{
\begin{itemize}
	\item Erster Auftritt von: Wendell, Kernspalt, Otto, Charles Montgomery Burns und Waylon Smithers.
	\item Homer erwähnt eine Tante Sally. Ob es sich hier um eine weitere Schwester von Marge handelt, ist unklar.
	\item Homer wird als technischer Überwacher vom Vater von Sherri und Terri entlassen.
	\item Homer wird erst mit der Wiedereinstellung Sicherheitsinspektor im Atomkraftwerk.
	\item Fehler: Mr. Smithers Hautfarbe ist in dieser Episode schwarz (braun) -- ein Fehler der Animationsfirma.
\end{itemize}
}


\subsection{Eine ganz normale Familie}\label{7G04}
Die Simpsons werden zur alljährlichen Gartenparty von Homers Chef Mr. Burns eingeladen. Und weil dieser zerstrittene Familien hasst, wollen ihm die Simpsons eine Familienidylle vorspielen. Das überaus schwierige Unternehmen scheitert. Deshalb meldet Homer seine Familie bei dem Psychotherapeuten Dr. Marvin Monroe\index{Monroe!Dr. Marvin} an. Für die Sitzungen gehen die gesamten Ersparnisse drauf, trotzdem misslingt die Therapie.

\notiz{
\begin{itemize}
	\item Dr. Marvin Monroes Telefonnummer aus dem Werbespot lautet 1-800-555-HUGS.
	\item Filmzitat: In einer Anspielung auf den Film \glqq A Clockwork Orange\grqq\ (Uhrwerk Orange) wird die Simpsons-Familie in ein grell weißes Labor mit vielen Knöpfen gesetzt und mit El\-ek\-tro\-den ver\-ka\-belt. Jeder hat die Mög\-lich\-keit, den anderen Stromschläge zu verpassen.
  \item Auffälligkeit in dieser Episode: Homer ist mehr oder weniger \glqq normal\grqq\ und der Rest der Familie benimmt sich daneben.
  \item Erste \glqq Itchy \& Scratchy\grqq -Szenen.
  \item Fehler: Moe hat schwarzes Haar, Barney ist blond und Mr. Smithers hat violette Haare.
\end{itemize}
}


\subsection{Bart schlägt eine Schlacht}\label{7G05}
Lisa hat für ihre Lehrerin etwas gebacken, doch in der Schule nimmt ihr ein frecher Bengel die kleinen Kuchen weg. Bart stellt ihn zur Rede. Leider bekommt der Junge von einem Mitschüler Schützenhilfe und dieser Nelson ist ein gefürchteter Schläger: Bart bekommt eine Abreibung. Doch Barts Großvater weiß Rat.

\notiz{
\begin{itemize}
  \item In Hermans Modell der Stadt ist der \glqq Kwik-E-Mart\grqq\ fälschlicherweise mit \glqq Quick-E-Mart\grqq\ beschriftet.
  \item Filmzitat: Während Bart seine Truppe trainiert, blasen Trompeten die Titelmusik aus dem Film \glqq Patton -- Rebell in Uniform\grqq .
  \item Lisa nennt ihre Lehrerin \glqq Mrs. Hoover\grqq\ statt \glqq Miss Hoover\grqq .
  \item Das Aussehen von Herman basiert auf dem von John Swartzwelder\footnote{Einer der Drehbuchautoren der Simpsons.}\index{Swartzwelder!John}.
  \item Erster Auftritt von Abe Simpson (siehe \ref{AbeSimpson}).
  \item Abe Simpsons will die Begriffe \glqq BH\grqq\ und \glqq geil\grqq\ sowie die Formulierung \glqq Tritt in die Familienjuwelen\grqq\ im Fernsehen nicht mehr hören.
  \item Herman verlor seinen rechten Arm, als er ihn zu weit in einen Häcksler steckte.
  \item Im Rahmen einer Studie zur Erforschung der Gehirnaktivität in humorvollen Fernsehserien wurde diese Folge den Probanden gezeigt \cite{BrainHumor}.
  \item Das Altersheim, in dem Abe Simpson lebt, heißt hier \glqq Springfield Retirement Home\grqq.
\end{itemize}
}


\subsection{Lisa bläst Trübsal}\label{7G06}
Lisa ist ziemlich deprimiert. Ihr ist alles gleichgültig und im Schulchor fällt sie vor allem durch ihre schrägen Töne auf. Eines Nachts spielt sie Saxophon. Da hört sie aus der Ferne ein anderes Saxophon, das einen Blues spielt. Lisa geht den Klängen nach und lernt den Saxophonisten Murphy kennen. Für den ist Blues das Mittel schlechthin, um seinen Schmerz auszudrücken. Die beiden spielen zusammen und so überwindet Lisa ihr Tief.

\notiz{
\begin{itemize}
  \item Erster Auftritt von Zahnfleischbluter Murphy.
  \item Das Videospiel, welches Bart gegen Homer spielt, heißt Slugfest\index{Slugfest}.
\end{itemize}
}

	
\subsection{Bart köpft Ober-Haupt}\label{7G07}
Bart lernt die Bande des berüchtigten Rowdys Jimbo kennen und möchte unbedingt Mitglied werden. Er beteiligt sich daran, die Statue des verehrten Stadtgründers Jebediah Springfield mit Steinen zu bewerfen. Doch damit nicht genug: In der Nacht schleicht sich Bart in den Park und sägt der Statue den Kopf ab. Die ganze Stadt ist empört -- selbst Jimbo geht dies zu weit. Zu allem Überfluss fängt Jebediahs Kopf in Barts Schultasche zu sprechen an.

\notiz{
\begin{itemize}
  \item Erster Auftritt von Dolph, Jimbo, Kearney und Reverend Lovejoy.
  \item Filmzitat: Nachdem Bart den Kopf der Jebediah-Springfield-Statue abgesägt hat, wacht er am nächsten Morgen mit dem Kopf neben sich im Bett auf -- wie es in \glqq Der Pate\grqq\ mit einem Pferdekopf geschieht.
  \item Literaturzitat: Der Kopf von Jebediah Springfield spricht zu Bart und steht dabei für sein Gewissen -- ganz ähnlich dem schlagenden Herzen in der klassischen Geschichte von Edgar Allan Poe \glqq Das verräterische Herz\grqq\ (The Telltale Heart).
  \item Der vollständige Name des Stadtgründers: Jebediah Obadiah Zachariah Jebediah Springfield.
  \item Janey hat einen Hund namens Pepper.
\end{itemize}
}

	
\subsection{Es weihnachtet schwer}\label{7G08}
Weihnachten steht vor der Tür und die Kleinen präsentieren Homer und Marge ihre Wunschzettel: Lisa wünscht sich ein Pony, Bart eine Tätowierung. Als Bart sich dann tatsächlich tätowieren lässt, geht für das Entfernen der Tätowierung das gesamte Weihnachtsgeld drauf. Inzwischen erfährt Homer, dass es diesmal keine Weihnachtsgratifikation gibt. So nimmt der geplagte Vater einen Nebenjob als Weihnachtsmann an.

\notiz{
\begin{itemize}
	\item Marges Weihnachtsbrief an die Verwandten: \glqq Liebe Freunde der Familie Simpson, wir haben in diesem Jahr schlimme, aber auch glückliche Zeiten erlebt. Zuerst das Schlimme: Unser Kätzchen Schneeball wurde völlig unerwartet überfahren und ist jetzt im Katzenhimmel. Jetzt haben wir Schneeball die zweite, das Leben geht eben weiter. Da ich gerade vom Leben schreibe, Opa lebt immer noch bei uns, ungenießbar wie immer. Maggie hat Laufen gelernt, Lisa kriegt lauter Einser und Bart, na ja, wir lieben Bart. Wir sind alle in hoch weihnachtlicher Stimmung. Homer lässt auch grüßen. Glückliche Feiertage wünschen Euch, Die Simpsons.\grqq \cite{SpringfieldAt}.
  \item Fehler der Animationsfirma: Moe hat schwarze Haare, Barney ist blond \cite{SpringfieldAt}.
  \item Patty und Selma werden eingeführt, Knecht Ruprecht kommt zur Familie. Ned und Todd Flanders, die Nachbarn der Simpsons werden eingeführt.
\end{itemize}
}



\subsection{Vorsicht, wilder Homer}\label{7G09}
Weil die Nachbarn mit ihren neuesten Wohnmobil protzen, will Homer es ihnen gleichtun. Und er macht denselben Fehler: Er begibt sich in die Hände eines Kredithais, der den Simpsons ein altes Wohnmobil aufschwatzt. Damit unternimmt die Familie eine Fahrt ins Grüne. Leider stürzen sie dann in eine Schlucht, was eine denkwürdige Kettenreaktion hervorruft. Am Ende der Reaktionskette steht Homer als wilder Affenmensch da.

\notiz{
\begin{itemize}
  \item Ned Flanders verdient nur 27 Dollar mehr pro Woche als Homer.
  \item Homer war in seiner Jugend Pfadfinder.
  \item Zunächst sieht sich Homer das Luxuswohnmobil \glqq Ultimate Behemoth\index{Behemoth}\grqq\ an, das er sich allerdings nicht leisten kann.
\end{itemize}
}

	
\subsection{Homer als Frauenheld}\label{7G10}
Bart hat sich eine \glqq Spionage-Kamera\grqq\ gekauft. Alles wird fotografiert. Eines Tages geht Marge mit den Kindern zum Essen. Im Nebenraum dieses Restaurants sitzt Vater Homer bei einem Herrenabend, der immer ausgelassener zu werden scheint. Schließlich tritt sogar die Bauchtänzerin Prinzessin Kashmir\index{Prinzessin Kashmir} auf. Homer wird von dieser genötigt, mit ihr auf dem Tisch zu tanzen. Barts Foto hat ernste eheliche Konsequenzen.

\notiz{
\begin{itemize}
  \item Homer wiegt 239 Pfund.
  \item Der Herrenabend wird anlässlich der Hochzeit von Eugene Fisk\index{Fisk!Eugene}, einem früheren Assistenten und jetzigen Vorgesetzten von Homer veranstaltet.
  \item Prinzessin Kashmir heißt eigentlich Shawna Tifton\index{Tifton!Shawna}.
  \item Auf den Preisschildern des Fotokopierers steht \glqq 5 Cents\grqq\ auf der Frontseite und am Münzeinwurf steht \glqq 10 Cents\grqq .
  \item Bart arrangiert die Aufschrift auf einem Schild von \glqq Cod Platter \$4.95\grqq\ (Kabeljau-Platte) um zu \glqq Cold Pet Rat \$4.95\grqq\ (Kalte Hausratte).
  \item Fehler: Auf der Kopie ist Homer nur mit der Tänzerin abgebildet, hinter dem Tisch standen aber noch andere Männer, die auf dem Foto nicht zu sehen sind.
\end{itemize}
}

	
\subsection{Der schöne Jacques}\label{7G11}
Marge hat ihren 34. Geburtstag. Die Kinder bringen ihr das Frühstück ans Bett. Homer stellt entsetzt fest, dass er vergessen hat, ein Geschenk zu kaufen. Sein Geschenk kommt nicht gut an: eine Bowlingkugel, in der sein Name eingraviert ist. Marge beschließt, ihm eins auszuwischen. Sie lernt bowlen -- bei dem charmanten Lehrer Jacques Brunswick\index{Brunswick!Jacques}. Nur langsam begreift die ganze Familie, dass sich hier eine ernsthafte Ehekrise anbahnt.

\notiz{
\begin{itemize}
  \item Marges Schuhgröße ist 46 (in der amerikanischen Originalfolge: 13AA).
	\item Filmzitat: Zu den Klänge von \glqq Up Where We Belong\grqq\ aus \glqq Ein Offizier und Gentleman\grqq\ schreitet Marge entschlossen durch das Atomkraftwerk und umarmt Homer. Er nimmt sie in die Arme und trägt sie hinaus, wobei er laut verkündet: \glqq Ich machs's mir mit der Frau, die ich über alles in der Welt liebe, auf meinem Rücksitz gemütlich und will die nächste Viertelstunde nicht gestört werden.\grqq .
	\item Erster Auftritt von Helen Lovejoy.
	\item Fehler I: Zu Beginn der Episode. als Marge mit ihrer Schwester telefoniert, wechselt ihre Halskette die Farbe, von rot auf weiß.
	\item Fehler II: Jacques hilft Marge beim Bowling. Seine Kugel wechselt die Farbe von blau auf grün.
	\item Fehler III: Marge kommt nach Hause. Homer schläft bereits. Es ist 22:00 Uhr. In der nächsten Szene ist es 23:00 Uhr.
	\item Diese Episode gewann 1990 einen Emmy.
\end{itemize}
}
	
\subsection{Der Clown mit der Biedermaske}\label{7G12}
Krusty, der Clown, ist das Idol vieler Kinder. Auch Bart gehört zur eingeschworenen Fangemeinde des Moderators einer Kindersendung. Eines Tages bricht für die Fans eine Welt zusammen: Krusty wird beschuldigt, den Kwik-E-Mart überfallen zu haben. Eine Videoaufzeichnung der Überwachungskamera und die Zeugenaussage Homers sind erdrückende Beweismittel. Doch Bart will Krustys Unschuld beweisen.

\notiz{
\begin{itemize}
	\item Krustys Karriere hat in Tupelo, Mississippi begonnen, genau wie die Karriere von Elvis.
	\item Krusty begann seine Karriere als Straßenpantomime.
	\item Krusty gesteht, nicht lesen und schreiben zu können.
	\item Krustys Gefangenennummer ist A113\footnote{A113 ist ein Insider-Witz, ein sogenanntes \glqq Easter Egg\grqq, das sich in zahlreichen Film- und Fernsehproduktionen wiederfindet. Der Raum A113 befindet sich am California Institute of the Arts, dort finden Vorlesungen zur Animation statt. Einige Absolventen bauen diese Raumnummer an verschiedenen Stellen in ihren Produktionen ein.}.
	\item In dieser Episode sind der Polizist Lou und Richter Synder weiß und nicht schwarz wie in den übrigen Folgen.
	\item Erster Auftritt von Sideshow Bob bzw. Tingel-Tangel Bob.
	\item Krusty erlitt 1986 eine Herzattacke während einer Sendung und hat seitdem einen Herzschrittmacher.
	\item Laut Matt Groening wurde Krusty dem christlichen Radio- und Fernsehclown Rusty Nails nachempfunden.
	\item Fehler: Während der Nachrichten, verändert sich das Scheibenmuster hinter Reverend Lovejoy.
\end{itemize}
}
	
\subsection{Tauschgeschäfte und Spione}\label{7G13}
Barts Lehrer schlägt vor, ihn im Rahmen eines Schüleraustauschprogramms nach Frankreich zu schicken. Homer und Marge sollen als Gegenleistung den kleinen Albaner Adil Hoxha\index{Hoxha!Adil} aufnehmen. Während Bart in Frankreich schwere Zeiten durchmacht, weil er auf dem Weingut der Gastfamilie arbeiten und im Stall bei den Tieren schlafen muss, entpuppt sich Adil als Spion, der Geheimnisse aus Homers Kernkraftwerk verrät. Adil wird auf der Stelle gegen einen amerikanischen Kinderspion ausgetauscht. Bart erweist sich in Frankreich als Held, denn er kann seine Gastfamilie der Weinpanscherei überführen.

\notiz{
\begin{itemize}
  \item Das Weingut in Frankreich, wo Bart seinen Austausch verlebt, heißt \glqq Chateau Maison\index{Chateau Maison}\grqq .
  \item Der Deckname von Adil lautet \glqq Spatz\index{Spatz}\grqq .
\end{itemize}
}

\section{Staffel 2}

\subsection{Frische Fische mit drei Augen}\label{7F01}
Lisa und Bart angeln an einem See in der Nähe des Kernkraftwerks, in dem ihr Vater arbeitet. Ein Journalist kommt vorbei. Er ist auf der Suche nach Sensationen. Ausgerechnet jetzt ziehen Lisa und Bart einen dreiäugigen Fisch aus dem Wasser. Fabrikbesitzer Burns versucht alles, um die Verschmutzung unter den Teppich zu kehren, doch Marge Simpson ist er nicht gewachsen.

\notiz{
\begin{itemize}
  \item Monty Burns und Mary Bailey\index{Bailey!Mary} kämpfen darum, Gouverneur eines Staates zu werden, dessen Name nie erwähnt wird.
  \item Während seiner Wahlkampagne posiert Burns auf einem Panzer, was an die Kampagne von Michael Dukakis (US-Präsidentschafts-Wahlkampf) erinnert.
  \item Mary Bailey heißt wie eine Figur in dem Film \glqq Ist das Leben nicht schön\grqq .
  \item Die Atomaufsichtsbehörde stellt 342 Verstöße im Springfielder Atomkraftwerk fest.
\end{itemize}
}

\subsection{Karriere mit Köpfchen}\label{7F02}
Der kahlköpfige Homer erfährt von einer neuartigen, vielversprechenden Haar\-wuchs\-kur \glqq Dimoxinil\index{Dimoxinil}\grqq . Leider ist die Kur mit 1000 Dollar sehr teuer. Doch Homer weiß sich zu helfen: Er betrügt die firmeneigene Krankenversicherung, um an das begehrte Elixier zu kommen. Über Nacht wächst sein Haar und plötzlich stellt er fest, dass er im Geschäftsleben auffallend mehr Chancen hat als zuvor. Er wird nämlich zu einer Nachwuchsführungskraft befördert. Leider ist der Erfolg nicht von Dauer.

\notiz{
\begin{itemize}
  \item Filmzitat: Homer rennt durch die Straßen und feiert sein neu gewachsenes Haar, eine Anspielung auf den Film \glqq Ist das Leben nicht schön\grqq .
  \item Der Name des Haarwuchsmittels Dimoxinil ist nicht zufällig dem eines existierenden Produkts, Minoxidil$^{TM}$, sehr ähnlich.
  \item Mr. Burns gibt an, 81 Jahre alt zu sein.
  \item In dieser Folge ist der erste Kuss zwischen zwei Männern in einer Zeichentrickserie zu sehen (Homer und Karl).
  \item Laut Rezept wurde Homer am 10. Mai 1955 geboren.
\end{itemize}
}
	
\subsection{Der Musterschüler}
Barts Versetzung ist gefährdet. Er versucht es mit Schummeln -- vergeblich, da bleibt ihm nichts anderes übrig, als zu büffeln. Er tut das auch, aber er erreicht trotzdem nicht die erforderliche Mindestanzahl von Punkten. Als er schließlich durch eindringliches Betteln bei der Lehrerin den rettenden Zusatzpunkt herausschindet, besteht er zwar -- doch hinterher ekelt es Bart vor sich selbst.

\notiz{Bart trägt in dieser Episode ein rotes T-Shirt anstatt des üblichen orangefarbenen.}
	
\subsection{Horror frei Haus}\label{7F04}
Wieder einmal ist Halloween: Heute erzählen die Kinder der Simpson-Familie einander grausige Schauergeschichten. 
\begin{itemize}
	\item \textbf{Das Haus der bösen Träume}\\ Die Simpsons ziehen in ein Spukhaus, komplett mit einem Dimensionsstrudel, blutenden Wänden und einer indianischen Begräbnisstätte.
	\item \textbf{Hungrig sind die Verdammten}\\ Außerirdische entführen die Simpsons und laden sie zu einem \glqq Festessen\grqq\ auf Rigel 4 ein.
	\item \textbf{Der Rabe}\\ Edgar Allen Poe's klassische Horrorgeschichte mit Simpsons-Besetzung.
\end{itemize}

\notiz{
\begin{itemize}
  \item Der Spielzeughase neben der Kiste, die Bart auspackt, sieht aus wie der Hase Binky\index{Binky} aus \glqq Life in Hell\grqq , einem Comic Strip von Matt Groening.
  \item Im Keller des Hauses sind u.\,a. die Grabsteine von \glqq Crazy Horse\grqq\ und \glqq Not So Crazy Horse\grqq\ zu sehen.	
  \item Filmzitat I: Maggie dreht ihren Kopf einmal ganz herum wie in \glqq Der Exorzist\grqq .
  \item Filmzitat II: Das Verhalten des verfluchten Hauses und die indianischen Gräber im Keller sind eine Parodie auf \glqq Poltergeist\grqq .
  \item Filmzitat III: Blut läuft an den Wänden herunter und der Kamin bewegt sich zur Mitte der Wand wie in \glqq The Amityville Horror\grqq .
  \item Filmzitat IV: Das Aussehen des verfluchten Hauses erinnert an das Haus in \glqq Psycho\grqq .
  \item Filmzitat V: Das verfluchte Haus implodiert wie in \glqq Der Untergang des Hauses Usher\grqq .
  \item Filmzitat VI: Das \glqq Au\grqq , das man hört, als die Fliege in die Fliegenfalle gerät, ist eine Parodie auf den Film \glqq Die Fliege\grqq .
  \item Fernsehzitat: Das Kochbuch \glqq Wie man für vierzig Menschen kocht\grqq\ parodiert die \glqq Twilight Zone\grqq -Episode \glqq To Serve Man\grqq\ (was sowohl \glqq Menschen servieren\grqq\ als auch \glqq für Menschen servieren\grqq\ bedeutet).
\end{itemize}
}


\subsection{Das Maskottchen}\label{7F05}
Der alljährliche Betriebsausflug des Atomkraftwerks führt zum Baseballspiel der Springfield Isotopes gegen die Shelbyvillians\index{Shelbyvillians}. Homer ist nicht gerade darüber erfreut, dass sich sein Chef, Mr. Burns, ausgerechnet neben ihn im Stadion setzt. Nachdem Mr. Burns Homer ein paar Biere ausgegeben hat, verstehen sie sich allerdings besser, aber die Heimmannschaft droht zu verlieren. Da beginnt der etwas angesäuselte Homer, die Zuschauer zum Anfeuern zu animieren. Homer wird als Maskottchen der \glqq Isotopes\grqq\ engagiert. Der Animateur avanciert zum lokalen Star. Daraufhin bekommt Homer ein Angebot aus der Oberliga, das er auf Zuraten von Marge annimmt. Doch die Großstädter können mit Homers Tanzeinlagen nichts anfangen und so kehren die Simpsons wieder nach Springfield zurück.

\notiz{
\begin{itemize}
	\item Filmzitat: Homers Abschiedsrede in der \glqq Dancin' Homer Appreciation Night\grqq , bei der er seine Mütze an sein Herz drückt, ist eine Anspielung auf Lou Gehrig in dem Film \glqq Pride of Yankees\grqq : \glqq \dots Heute, wo ich nach Capital City abreise, halte ich mich selbst für das glücklichste Maskottchen der Welt!\grqq .
	\item Tony Bennet\index{Bennet!Tony} singt den \glqq Capital City\grqq -Song, eine Variation von \glqq New York, New York\grqq . Tony Bennet ist der erste Gaststar bei den Simpsons, der sich selbst spielt.
	\item Antoine \glqq Tex\grqq\ O'Hara\index{O'Hara!Antoine} ist der Besitzer der Springfield Isotopes. Dave Rosenfield\index{Rosenfield!Dave} ist der Eigentümer der Baseballmannschaft aus Capital City.
\end{itemize}
}

	
\subsection{Der Teufelssprung}\label{7F06}
Homer und Bart gehen auf ein Konzert der Schulband. Bei diesem Konzert spielt Lisa ihr erstes Solo. Die beiden können das letzte Musikstück kaum abwarten: Sie wollen nämlich im Anschluss gleich eine \glqq Monster-Truck-Show\grqq\ besuchen. Sie sind fasziniert von den spektakulären Stunts der Artisten auf ihren Feuerstühlen. Bart ist so beeindruckt, dass er beschließt, mit seinem Skateboard über den nahe gelegenen Canyon zu springen.

\notiz{
\begin{itemize}
	\item Filmzitat: Barts Auftauchen im hitzigen Dunst, kurz vor dem Sprung über die Schlucht, erinnert an \glqq Lawrence von Arabien\grqq .
   \item Erstauftritt von Dr. Julius Hibbert.
   \item Todd Flanders spielt im Schulorchester Geige.
   \item Der Name des Stuntman lautet Captain Lance Murdock\index{Murdock!Lance}. Diese Figur ist eine Anspielung auf Evel Knievel\footnote{Robert Craig \glqq Evel\grqq\ Knievel, jun. war ein amerikanischer Motorradstuntman (geboren am 17. Oktober 1938 in Butte, Montana, USA; gestorben am 30. November 2007 in Clearwater, Florida, USA). Durch seine spektakulären Motorradsprünge und Stuntshows erlangte er weltweit Berühmtheit.}.
\end{itemize}
}

\subsection{Bart bleibt hart}\label{7F07}
Das alljährliche Truthahnessen ist für die Familie Simpson immer wieder ein Horrortrip aufgrund der lieben Verwandten, die sich zu diesem Anlass immer wieder einstellen. In der Hektik der Vorbereitung kommt es zwischen Lisa und Bart zu einem heftigen Streit. Als Bart bestraft wird, reißt er aus. Er streunt durch die Stadt und landet in einem Obdachlosenasyl. Für Bart ist dies eine ganz neue Erfahrung.

\notiz{
\begin{itemize}
	\item Literaturzitat: Lisas Gedicht erinnert stark an \glqq Howl\grqq\ des Poeten Allen Ginsberg. Sie hat außerdem ein Buch von ihm neben Jack Kerouacs \glqq On the Road\grqq\ und eine Sammlung von Gedichten von Edgar Allan Poe in ihrem Regal.
	\item In dieser Folge wird das Altenheim, in dem Abe Simpson wohnt, als \glqq Springfield Retirement Home\grqq\ bezeichnet. In spä\-te\-ren Fol\-gen heißt es dann \glqq Springfield Retirement Castle\grqq .
	\item Bei der Thanksgiving-Parade ist eine Bart-Simpson-Puppe zu sehen.
	\item Die Dallas Cowboys sind Homers Lieblings-Football-Mannschaft.
\end{itemize}
}


\subsection{Der Wettkampf}\label{7F08}
Homer Simpson ist genervt. Grund dafür ist der Nachbar Ned Flanders, der Saubermann mit der perfekten Familie. Flanders Frau umsorgt ihren Gatten, die Söhne Rod und Todd sind ein Vorbild für alle Kinder und Flanders Rasen verunstaltet kein Unkräutlein. Homer wünscht sich nur eines: diesem Typen endlich einmal eins auswischen zu können. Bei einem Minigolfturnier ergibt sich eine gute Gelegenheit. Weil Flanders Sohn daran teilnimmt, versucht Homer, Bart für seinen Feldzug einzuspannen.

\notiz{
\begin{itemize}
  \item Ned Flanders hat vier Schnellwähltasten seines Telefons folgendermaßen belegt: Reverend (Work), Reverend (Home), Recycling Center und Book Mobile.
  \item In dieser Episode wird der Familienname des Reverends mit \glqq Gottlieb\grqq\ übersetzt. In spä\-te\-ren Folgen heißt er immer Lovejoy.
  \item Da weder Bart noch Todd das Minigolfturnier gewinnen (sie einigen sich auf ein Unentschieden) müssen Ned und Homer in den Sonntagskleidern ihrer Frauen den Rasen des jeweils anderen mähen.
  \item Flanders unterschreibt die Wette mit der rechten Hand. In späteren Folgen ist er allerdings Linkshänder (vergleiche \glqq Ein Fluch auf Flanders\grqq , siehe \ref{7F23}).
  \item Todd Flanders ist zehn Jahre alt.
  \item Homer stellt beim Betreten von Neds Keller fest, dass er und Ned Flanders bereits seit acht Jahren Nachbarn sind und er noch nie in seinem Haus war. 
  \item Homer fordert Bart auf, seinen Putter \glqq Marlene\index{Marlene}\grqq\ zu nennen.
\end{itemize}
}

	
\subsection{Das Fernsehen ist an allem schuld}\label{7F09}
Maggie verfällt auf die Idee, Vater Homer einen Hammer auf den Kopf zu schlagen. Schuld daran ist ein Zeichentrickfilm, den das Baby im Fernsehen gesehen hat. Für Marge ist der Fall klar: Sie schreibt einen geharnischten Brief an die Fernsehanstalt, in dem sie gegen jegliche Gewalt in Zeichentrickfilmen aufs Schärfste protestiert. Die Antwort bringt Marge erst recht auf die Palme.

\notiz{
\begin{itemize}
	\item Filmzitat: Die Szene, in der Maggie Homer mit dem Hammer angreift, erinnert sehr stark an die Duschszene aus dem Film \glqq Psycho\grqq .
	\item David Silverman diente als Vorlage für einen Itchy \& Scratchy Animateur. Der Regisseur der Folge, Rich Moore, diente als Vorlage für den Animateur, der Marge als Eichhörnchen zeichnet.
	\item Kent Brockman hat einen Emmy gewonnen.
	\item Eines von Homers Handwerkerbüchern hat den Titel \glqq Trojan Horse\grqq .
\end{itemize}
}

	
\subsection{Bart kommt unter die Räder}\label{7F10}
Homers Chef Mr. Burns fährt Bart mit dem Auto an. Als er wieder zu Bewusstsein kommt, liegt er im Krankenhaus -- umgeben von seiner Familie und dem ausgebufften Rechtsanwalt Lionel Hutz. Der fordert von Mr. Burns eine Million Dollar Schadensersatz. Nun muss Bart nur noch den Schwerverletzten mimen. Im Zeugenstand kann Mr. Burns Anwalt Marge entlocken, dass Bart gar nicht so schwer verletzt ist, wie es äußerlich erscheint. Somit muss Mr. Burns statt der geforderten einen Million an Schadenersatz nichts bezahlen.

\notiz{
\begin{itemize}
	\item Filmzitat: Als Bart im Krankenhaus sein Bewusstsein wieder erlangt, zeigt er auf jeden, der während seiner Vision vom Jenseits auftauchte -- genau wie Dorothy am Ende von \glqq The Wizard of Oz\grqq\ (Das zauberhafte Land).
	\item Burns Luxusauto ist ein 1948er Rolls Royce.
	\item Der Teufel und Lionel Hutz werden eingeführt.
	\item Dr. Hibbert wurde an der John Hopkins Medical School ausgebildet.
	\item In dieser Folge heißt der Richter Moulton\index{Moulton}; später wird er Richter Synder genannt.
\end{itemize}
}

	
\subsection{Die 24-Stunden-Frist}\label{7F11}
Auf Marges Vorschlag hin gehen die Simpsons essen. Weil man mal etwas Neues ausprobieren möchte, fällt die Wahl auf das japanisches Restaurant \glqq The Happy Sumo\grqq . Homer schmeckt es so gut, dass er die ganze Karte kosten will: Er bestellt auch einen Fisch, der bei falscher Zubereitung absolut tödlich ist. Weil der Chefkoch Sensei\index{Sensei} nicht gestört werden will, sieht sich der Lehrling gezwungen, das Gericht selbst zu zubereiten.

\notiz{
\begin{itemize}
	\item Homers Liste von Sachen, die er vor dem Tod noch erledigen will, geschrieben auf \glqq Dumb Things I Gotta Do Today\grqq-Notizpapier (Dämliche Sachen, die ich heute erledigen muss): 
\begin{enumerate}
	\item Liste erstellen (schon durchgestrichen);
	\item Ein reichhaltiges Frühstück; 
	\item Videoaufnahme für Maggie; 
	\item Mann-zu-Mann-Gespräch mit Bart; 
	\item Lisa beim Saxophon spielen zuhören; 
	\item Das Begräbnis organisieren; 
	\item Mit Dad Frieden schließen; 
	\item Bier trinken mit den Jungs in der Bar; 
	\item Dem Boss die Meinung sagen; 
	\item Drachen fliegen; 
	\item Einen Baum pflanzen; 
	\item Ein letztes Abendessen mit meiner geliebten Familie; 
	\item Mit Marge intim sein (\glqq be intimate\grqq ).
\end{enumerate}
\item Chefkoch Sensei und Mrs. Krabappel vergnügen sich auf dem Rücksitz eines Autos.
\item Filmzitat: Homer läuft gegen Ende Heim zu Marge wie Dustin Hoffman in \glqq Die Reifeprüfung\grqq\ zur Kirche.
\end{itemize}
}

\subsection{Wie alles begann}\label{7F12}
Die Simpsons sitzen gemütlich beim Fernsehen zusammen, als die Glotze plötz\-lich den Geist aufgibt. Homer versucht verzweifelt, das Ding zu reparieren -- vergeblich. Für alle Familienmitglieder eine Katastrophe. Nur Marge findet das gar nicht so schlimm. Sie erinnert sich an früher, als sie nur selten fernsahen. Gegen den Protest von Bart erzählt sie die Geschichte, wie sie Homer kennen lernte.

\notiz{
\begin{itemize}
	\item Erster Auftritt von McBain alias Rainier Wolfcastle\index{Wolfcastle!Rainier}.
	\item Erster Auftritt von Artie Ziff\index{Ziff!Artie}.
\end{itemize}
}

	
\subsection{Das achte Gebot}\label{7F13}
Als Homer von Ned Flanders erfährt, dass ein Angestellter der Kabelfirma illegal für 50 Dollar einen Kabelanschluss legt, lässt er sich an das Kabelnetz anschließen. Als Erste hat Marge ein schlechtes Gefühl dabei. Inzwischen lernt Bart in der Sonntagsschule bei Mrs. Albright\index{Albright} die zehn Gebote. Er lernt, dass bei einem Verstoß dagegen die Hölle droht. Bart freut sich schon darauf. Aber Lisa ist gar nicht wohl bei der Sache. Sie beschließt, beim Pfarrer zu beichten. Dieser Rät ihr, sich der Technik zu verwehren. Homer kann sich dem zunächst nicht anschließen, da ein wichtiger Boxkampf stattfindet und er viele Freunde und Bekannte zu sich nach Hause eingeladen hat. Aber schließlich schneidet er nach dem Ende des Boxkampfs das Kabel durch.

\notiz{
\begin{itemize}
	\item Die beiden Boxer sind nach den Vorbildern von Mike Tyson (Tatum) und Buster Douglas (Watson) gezeichnet.
	\item Tatum wurde in Springfield geboren und er wurde u.\,a. wegen Raubüberfalls und Totschlags zu einer Haftstrafe verurteilt.
	\item Filmzitat: Nachdem Homer herausgefunden hat, dass der Kabelmann illegale Anschlüsse einrichtet, schmeißt Homer sich vor seinen Truck und spielt den Angefahrenen, um ihn anzuhalten -- eine Anspielung auf Hitchcocks \glqq Der unsichtbare Dritte\grqq .
	\item Diese Folge gewann 1991 den Emmy für die beste animierte Serie.
\end{itemize}
}

	
\subsection{Betragen mangelhaft}\label{7F14}
Der Simpson-Hund, Knecht Ruprecht, der eigentlich Bart gehört, entwickelt sich immer mehr zu einer Plage: Er macht alles kaputt, randaliert im Garten und frisst alles, was nicht niet- und nagelfest ist. Marge und Homer beschließen, den Hund wegzugeben. Schließlich kann Bart sie breitschlagen, das nicht zu tun und Knecht Ruprecht erhält noch eine Chance: Er muss sich in einer Hundeschule von Emily Winthrop\index{Wintrop!Emily} einem Erziehungslehrgang unterwerfen.

\notiz{
\begin{itemize}
	\item Filmzitat: Als Lisa mit der Decke fertig ist, berühren sie und Marge sich mit den Zeigefingern wie in \glqq E.T.\grqq .
	\item Nach Auskunft von Kent Brockman ist die kriminelle Babysitterin Mrs. Botz\index{Botz} (\glqq Der Babysitter ist los\grqq , siehe \ref{7G01}) aus dem Gefängnis geflohen.
	\item Homer kauft sich wie Ned Flanders für 125 Dollar Turnschuhe der Marke Assassins\index{Assassins}.
	\item Laut Apu handelt es sich beim Kwik-E-Markt um einen Familienbetrieb.
	\item Lisa Lieblingsschulfach ist nach eigener Aussage Arithmetik.
	\item Lisa hat die Mumps.
\end{itemize}
}

	
\subsection{Der Heiratskandidat}\label{7F15}
Marges Schwester Selma kommt zu Besuch. Sie erzählt Marge, dass sie einen Mann sucht. Homer, der Marge einen Gefallen schuldet, wird beauftragt, einen geeigneten Ehemann zu finden. Als er in die Schule zum Direktor gerufen wird, ist dies für Homer ausnahmsweise ein Glücksfall: Direktor Skinner wäre der ideale Mann für Selma. Homer lädt Skinner zum Abendessen ein. Doch dort hat der Kandidat nur Augen für Selmas Zwillingsschwester Patty und die hasst Männer.

\notiz{
\begin{itemize}
	\item Die \glqq Happy Hour\grqq\ bei Moe geht von 17:00 bis 17:30 Uhr.
  \item Filmzitat I: Homer futuristische Analyse der potenziellen Männer für Selma parodiert den Film \glqq Terminator\grqq .
	\item Filmzitat II: Rektor Skinner verkündet \glqq Morgen ist ein neuer Schultag\grqq\ -- in Anlehnung an das Schlusszitat aus \glqq Vom Winde verweht\grqq\ (\glqq Morgen ist ein neuer Tag\grqq ).
	\item Filmzitat III: Skinner trägt Patty die Stufen des Glockenturms hoch -- so wie Quasimodo Esmeralda in \glqq Der Glöckner von Notre Dame\grqq .
	\item Filmzitat IV: Als Skinner die Treppen der Springfielder Grundschule hoch steigt, erinnert das an den Schluss von Hitchcocks \glqq Vertigo\grqq .
	\item Erster Auftritt: Hans Moleman\index{Maulwurf!Hans}\index{Moleman!Hans} (Hans Maulwurf), der eigentlich Ralph Melish\index{Melish!Ralph} heißt. 
	\item Patty ist zwei Minuten jünger als Selma und zurzeit sind beide 40 Jahre alt.
	\item Patty und Selma wohnen im Appartement mit der Nummer 1599.
	\item Barney Gumble war in der Army.
\end{itemize}
}

	
\subsection{Ein Bruder für Homer}\label{7F16}
Homers Vater verrät seinem Sohn ein Geheimnis: Homer hat einen unehelichen Halbbruder, Herbert Powell\index{Powell!Herbert}. Homer macht sich auf die Suche nach dem Bruder, den sein Vater im Waisenhaus von Shelbyville abgegeben und von dem er nie wieder gehört hat. Und Homer hat Erfolg: Er findet den Verwandten in Detroit und die Simpsons machen sich auf den Weg. Die Überraschung ist riesengroß, als sich der Halbbruder als Millionär entpuppt.

\notiz{
\begin{itemize}
	\item Der Direktor des Waisenhauses ist der lang vermisste Zwillingsbruder von Dr. Hibbert (Zahnfleischbluter Murphy ist ein weiterer Bruder der beiden).
	\item Herbert Powell ging aus einer Affäre von Abraham Simpson und einer Schaustellerin hervor. Zu dieser Zeit war Abe allerdings schon mit seiner späteren Frau Mona zusammen.
\end{itemize}
}

\subsection{Die Erbschaft}\label{7F17}
Opa Simpson hat im Altersheim noch eine Verehrerin, Beatrice Simmons\index{Simmons!Beatrice}, gefunden. Ausgerechnet an deren Geburtstag entführt Homer seinen Vater auf eine Safari. Als Opa Simpson am nächsten Tag ins Altersheim zurückkommt, ist Beatrice gestorben. Sie hat ihm \$106.000 vermacht. Opa Simpson beschließt, das Geld einem guten Zweck zuzuführen. Doch sein Freund Jasper hat eine bessere Idee: Man könnte damit in die Spielbank gehen, um das Geld noch zu vermehren.

\notiz{
\begin{itemize}
	\item Erstauftritt von Professor Dr. John Frink.
	\item Prof. Frink sagt, er sei verheiratet.
	\item Abe wohnt im Zimmer 18 und Beatrice im Zimmer 27.
	\item Abe kauft sich bei Herman den Hut, welchen angeblich Napoleon getragen hat, für 400 Dollar.
	\item Abe sagt, er habe nur noch eine funktionierende Niere.
	\item Kent Brockman gibt an, dass er mit der Wetterfee Stefanie verheiratet war oder ist.
	\item Abe erwähnt Bea gegenüber, dass er Witwer ist und einen Sohn hat. In Wirklichkeit ist seine Frau Mona nicht gestorben, sondern hat ihn verlassen (\glqq Wer ist Mona Simpson?\grqq , \ref{3F06}) und er hat außerdem zwei Söhne, wie in der Episode \glqq Ein Bruder für Homer\grqq\ (siehe \ref{7F16}) zu sehen ist und eine Tochter in England (\glqq Die Queen ist nicht erfreut!\grqq , \ref{EABF22}).
	\item Die Szene, in der Abe im Cafe sitzt, ist eine Anspielung auf das Gemälde \glqq Nighthawks\grqq\ von Edward Hopper\footnote{Edward Hopper (geboren am 22. Juli 1882 in Nyack, New York; gestorben am 15. Mai 1967 in New York City, New York) war ein US-amerikanischer Maler.}.
\end{itemize}
}
	
	
\subsection{Marges Meisterwerk}\label{7F18}
Als Homer im Vergnügungspark Mt. Splashmore in der Wasserrutsche steckenbleibt, beginnt er, ernsthaft über eine Abmagerungskur nachzudenken. Währenddessen entdeckt Marge ihr Zeichentalent wieder. Ihr Porträt des rundlichen Homer gewinnt den ersten Preis der Kunstausstellung von Springfield. Doch dann erhält sie einen Auftrag für ein Porträt von Mr. Burns und der Mann ist wirklich abstoßend hässlich.

\notiz{
\begin{itemize}
	\item Filmzitat I: Homers Versprechen \glqq Der ewige Gott ist mein Zeuge: Ich werde mich niemals wieder satt essen!\grqq\ ist im amerikanischen Original eine Parodie auf Scarlett O'Hara in \glqq Vom Winde verweht\grqq .
	\item Filmzitat II: Homers Fitnessübungen werden von einer Musik begleitet, die an das Titelthema von \glqq Rocky\grqq\ erinnert.
	\item Der Mallehrer von Marge in der Volkshochschule ist Professor Lombardo\index{Lombardo}.
	\item Fehler: Als Marge in der Highschool das Portrait von Ringo Star malt, ist der Hintergrund rot. Als das Bild allerdings Ringo aufhängt, ist der Hintergrund plötzlich blau.
\end{itemize}
}

	
\subsection{Der Aushilfslehrer}\label{7F19}
Lisas Klassenlehrerin Miss Hoover wird krank. Ihr Vertreter Mr. Bergstrom\index{Bergstrom} gewinnt mit seinen Lehrmethoden im Handumdrehen Lisas Zuneigung. Gerade als Lisa ihren Lieblingslehrer zum Abendessen nach Hause einladen will, kommt Miss Hoover wieder zurück. Bart hat ein Problem: Er tritt bei der Klassensprecherwahl gegen den Streber Martin an. Fast alle sind für Bart, vergessen aber, ihm ihre Stimme zu geben.

\notiz{
\begin{itemize}
  \item Bart erhält bei der Wahl keine Stimme. Martin erhält dagegen zwei Stimmen (seine eigene und die Stimme von Wendell).
	\item Historisches Zitat: Als Martin die Wahl gewinnt, posiert er mit einer vorab gedruckten Ausgabe des \glqq Fourth Gradian\grqq\ (Der Viertklässler), dessen Schlagzeile \glqq Simpsons Defeats Prince\grqq\ (Simpson schlägt Prince) das in den USA bekannte Foto \glqq Dewey schlägt Truman\grqq\ parodiert.
	\item Mrs. Krabappel gibt an, noch verheiratet zu sein.
	\item Den Zettel, auf dem steht: \glqq Du bist Lisa Simpson (You are Lisa Simpson)\grqq\ sieht man nochmal in der Folge \glqq Klug \& Klüger\grqq\ (siehe \ref{FABF09}).
\end{itemize}
}

	
\subsection{Kampf dem Ehekrieg}\label{7F20}
Nachdem Homer sich völlig betrunken auf der eigenen Party daneben benommen hat, ist Gattin Marge mächtig sauer. Er grölt herum und guckt Maude Flanders sogar in den Ausschnitt. Sie meldet sich und Homer auf der Stelle zu einem Eheberatungswochenende bei Reverend Lovejoy an. Doch was versteht ein Reverend schon von der Ehe! Die Beratung wird zu einer Generalanklage gegen Homer. Er stiehlt sich davon und geht angeln.

\notiz{
\begin{itemize}
	\item Reverend Lovejoy ist seit zehn Jahren als Eheberater tätig.
	\item Filmzitat: Flanders mixt sich flott einen Drink und ahmt dabei Bewegungen aus \glqq Cocktail\grqq\ nach.
	\item Literaturzitat I: Als Marge Homer fragt, ob er sich daran erinnert, wie er sich auf der Party benommen hat, erinnert er sich an eine sprachgewandte Diskussion inmitten einer intellektuellen Festgesellschaft, ähnlich dem sogenannten \glqq Algonquin Round Table\grqq , einem wöchentlichen Treffen berühmter Literaten und Intellektueller im New Yorker Hotel Algonquin, das vor allem in den 20er-Jahren für seine intelligenten Gespräche berühmt war.
	\item Literaturzitat II: John\index{John} und Gloria\index{Gloria} sind das vierte Paar beim Seminar. Die beiden quälen sich gegenseitig wie in dem berühmten Roman \glqq Wer hat Angst vor Virginia Woolf?\grqq .
	\item Literaturzitat III: Homers Kampf mit General Sherman\index{General Sherman} erinnert an \glqq Moby Dick\grqq\ und \glqq Der alte Mann und das Meer\grqq .
	\item Homer sagt, bei der Hochzeit von ihm und Marge sei viel Champagner getrunken worden. In der Episode \glqq Scheide sich, wer kann\grqq\ (siehe \ref{4F04}) ist in der Rückblende zu sehen, dass beide ohne Verwandte und Bekannte ihre Hochzeit gefeiert haben.
	\item Erstauftritt von Snake.
\end{itemize}
}

	
\subsection{Drei Freunde und ein Comic-Heft}\label{7F21}
Auf einer Comic-Ausstellung entdeckt Bart ein Heft, das er unbedingt haben möchte: Die Erstausgabe von \glqq Radioactive Man\grqq . Bart würde sein gesamtes Barvermögen von dreißig Dollar investieren, doch der Händler verlangt 100 Dollar für das Heft. Der Rat von Marge: Bart soll einen Job annehmen. Weil man mit Arbeit nicht reich wird, legen er, Milhouse und Martin zusammen. So fangen die Probleme an, denn: Wem gehört nun der \glqq Radioactive Man\grqq ?

\notiz{
\begin{itemize}
  \item In dieser Episode erscheinen erstmals die Geschäfte \glqq Krusty Burger\grqq\ und der \glqq Android's Dungeon Comic Book Store\grqq .
	\item Fernsehzitat: In der Szene, in der Bart vorgeschlagen wird, dass er sich einen Job suchen soll, hört man Bart als Erwachsenen über seine Jugend sprechen, wie in der Serie \glqq Wunderbare Jahre\grqq .
	\item Filmzitat I: Das Misstrauen, dass die Jungs und Bart untereinander entwickeln, ist -- einschließlich Barts Paranoia -- eine Hommage an den Klassiker \glqq Der Schatz der Sierra Madre\grqq .
	\item Filmzitat II: Die Szene, in der Mrs. Glicks Schatten sich langsam über Bart senkt und er fleht \glqq Nein! Nicht wieder Jod! Brennen Sie Bakterien mit 'ner Kerze raus, amputieren Sie mir den Arm, aber nicht schon wieder -- Aaaaah!\grqq\ erinnert an eine Szene aus \glqq Vom Winde verweht\grqq , in welcher der Arm eines Soldaten amputiert wird.
	\item Der Comicbuchverkäufer gibt an, Ägyptologie und Byzantismus studiert zu haben.
	\item Asa Glick\index{Glick!Asa}, der Bruder von Mrs. Glick, ist im Zweiten Weltkrieg gestorben, weil er eine Handgranate nicht loslassen wollte.
  \item Der Schauspieler, der Radioactive Man gespielt hat, hieß Dirk Richter\index{Richter!Dirk} und Fallout Boy wurde von Buddy Hodges\index{Hodges!Buddy} verkörpert.
  \item Das erste Radioactive-Man-Comic erschien im November 1952.
\end{itemize}
}

\subsection{Der Lebensretter}\label{7F22}
Homers Chef Mr. Burns geht es ziemlich dreckig: Er leidet an Blutarmut und braucht dringend eine Bluttransfusion. Leider hat er die äußerst seltene Blutgruppe Null-Null-Negativ. Und dafür ist weit und breit kein Blutspender aufzutreiben -- bis auf Bart. So wird Bart zum Lebensretter des großen Mr. Burns. Homer sieht das Dankeschön in Form von Geldbündeln auf die Familie zukommen, stattdessen schenkt Mr. Burns Bart den Kopf des olmekischen Gottes Xt'Tapalataketel\index{Xt'Tapalataketel}.

\notiz{
\begin{itemize}
	\item Otto summt \glqq Iron Man\grqq\ von Black Sabbath, während er sich für die Blutabnahme wäscht. Nelson summt in der Folge \glqq Die Perlen-Präsidentin\grqq\ (siehe \ref{EABF20})  ebenfalls \glqq Iron Man\grqq .
	\item Der Ghostwriter, den Burns anheuert (und feuert) schreibt \glqq Zum Teufel, ich kann's nicht\grqq , eine Anspielung auf Sammy Davis, Jr.'s \glqq Yes I Can\grqq .
	\item Mr. Burns gibt an, dass ihm Waylon Smithers eine Niere gespendet hat.
	\item Blutgruppen:
	\begin{itemize}
	  \item Homer: A+
	  \item Bart: 00-
	  \item Burns: 00-
	  \item Smithers: B+ 
  \end{itemize}
  \item Mr. Burns kauft den Kopf Xt'Tapalataketel bei \glqq Plunderer Pete's\grqq\ für 32.000 US-Dollar.
  \item Der Titel des Buches von Mr. Burns über seine Genesung trägt den Titel \glqq Will There Be A Rainbow?\grqq\ (Wird es einen Regenbogen geben?).
  \item Carl Carlson gibt an, Homers Vorgesetzter zu sein.
  \item Mr. Burns wohnt in 1000 Mammon Street.
\end{itemize}
}


\section{Staffel 3}

\subsection{Ein Fluch auf Flanders}\label{7F23}
Nachbar Flanders lädt die Familie Simpson zu einer Grillparty ein, um eine Neuigkeit zu verkünden: Er wird seinen bisherigen Job aufgeben, weil er einen eigenen Laden eröffnen will, in dem es nur Artikel für Linkshänder gibt. Diese Idee erfüllt Homer Simpson mit grimmigem Neid und er spricht insgeheim einen Fluch über Flanders Plan aus. Der Fluch geht derart gnadenlos in Erfüllung, dass es schließlich selbst Homer zu viel wird.

\notiz{
\begin{itemize}
  \item Ned Flanders arbeitete vorher in der pharmazeutischen Industrie.
  \item Ned Flanders hat eine Schwester, die in Capitol City lebt.
  \item Filmzitat: Mehrere Elemente der letzten Szene, unter anderem Neds und Maudes Kleidung sowie Homers Trinkspruch an die Sänger sind eine Hommage an das Ende von \glqq Ist das Leben nicht schön!\grqq .
\end{itemize}
}
	
\subsection{Die Geburtstagsüberraschung}\label{7F24}
Barts Mütze in der Waschmaschine verfärbt die Arbeitshemden von Homer rosa, dadurch fällt Homer am Arbeitsplatz auf. Mr. Burns wittert einen Abweichler, der psychiatrisch untersucht werden muss. Homer lässt den psychologischen Test von Bart ausfüllen -- mit durchschlagendem Erfolg: Am nächsten Tag wird Homer in die Irrenanstalt eingeliefert. Dort lernt er einen Weißen kennen, der sich für Michael Jackson hält.

\notiz{
\begin{itemize}
	\item Michaels Story: Vertragliche Verpflichtungen verhinderten, dass der echte Michael Jackson seine Rolle als \glqq Michael Jackson\grqq\ sprechen durfte. So entstand das mysteriöse Pseudonym \glqq John Jay Smith\index{Smith!John Jay}\grqq .
	\item Matt Groening gab an, dass ihn Michael Jackson angerufen habe und bat, in der Serie einen Gastauftritt zu haben. Aus Vertragsgründen durfte Michael Jackson nicht das Lied \glqq Happy Birthday Lisa\grqq\ singen. Stattdessen wurde es von einem Imitator eingesungen, während Michael Jackson im Studio anwesend war (siehe \cite{Tel18}).
	\item 1952 besuchte der Dalai Lama Springfield. Daraufhin wurde die Bundesstraße 401 in die \glqq Dalai Lama Schnellstraße\grqq\ umbenannt.
	\item Maggie ist mit dem Hasen Binky\index{Binky} aus \glqq Life in Hell\grqq\ zu sehen.
	\item Unter den Wartenden vor dem Haus der Simpsons hält ein Mann ein Schild mit der Aufschrift \glqq John 3:16\grqq\ hoch. Diese Aufschrift bezieht sich auf das Johannes-Evangelium Kapitel 3, Vers 16.\footnote{Nach der Einheitsübersetzung lautet dieser: \glqq Denn Gott hat die Welt so sehr geliebt, dass er seinen einzigen Sohn hingab, damit jeder, der an ihn glaubt, nicht verloren geht, sondern ewiges Leben hat \cite{Bibel2016}.\grqq}
\end{itemize}
}

	
\subsection{Einmal Washington und zurück}\label{8F01}
Bei einem Aufsatzwettbewerb soll Amerika als das Land der unbegrenzten Mö\-glich\-kei\-ten gepriesen werden. Die Sieger fahren nach Washington. Natürlich ist Lisa dabei. Zu ihrer großen Enttäuschung muss sie hier den Unterschied zwischen Theorie und Praxis erkennen. Als sie einen Politiker beobachtet, der Bestechungsgelder annimmt, beschließt sie, ihre Rede zu ändern. Lisa deckt den Skandal schonungslos auf, was ihr allerdings keinen Preis einbringt.

\notiz{
\begin{itemize}
	\item Der Originaltitel ist eine Anspielung auf den Film \glqq Mr. Smith goes to Washington\grqq .
	\item Um den Kongressabgeordneten Bob Arnold\index{Arnold!Bob} zu überführen, schlägt ein FBI-Beamter zum Schein vor, im Kopf von Theodore Roosevelt nach Öl zu bohren.
	\item Eine Karte erweckt den Eindruck, Springfield läge in Utah. Wenn aber derselbe Kongressabgeordnete, der die Abholzgenehmigung für den Springfielder Wald gibt, auch für Mount Rushmore zuständig ist, befände sich Springfield in South Dakota.
	\item Die Simpsons sind in Washington im Watergate-Hotel untergebracht.
	\item Die Simpsons begegnen bei Ihrer Tour durch das Weiße Haus Barbara Bush.
\end{itemize}
}

\subsection{Alpträume}\label{8F02}
Die Süßwarenausbeute der Simpson-Kinder an Halloween ist ungeheuer. Noch gigantischer allerdings ist die Fresssucht, mit der Homer, Bart und Lisa alles auf einmal vertilgt haben. In der Nacht werden sie deshalb von schweren Alpträumen geplagt. 
\begin{itemize}
	\item \textbf{The Monkey's Paw}\\ Die Simpsons kaufen auf einer Auslandsreise eine Affenpfote, die ihrem Besitzer vier Wünsche erfüllt -- mit unangenehmen Nebenwirkungen.
   \item \textbf{Bart the Monster}\\ In einer Parodie auf \glqq Twilight Zone\grqq\ muss jeder in Springfield immer glücklich sein oder ein Monster wird ihm etwas antun. Das Monster heißt Bart Simpson, der Junge mit den unheimlichen Kräften.
   \item \textbf{Homer's Brain}\\ Homer wird entlassen und findet einen neuen Job als Totengräber. Er schläft im Grab ein und wird von Mr. Burns für ein Cyborg-Experiment aufgesammelt.
\end{itemize}

\notiz{\begin{itemize}
	\item Auf Lisas T-Shirt steht \glqq I Kissed the Balmoujelaoud\grqq , eine Anspielung auf die \glqq I kissed the Blarney Stone\grqq -Shirts, die in Irland verkauft werden.
	\item Filmzitat I: Die Szene, in der Homer am Flughafen von Marokko erwischt wird, als er versucht, Ramsch-Souvenirs außer Landes zu schmuggeln, indem er sie mit Klebeband an seiner Brust befestigt, ist eine Parodie auf den versuchten Drogenschmuggel am Anfang von \glqq Midnight Express\grqq .
	\item Filmzitat II: Homers Traum von einer Hirntransplantation ist eine Parodie auf die alten Frankenstein-Filme.
	\item Filmzitat III: Das Ende der Geschichte ist eine Parodie auf den Film \glqq The Thing with Two Heads\grqq .
	\item Fernsehzitat I: Johnny Carson tritt im Fernseher der Simpsons auf in seiner Verkleidung als Karnac, der Große. Er hält einen Umschlag an seine Stirn und sagt \glqq Geraldo Rivera, Madonna und ein verendeter Hirsch\grqq . Niemand stellt eine Frage.
	\item Fernsehzitat II: Die Geschichte des Jungen, der fremde Gedanken lesen kann und mit seinen eigenen Gedanken Verwüstungen anrichtet, stammt aus der Folge \glqq It's A Good Life\grqq\ aus der Serie \glqq Twilight Zone\grqq\ mit Bill Murray.
\end{itemize}
}

	
\subsection{Verbrechen lohnt sich nicht}\label{8F03}
Bart gerät in die Klauen einer Mafiabande. Doch aus dem Gefangenen wird sehr schnell ein gefragter Junge: Bart weiß die besten Drinks zu mixen und beim Pferderennen die richtigen Tipps zu geben. Eines Tages kommt Bart zu spät zum Dienst bei den Mafiosi, weil er bei Rektor Skinner nachsitzen musste. Die Bande verspricht, sich um den Rektor zu kümmern. Bart bekommt eine Mordanklage an den Hals.

\notiz{
\begin{itemize}
  \item Der lächelnde Azteke in Troy McClures Kurzfilm über Schokolade ist fast identisch mit dem Maskottchen des Baseballteams \glqq Cleveland Indians\grqq .
	\item Filmzitat: Barts Job als Helfer bei Fat Tony und seinen Gangsterfreunden ist eine Parodie auf die Karriere des jungen Henry Hill in \glqq Goodfellas\grqq\ von Martin Scorsese.
	\item Fat Tonys Gefangenennummer ist 8F03 genau wie der Produktionscode dieser Folge.
	\item In dieser Episode wird enthüllt, dass Fat Tony eigentlich Anthony Williams heißt\index{Williams!Anthony} \cite{SpringfieldShopper2}.
	\item Lionel Hutz ist Bart als Pflichtverteidiger zugeteilt.
	\item Clancy Wiggum wird in dieser Episode als Inspektor bezeichnet.
\end{itemize}
}

	
\subsection{Der Ernstfall}\label{8F04}
Durch eine winzig kleine Unachtsamkeit verursacht Homer einen Störfall im Kernkraftwerk: Eine Kernschmelze steht kurz bevor. Homer sitzt im Schaltzentrum und weiß nicht, was er tun soll. Es sind einfach zu viele Knöpfe am Schaltpult. Schließlich bemüht er einen Abzählreim und drückt dann auf den übrig gebliebenen Knopf. Die Kernschmelze wird verhindert und Homer ist der Held des Tages.

\notiz{
\begin{itemize}
  \item Erster Auftritt von Luann Van Houten.
  \item Otto sagt, dass sein Bruder seine Ex-Freundin geheiratet hat.
  \item Der Besitzer des Kernkraftwerks in Shelbyville ist Aristotle Amadopoulis\index{Amadopoulis!Aristotle}.
  \item Waylon Smithers hat einen Yorkshire Terrier namens Hercules\index{Hercules}.
  \item Filmzitat: Der Countdown für die Kernschmelze stoppt bei 0:07 Minuten, genau wie im James Bond Film \glqq Goldfinger\grqq .
\end{itemize}
}
	
\subsection{Der Vater eines Clowns}\label{8F05}
Krusty, der Lieblingsclown der Simpsons, wird zum Abendessen eingeladen. Der Abend wird alles andere als lustig: Anstatt seine Gastgeber zu unterhalten, klagt er ihnen sein Leid: Krusty ist ein ganz normaler Mensch und als solcher hat er ebenfalls Schwierigkeiten mit seinem Vater. Dieser ist ein weiser Rabbi, der seinen Sohn wegen dessen Berufswahl verstoßen hat. Die Simpsons wollen ihrem lieben Krusty helfen.

\notiz{
\begin{itemize}
	\item Filmzitat: Das Thema der Geschichte, ein Rabbi, der seinem Sohn verbieten will, Karriere als Entertainer zu machen, erinnert an den Film \glqq The Jazz Singer\grqq .
	\item Barts Brief an Krusty: Lieber Krusty, hier schreibt Dir Bart Simpson, Krusty-Fanclub Nummer 16302. Ich gebe mit vorzüglicher Hochachtung mein Abzeichen zurück. Ich hatte schon immer den Verdacht, dass sich nichts im Leben wirklich lohnt. Jetzt weiß ich es sicher. Leck mich, Bart Simpson.
	\item Jackie Mason, der Krustys Vater, Rabbi Hyman Krustofski\index{Krustofski!Hyman}, spricht, ist der erste Gaststar bei den Simpsons, der für seinen Auftritt einen Emmy gewonnen hat.
	\item Krustys Chefsekretärin heißt Miss Pennycandy\index{Pennycandy}, eine Anspielung auf Miss Moneypenny aus James Bond.
	\item Rabbi Krustofsky und Bart debattieren am Ende der Folge über den Talmud. Bei der Abfassung dieses Textes zogen die Simpsons-Autoren drei echte Rabbis zurate (siehe \cite{Reiss19}).
\end{itemize}
}

	
\subsection{Lisas Pony}\label{8F06}
Lisa soll in der Talentshow der Grundschule auftreten. Homer vermasselt ihr alles, weil er es nicht schafft, ihr ein Saxophonblatt zu kaufen. Lisa ist sauer. Um ihre Liebe wiederzugewinnen, kauft ihr Homer ein Pony, das Lisa Prinzessin tauft. Weil sich die Familie diese Ausgabe nicht leisten kann, nimmt Homer einen weiteren Job an. Er übernimmt in Apus Laden die Nachtschicht. Lisa sieht, dass sich ihr Vater ruiniert und gibt das Pony zurück.

\notiz{
\begin{itemize}
	\item Ein Kalender im Hintergrund zeigt den 07. November an, das Datum der Erstausstrahlung der Episode.
	\item Filmzitat: In einem Zukunftstraum gönnt sich ein affenartiger Homer ein Schläfchen, während er auf dem Monolith aus \glqq 2001: A Space Odyssey\grqq\ liegt. Während die anderen Affen Werkzeug entdecken, erfindet Homer das \glqq sich verdrücken\grqq .
	\item Apu scheint ein Verhältnis mit Prinzessin Kashmir zu haben. 
\end{itemize}
}
	
\subsection{Das Seifenkistenrennen}\label{8F07}
Bei einem Eignungstest für Väter schneidet Homer einfach katastrophal ab: Er erreicht null Punkte und erweist sich damit als einer der schlechtesten Väter aller Zeiten. Auf den Rat eines Fachmanns hin versucht Homer herauszubekommen, was seinen Sohn Bart wirklich interessiert: Seifenkistenrennen. Also gründen Bart und Homer das Simpson-Team. Doch das erste Rennen verläuft alles andere als gut.

\notiz{
\begin{itemize}
	\item Der Originaltitel spielt auf den Film \glqq Days of Thunder\grqq\ (Tage des Donners) mit Tom Cruise an.
	\item Musikzitat: Als Bart auf die Tribüne blickt und sieht, dass sein Vater aufgestanden ist, um ihm beim Rennen zu unterstützen, ertönt das Hauptthema des Films \glqq The Natural\grqq\ (Der Unbeugsame).
	\item Das Reinigungsmittel, das Dr. Nick Rivieria im Home-Shopping-Kanal anbietet, heißt Spiffy\index{Spiffy}.
	\item Laut Marge ist Steve McQueen\index{McQueen!Steve} Homers Held.
\end{itemize}
}

	
\subsection{Das Erfolgsrezept}\label{8F08}
Moes Kneipe läuft schlecht und deshalb liefern auch die Brauereien nicht mehr. Um nicht völlig auf dem Trockenen zu sitzen, mixt sich Homer einen Drink zusammen, den er an einem langweiligen Dia-Abend mit seinen Schwägerinnen erfunden hat: \glqq Flaming Homer\index{Flaming Homer}\grqq\ schmeckt wirklich hervorragend. Doch der fiese Moe klaut ihm das Rezept, tauft den Drink in \glqq Flaming Moe\index{Flaming Moe}\grqq\ um. Prompt platzt seine Kneipe aus allen Nähten.

\notiz{
\begin{itemize}
	\item Maggies Lautäußerung \glqq Moe\grqq\ ist vielleicht ihr erstes gesprochenes Wort, wenn man Halluzinationen zählt. Sie sagt es, als Homer kurzzeitig durchdreht und jeder um ihn herum plötzlich aussieht wie Moe und seinen Namen sagt. (Maggie \glqq sprach\grqq\ außerdem in Barts Albtraum in \glqq Bart bleibt hart\grqq . Dort sagte sie zu Bart: \glqq Du bist schuld, dass ich nicht sprechen kann\grqq\ -- allerdings ohne, dass sich ihre Lippen bewegten.).
	\item Aerosmith ist die erste Band, die bei den Simpsons auftritt und gleichzeitig performt.
	\item Moe heißt eigentlich Morris.
	\item Das Feuer im Reifenhaufen brennt seit 1966.
	\item Moe kauft von seinem letzten Geld den Liebestestautomat.
\end{itemize}
}

	
\subsection{Kraftwerk zu verkaufen}\label{8F09}
Burns hat die Nase voll. Als er aus Deutschland ein interessantes Angebot für sein Kraftwerk bekommt, verkauft er dieses. Die Deutschen verbessern den Sicherheitsstandard. Auch die Arbeitsplätze bleiben erhalten -- bis auf den von Homer Simpson. Homer macht Burns klar, dass Geld nicht alles ist. Dieser kann seinen Reichtum plötzlich nicht mehr genießen, da er niemanden herumkommandieren kann.

\notiz{
Smithers besucht Burns Haus, in dem sein früherer Boss Bienen züchtet und wird von mehreren Bienen gestochen, ohne eine Reaktion zu zeigen. In \glqq 22 Kurzfilme über Springfield\grqq\ (siehe \ref{3F18}) stirbt Smithers beinahe, nachdem er von einer einzigen Biene gestochen wurde.
}

	
\subsection{Blick zurück aufs Eheglück}\label{8F10}
Seine Kinder Bart und Lisa sind nicht gerade begeistert, doch Homer hat sich in den Kopf gesetzt, ihnen die Geschichte seiner Ehe zu erzählen -- in aller epischer Ausführlichkeit. Obwohl Bart und Lisa während der ausufernden Erzählung zweimal weglaufen, kommen sie doch immer wieder zurück. Aus dem einfachen Grund, weil die Geschichte ja doch ganz nett ist und Homer ein sympathischer Kerl ist.

\notiz{
Smithers war einmal Mitglied der Alpha-Tau-Bruderschaft.
}

	
\subsection{Wer anderen einen Brunnen gräbt}\label{8F11}
Bart wird zehn Jahre alt. Leider ist unter den Geschenken nichts Brauchbares. Bis auf eines: Ein Funkgerät, mit dem er sich in jedes Radiogerät einschalten kann. Nach ein paar harmlosen Streichen hat er eine Super-Idee: Er versenkt den Verstärker und den Lautsprecher in einem tiefen Brunnen und ruft um Hilfe. Er kolportiert, ein kleiner Waisenjunge namens Timmy O'Toole\index{O'Toole!Timmy} zu sein, den Rektor Skinner verstoßen hat. Die Aktion hat fatale Folgen.

\notiz{
\begin{itemize}
	\item Die Fernsehwerbung für das \glqq Superstar-Celebrity-Microphone\grqq\ ist eine Parodie auf die Ranco Mr. Microphone-Werbung aus den 70er-Jahren.
	\item Zitat I: Das Wall E. Weasel's Restaurant ist mit seiner verrückten Kombination aus Spielen, Pizza und alten mechanischen Figuren eine Parodie auf die Imbiss-Kette \glqq Chuck E. Cheese Pizza Time Theater\grqq .
	\item Zitat II: \glqq We're Sending Our Love down the Well\grqq\ (Wir senden all' unsere Liebe in den Brunnen) ist inspiriert von dem Benefiz-Projekt USA for Africa (\glqq We Are The World\grqq ) aus dem Jahr 1985.
	\item Ursprünglich wollten die Produzenten der Simpsons Bruce Springsteen als Gaststar. Als dieser absagte, fragten sie Sting und dieser nahm gerne an.
\end{itemize}
}

	
\subsection{Der Wettkönig}\label{8F12}
Lisa beklagt sich, dass sich ihr Vater nicht für sie interessiere. Marge rät ihr, sie solle auf Homer zugehen und für dessen Hobbys Interesse bekunden. Lisa beherzigt diesen Rat und wird zur absoluten Football-Expertin. Jede Woche gibt sie todsichere Tipps ab, über die Homer sehr erfreut ist: Nun gewinnt er wenigstens beim Wetten. Lisa entdeckt, dass sich ihr Vater nur für ihre Tipps interessiert.

\notiz{
\begin{itemize}
	\item Die Simpsons speisen zum zweiten Mal im \glqq Goldenen Trüffel\grqq . Ihr erster Ausflug dorthin war in der Affenpfoten-Episode in \glqq Alpträume\grqq\ (siehe \ref{8F02}).
	\item Barney erwähnt, dass aus Norwegen seine Mutter zu Besuch komme.
	\item Professor Dr. Frink stellt den Gamble-Tron 2000\index{Gamble-Tron 2000} vor, einen Computer, der Football-Ergebnisse voraussagen könne.
	\item Filmzitat: Als Homer sagt, \glqq Früher habe ich den Gestank deiner Schweiß\-füße gehasst, aber jetzt ist es der Geruch des Siegers\grqq\ parodiert er damit einen Monolog aus \glqq Apocalypse Now\grqq .
\end{itemize}
}

	
\subsection{Der Wunderschläger}\label{8F13}
Mr. Burns wettet um \$ 1.000.000 mit Aristotle Amadopoulis\index{Amadopoulis!Aristotle}, dem Besitzer des Atomkraftwerks in Shelbyville, dass seine betriebseigene Softballmannschaft die andere Mannschaft besiegt. In der Regel ist diese Mannschaft nicht besonders erfolgreich, aber Mr. Burns gibt Profispielern Alibi-Jobs im Kraftwerk. Außerdem tritt Homer mit einem selbstgebastelten Schläger an, der angeblich Wunder wirkt. Tatsächlich schafft die Mannschaft einen sensationellen Durchmarsch. Nur noch ein Spiel ist zu gewinnen, dann gehört der Pokal ihnen. Ausgerechnet Burns macht ihnen einen Strich durch die Rechnung.

\notiz{
\begin{itemize}
  \item In der vergangenen Saison gewann das Softballteam zweimal und hat 28 Spiele verloren.
  \item Als Ozzie Smith in das Loch fällt, ist die Gleichung $E = mc^2$ zu sehen.
  \item Filmzitat: Homer schnitzt seinen \glqq Wunderschläger\grqq\ aus einem Baum, der von einem Blitz getroffen worden ist, so wie Roy Hobbs in \glqq The Natural\grqq\ (Der Unbeugsame).
  \item Ein Woche nach Ausstrahlung der Originalfolge berichtete die Los Angeles Times von einem achtjährigen Jungen, der seinen Freund vor dem Ersticken gerettet hatte. Auf die Frage, wo er den Griff gelernt habe, antwortete er: \glqq Der war auf einem Poster bei den Simpsons.\grqq
\end{itemize}
}

\subsection{Wenn Mutter streikt}\label{8F14}
Marge kann nicht mehr. Weil sie ständig von ihrer Familie malträtiert wird, erleidet sie einen Nervenzusammenbruch. Eine wohlverdiente Kur ist fällig. Als die Kinder zu den Tanten Patty und Selma verfrachtet werden sollen, wehrt sich Maggie mit Händen und Füßen dagegen. Während sie bei Homer bleiben darf, müssen sich Bart und Lisa mit den Eigenheiten der beiden Tanten herumschlagen.

\notiz{
\begin{itemize}
	\item Der Originaltitel \glqq Homer Alone\grqq\ spielt auf den Kassenschlager \glqq Home Alone\grqq\ (Kevin allein zu Hause) an.
	\item Nummer von Marges Polizeifoto: 50798.
	\item Der Kopf von Xt'Tapalataketel\index{Xt'Tapalataketel} (das Geschenk aus \glqq Der Lebensretter\grqq ) ist noch immer im Keller der Simpsons (siehe \ref{7F22}).
	\item In dieser Episode heißt Arnie Pye mit Nachnamen Angel. Er ist nicht nur als Reporter in der Luft zu sehen, sondern er fliegt den Helikopter auch selbst.
\end{itemize}
}
	
\subsection{Der Eignungstest}\label{8F15}
Bart und Lisa machen bei einem Berufstest in der Schule mit. Laut Testergebnis soll Bart Polizist werden. Seymour Skinner setzt ihn als Hüter von Recht und Ordnung in der Schule ein. Lisa gefällt das Ergebnis des Tests nicht: Demnach wäre Hausfrau das Richtige für sie. Lisa wird immer muffiger und hat zu nichts mehr Lust. Schließlich klaut sie aus Frust die Lehrerschulbücher mit den richtigen Antworten. Polizist Bart kommt ihr auf die Schliche. Doch um Lisas Zukunft nicht zu gefährden, behauptet er Skinner gegenüber, er habe die Bücher geklaut.

\notiz{
\begin{itemize}
	\item Filmzitat: Als Rektor Skinner Lisa fragt, wogegen sie rebelliert, antwortet sie mit \glqq Was hast 'n anzubieten?\grqq\ -- wie Marlon Brando in \glqq Die Faust in der Tasche\grqq . Sie hat ebenso einen Zahnstocher in ihrem Mund.
	\item Marge wollte als kleines Mädchen Astronautin werden.
	\item Der Eignungstest ergibt: Janey Architektin, Ralph Wiggum wird Fischverwerter, Martin Prince Systemanalytiker und Milhouse Van Houten militärischer Hardliner.
	\item Edna Krabappel hat in Harvard studiert.
\end{itemize}
}

	
\subsection{Die Kontaktanzeige}\label{8F16}
Barts Lehrerin Mrs. Krabappel ist kürzlich geschieden worden und sucht bisher vergeblich über Kontaktanzeigen einen neuen Lebensgefährten. Als Bart eines Tages von ihr mit Nachsitzen bestraft wird, weil er es in der Schule mal wieder zu bunt getrieben hat, entdeckt er in ihrem Schreibtisch die Anzeigen. Er beschließt, ihr einen Streich zu spielen: Er schreibt ihr als Antwort auf ihre Anzeige einen Liebesbrief.

\notiz{
\begin{itemize}
  \item Marge erwähnt, dass ihr Vater ein Fotograf für Babyfotos war, nachdem er die Marine verlassen hatte. In der Episode \glqq Angst vorm Fliegen\grqq\ (siehe \ref{2F08}) sagt sie dann, dass ihr Vater Flugbegleiter war.
  \item Gemäß dieser Folge wohnen die Simpsons in 94 Evergreen Terrace.
  \item Bart benutzt den Vornamen Woodrow\index{Woodrow} für seine Kontaktanzeige und als Bild legt er ein Foto von Gordie Howe\index{Howe!Gordie}\footnote{Gordie Howe (geboren am 31. März 1928 in Floral, Saskatchewan; gestorben am 10. Juni 2016 in Toledo, Ohio) war ein ehemaliger kanadischer Eishockeyspieler, der oft \glqq Mr. Hockey\grqq\ genannt wird.} bei.
  \item Homer schickte Marge 1978 eine Postkarte mit Liebesgrüßen aus der Duff Brauerei in Capital City.
  \item Das Antwortschreiben, das Bart von Mrs. Krabappel erhält, ist an Woodrow, 94 Evergreen Terrace adressiert.
  \item Rod und Todd Flanders sind in der Grundschule Springfield zu sehen. In der Folge \glqq Schlaflos mit Nedna\grqq\ (siehe \ref{PABF15}) sind beide auf einer anderen Schule.
\end{itemize}
}

	
\subsection{Auf den Hund gekommen}\label{8F17}
Knecht Ruprecht, der Hund der Simpsons, wird krank. Die Operation, die ihn retten könnte, kostet 750 Dollar. Eine Menge Geld, welches die Simpsons nicht haben. Schließlich paukt Marge einen drastischen Sparplan durch und Knecht Ruprecht kann gerettet werden. Doch alle Familienmitglieder nehmen es ihrem Hund ziemlich übel, dass sie sich seinetwegen derart einschränken müssen. Daraufhin läuft der Hund weg und landet schließlich über das Tierheim bei Mr. Burns, der aus Knecht Ruprecht einen Killerhund macht. Als er aber Bart sieht, erkennt er ihn wieder und rettet ihm dadurch das Leben.

\notiz{
\begin{itemize}
	\item Telefonnummer der Simpsons (sichtbar auf den Flyers): KLS-3457.
	\item Ned Flanders trägt seine Assassins-Schuhe\index{Assassins}. Er hat sie in \glqq Betragen mangelhaft\grqq\ (siehe \ref{7F14}) gekauft.
	\item Filmzitat: Burns verpasst Knecht Ruprecht eine Gehirnwäsche, indem er ihn einsperrt, seine Augen gewaltsam offen hält und ihn zwingt, Filme anzusehen, die ein Hund fürchterlich finden würde -- so wie in dem Film \glqq A Clockwork Orange\grqq\ (Uhrwerk Orange).
	\item Kent Brockman gewinnt im Lotto 130 Millionen Dollar mit den Lottozahlen: 17, 3, 25, 41, 38 und 49.
	\item Kent Brockman hat ein Jahresgehalt von 500.000 Dollar bei Kanal 6.
\end{itemize}
}

	
\subsection{Homer auf Abwegen}\label{8F19}
Als sich Homer bei einem Kinobesuch ziemlich danebenbenimmt, lässt Marge ihn das deutlich spüren. Beleidigt unternimmt er eine Kneipentour, bei der er die Kellnerin Lurleen Lumpkin\index{Lumpkin!Lurleen} kennenlernt. Lurleen singt Country-Lieder, wovon Homer begeistert ist. Er will sie groß rausbringen. Er wird deshalb ihr Manager und nennt sich Colonel Homer. Während Lurleen immer populärer wird, leidet Marge ziemlich unter dieser \glqq musikalischen\grqq\ Beziehung ihres geliebten Homer.

\notiz{
\begin{itemize}
	\item Filmzitat: Als er Spittle County\index{Spittle County} betritt, kommt Homer an einem Kind vorbei, das auf einer Veranda Banjo spielt -- so wie in dem Actiondrama \glqq Deliverence\grqq\ (\glqq Beim Sterben ist jeder der Erste\grqq ).
	\item Krusty gibt an, eine Schwester zu haben, mit der wohl Sideshow Mel eine Affäre hatte.
	\item Die Kneipe, in der Homer Lurleen kennenlernt, heißt \glqq Beer \grq N\grq\ Brawl\index{Bier-N-Brawl}\grqq\ und dort wird das Bier der Marke Fudd\index{Fudd} ausgeschenkt.
	\item Dies ist die einzige Folge, die Matt Groening alleine geschrieben hat. Sonst hat er nur noch bei den Episoden \glqq Der Babysitter ist los\grqq\ (siehe \ref{7G01}) und \glqq Bart köpft Oberhaupt\grqq\ (siehe \ref{7G07}) als Co-Autor mitgewirkt.
	\item Homers Name als Manager, Colonel Homer, ist eine Anspielung auf Colonel Tom Parker, den Manager von Elvis Pressley.
\end{itemize}
}

\subsection{Bis dass der Tod euch scheidet}\label{8F20}
Tante Selma will unbedingt heiraten. Weil sie niemanden findet, schreibt sie einem Häftling. Als sie ihren Bräutigam der Familie vorstellt, bekommt Bart einen Schock: Bei dem Mann handelt es sich um Sideshow Bob, den Bart ins Gefängnis gebracht hat. Während Sideshow Bob der Familie glaubhaft den Geläuterten vorspielt, bleibt Bart misstrauisch. Zu Recht, denn Sideshow hat vor, Tante Selma wegen ihres Geldes umzubringen.

\notiz{
\begin{itemize}
	\item Selma arbeitet an Schalter 6 der Zulassungsstelle.
	\item Fernsehzitat: Bobs TV-Versöhnung mit Krusty bei seinem \glqq Telethon for Motion Sickness\grqq\ (Telemarathon für die Opfer von Reisekrankheiten) ist eine Parodie auf die Wiedervereinigung von Jerry Lewis und Dean Martin bei einem \glqq Jerry's Labor Days\grqq -Telethon (eine Sendung, in der live um Spenden per Telefon gebeten wird und Prominente die Anrufe entgegennehmen).
	\item Premiere: Der Charakter, der früher als \glqq Jailbird\index{Jailbird}\grqq\ bekannt war, wird hier zum ersten Mal \glqq Snake\index{Snake}\grqq\ genannt.
	\item Sideshow Bobs Häftlingsnummer lautet 24601.
	\item Selma hat als Kind ihren Geruchs- und Geschmackssinn verloren.
	\item Vor dem Haus der Simpsons ist ein Verkehrsschild zu sehen, aus dem zu lesen ist, dass Shelbyville 34 Meilen entfernt ist.
	\item Selma und Patty sind 41 Jahre alt.
	\item Auf dem Autokennzeichen des Hochzeitautos steht: IH8 Bart (Ich hasse Bart).
	\item Sideshow Bob gewinnt einen Emmy.
\end{itemize}
}

	
\subsection{Der Fahrschüler}\label{8F21}
Schulbusfahrer Otto fährt noch schlimmer als gewöhnlich und fährt den Bus zu Schrott. Dabei stellt sich heraus, dass Otto gar keinen Führerschein hat. Rektor Skinner wirft Otto raus. Die Simpsons erbarmen sich und nehmen ihn auf, aber das geht nicht lange gut. Schließlich muss mit mehr oder weniger Gewalt durchgesetzt werden, dass Otto einen Führerschein bekommt, damit ihn die Simpsons wieder loswerden.

\notiz{
\begin{itemize}
  \item Die Spi\"nal-Tap\index{Spi\"nal Tap}-Welttournee umfasst die folgenden vier Städte: London, Paris, München und Springfield.
  \item Bart bekommt von Homer eine E-Gitarre für Linkshänder. Später in der Folge ist zu sehen, wie Otto mit der Gitarre in der Garage der Simpsons spielt und dort ist es plötzlich eine Gitarre für Rechtshänder.
  \item Ottos Vater ist Admiral.
  \item In dieser Episode erfährt der Zuschauer, dass Otto mit vollem Namen Otto Mann\index{Mann!Otto} heißt. Auf dem Führerschein ist der Name Otto Mans\index{Mans!Otto} zu lesen.
  \item Laut Führerschein wurde Otto am 18. Januar 1963 geboren.
\end{itemize}
}

	
\subsection{Liebe und Intrige}\label{8F22}
Barts Freund Milhouse verliebt sich in die neue Mitschülerin Samantha Stanky\index{Stanky!Samantha}, die ebenfalls bald auf ihren Verehrer aufmerksam wird. Sehr zum Ärger von Bart werden die beiden ein Paar. Weil Milhouse nun keine Zeit mehr für ihn hat, wird Bart immer eifersüchtiger. Schließlich verpfeift er Samantha bei ihrem strengen Vater. Dieser reagiert sofort und steckt das Mädchen in eine strenge Klosterschule.

\notiz{
\begin{itemize}
	\item Filmzitat: Die Eröffnungsszene, in der Bart versucht, Homers Kleingeldglas zu stehlen, ist eine Parodie auf den Anfang von \glqq Jäger des verlorenen Schatzes\grqq .
	\item Samanthas Vater verkauft Alarmanlagen.
\end{itemize}
}

	
\subsection{Der vermisste Halbbruder}\label{8F23}
Bei der Gesundheitsuntersuchung im Kraftwerk wird festgestellt, dass Homer unfruchtbar ist. Aus Angst vor einer Klage überreicht Mr. Burns Homer eine Auszeichnung und einen Scheck über 2000 Dollar. Als Herbert Powell\index{Powell!Herbert}, Homers Halbbruder, dies in der Zeitung ließt, kehrt er zurück und bittet die Simpsons um Hilfe bei einer Erfindung, die er gemacht hat. Homer gibt ihm das Geld, welches er bekommen hat. Die Erfindung, ein Baby-Übersetzer, ist ein voller Erfolg und Herbert ist wieder reich.

\notiz{
\begin{itemize}
	\item Filmzitat: Homers psychedelischer, traumartiger Rausch während er im \glqq Spine Melter 2000\grqq\ (Wirbelrüttler 2000) sitzt, erinnert an eine Szene kurz vor Schluss aus \glqq 2002: Odyssee im Weltall\grqq .
	\item Einer der Landstreicher, der um das Feuer sitzt, ist angezogen wie Charlie Chaplins Figur \glqq Tramp\grqq\ und verspeist einen Schuh. Ein anderer ist ein Landstreicher-Clown.
	\item Musikzitat: Homers Satz (im amerikanischen Original) \glqq S'cuse me while I kiss the sky\grqq\ stammt aus dem Lied \glqq Purple Haze\grqq\ von Jimi Hendrix.
	\item Professor Frink gibt an, dass er verheiratet ist und einen Sohn hat.
	\item Fehler: Als Herb mit den anderen Landstreichern spricht, sagt er, er habe seinen Autos japanische Namen gegeben. Doch in der Episode \glqq Ein Bruder für Homer\grqq\ (siehe \ref{7F16}) haben seine Autos die Namen wilder Tiere.
\end{itemize}
}

\section{Staffel 4}

\subsection{Krise im Kamp Krusty}\label{8F24}
Bart und Lisa haben nur einen Wunsch: Sie wollen in das Sommerlager \glqq Kamp Krusty\grqq\ fahren, das der lustige Fernseh-Clown Krusty leitet. Doch als sie dort ankommen, erleben sie eine Überraschung: Gangster haben die Herrschaft im Lager übernommen. Die Kinder werden wie Gefangene gehalten. Bart hat ziemlich schnell die Nase voll davon. Den Simpson-Kindern ist klar, dass gegen diese üblen Zustände etwas unternommen werden muss.

\notiz{
\begin{itemize}
  \item Während Barts Traum läuft das Lied \glqq School's Out\grqq\ von Alice Cooper.
  \item Filmzitat I: Lisa besticht einen Fremden auf einem Pferd mit einer Flasche Alkohol, damit er ihren Brief nach Hause schmuggelt -- so wie Meryl Streep in \glqq Die Geliebte des Französischen Lieutenants\grqq .
  \item Filmzitat II: Das niedergerissene Kamp Krusty hat Ähnlichkeit mit Kurzts Feldlager in \glqq Apocalypse Now\grqq . 
  \item Der Leiter des Kamp Krusty, Mr. Black\index{Black}, war 15 Jahre lang Präsident von Euro-Krustyland.
\end{itemize}
}

	
\subsection{Bühne frei für Marge}\label{8F18}
Eine Musical-Version von \glqq Endstation Sehnsucht\grqq\ wird vorbereitet. Marge geht zum Vorsprechen. Als der Regisseur, Llewellyn Sinclair\index{Sinclair!Llewellyn}, Marge im Gespräch mit ihrem Mann Homer zufällig belauscht, weiß er, dass er die geeignete \glqq Blanche\index{Blanche}\grqq\ gefunden hat. Homer, dem Marges Schauspiel-Engagement gar nicht gefällt, gebärdet sich als Pascha, was wiederum Marge derart ärgert, dass sie das richtige \glqq Feuer\grqq\ in ihre Rolle einbringt.

\notiz{
\begin{itemize}
  \item Filmzitat I: Als der Öffner einer Dose abbricht, steht Homer draußen im Hof und ruft \glqq Marge!\grqq\ wie Marlon Brando es in \glqq Endstation Sehnsucht\grqq\ getan hat.
  \item Filmzitat II: Die Sequenzen, als sich Maggie und die anderen Babys ver\-bün\-den, um ihre Schnuller zurück zu bekommen, sind an \glqq Gesprengte Ketten\grqq\ angelehnt.
  \item Filmzitat III: Homer holt Maggie am Premierenabend von der Kinderbetreuung ab. Die anderen Babys sitzen um sie herum. Das einzige Geräusch ist das Saugen an den Schnullern. Homer bewegt sich durch die Kinder wie Rod Taylor in \glqq Die Vögel\grqq .
  \item Alfred Hitchcock geht am Gebäude der Kinderbetreuung vorbei, als Homer Maggie abholt.
  \item Die Kinderkrippe wird von Llewellyn Sinclairs Schwester geleitet.
  \item Die Produktionsfirma wechselte von Klasky-Csupo\index{Klasky-Csupo} zu Film Roman\index{Film Roman}, welche auch \glqq Family Guy\grqq\ \index{Family Guy} produziert.
\end{itemize}
}

	
\subsection{Ein gotteslästerliches Leben}\label{9F01}
Sehr zum Unwillen seiner Frau Marge hat es Homer satt, jeden Sonntag in die Kirche zu gehen. Weder Marge noch Reverend Lovejoy können den hartnäckigen Häretiker umstimmen. Er bleibt am Sonntag allein zu Haus, liegt rauchend auf dem Sofa und schläft schließlich ein. Wenig später steht das Haus in Flammen. Ein Zeichen von Gottes Unwillen über Homers lasterhaftes Leben? Schließlich wird Homer von Ned Flanders aus dem brennenden Haus gerettet.


\notiz{
\begin{itemize}
  \item Filmzitat I: Die Szene in der die Familie Simpson zur Kirche fährt, bei der Maggie mit der Zunge am Kindersitz kleben bleibt, ähnelt der Szene in \glqq Dumm und Dümmer\grqq\ als Jeff Daniels mit seiner Zunge am Sessellift kleben bleibt.
  \item Filmzitat II: Während Ned Homer rettet, brennt der Boden unter Flanders Füßen wie in der Rettungsszene in \glqq Backdraft\grqq .
  \item Filmzitat III: Homer tanzt in seiner Unterwäsche wie Tom Cruise in \glqq Lockere Geschäfte\grqq .
  \item Moe ist Schlangenbeschwörer.
  \item In Homers erster Traumsequenz hat Gott vier Finger und einen Daumen -- in Homers Traum kurz vor Schluss hat Gott nur drei Finger. In späteren Episoden hat Gott immer fünf Finger.
  \item Die Mutter von Milhouse ist Mitglied der Freiwilligen Feuerwehr in Springfield, ebenso wie Clancy Wiggum, Otto, Barney, Krusty und Apu.
  \item Fehler: Als Marge den Parkplatz verlassen will, kann man das Auto zweimal sehen. Erst ist es sauber und dann liegt massig Schnee darauf.
\end{itemize}
}
	
\subsection{Lisa, die Schönheitskönigin}\label{9F02}
Weil sich Lisa wieder einmal ausgesprochen hässlich findet, meldet Homer sie zu einem Schön\-heits\-wett\-be\-werb an. Das soll ihr Selbstvertrauen stärken. Und tatsächlich -- das Mädchen wird Zweite in der gnadenlosen Konkurrenz. Doch es kommt noch besser: Amber Dempsey\index{Dempsey!Amber}, die Siegerin, wird bei ihrem ersten Auftritt von einem Blitz getroffen. Selbstverständlich übernimmt Lisa nun die Verpflichtungen der Siegerin. Ebenso klar ist, dass dies nicht ihr Ding ist, da sie Werbung für Zigaretten machen soll.

\notiz{
\begin{itemize}
  \item Homer gibt an, dass er 36 Jahre alt ist und 216 Pfund wiegt.
  \item Um Lisa für den Schönheitswettbewerb anmelden zu können, verkauft er Barney den gewonnenen Rundflug mit dem Duff-Zeppelin für 250 Dollar.
  \item Im Springfielder Wachsfigurenkabinett sind die Köpfe von Mr. T, Dr. Ruth und Ronald Reagan zu sehen.
  \item Eine Filiale der Handelskette Sh\o p wird eröffnet.
  \item Die Szene, in der Bob Hope und Lisa vor den aufgebrachten Soldaten mit dem Hubschrauber gerettet werden, ist eine Anspielung auf den Film \glqq Apocalypse Now\grqq\ von 1979.
\end{itemize}
}

	
\subsection{Bart wird bestraft}\label{9F03}
Homer kommt immer mehr zu der Ansicht, dass aus seinem Sohn Bart nichts Rechtes werden will. Der Vater beschließt schweren Herzens, seinen Sohn Bart endlich einmal richtig zu bestrafen, da er seine Nachsicht bisher immer nur schamlos ausgenutzt hat. Als der neue Zeichentrick-Kultfilm \glqq Itchy \& Scratchy\grqq\ ins Kino kommt, verbietet er Bart, diesen Film zu sehen -- eine Strafe, die Bart gar nicht gut findet.

\notiz{
\begin{itemize}
  \item Der Vater von Martin Prince sitzt in der Klasse von Lisa, obwohl Martin mit Bart in die Klasse geht.
  \item Filmzitat: Nachdem Bart eine James Bond Figur in die Mikrowelle gestellt hat, streichelt er die Katze wie der Bösewicht Blofeld im James Bond Film \glqq Dr. No\grqq .
  \item Klassisches Cartoon-Zitat: Der erste Itchy \& Scratchy Kurzfilm, \glqq Steamboat Itchy\grqq , parodiert den ersten Mickey Maus Cartoon, \glqq Steamboat Willie\grqq .
  \item Lisa sagt, im Itchy \& Scratchy Film hatten Dustin Hoffman und Michael Jackson Gastauftritte, die allerdings nicht genannt werden. Dustin Hoffmann hatte bei den Simpsons in der Folge \glqq Der Aushilfslehrer\grqq\ (siehe \ref{7F19}) einen Gastauftritt und Michael Jackson in \glqq Die Geburtstagsüberraschung\grqq\ (siehe siehe \ref{7F24}). Beide wurden auch bei den Simpsons nicht namentlich genannt.
  \item Im Aztec Theater der Zukunft wird \glqq Soylent Green\index{Soylent Green}\grqq\ verkauft. Soylent Green ist in dem Film \glqq Jahr 2022\dots die überleben wollen\grqq\ das einzige Nahrungsmittel der Zukunft.
\end{itemize}
}
	
\subsection{Bösartige Spiele}\label{9F04}
Wieder einmal ist Halloween: Die Einwohner von Springfield spielen an diesem Tag völlig verrückt. Natürlich macht auch Familie Simpson bei dem Unfug kräftig mit. 
\begin{itemize}
	\item \textbf{Die Klauen des Clowns}\\ Homer schenkt Bart eine Krusty-Puppe, die sich selbständig macht und nur Böses im Sinn hat. Die Simpsons erleben die unheimlichsten Abenteuer mit der Krusty-Puppe -- bis hin zur Verfolgung durch Zombies. Schließlich aber landen sie dann doch wieder vor ihrem Fernseher -- wenn auch völlig entnervt.
	\item \textbf{King Homer}\\ Mr. Burns und Smithers reisen auf eine Insel, um einen riesigen Affen (Homer) zu fangen. Als Köder nehmen sie Marge mit. Mr. Burns gelingt es schließlich, den Affen von der Insel nach Springfield zu entführen und dort als achtes Weltwunder zur Schau zu stellen. Homer kann sich losreißen und läuft Amok. Doch schließlich heiraten Homer und Marge noch.
  \item \textbf{Dial 'Z' for Zombies}\footnote{Der Titel dieser Teilepisode ist eine Anspielung auf den Hitchcock Film \glqq Dial M For Murder\grqq .}\\ Bart bekommt die Aufgabe, ein Buch zu lesen. Er wählt eines aus der Okkultismus-Abteilung und versucht, die tote Schneeball I wiederzubeleben, stattdessen bringt er die Toten dazu, sich aus ihren Gräbern zu erheben.
\end{itemize}

\notiz{
\begin{itemize}
  \item Die Affenpfote aus \glqq Alpträume\grqq (siehe \ref{8F02}) liegt auf dem Verkaufstresen des \glqq House of Evil\grqq .
  \item Fernsehzitat: Die Krusty Puppe, die gut zu ihrem jungen Besitzer ist, aber böse zu dessen Vater, ist eine Parodie auf die Episode \glqq Living Doll\grqq\ aus der TV-Serie \glqq Twilight Zone\grqq\ mit Telly Savalas. Zitat: \glqq My name is Talking Tina and I love you very much\grqq\ (Mein Name ist \glq Talkin Tina\grq\ und ich habe Dich sehr gern.).
  \item Der Ausruf des Inselhäuptlings \glqq Mosi Tatupu! Mosi Tatupu!\grqq\ wird über\-setzt als \glqq Die blauhaarige Frau wird ein ideales Opfer\grqq . In Wirklichkeit war \glq Mosi Tatupu\grq\ aus Samoa in den 70ern ein Footballspieler für den USC und setzte seine Karriere in der NFL mit den New England Patriots fort.
  \item Filmzitat I: Die gesamte Geschichte \glqq King Homer\grqq\ ist eine Parodie auf \glqq King Kong\grqq .
  \item Filmzitat II: Diese Geschichte über Zombies, die aus Gräbern auferstehen und menschliches Fleisch und Organe verschlingen, parodiert George Romeros Horrorklassiker \glqq Night of the Living Dead\grqq\ (Die Nacht der lebenden Toten).
  \item Auf dem Friedhof trägt Bart das Plattencover von Michael Jacksons \glqq Thri\-ller\grqq\ auf dem Kopf, während er die Zombies zum Leben erweckt. Im Video zu \glqq Thriller\grqq\ geht es ebenfalls um Zombies, die aus ihren Gräbern auferstehen.
  \item Ein Zombie kommt aus dem Grab mit der Aufschrift \glqq Jay Kogen\grqq . Jay Kogen ist ein Autor/Produzent der Simpsons.
  \item Um die Zombies zu befreien, singt Bart im Original \glqq Cullen, Rayburn, Narz, Trebek!\grqq\ -- allesamt Gameshow-Moderatoren, und \glqq Zabar, Kresge, Caldor, Walmart\grqq , die Namen von Supermärkten.
  \item Als er Lisa in eine Schnecke verwandelt, singt Bart im Original \glqq Kolchak, Mannix, Banacek, Dano!\grqq\ (alles TV-Detektive aus den 70ern), und um die Zombies wieder loszuwerden \glqq Trojan, Ramses, Magnum, Sheik!\grqq\ (vier Markennamen von Kondomen).
\end{itemize}
}

	
\subsection{Marge muss jobben}
Das Haus der Simpsons ist auf einer Seite abgesackt. Für die Renovierung ihres Hauses brauchen die Simpsons eine Menge Geld (8500 Dollar). Also erklärt sich Marge bereit, einen Job im Atomkraftwerk anzunehmen. Mr. Smithers stellt Marge aufgrund eines von Lisa auffrisierten Lebenslaufs ein. Mr. Burns verliebt sich in Marge. Doch dann erfährt er, dass Marge die Frau seines Angestellten Homer Simpson ist, daraufhin entlässt er Marge. Marge und Homer drohen mit juristischen Schritten. Mr. Burns gelingt es, sich auf ungewöhnliche Art aus der Affäre zu ziehen.

\notiz{
\begin{itemize}
  \item Ned Flanders raucht Pfeife.
  \item Marge übernimmt den Job von Jack Marley\index{Marley!Jack} in Sektor 7G, der nach über 45-jähriger Firmenzugehörigkeit in den Ruhestand versetzt wird.
  \item Ein Foto in Smithers Büro zeigt Mr. Burns, wie er Elvis die Hand gibt.
  \item Wenn Homer Fehler macht, schiebt er diese auf Tibor\index{Tibor}, einen von ihm erfundenen Angestellten.
  \item Das Lied \glqq It's Mr. Burns\grqq , welches Mr. Smithers bei der Pensionierung Jack Marleys singt, ist aus dem Film \glqq Citizen Kane\grqq .
  \item Das Fundament des Simpson Hauses übernimmt die Firma Surly Joe's\index{Surly Joe's}.
  \item Als Abe babysittet und denkt, Maggie sei krank, versucht er die Diagnose mit dem \glqq Guide To Infantile Destress\grqq\ von Dr. Washburn Asbestpillen zu stellen.
\end{itemize}
}

	
\subsection{Laura, die neue Nachbarin}\label{9F06}
In das Nachbarhaus der Simpsons ziehen neue Nachbarn ein: Mrs. Ruth Powers\index{Powers!Ruth} mit ihrer überaus hübschen Tochter Laura. Bart verliebt sich auf der Stelle in die schöne Laura\index{Powers!Laura}. Da gibt es allerdings ein Problem: Zu seinem großen Leidwesen hat sich Laura schon in den Draufgänger Jimbo Jones\index{Jones!Jimbo} verknallt. Mit einem Trick versucht Bart, am Ansehen des Schürzenjägers zu kratzen. Doch verliebte Frauen haben das nicht gern. Bart muss sich unbedingt eine andere Strategie ausdenken.

\notiz{
\begin{itemize}
  \item Filmzitat: Die Gerichtsszene, in der Gerichtsvollzieher ganze Säcke voller Briefe an den Weihnachtsmann herein bringen, ist eine Parodie auf den Film \glqq Das Wunder in der 34. Straße\grqq .
  \item Bart erzählt Moe, dass die Adresse der Simpsons \glqq 1094 Evergreen Terrace\grqq\ lautet (tat\-säch\-lich lautet sie \glqq 742 Evergreen Terrace\grqq ).
  \item Homer verklagt das Restaurant \glqq The Frying Dutchman\grqq , weil er nicht so viel essen durfte, wie er eigentlich gekonnt hätte.
  \item Die ehemaligen Nachbarn der Simpsons waren die Winfields\index{Winfield}, in deren Haus Ruth Powers einzieht.
  \item Marge ist allergisch gegen Meeresfrüchte jeglicher Art.
  \item Sara Gilbert, welche Laura Powers im englischen Original die Stimme lieh, spielte die Darlene Conner in der Serie \glqq Roseanne\grqq .
\end{itemize}
}
	
\subsection{Einmal als Schneekönig}\label{9F07}
Homer Simpson hat einen sagenhaften Unfall gebaut. Als Ersatz für seinen Wagen, den er zu Schrott gefahren hat, kauft sich Homer einen Schneepflug. Damit verdingt er sich als Schnee\-räum\-er, um die Raten abzahlen zu können. Das Geschäft läuft gut und deshalb tut es ihm sein Freund Barney bald nach. Weil dessen Schneepflug aber viel größer ist, wird Homer bald arbeitslos. Und das war irgendwie nicht in der Kalkulation Homers enthalten. Dagegen muss etwas unternommen werden.

\notiz{
\begin{itemize}
  \item Homer kauft den Schneepflug bei Kumatsu Motors\index{Kumatsu Motors}, das ist jene Firma, welche Herb Powells Firma in der Episode \glqq Ein Bruder für Homer\grqq\ (siehe \ref{7F16}) ersetzt.
  \item Homers Telefonnummer aus der Werbung: KL5-3226
  \item Barneys Telefonnummer aus der Werbung: KL5-4796
  \item Als Homer eine neue Fernsehwerbung braucht, geht er zu \glqq McMahon\index{McMahon} und Tate\grqq , der Werbeagentur von Darren Stevens in der Serie \glqq Bewitched\grqq\ (Verliebt in eine Hexe, US-TV-Serie 1964-72).
  \item Filmzitat: Nachdem Homer einen schulfreien Tag verhindert hat, indem er mit seinem Schneepflug den Weg für den Schulbus frei gemacht hat, wird Bart in einen Hinterhalt gelockt und mit Schneebällen \glqq durchlöchert\grqq\ wie Sonny Corleone in \glqq Der Pate\grqq .
  \item Barney sagt, dass sein Vater tot sei.
  \item Barney spendet dem Tanzmuseum in Shelbyville 50.000 Dollar.
  \item Dan Castellaneta, die Stimme von Homer im Original, gewann für diese Folge 1993 einen Emmy.
\end{itemize}
}

	
\subsection{Am Anfang war das Wort}\label{9F08}
Die kleine Lisa Simpson will endlich wissen, welches ihr erstes Wort war, das sie gesprochen hat. Als Marge und Homer darüber nachdenken, erinnern sie sich auf diese Weise an die vergangenen Jahre und was da alles so abging. Dass da innerhalb der Familie nicht immer alles Gold war, was glänzte, versteht sich. Am schlimmsten aber haben die Eltern die ständigen Streitereien ihrer Kinder empfunden. Lisa und Bart erfahren auf diese Weise einiges, was sie gar nicht gerne hören.

\notiz{
\begin{itemize}
  \item Filmzitat: Die Musik im \glqq Itchy \& Scratchy\grqq -Cartoon \glqq 100-Yard Gash\grqq\ stammt von Vangelis Soundtrack zu \glqq Chariots of Fire\grqq\ (Die Stunde des Siegers).
  \item Obwohl Abe während des Gesprächs mit Homer in der Rückblende jünger aussieht, hängt ein Bild seines älteren Ichs an der Wand von Homer und Marges erster Wohnung.
  \item Lisa wurde im Jahr 1984 geboren.
  \item Drederick Tatum ist bei den Olympischen Spielen als Boxer zu sehen.
  \item In dieser Episode sagt Maggie das Wort \glqq Daddy\grqq , als niemand in ihrem Zimmer ist.
  \item Krusty der Clown wurde 1990 in der Episode \glqq Der Clown mit der Biedermaske\grqq\ (siehe \ref{7G12}), als Analphabet beschrieben. Im Jahr 1984 ist er offenbar in der Lage, ein Telegramm zu entziffern, das ihn über den Boykott der Russen informiert.
\end{itemize}
}

	
\subsection{Oh Schmerz, das Herz!}\label{9F09}
Homer Simpson soll seiner Faulheit wegen entlassen werden. Als ihm dies schonend beigebracht wird, erleidet er tatsächlich einen Herzanfall. Homer muss sofort operiert werden. Doch da gibt es ein Problem: Wie soll man die 40.000 Dollar Operationskosten aufbringen? Da sieht er im Fernsehen den Werbespot eines Dr. Rivieras. Riviera bietet jede beliebige Operation zum Festpreis von 129,95 Dollar an. Soviel gäbe das Familienbudget für eine Herzoperation schon her.

\notiz{
  \begin{itemize}
    \item Die Fingerpuppen, mit denen Homer seine Bypass-Operation erklärt, sehen aus wie die \glqq Life in Hell\grqq -Figuren Akbar\index{Akbar} und Jeff\index{Jeff} ohne Filzkappen.
    \item Ned Flanders spendet im Krankenhaus eine Niere und einen Lungenflügel.
    \item Fehler: Reverend Lovejoy wohnt in der Evergreen Terrace neben dem Haus mit der Nummer 742, das ist aber die Hausnummer der Simpsons und diese wohnen nicht neben Reverend Lovejoy. Denn die Nachbarn der Simpsons sind Ned Flanders mit Familie und Ruth Powers.
  \end{itemize}
}

	
\subsection{Homer kommt in Fahrt}\label{9F10}
Die Stadt hat von dem Umweltverschmutzer Mr. Burns drei Millionen Dollar Schadenersatz erhalten. Nun denkt man an höchster Stelle darüber nach, was man mit dem vielen Geld anstellen könnte. Lyle Lanley\index{Lanley!Lyle}, ein skrupelloser Geschäftemacher, redet den Bürgern ein, dass sie unbedingt eine Einschienenbahn brauchen. Homer bewirbt sich als Zugführer. Die Bahn wird gebaut und Homer bekommt den Job. Dass er dabei gehörig ins Schwitzen kommen wird, hätte er sich nicht träumen lassen.

\notiz{
\begin{itemize}
  \item Filmzitat I: Lyle Lanley benimmt sich wie der reisende Verkäufer Professor Harold Hill in \glqq The Music Man\grqq\ -- mit der Ausnahme, dass Hill nicht nach Springfield, sondern nach River City in Iowa ging und mehr daran interessiert war, dort eine Band zu gründen, als der Stadt eine Einschienenbahn aufzuschwätzen. Das Einschienenbahn-Lied enthält außerdem Elemente des Liedes \glqq Trouble\grqq\ aus \glqq The Music Man\grqq .
  \item Filmzitat II: Die Szene, in der Mr. Burns und Smithers sich darauf vorbereiten, illegal Atommüll zu entsorgen, wird durch eine Variation von \glqq Axel F (Axel's Theme)\grqq\ aus \glqq Beverly Hills Cop\grqq\ begleitet. Eine ähnliche Adaption des Liedes wurde schon in \glqq Der Eignungstest\grqq\ (siehe \ref{8F15}) verwendet.
  \item Auf der Karte, die Lanley den Bürgern von Springfield zeigt, liegt Ogdenville im Westen der USA. Neben Ogdenville\index{Ogdenville} verkaufte er die Monorail auch an die Städte North Haverbrook\index{North Haverbrook} und Brockway\index{Brockway}.
  \item Das Geburtshaus von Jebediah Springfield wird durch das Umfallen des ältesten Baums in Springfield zerstört. Allerdings heißt es in der Folge \glqq Auf zum Zitronenbaum\grqq\ (siehe \ref{2F22}), dass Jebediah Springfield, das Land um Springfield erst in Besitz genommen hat. 
  \item Krusty hat offensichtlich einen unehelichen Sohn.
  \item Lurleen Lumpkin\index{Lumpkin!Lurleen}, die Country-Sängerin aus \glqq Homer auf Abwegen\grqq\ (siehe \ref{8F19}), ist eine der Ehrengäste bei der Jungfernfahrt der Einschienenbahn, nachdem sie gerade eine Entziehungskur gemacht hatte.
  \item Fehler: Dr. Hibbert ist während der Monorailkatastrophe in seiner Praxis zu sehen, aber kurz davor war er noch selbst in der Einschienenbahn.
\end{itemize}
}

	
\subsection{Selma will ein Baby}\label{9F11}
Immer wenn Homer Simpson den Vergnügungspark \glqq Duff Gardens\grqq\ aufsuchen will, kommt etwas dazwischen. Schließlich gehen die Kinder mit ihrer Tante Selma ohne Homer dorthin. Doch ein Vergnügen ist das nicht, denn Tante Selma ist ununterbrochen auf der Suche nach einem Mann. Sie will unbedingt ein Baby haben. Dass sie mit ihren Plänen auf einem Rummelplatz nicht gerade am richtigen Platz ist, ist allen klar -- außer Selma natürlich. Bart und Lisa haben an diesem Tag nicht viel zu lachen.

\notiz{
\begin{itemize}
  \item Selma hatte mit Hans Maulwurf ein Rendezvous.
  \item Hans Maulwurf ist laut seines Führerscheins 31 Jahre alt.
  \item Barney ist in der Springfielder Samenbank zu sehen.
  \item Die Spermabank in Springfield wurde 1858 gegründet.
  \item In dieser Episode halluziniert Lisa erstmals unter Drogeneinfluss. Das zweite Mal (im Stil des Beatles-Films \glqq Yellow Submarine\grqq ) passiert ihr in der Folge \glqq Prinzessin von Zahnstein\grqq\ (siehe (\ref{9F15}).
  \item Marges Tante Gladis verstirbt.
\end{itemize}
}

	
\subsection{Großer Bruder -- Kleiner Bruder}\label{9F12}
Weil sich Bart von seinem Vater Homer vernachlässigt fühlt, wendet er sich an die Organisation der \glqq Großen Brüder\grqq . Dort wird ihm Tom\index{Tom} als Betreuer zugewiesen. Natürlich ist Homer von dieser Aktion seines Sohnes schwer enttäuscht und er wendet sich ebenfalls an die Organisation. Homer kümmert sich nun um Pepi\index{Pepi}. Homer erhält eine Lektion, die nicht nur für ihn außerordentlich lehrreich ist. Am Ende stellen er und sein Sohn Bart gleichermaßen fest, was sie aneinander haben. Unterdessen wird Lisa süchtig nach der Corey\index{Corey} Hotline, eine teure Telefonnummer.

\notiz{
\begin{itemize}
	\item Filmzitat: Milhouse schreibt die Botschaft \glqq Trab pu Kcip! Trab pu Kcip!\grqq\ (Pick up Bart = Bart abholen) an die Wand -- in Rückwärtsschrift wie das Kind in dem Film \glqq The Shining\grqq .
	\item Um in der Schule Geld zu sparen, will Rektor Skinner Physik, Musik und Kunsterziehung streichen.
	\item Die Telefonnummer der großen Brüder lautet KL5 - 2233.
	\item Ursprünglich war Tom Cruise für die Rolle des Tom vorgesehen, doch dieser sagte ab, sodass Phil Hartman Tom gesprochen hat.
	\item Bart und Tom schauen sich im Fernsehen \glqq Ren und Stimpy\grqq\ an.
\end{itemize}
}

	
\subsection{Ralph liebt Lisa}\label{9F13}
Es ist Valentinstag und Lisa schreibt Ralph aus Mitleid eine Karte, weil er sonst keine Karte bekommen hätte. Ralph macht ihr daraufhin Geschenke, u.\,a. eine Eintrittskarte zu Krustys 29-jährigem Bühnenjubiläum. Ralph glaubt, Lisa sei seine Freundin. Doch die amouröse Angelegenheit verläuft sehr einseitig, denn Lisa ist nicht annähernd so begeistert von Ralph, wie dieser von ihr. Die Wende bringt dann jedoch eine Schulaufführung: Als Ralph George Washington spielt und dabei von anderen Mädchen Begeisterungsstürme erntet, besinnt sich Lisa eines Besseren.

\notiz{
\begin{itemize}
  \item Willie gibt an, dass sein Vater wegen Schweinediebstahls gehängt worden sei.
  \item In dieser Episode wird enthüllt, dass Ralph der Sohn von Polizeichef Clancy Wiggum ist.
  \item Fehler: Als Krusty Ralph in seiner Show befragt, ist auf Lisas Kleid kein Schokofleck mehr zu sehen, obwohl kurz vorher Eis darauf gefallen war.
\end{itemize}
}

	
\subsection{Keine Experimente}
Homer Simpson hat einen wahrhaft schwierigen Entschluss gefasst: Er will auf Bitte von Marge das Biertrinken für 30 Tage aufgeben, nachdem er bei einem Besuch der Duff Brauerei seinen Führerschein abgeben musste. Obwohl die Anfechtungen an allen Ecken und Enden lauern, bleibt Homer standhaft. In der Schule findet inzwischen ein \glqq Jugend forscht\grqq -Wettbewerb statt. Bart nimmt mit einem dressierten Hamster teil, der ein Flugzeug fliegen kann. Das ist ein ähnlich mutiges Projekt wie der Entschluss seines Vaters zur Enthaltsamkeit. Und irgendwie enden auch beide Projekte auf ähnliche Weise.

\notiz{
\begin{itemize}
  \item Sorten von Duff: Duff, Duff Lite, Duff Dry (herb), Raspberry Duff (Himbeere), Duff Dark (Dunkel), Tartar-Control Duff (gegen Zahnstein), Lady Duff, Duff Gummi Bears (Gummibären).
  \item Filmzitat: Am Ende der Folge fährt Homer mit Marge auf der Lenkstange Fahrrad. Dazu läuft der Song \glqq Raindrops Keep Falling On My Head\grqq\ genau wie in dem Film \glqq Butch Cassidy and the Sundance Kid\grqq .
  \item Nach seinem Führerschein mit der Nummer \#C4043243, der von dem Richter ungültig gemacht wird, ist Homer 1,83 m , wiegt 108 kg, hat blaue Augen und kein Haar. Geburtsdatum: 12.05.1956. Laut Episode \glqq Karriere mit Köpfchen\grqq\ (siehe \ref{7F02}) wurde er aber 1955 geboren.
  \item Das Staatskürzel auf Homers Führerschein lautet NT.
  \item Die Sorten Duff, Duff Lite und Duff Dry fließen in der Duff-Brauerei alle aus ein und derselben Leitung -- es sind also verschiedene Namen für dasselbe Produkt.
  \item Hans Maulwurf sagt bei ein Anonymen Alkoholikern, dass er erst 31 Jahre alt ist.
  \item Bei der Polizeischulung sind außer Homer unter anderem noch Bernice Hibbert und Ruth Powers.
  \item Ned Flanders gibt an, seit 4000 Tagen keinen Tropfen Alkohol mehr getrunken zu haben.
  \item In dieser Episode hat Sarah Wiggum\index{Wiggum!Sarah}, die Frau von Polizeichef Wiggum, ihren ersten Auftritt.
\end{itemize}
}
	
\subsection{Prinzessin von Zahnstein}\label{9F15}
Die kleine Lisa Simpson braucht unbedingt eine neue Zahnspange. Eine horrende Ausgabe für Homer, da die Angestellten des Kernkraftwerks keine Zahnarztzusatzversicherung mehr haben. Als der neue Tarifvertrag ausgehandelt wird, macht sich Homer für die Wiedereinführung der Versicherung stark und wird daraufhin prompt zum neuen Gewerkschaftsvorsitzenden ernannt. Schließlich kriegt der neue Vorsitzende auch Mr. Burns klein und die kleine Lisa von Zahnarzt Dr. Wolfe\index{Wolfe} eine tolle, fast unsichtbare Spange.

\notiz{
\begin{itemize}
  \item Filmzitat: Lisa zerstört einen Spiegel, nachdem sie darin ihre neue Zahnspange gesehen hat -- wie Joker in dem Film \glqq Batman\grqq .
  \item Als Lisa durch ihren psychedelischen Narkosetraum fliegt, rast das Wort \glqq HATRED\footnote{engl.: Hass}\grqq\ vorbei (in dem Film \glqq Yellow Submarine\grqq\ ist es \glqq LOVE\grqq ).
  \item Diese Episode ist die letzte, welche von Jay Kogen und Wally Wolodarsky geschrieben worden ist.
  \item Als Gaststar für den Zahnarzt war ursprünglich Anthony Hopkins vorgesehen, doch dieser sagte ab. Anschließend fragte man Clint Eastwood, ob er die Rolle übernehmen wolle und auch er sagte ab.
\end{itemize}
}

	
\subsection{Wir vom Trickfilm}\label{9F16}
Lisa und Bart haben das Drehbuch zu einer Folge der Zeichentrickfolge Itchy \& Scratchy geschrieben. Das Buch wird abgelehnt, weil die beiden noch Kinder sind! Daraufhin setzen Lisa und Bart den Namen ihres Großvaters auf die Titelseite. Prompt haben sie Erfolg. Und sie gewinnen sogar einen Preis. Nur der Großvater ist unzufrieden. Er hat zum ersten Mal einen Zeichentrickfilm gesehen und findet ihn widerlich und gewalttätig. Währenddessen muss Homer seinen Highschool-Abschluss nachholen.

\notiz{
\begin{itemize}
	\item Unter den Anwesenden bei der Preisverleihung sind auch viele Simp\-sons-Autoren und Matt Groening.
	\item Homer hatte 1974 seinen Highschool-Abschluss nicht gemacht.
	\item Mr. Dondelinger\index{Dondelinger} war, als Homer und Marge auf die Highschool gingen, ihr Rektor.
	\item Sowohl Lisa als auch Abe können Schreibmaschine schreiben.
\end{itemize}
}
	
\subsection{Nur ein Aprilscherz}\label{9F17}
1. April: Homer gelingt es mehrfach, seine Kinder in den April zu schicken. Bart ist richtig wütend darüber. Er sinnt auf Rache. Sein übler Scherz führt dazu, dass Homer anschließend sieben Wochen im Koma liegt. Dabei passieren Erlebnisse aus der Vergangenheit Revue. Thema: Wie Homer und die Kinder versucht haben, sich gegenseitig eins auszuwischen. Nach sieben Wochen erwacht Homer gesund aus dem Koma. Nur fünf Prozent seines Gehirnes sind verschwunden.

\notiz{
\begin{itemize}
  \item Dies ist die erste sogenannte Clip Show.
  \item Filmzitat: Barneys Versuch, Homer mit einem Kissen zu ersticken und seine anschließende Flucht aus dem Krankenhaus parodieren die letzte Szene aus \glqq Einer flog übers Kuckucksnest\grqq .
\end{itemize}
}
	
\subsection{Das Schlangennest}\label{9F18}
Bart wird von der Schule verwiesen, weil er Rektor Skinner während einer Schulinspektion in eine missliche Lage gebracht hat und es ist Knüppeltag\index{Knüppeltag}\footnote{Der Knüppeltag geht angeblich auf dem 10. Mai 1775 zurück, als Jebediah Springfield zum ersten Mal eine Schlange erschlug. Seitdem versammeln sich die Springfielder an diesem Tag, um Schlangen aus der Umgebung in das Stadtzentrum zu treiben und sie dort mit einem Knüppel zu töten.
Doch Bart findet heraus, dass Jebediah Springfield an diesem Tag in der Schlacht von Ticonderoga\index{Ticonderoga} kämpfte und somit gar keine Schlange erschlagen konnte. Stattdessen wurde dieser Tag 1924 als Vorwand eingeführt, die Iren zu verprügeln.} in Springfield. Die Bewohner des Städtchens sind aufgerufen, alle Schlangen in der Umgebung totzuschlagen. Die kleine Lisa ist darüber sehr empört, was die übrigen Bewohner allerdings wenig beeindruckt. Doch Lisa und Bart haben einen Plan: Sie wissen, dass Schlangen Bodenvibrationen spüren. Sie überzeugen den Sänger Barry White\index{White!Barry}, über Lautsprecher die Schlangen mit seiner Stimme in das Haus der Simpsons zu locken. Dort sind sie erst mal vor den Städtern sicher und Bart darf wieder zur Schule gehen.

\notiz{
\begin{itemize}
	\item Der Autor von \glqq The Truth about Whacking Day\grqq\ (Die Wahrheit über den Knüppeltag) ist Bob Woodward, einer der Journalisten, welcher die Watergate-Affäre ins Rollen brachte.
	\item Der Knüppeltag findet am 10. Mai statt.
	\item Erster Auftritt von Oberschulrat Chalmers.
\end{itemize}
}

	
\subsection{Krusty, der TV-Star}\label{9F19}
Krusty, der allseits beliebte Fernseh-Clown, hat Konkurrenz bekommen. Gabbo\index{Gabbo} stiehlt Krusty zur selben Sendezeit auf einem anderen Kanal die Show. Als Krusty daraufhin seinen Job verliert, beschließen seine Fans Lisa und Bart Simpson, dagegen etwas zu unternehmen. Sie organisieren ein Prominenten-Special, um Krusty ein Comeback zu ermöglichen. Bart und Lisa beweisen, was man alles erreichen kann, wenn man etwas nur richtig will. Krusty ist gerührt von soviel Treue.

\notiz{
\begin{itemize}
  \item In der Krusty-Show heißt die neue Trickserie \glqq Arbeiter und Parasit\grqq .
  \item Luke Perry ist ein Halbbruder von Krusty.
  \item Dafür, dass Bart und Lisa Krusty zu seinem Comeback verholfen haben, bekommen sie 50 \% vom Gewinn des T-Shirt-Verkaufs.
\end{itemize}
}

	
\subsection{Marge wird verhaftet}\label{9F20}
Mehrere Bürger von Springfield, darunter auch Homer Simpson, bestellen sich einen ganz besonderen Ent\-safter aus Japan. Leider wird mit dem neuen Küchen\-gerät auch ein Virus eingeschleppt. Doch das ist nicht das Schlimmste. Etwas ganz und gar Undenkbares geschieht: Die gute Marge Simpson wird beim Ladendiebstahl erwischt. Sie hatte vergessen, eine Flasche Schnaps zu bezahlen. Sie wird zu 30 Tagen Gefängnis verurteilt und das kurz vor dem jährlichen Wohltätigkeitsjahrmarkt. Dort waren Marges Marshmallows immer die Hauptattraktion.

\notiz{
\begin{itemize}
  \item Der Moderator der Erfindershow sieht aus wie Troy McClure wird aber Kevin McClure\index{McClure!Kevin} genannt.
  \item Enthüllung durch Dr. Hibbert: Marge hat Schwimmhäute zwischen den Zehen.
  \item Einem Schild zufolge, das Snake passiert, als er die Stadt mit dem Kwik-E-Mart verlässt, ist Shelbyville 47 Meilen, North Haverbrook 63 und Mexico City 678 Meilen von Springfield entfernt.
  \item Die Inschrift auf dem Siegel des Bürgermeisters lautet: \glqq Corruptus in Extremis\grqq .
  \item Apu kennt die ersten 40 Nachkommastellen der Zahl $\pi$. Im englischen Original kennt Apu allerdings die ersten 40.000 Nachkommastellen von $\pi$.
  \item Ned Flanders sieht im Fernsehen eine Folge der \glqq Schrecklich netten Familie\grqq .
  \item Lionel Hutz hat den Sohn von Richter Synder mehrmals überfahren.
\end{itemize}
}

\section{Staffel 5}

\subsection{Homer und die Sangesbrüder}\label{9F21}
Lisa und Bart vertreiben sich die Zeit auf dem Flohmarkt. Dort entdecken sie zu ihrem Erstaunen eine Langspielplatte ihres Vaters, der vor acht Jahren (im Sommer 1985) mit seiner Band sehr erfolgreich war. Damals hatte Homer zusammen mit Rektor Skinner, Apu und Barney\footnote{Ursprünglich gehörte Chief Wiggum den Überspitzen\index{"Uberspitzen} an, er wurde durch Barney ersetzt.} als \glqq Barbershop\index{Barbershop}-Quartett\grqq\ den Grammy gewonnen. Doch die Zeiten des Musikerruhms sind längst passé. Das einzige, was den Kindern bleibt, ist sich zu wundern, wo das viele Geld, das ihr Vater damals verdient hat, geblieben ist.

\notiz{ 
\begin{itemize}
	\item Nigels Grund, warum Wiggum das Quartett verlassen muss: Viel zu Heavy Metal (im Original ist Wiggum \glqq Viel zu sehr Village People\grqq ).
	\item Das zweite Album der Überspitzen hieß \glqq Bigger Than Jesus\grqq\ (Größer als Jesus).
	\item 1985 hieß Moes Kneipe \glqq Moe's Cavern\grqq .
	\item Auf dem Dach von Moes Taverne sind die Graffiti \glqq FOR A GOOD TIME CALL EDNA KRABAPPLE! 555 6921\grqq\ und \glqq El Barto\grqq\ zu lesen. 
	\item Literaturzitat: Die Gefangenennummer 24601 entstammt aus dem Roman \glqq Die Elenden\grqq\ von Victor Hugo. Rektor Skinner trug während seiner Kriegsgefangenschaft zwei Jahre lang eine Gefangenenmaske mit dieser Nummer. Diese Nummer hatte ebenfalls Sideshow Bob in der Folge \glqq Bis dass der Tod euch scheidet\grqq\ (siehe \ref{8F20}) als Gefangenennummer.
	\item Fehler: An der Kirchentafel steht, dass die \glqq Be Sharps\grqq\ dort singen. Doch dieser Name wurde erst erfunden, als Barney schon zur Gruppe gehört hatte und hier singt noch Wiggum mit. 
\end{itemize}
}

	
\subsection{Am Kap der Angst}\label{9F22}
Tingeltangel-Bobs Entlassung aus dem Gefängnis steht kurz bevor. Aus seiner Zelle schreibt er Drohbriefe an Bart Simpson, der sich fast zu Tode ängstigt. Die Simpsons wenden sich in ihrer Not an das FBI, das sie in ihr Zeugenschutzprogramm aufnimmt, ihnen den neuen Namen \glqq Thompson\grqq\ \index{Thompson} gibt und sie in eine andere Stadt umsiedelt. Doch Tingeltangel-Bob macht sie in ihrem neuen Domizil, einem Hausboot, ausfindig. Er dringt in Barts Zimmer ein, um ihn zu ermorden.

\notiz{
\begin{itemize}
  \item Filmzitat: Diese Episode parodiert den Film \glqq Cape Fear\grqq\ und enthält Elemente aus \glqq Psycho\grqq\ (Tingeltangel-Bob wohnt am Terrorsee im Bates Motel).
  \item Der späte deutsche Erstausstrahlungstermin resultierte daraus, dass die Episode zuvor als nicht geeignet für Jugendliche eingestuft wurde. Später bekam sie aber eine FSK-6-Freigabe und wurde ausgestrahlt.
  \item Tingeltangel-Bob hat die Gefangenennummer 1211 und sein Zellenkamerad ist Snake.
  \item Tingeltangel-Bob hat auf der Brust \glqq Die Bart, Die\grqq\ (Stirb Bart, stirb) eintätowiert.
  \item \glqq Jemand, der deutsch spricht, kann kein schlechter Mensch sein,\grqq\ sagt ein Mitglied des Bewährungsausschusses.
\end{itemize}
}

	
\subsection{Kampf um Bobo}\label{1F01}
Mr. Burns ist schon früh von zu Hause weggegangen, um Milliardär zu werden. Nun hat er zwar sein Ziel erreicht, doch der steinreiche Mann wünscht sich nur eines: Seinen Teddy Bobo, den er damals zurücklassen musste. Mr. Burns lässt landesweit nach Bobo\index{Bobo} fahnden. Zufällig findet Bart den Teddy in einer Portion Gefriereis und schenkt ihn Maggie. Als Homer dahinter kommt, versucht er, das Geschäft seines Lebens zu machen. Doch er rechnet nicht mit Maggies gutem Herzen und das versaut Homer den großen Deal.

\notiz{
\begin{itemize}
  \item Filmzitat I: Die Eröffnungsszene mit Mr. Burns und die Szene, in der seine Schneekugel zerbricht, ist eine Parodie auf \glqq Citizen Kane\grqq .
  \item Filmzitat II: Die Wachen vor Burns Anwesen marschieren und singen wie die Wachen in \glqq The Wizard of Oz\grqq\ (Das zauberhafte Land).
  \item Mr. Burns hat am 15. September Geburtstag.
  \item Die ehemaligen US-Präsidenten Jimmy Carter und George Bush sen. werden nicht zur Geburtstagsfeier von Mr. Burns reingelassen.
\end{itemize}
}

	
\subsection{Homer an der Uni}\label{1F02}
Die Behörde für Atomsicherheit hält Homer Simpson für ein Sicherheitsrisiko und deshalb schickt ihn Mr. Burns auf die Universität. Doch anstatt etwas zu lernen, hat Homer nur studentischen Unsinn im Kopf. Seine Studienkollegen, Gary\index{Gary}, Doug\index{Doug} und Benjamin\index{Benjamin}, revanchieren sich bei ihm dafür, indem sie ihn per Computer zum Einser-Kandidaten machen. Doch Marge durchschaut das Spiel und bewegt ihren Mann dazu, den Kurs zu wiederholen. Homer muss in den sauren Apfel beißen.
Nebenbei verhindern die drei Kommilitonen, dass Lisa und Bart das Ende der Itchy \& Scratchy Folge sehen, in der Scratchy erstmals gewinnt.

\notiz{
\begin{itemize}
  \item Fernsehzitat I: Die Streber arbeiten im \glqq Room 222\grqq\ (Zimmer 222) -- der Titel einer beliebten US-Fernsehserie aus den 70er-Jahren über ein paar College-Studenten.
  \item Fernsehzitat II: Mr. Burns versucht die Agenten der Atomaufsicht zu überzeugen, die Waschmaschine und den Trockner zu nehmen, die neben Smithers stehen oder sie gegen eine geheimnisvolle Schachtel einzutauschen -- wie in der klassischen US-Spielshow \glqq Let's Make a Deal\grqq .
  \item In der Itchy \& Scratchy Folge \glqq Burning Down The Mouse\grqq\ wird offensichtlich Itchy von Scratchy besiegt.
  \item Die Springfielder Universität wurde 1952 gegründet.
  \item Diese Episode ist die Letzte, die Conan O'Brien geschrieben hat.
  \item Mr. Burns hat einen Lehrstuhl an der Springfielder Universität.
\end{itemize}
}

	
\subsection{Die rebellischen Weiber}\label{1F03}
Homer kann Marge nicht zu einem Ballettabend begleiten, weil seine Arme in einem Getränke- und einem Süßigkeitenautomaten steckengeblieben sind. Dies hat üble Konsequenzen: Marge nimmt Ruth Powers mit und die beiden freunden sich an. Dabei kommt heraus, dass Ruth den Wagen ihres Ex-Mannes gestohlen hat, weil dieser die Alimente für die gemeinsamen Kinder nicht bezahlen will. Als Chief Wiggum und Homer den gestohlenen Wagen entdecken, begehen die beiden Frauen Fahrerflucht und werden von den beiden verfolgt. Als die Verfolgungsjagd an einer Klippe endet, stürzen nicht die Frauen in eine Schlucht sondern die beiden Männer.

\notiz{
\begin{itemize}
  \item Radiozitat: Die Eröffnungsszene, in der ein Komiker auftritt, den die Simpsons nicht sehr lustig finden, parodiert Garrison Keillor\index{Keillor!Garrison} und seine aus Minnesota stammende Radio- und Fernsehshow \glqq A Prairie Home Companion\grqq .
  \item Die Summe, die beim Spendenmarathon zusammenkommt, beträgt nur \glqq 0,000,023.58\grqq\ -- also etwas mehr als 23 Dollar.
  \item Marges Begegnung mit Bürgermeister Quimby im Nachtclub \glqq The Hate Box\index{Hate Box}\grqq\ (Die Hass\-box) und seine Behauptung, dass er mit seinem Neffen dort war, sind eine Anspielung auf die Vergewaltigungsanschuldigung gegen den US-Senator Teddy Kennedy und seinen Neffen William Kennedy Smith.
  \item Ruth schiebt eine Kassette ins Autoradio, die \glqq Welcome to the Jungle\grqq\ von \glqq Guns N' Roses\grqq\ spielt.
  \item In dieser Episode benutzt Lionel Hutz auch den Namen Miguel Sanchez\index{Sanchez!Miguel}. Laut Kent Brockman verwendet Lionel Hutz auch noch den Namen Dr. Nguyen Van Phuoc\index{Van Phuoc!Dr. Nguyen}.
  \item Als Homer Lenny anruft, ist dieser zu sehen, als er einer Frau die Beine rasiert.
\end{itemize}
}

	
\subsection{Die Fahrt zur Hölle}
Wieder einmal steht der wichtige amerikanische Feiertag Halloween vor der Tür. Aus diesem Anlass unternimmt Bart einen kleinen Rundgang durch die Gemäldegalerie der Familie Simpson. Zu allen Bildern gibt es eine mehr oder weniger gruselige Geschichte: 
\begin{itemize}
	\item \textbf{Der Teufel und Homer Simpson}\\ Homer gibt seine Seele für einen einzigen Donut auf. Jetzt will der Teufel, bei dem es sich um Ned Flanders handelt, sich Homers Seele holen. Wie sich allerdings herausstellt, hatte Homer seine Seele schon vor langer Zeit seiner Frau Marge versprochen.
	\item \textbf{Terror in ein Meter sechzig Höhe}\\ Bart Simpson hat einen Albtraum von einem Busunglück und dieser Albtraum scheint wahr zu werden, als ein verspielter Gremlin damit beginnt, den Bus Stück für Stück auseinander zu nehmen.
   \item \textbf{Bart Simpson's Dracula}\\ Mr. Burns ist ein Vampir, er will Blut und er verwandelt Bart in einen Vampir. Wie sich zeigt, ist nicht Mr. Burns der Obervampir sondern Marge.
\end{itemize}

\notiz{
\begin{itemize}
  \item Die Geschworenenbank der Verdammten: Benedict Arnold, Lizzie Borden, Richard Nixon, John Wilkes Booth, Blackbeard der Pirat, John Dillinger und die Mannschaft der 1976er Philadelphia Flyers.
  \item Fernsehzitat: Die zweite Story parodiert die Folge \glqq Nightmare at 20.000 Feet\grqq\ aus der Serie \glqq Twilight Zone\grqq\ mit William Shatner. Sie spielt allerdings nicht in einem Schulbus sondern in einem Flugzeug.
  \item Filmzitat: Der Titel der dritten Geschichte (\glqq Bart Simpson's Dracula\grqq ) und der größte Teil der Handlung sind eine Parodie auf Bram Stokers\index{Stoker!Bram} \glqq Dracula\grqq\ bzw. die Film-Adaption von Francis Ford Coppola.
  \item Das Vorwort von Mr. Burns Buch \glqq Yes, I am a Vampire\grqq\ (Ja, ich bin ein Vampir) wurde von (dem amerikanischen TV-Moderator, Komiker und Autor) Steve Allen\index{Allen!Allen} geschrieben.
  \item Lisa schüttet absichtlich ein Glas mit Blut über sich und Bart, aber kurz nachdem sie sich vom Tisch weggeschlichen haben, sieht ihre Kleidung wieder sauber aus.
\end{itemize}
}


\subsection{Bart, das innere Ich}
Als Krusty ein Trampolin verschenkt, reißt sich Homer Simpson das Sportgerät sofort unter den Nagel. Pech für ihn, dass das Trampolin nur Unglück bringt. Doch das wäre nicht das Schlimmste: Weil Homer sich wieder einmal bei seiner Frau Marge beschwert, dass sie zu viel nörgelt, beschließt diese daraufhin, ein Selbstfindungsseminar bei Brad Goodman\index{Goodman!Brad} mitzumachen. Die Familie begleitet sie zu seinem Seminar in Springfield und Bart lässt dann einen folgenschweren Satz los: Er werde in Zukunft nur noch dasjenige tun, wonach ihm zumute ist. Von nun an versucht jeder, so wie Bart zu sein.

\notiz{
\begin{itemize}
  \item Filmzitat: Die Kinder, die sich beim Trampolinspringen verletzt haben, liegen in einer Reihe wie die verletzten Soldaten in \glqq Vom Winde verweht\grqq .
  \item Krustys Adresse: 534 Center Street.
  \item In der Gratis-Sektion des Springfield Shopper steht die Anzeige \glqq Fat Boy Bomb Free!!! Call Herman K19-4327\grqq\ (Fat Boy-Atombombe gratis!). Herman hat auch in \glqq Homer und die Sangesbrüder\grqq\ (siehe \ref{9F21}) und in \glqq Die Springfield Bürgerwehr\grqq\ (siehe \ref{1F09}) Atombomben zum Verkauf angeboten.
\end{itemize}
}

	
\subsection{Auf Wildwasserfahrt}\label{1F06}
Bart Simpson ist nun doch bei den Jungpfadfindern gelandet. Als eine Vater-Sohn-Wild\-was\-ser\-fahrt veranstaltet wird, muss auch Homer Simpson daran teilnehmen. Vater und Sohn Simpsons sitzen mit Nachbar Flanders und dessen Sohn Rod in einem Boot. Zu ihrem großen Pech werden die vier schließlich aufs offene Meer hinausgetrieben. Dem Verhungern und Verdursten nahe landen sie schließlich auf einer Bohrinsel und dort wartet schon ein Krusty-Burger-Schnellimbiss auf sie.

\notiz{
\begin{itemize}
  \item Als Lisa sagt, dass Trickserien nicht immer realistisch sein müssen, sieht man Homer am Fenster vorbeigehen, obwohl er auf der Couch neben Lisa sitzt.
  \item Beim Baden der alten Leute von den Pfadfindern ist Jasper derjenige, der gebadet werden soll.
  \item Im Polizeirevier ist ein Fahndungsaufruf nach Fat Tony und Mrs. Botz, der Babysitterin aus \glqq Der Babysitter ist los\grqq\ (siehe \ref{7G01}) zu sehen.
  \item Martin Prince ist zu sehen, wie er das Videospiel \glqq My Dinner with André\grqq\ spielt.
\end{itemize}
}

	
\subsection{Homer liebt Mindy}\label{1F07}
Kraftwerksbesitzer Mr. Burns sucht eine weibliche Angestellte. Als sich die hübsche Mindy Simmons\index{Simmons!Mindy} um die Stelle bewirbt, stellt Mr. Burns sie sofort ein. Schon bald darauf fühlt sich Homer Simpson von Mindy ständig in Versuchung geführt. Als er zusammen mit Mindy nach Capital City zu einem Kongress geschickt wird, wird es gefährlich: Mindy hat dieselben Vorlieben wie er: Essen, Bier und Fernsehen. Schließlich fällt Homer ein Ausweg aus der Situation ein: Er lässt Marge nachkommen.

\notiz{
\begin{itemize}
  \item Homer und Mindy werden Energie-König und Energie-Königin auf der Energiemesse.
  \item Außerdem gewinnen beide ein romantisches Abendessen im Restaurant \glqq Madame Chao's\index{Madame Chao's}\grqq .
  \item \glqq Die Geburt der Venus\grqq\ Gemälde von Sandro Botticelli (1482) -- Homers Vision von Mindy in der Muschelschale.
\end{itemize}
}

	
\subsection{Vom Teufel besessen}\label{1F08}
Die Stadtregierung von Springfield hat auf Vorschlag von Seymour Skinner einen folgenschweren Entschluss gefasst: Man will endlich das Glücksspiel legalisieren. Mr. Burns lässt sich die Chance, Geld zu verdienen, nicht entgehen und lässt daraufhin ein riesiges Casino errichten. Es dauert nicht lange und die Bevölkerung von Springfield ist vom Spielteufel besessen. Selbst Marge Simpson ist so gut wie nie mehr zu Hause. Sie steht nur noch am einarmigen Banditen und spielt. Schließlich platzt Homer der Kragen.

\notiz{
\begin{itemize}
  \item Filmzitat: Die Szene am Black-Jack-Tisch mit Homer und einem autistischen Spieler parodiert \glqq Rain Man\grqq .
  \item Die Pornofilme \glqq Sperms on Endearment\grqq\ (Liebevolles Sperma) und \glqq I'll Do Anyone\grqq\ (Ich mach's mit jedem) sind Parodien auf Filme, die Simp\-sons-Pro\-du\-zent James L. Brooks produziert hat: \glqq Terms of Endeament\grqq\ (Zeit der Zärtlichkeit) und \glqq I'll Do Anything\grqq\ (Geht's hier nach Hollywood?).
  \item In der Schule verkleidet sich Lisa als Florida (allerdings als Floreda\index{Floreda} geschrieben) und Ralph als Idaho.
  \item Der reiche Texaner mit dem großen Cowboyhut ist Senator.
  \item Der Fichtenelch, mit dem Burns in das Kraftwerk fliegen will und sein Verfolgungswahn sind eine Anspielung auf Howard Hughes\index{Hughes!Howard}\footnote{Howard Robard Hughes (geb. 24. Dezember 1905 in Houston, Texas; gest. 5. April 1976 in einem Flugzeug über Texas) war ein US-amerikanischer Unternehmer. Er war der Haupterbe der einträglichen Hughes Tool Company, Filmproduzent und Luftfahrtpionier \cite{HowardHughes}.}, der 1944 das bisher größte jemals gebaute Flugzeug entwarf.
\end{itemize}
}

	
\subsection{Die Springfield Bürgerwehr}\label{1F09}
Das Städtchen Springfield zittert vor dem sogenannten \glqq Katzeneinbrecher\grqq . Auch die Familie Simpson ist davon betroffen, denn er hat Lisas Saxophon, Bart tragbaren Fernseher und Marges Halskette gestohlen. Deshalb versucht Homer, eine Bürgerwehr zu organisieren. Deren Mitglieder hoffen, dem Einbrecher dadurch das Handwerk legen zu können. Doch als dies nicht gelingt, wendet sich die Bevölkerung gegen ihren Begründer Homer Simpson. Aber dann erhält dieser unerwartete Schützenhilfe von Grandpa: Als der Täter entlarvt wird, ist ganz Springfield überrascht.

\notiz{
\begin{itemize}
  \item Es hängen noch Bilder von Mrs. Botz\index{Botz} und Fat Tony an der Wand der Polizeistation.
  \item Neben Homer gehören noch Barney, Apu, Skinner und Moe der Springfielder Bürgerwehr an.
  \item Jimbo sprüht \glqq Carpe Diem\grqq\ (Nutze den Augenblick) an eine Wand.
  \item Barney macht sich mit Patty und Selma in einem Flugzeug auf den Weg zum versteckten Geld.
  \item Homer sagt, er sei 36 Jahre alt.
  \item Filmzitat: Molloy\index{Molloy} schickt die Springfielder Bürger auf eine Jagd nach einem Schatz, der unter einem großen Buchstaben vergraben ist, wie in dem Film \glqq It's a Mad Mad Mad Mad World\grqq\ (Eine total, total verrückte Welt). Am Schluss gibt es viele Anspielungen auf den starbesetzten Film: Die Musik, die verrückte Suche nach dem Geld und die Anspielung auf die Szene, in der Phil Silvers\index{Silvers!Phil} mit seinem Cabriolet in einen Fluss fährt.
  \item Fehler: Chief Wiggum sagt, die Simpson wohnen in 732 Evergreen Terrace.
\end{itemize}
}

	
\subsection{Apu der Inder}\label{1F10}
Als Apu, der Inder, verdorbenes Fleisch an die Simpsons verkauft, landet Homer mit einer schweren Lebensmittelvergiftung im Krankenhaus. Dies führt dazu, dass Apu seinen Job verliert und entlassen wird. Doch Apu will seine Tat wieder gutmachen: Er bietet den Simpsons an, sich ihnen als Diener zur Verfügung zu stellen. Klar, dass Familie Simpson darüber mehr als begeistert ist und dieses verlockende Angebot nicht ausschlägt.

\notiz{
\begin{itemize}
  \item Der nachdatierte Schinken, den Homer isst, ist ursprünglich im Februar 1989 abgelaufen.
  \item Kent Brockman moderiert nicht nur die Nachrichten sondern auch die Verbrauchersendung \glqq Bite Back\grqq\ (Beiße zurück). Diese ist ähnlich der Sendung von David Horowitz \glqq Fight Back\grqq .
	\item Der Kwik-E-Mart akzeptiert keine Schecks von folgenden Personen: Chief Wiggum, Reverend Lovejoy, Homer J. Simpson, Homer S. Simpson, H. J. Simpson, Homer Simpson, Homer J. Fong.
	\item Simpsons-Autor Jeff Martin gab Apu den Nachnamen \glqq Nahasapeemapetilon\grqq , da einer seiner Schulfreunde so heiß.
\end{itemize}
}

	
\subsection{Bart wird berühmt}\label{1F11}
Nach der Schule arbeitet Bart als Assistent von Krusty dem Clowns. Doch dann passiert Bart ein Missgeschick: In einer der Shows stolpert er und bringt dadurch eine ganze Kulisse zum Einsturz. Als er danach aus lauter Verzweiflung sagt: \glqq Ich hab nichts gemacht!\grqq , wird Bart schlagartig berühmt. Der Ausspruch wird zum geflügelten Wort -- Bart macht eine sensationelle Karriere. Doch leider ist diese genauso schnell beendet, wie sie begonnen hat.

\notiz{
\begin{itemize}
  \item Die Melodie, die Bart pfeift und die Marge so schrecklich findet, ist die Titelmusik der Simpsons.
  \item Bart tritt in der Late Night Show von Conan O'Brien auf. Conan O'Brien schrieb, bevor er eine eigene Talkshow hatte, mehrere Folgen für die Simpsons.
  \item Im Aufnahmestudio, in dem Bart seinen Rap aufnimmt, hängen goldene Schallplatten von Zahnfleischbluter Murphy und der Larry Davis Experience.
  \item Sideshow Mel ist allergisch gegen Milchfett.
\end{itemize}
}

	
\subsection{Lisa kontra Malibu Stacy}\label{1F12}
Lisa ärgert sich sehr, weil die sprechende \glqq Malibu Stacy\index{Malibu Stacy}\index{Stacy!Malibu}\grqq -Puppe nur blödsinnige und doofe Sätze von sich gibt. Daher beschließt sie, eine neue, intelligentere Puppe mithilfe der ursprünglichen Schöpferin von Malibu Stacy, Stacy Lovell\index{Lovell!Stacy}, auf den Markt zu bringen: Es entsteht ein ent\-zück\-en\-des Geschöpf, das mit Lisas Stimme lauter vernünftige Lebensweisheiten von sich gibt. Der Name der kompetenten Lebensberaterin ist Lisa Löwenherz\index{Löwenherz!Lisa}. Lisas Puppe wird zum Verkaufsschlager, jedoch nicht allzu lange, denn die Konkurrenz schläft nicht!

\notiz{
\begin{itemize}
  \item Filmzitat I: Das \glqq We Love You, Matlock\grqq -Lied parodiert \glqq We Love You, Conrad\grqq\ aus dem Film \glqq Bye Bye Birdie\grqq .
  \item Filmzitat II: Homers musikalische Einlage auf dem riesigen Keyboard im Spielzeugladen erinnert an Tom Hanks in dem Film \glqq Big\grqq .
  \item Smithers hat die weltgrößte Malibu-Stacy-Sammlung.
  \item Waylon Smithers wohnt in einem Appartement mit der Nummer 19.
\end{itemize}
}

	
\subsection{Homer der Weltraumheld}\label{1F13}
Homer wird eine große Ehre zuteil: Die NASA beschließt, ihn mit noch zwei anderen Astronauten in einer Raumfähre ins Weltall zu schicken, um dort Forschungen zu betreiben. Ursprünglich sollte eigentlich Barney den Flug antreten, aber nachdem dieser nach dem Genuss von alkoholfreien Sekt nicht mehr in der Lage dazu ist, wird Homer ins All geschickt. Dass im All andere Gesetzmäßigkeiten herrschen, muss Homer sehr bald feststellen: Als er ahnungslos eine Tüte Kartoffelchips öffnet, kommt es in dem Raumschiff zu Komplikationen. Da schwingt sich Homer zum Weltraumhelden auf und kann gerade noch im letzten Moment eine Katastrophe verhindern.

\notiz{
\begin{itemize}
  \item Serienzitate: Als die Weltraumbehörde NASA in der Besprechung über das Fernsehen reden, laufen im TV gerade Ausschnitte aus \glqq Hör mal wer da hämmert\grqq\ (Tim überrollt Wilson mit einem kleinen Traktor zum Rasenmähen) und \glqq Eine schrecklich nette Familie\grqq\ (Al und Peggy sitzen am Sofa und Al drückt die Klospülung am Klo, welches neben ihm steht).
  \item Filmzitat: Mehrere Szenen sind aus dem Film \glqq The right stuff\grqq\ (Der Stoff aus dem die Helden sind) übernommen wurden, u.\,a. Homers und Barneys Training, sowie der Gesang der Mannschaft beim Wiedereintritt in die Atmosphäre.
  \item Homer kennt die Telefonnummer der NASA und von Präsident Clinton.
  \item Der Name des Spaceshuttles lautet \glqq Corvair\grqq . Der Chevrolet Corvair aus den 60er-Jahren war als einer der unsichersten Wagen bekannt.
  \item Bei der Wahl zum \glqq Mitarbeiter des Monats\grqq\ ist Homer sauer, weil er noch nie diesen Titel inne haben durfte. Das stimmt aber nicht, denn in der Folge \glqq Der Ernstfall\grqq\ (siehe \ref{8F04}) wurde er, nachdem er die Kernschmelze verhindern konnte, \glqq Mitarbeiter des Monats\grqq .
\end{itemize}
}
	
\subsection{Homie und Neddie}
Nachbar Flanders ergattert zwei der überaus begehrten Football-Karten. Als Flanders Homer Simpson einlädt, ist die Feindschaft zwischen den Nachbarn endgültig beendet. Von nun an verbringen sie jede freie Minute gemeinsam und das wird selbst Flanders schon bald zu viel. Er versucht, den lästigen Homer loszuwerden, doch das ist nicht so einfach. Vor allem, als Homer schließlich in der Kirche vor allen Gläubigen erklärt, es gäbe keinen besseren Menschen als Ned Flanders.

\notiz{
\begin{itemize}
  \item Filmzitat I: \glqq Na, wo bleibt denn Ihr Messias, Flanders?\grqq\ parodiert ein Zitat aus \glqq The Ten Commandments\grqq\ (Die zehn Gebote).
  \item Filmzitat II: Homer jagt mit Golfschlägern hinter Flanders Auto her und hakt sich darin fest wie in \glqq Terminator 2\grqq .
  \item Flanders Nummernschild lautet \glqq JHN 143\grqq . Das könnte für Johannes 14:3\footnote{Und wenn ich hingehe, euch die Stätte zu bereiten, will ich wiederkommen und euch zu mir nehmen, damit ihr seid, wo ich bin \cite{Bibel}.} aus dem Neuen Testament stehen.
  \item Da Vincis Bild \glqq Das letzte Abendmahl\grqq\ hängt in Flanders Zimmer.
  \item Moe liest im Krankenhaus kranken Kindern vor und Mittwochs hat seine Kneipe geschlossen, weil er den Obdachlosen Geschichten vorliest.
  \item Die Springfield Atoms schlagen die Shelbyville Sharks beim Football in Shelbyville mit 27 zu 24.
  \item Stan Taylor\index{Taylor!Stan} ist der Quarterback der Springfield Atoms.
\end{itemize}
}

	
\subsection{Bart gewinnt Elefant!}\label{1F15}
Bart hat von dem Radiosender \glqq KBBL\index{KBBL}\grqq\ einen Elefanten gewonnen. Als ihm der Sender den Gewinn in Geld (\$ 10.000) auszahlen will, besteht Bart auf dem Elefanten. Das führt nicht nur zu einigem Durcheinander in Springfield, sondern auch zur völligen Ebbe in der Haushaltskasse der Simpsons: Der Elefant frisst ihnen buchstäblich die Haare vom Kopf. Vater Homer braucht schnell eine gute Idee, wie er seinen Sprössling umstimmen kann. Doch Bart besteht auf seinem neuen Schmusetier -- zunächst jedenfalls.

\notiz{
\begin{itemize}
	\item Filmzitat: Stampfi\index{Stampfi} sieht neugierig direkt in das Wohnzimmerfenster der Simpsons, so wie der T-Rex in \glqq Jurassic Park\grqq\ in den Jeep geschaut hat.
	\item Erster Auftritt von Cletus Del Roy Spuckler.
	\item Stampfi ist wieder in der Folge \glqq Marge -- oben ohne\grqq\ (siehe \ref{DABF18}) zu sehen.
	\item Fehler: Als Stampfi im Garten der Simpsons angekettet wird, tritt er nacheinander auf einen Rasenmäher, ein Fahrrad, die Hundehütte und eine Mülltonne. Wenige Augenblicke später ist die Hundehütte wieder ganz.
\end{itemize}
}

	
\subsection{Burns Erbe}\label{1F16}
Bart Simpson hat das große Los gezogen: Der steinreiche, aber kinderlose Mr. Burns sucht einen Stammhalter und Erben. Er lässt alle Kinder der Stadt bei sich vorsprechen. Und gerade Bart Simpson, in seiner unverblümten Art, gewinnt das Herz des alten Mannes, als er Fensterscheiben seines Anwesen mit Steinen einwirft. Geld, Gold und ein sorgenfreies Leben vor Augen, kehrt Bart seiner Familie den Rücken und zieht in Mr. Burns Anwesen. Doch als dieser von Bart verlangt, seinen Vater zu entlassen, gehen Bart die Augen auf.

\notiz{
\begin{itemize}
  \item Literaturzitat: Ein Junge mit Londoner Akzent sagt zu Burns \glqq Aber wieso, heute ist Weihnachten\grqq . Dies parodiert eine Szene mit Ebeneezer Scrooge\index{Scrooge!Ebeneezer} in Charles Dickens klassischer Weihnachtsgeschichte \glqq A Christmas Carol\grqq\ (Ein Weihnachtsmärchen).
    \item Krusty hat 1982 in seiner Show von einem Zettel abgelesen, obwohl er angeblich Analphabet ist.
  \item Filmzitat I: Die Szene, in der Moe vor dem Spiegel ein Selbstgespräch führt, ist eine Anspielung auf den Film \glqq Taxi Driver\grqq .
  \item Filmzitat II: Die Gehirnwäsche von Hans Maulwurf wird in einem Zimmer mit der Nummer 101 durchgeführt genau wie im Film \glqq 1984\grqq .
  \item Edna ist mit Jacques\index{Jacques} aus der Folge \glqq Der schöne Jacques\grqq\ (siehe \ref{7G11}) beim Knutschen zu sehen.
  \item Als Bart bei Mr. Burns ist, ersetzt Hans Maulwurf Bart bei den Simpsons.
  \item Mr. Burns fordert von Lenny, ihm zum begründen, warum er ihn nicht entlassen sollte, ohne den Buchstaben \glqq e\grqq\ zu verwenden. Dies könnte eine Anspielung auf die Novelle \glqq Gadsby\grqq\ von Ernest Wright sein. Denn diese enthält 50.000 Wörter. Aber keines der verwendeten Wörter enthält den Buchstaben \glqq e\grqq .
\end{itemize}
}
	
\subsection{Freund oder Feind!}\label{1F18}
Bart hat für ein Referat seinen Hund Knecht Ruprecht in die Schule mitgenommen. Als dieser später in den Lüftungsschacht krabbelt und ein Chaos verursacht, wird Rektor Skinner von Oberschulrat Chalmers entlassen. Ned Flanders wird Skinners Nachfolger und Skinner geht wieder zur Armee zurück. Dennoch gelingt es Bart, Skinner wieder zur Rückkehr an die Grundschule zu bewegen und schließlich wird Ned Flanders als Rektor gefeuert, weil er morgens zu einem Gebet aufruft.

\notiz{
\begin{itemize}
  \item Filmzitat: Die Szene, in der Willie in den Be\-lüft\-ungs\-schäch\-ten nach Barts Hund, Knecht Ru\-precht, sucht und von Skinner auf dem Radarmonitor beobachtet wird, ist eine Parodie auf den Film \glqq Alien\grqq .
  \item Mrs. Krabappels Satz \glqq Er denkt, er sei ein Mensch\grqq , nachdem Knecht Ruprecht geniest hat, hat Bart schon mal in \glqq Bart gewinnt Elefant\grqq\ (siehe \ref{1F15}) gesagt, als der Elefant sich am Haus der Simpsons gerieben hat.
  \item Ned Flanders ist Vorsitzender der Lehrervereinigung.
  \item Fehler: Janey ist in Barts Klasse zu sehen, obwohl sie eine Klassenkameradin von Lisa ist.
\end{itemize}
}

	
\subsection{Bart packt aus}\label{1F19}
Kellner LaCoste sieht schlimm aus: Angeblich ist er von Freddy Quimby\index{Quimby!Freddy}, dem Neffen des Bürgermeisters, verprügelt worden. Da es keinerlei Zeugen gibt, sieht es ziemlich schlecht aus für Freddy. Alle Welt glaubt dem arg zugerichteten Kellner. Doch in Wirklichkeit ist LaCoste\index{LaCoste} nur ausgerutscht und unglücklich gefallen. Bart, der an diesem Tag die Schule geschwänzt hatte, war heimlich Zeuge gewesen. Nun sitzt er in der Zwickmühle: Entlastet er Freddy, kommt auf, dass er geschwänzt hat.

\notiz{
\begin{itemize}
  \item In dieser Episode stört Apu zum zweiten Mal Skinners Begeisterung. Das erste Mal war in \glqq Freund oder Feind\grqq\ (siehe \ref{1F18}), als Apu Skinner mitteilt, dass sein Roman bloß ein Abklatsch von \glqq Jurassic Park\grqq\ ist.
  \item Der Gerichtszeichner ist Matt Groening\index{Groening!Matt}.
  \item Die Szene, in der Rektor Skinner Bart durch den Fluss hindurch verfolgt, parodiert \glqq Terminator\grqq .
  \item Zu den Geschworenen im Prozess gegen Freddy Quimby gehören u.\,a. Homer Simpson, Apu Nahasapeemapetilon, Ned Flanders, Hans Maulwurf, Seymour Skinner, Jasper und Selma Bouvier.
  \item Apu ist Geschworener bei Gericht. Er wird aber erst in der Episode \glqq Volksabstimmung in Springfield\grqq\ (siehe \ref{3F20}) amerikanischer Staatsbürger, vorher kann er kein Geschworener sein.
  \item Das Ergebnis der Überprüfung, ob Barts Entschuldigungsschreiben echt ist, wird auf einem Lochkartendrucker ausgegeben.
  \item Laut Kent Brockman ist in dem Bundesstaat, in dem Springfield liegt, die Live-Übertragung aus Gerichtssälen nicht erlaubt.
\end{itemize}
}
	
\subsection{Ehegeheimnisse}\label{1F20}
Homer unterrichtet an der Abendschule als Fachmann für Ehefragen. Leider lässt er sich hinreißen und plaudert aus dem Privatleben. Marge ist stocksauer, kennt doch die ganze Stadt nun das Intimleben der Simpsons. Sie setzt Homer vor die Tür. Selbstverständlich hält diese Strafe nicht lange vor. Am Ende ist sie recht froh, dass sie Homer wieder liebevoll in die Arme schließen kann.

\notiz{
\begin{itemize}
  \item Der Pokerrunde, die sich in Lennys Haus treffen, gehören Lenny, Homer, Carl, Barney und Moe an.
  \item Waylon Smithers gesteht, dass er einmal verheiratet war.
  \item Homers Kursteilnehmer sind u.\,a. Barney, Willie, Otto, Waylon Smithers, Sideshow Mel, Rektor Skinner, Apu, Carl, Lionel Hutz, Edna Krabappel, Bernice Hibbert, Prinzessin Kashmir und Moe.
  \item Fehler: Beim Pokern hat Homer einen Straight-Flush\footnote{zweitbeste Hand beim Pokern} aber Moe sagt: \glqq Du hast ja einen Royal Flush\footnote{beste Hand beim Pokern} Homer\grqq .
  \item Folgende Unterrichte werden u.\,a. in der Abendschule angeboten:
\begin{itemize}
	\item \glqq Turn A Man Into Putty In Your Hands\grqq\ von Patty und Selma,
	\item \glqq Funk Dancing For Self-Defense\grqq\ von Moe,
	\item \glqq How To Chew Tobacco\grqq\ von Lenny,
	\item \glqq How To Eat An Orange\grqq\ von Hans Maulwurf und
	\item \glqq Secrets of a Successful Marriage\grqq\ von Homer Simpson.
\end{itemize}
\end{itemize}
}

	
\subsection{Liebhaber der Lady B.}\label{1F21}
Marges Mutter und Homers Vater finden sich gegenseitig ziemlich sympathisch. Bald gehen ihre Gefühle über die Sympathie hinaus. Auch Marge und Homer würden es gern sehen, wenn Mrs. Bouvier und Grandpa den Bund der Ehe eingingen. Aber plötzlich macht Mr. Burns Mrs. Bouvier den Hof. Schließlich treten die beiden vor den Altar. Da erscheint plötzlich Grandpa. Wie weiland der junge Dustin Hoffman in \glqq Reifeprüfung\grqq\ brennt der alte Abe mit der lieblichen Braut durch.

\notiz{
\begin{itemize}
	\item Filmzitat: Die Hochzeitszeremonie in der First Church of Springfield zwischen Mr. Burns und Mrs. Bouvier ist gerade in vollem Gange, als Grandpa aus dem verglasten Orgelzimmer etwas ruft. Grandpa läuft mit Mrs. Bouvier dann nach draußen in einen Bus, dazu läuft eine umgetextete Version des Songs \glqq The Sound of Silence\grqq\ aus dem Film \glqq The Graduate\grqq\ (Die Reifeprüfung), der hier auch parodiert wird.
	\item Troy McClure ist auf dem Fernsehsender IBN\index{IBN} (Impulse Buying Network) zu sehen.
	\item Milhouse hat in dieser Episode die Windpocken.
	\item Marges Mutter wohnt in den Hal Roach Apartments.
\end{itemize}
}

\section{Staffel 6}

\subsection{Lisas Rivalin}\label{1F17}
Lisa hat eine neue Schulkameradin: Allison Taylor\index{Taylor!Allison} ist in allen Fächern besser als Lisa. Das macht den Umgang miteinander schwierig. Einerseits möchte sich Lisa gerne mit Allison anfreunden, andererseits sinnt sie auf Rache wegen der Zurücksetzung durch Allison. In der Schule wird ein Diorama-Wettbewerb veranstaltet. Lisa nimmt sich fest vor, bei dieser Gelegenheit die Neue endlich zu schlagen. Dabei schreckt sie auch vor äußerst unfairen Mitteln nicht zurück, trotzdem gewinnt sie den Wettbewerb nicht sondern Ralph Wiggum. Unterdessen glaubt Homer, das große Geschäft zu machen, als er aus einem verunglückten LKW Zucker klaut. 

\notiz{
\begin{itemize}
	\item Nachdem Milhouse vom Staudamm gesprungen ist und  \glqq Mei\-ne Bril\-le!\grqq\ gerufen hat, sind die Brillengläser beschädigt und mit Klebestreifen zusammengeklebt.
	\item Hans Maulwurf war der Fahrer des verunglückten Zuckerlasters.
	\item Bart liest die Zeitschrift \glqq Bad Boy's Life\index{Bad Boy's Life}\grqq .
	\item Allison hat ein Autogramm von Zahnfleischbluter Murphy in ihrem Zimmer hängen.
\end{itemize}
}

	
\subsection{Ein grausiger Verdacht}\label{1F22}
Die Simpsons haben einen neuen Swimmingpool. Plötzlich sind Bart und Lisa die beliebtesten Kinder der Stadt. Beim Spielen hat Bart etwas Pech. Er bricht sich ein Bein und muss das Bett hüten. Von seinem Fenster aus beobachtet er seinen Nachbarn Flanders, der etwas in seinem Garten vergräbt. Und plötzlich fällt Bart auf, dass Mrs. Flanders seit geraumer Zeit verschwunden ist. Ein grausiger Verdacht keimt in ihm: Haben die Flanders ein Familienmitglied beseitigt?

\notiz{
\begin{itemize}
  \item Filmzitat: Verschieden Elemente der Geschichte um den \glqq Mord\grqq , u.\,a. die Musik, parodieren Hitchcocks \glqq Rear Window\grqq\ (Das Fenster zum Hof).
  \item An der Wand von Dr. Hibberts Trainingsraum (durch Barts Teleskop gesehen) hängen Familienportraits, auf denen alle weiß sind.
  \item Der Originaltitel der Episode \glqq Bart Of Darkness\grqq\ ist eine Anspielung auf Joseph Conrads Kurzgeschichte \glqq Heart of Darkness\grqq .
  \item Eddie, der Polizist, fliegt den Polizeihubschrauber.
\end{itemize}
}
	
\subsection{Der unheimliche Vergnügungspark}\label{2F01}
Heute machen die Simpsons einen Ausflug ins Itchy \& Scratchy-Land, den be\-rühm\-ten Hor\-ror-Ver\-gnügungspark. Plötzlich geschieht Unglaubliches: Die Roboter des Vergnügungsparks greifen die Besucher an. Eine Panik bricht aus. Durch einen Zufall kommt Familie Simpson dahinter, wie man die wild gewordenen Roboter ausschalten kann. Im letzten Augenblick können sie das Schlimmste verhindern. Die Simpsons werden die Helden des Vergnügungsparks.

\notiz{
\begin{itemize}
  \item Filmzitat I: Die Familie ist von Amok laufenden Robotern in einem Ver\-gnüg\-ungs\-park umzingelt, wie in \glqq Westworld\grqq .
  \item Filmzitat II: Frinks Kommentar über die Chaostheorie und das Zeichen auf den Parkfahrzeugen sind eine Anspielung auf \glqq Jurassic Park\grqq .
  \item Filmzitat III: Die Vogelattacke auf Hans Maulwurf in der Telefonzelle ist eine Parodie auf \glqq Die Vögel\grqq .
  \item Filmzitat IV: Der Pinitchio-Film\index{Pinitchio} ist eine Anspielung auf Pinocchio.
  \item An Itchys 70er Disco hängt ein kleines Schild \glqq Est. 1980\grqq\ (Gegr. 1980).
\end{itemize}
}
	
\subsection{Tingeltangel-Bob}\label{2F02}
Es steht die Bürgermeisterwahl in Springfield an. Bürgermeister Quimby will seinen Großmut beweisen: Er erlässt dem Schwerverbrecher Tingeltangel-Bob die restliche Haftzeit. Womit Quimby allerdings kaum gerechnet haben dürfte: Tingeltangel-Bob stellt sich anschließend selbst als republikanischer Kandidat zur Wahl und gewinnt. Er wird der neue Bürgermeister. Bart und Lisa kommt dies nicht ganz geheuer vor. Sie entdecken, dass alle Stimmen für Tingeltangel-Bob von Bürgern stammen, die schon lange tot sind. Ebenso erhält er Stimmen von verstorbenen Tieren.

\notiz{
\begin{itemize}
	\item Die Führungsspitze der republikanischen Partei in Springfield: Mr. Burns, Birch Barlow\index{Barlow!Birch}, ein grüner, vampirartiger Typ, der blauhaarige Anwalt (eingeführt in \glqq Bart kommt unter die Räder\grqq, \ref{7F10}), Dr. Hibbert, Rainier Wolfcastle und der texanische Millionär.
	\item Neben Tingeltangel-Bob hat Bart auch noch Dr. Demento\index{Demento!Dr.} zum Todfeind.
	\item In dieser Episode erfährt man, dass Tingeltangel-Bob in Yale studiert hat.
	\item Joe Quimby befindet sich ein seiner sechsten Legislaturperiode.
\end{itemize}
}
	
\subsection{Furcht und Grauen ohne Ende}\label{2F03}
In einer Parodie auf die \glqq Outer Limits\grqq , präsentieren die Simpsons drei Horrorgeschichten:
\begin{itemize}
	\item \textbf{Das Shining}\\ Die Familie Simpson arbeitet in einem entlegenen Anwesen von Mr. Burns als dessen Hausangestellte. Mr. Burns hat eine einfache Methode entwickelt, um sie zum Arbeiten zu zwingen: Er schaltet das Fernsehen ab und schafft sämtliche Biervorräte aus dem Haus. Homer Simpson dreht dabei durch und will schließlich seine Familie umbringen, nur Barts \glqq Shining\grqq\ kann die Familie retten.
	\item \textbf{Zeit und Strafe}\\ Ned Flanders wird der Führer einer George Orwellschen Welt, nachdem Homer mit einem Toaster die Vergangenheit veränderte.
	\item \textbf{Nightmare Cafeteria}\\ Die Überfüllung des Arrest-Raums in der Schule und Budget-Kürzungen in der Cafeteria führen dazu, dass Rektor Skinner eine einzigartige Lösung einfällt, er verarbeitet Schüler zu Essen. Es stellt sich allerdings heraus, dass alles nur ein Traum von Bart war. Jedoch wird die Haut der Simpsons am Ende von einem unnormalen Nebel zerfressen.
\end{itemize}

\notiz{
\begin{itemize}
  \item Running Gag: Willie kriegt in allen drei Geschichten dieser Folge eine Axt in den Rücken.
  \item Filmzitat: Die Story parodiert Stephen Kings Buch \glqq The Shining\grqq\ bzw. Stanley Kubricks Verfilmung von dem Blut, das aus dem Fahrstuhl läuft, bis hin zu einem wahnsinnigen Homer, der \glqq Hiiiiiiiiiiiiiiiier ist Johnny!\grqq\ ruft.
  \item Die Nachricht, die in dem dunklen Raum, den Marge entdeckt, über alle Wände geschrieben steht: \glqq Ohne Fernsehen und ohne Bier dreht Homer durch\grqq .
  \item Maggie legt mit ihren Buchstabenwürfel die Worte \glqq Red Rum\grqq\ (Roter Rum).
  \item Im Polizeirevier hängen Steckbriefe von Sideshow Bob, Fat Tony und Snake.
  \item TV- und Literaturzitat: Der Junge und der Hund, die Homer im Zeitstrudel korrigieren, sind Sherman\index{Sherman} und Mr. Peabody\index{Peabody} aus \glqq Peabody's Improbable History\grqq\ (Peabodys unwahrscheinliche Geschichte), einer Zeichentrickreihe, die regelmäßig in der \glqq Rocky and Bullwinkle Show\grqq\ vorkam. Der Hund, Mr. Peabody, reist mit seinem Besitzer Sherman stets durch die Zeit und zitiert wichtige historische Ereignisse.
  \item Homer bezeichnet sich selbst als den ersten \glqq Nicht-Bra\-si\-li\-an\-er, der rück\-wärts durch die Zeit reist\grqq . Damit spielt er auf den brasilianischen Autor Carlos Castaneda an, einer der Väter der \glqq New Age\grqq -Bewegung, der sich durch Drogen inspirieren ließ.
  \item Das Wort \glqq Oktoberfest\grqq\ ist so geschrieben: \glqq O\"{K}T"OB"ERF"ES\"{T}\grqq .
\end{itemize}
}
	
\subsection{Barts Freundin}\label{2F04}
Bart verliebt sich in Jessica\index{Lovejoy!Jessica}. Die Tochter von Reverend Lovejoy ist überaus wohlerzogen und sittsam. Natürlich lässt die brave Jessica den Simpson-Rowdy erst mal abblitzen. Umso größer ist dann aber Barts Überraschung: Es stellt sich nämlich heraus, dass Jessica in Wirklichkeit auch ein ziemlicher Wildfang ist. Der Beweis: Während des Gottesdienstes plündert sie ungeniert den Klingelbeutel. Leider fällt der Verdacht auf Bart. Nun wäre eigentlich ein echter Liebesbeweis fällig.

\notiz{
\begin{itemize}
  \item Willie feiert das \glqq Scotchoktoberfest\index{Scotchoktoberfest}\grqq .
	\item In dieser Folge ist zum zweiten Mal ein Springfielder angekettet und gesichert wie Hannibal Lecter in  \glqq Das Schweigen der Lämmer\grqq\ (Das erste Mal war in \glqq Homer kommt in Fahrt\grqq, siehe \ref{9F10}).
	\item Fehler: Beim Ankreuzen der Tage, die Bart Jessica nicht sehen will, hat der April 31 Tage und der Mai nur 30. In der nächsten Einstellung ist dann der Kalender wieder in Ordnung.
\end{itemize}
}

	
\subsection{Lisa auf dem Eise}\label{2F05}
Damit Lisa in Sport nicht die Note ungenügend bekommt, muss sich sich einem Sportverein anschließen. Nach mehreren Versuchen entscheidet sie sich für Eishockey und spielt für die Kwik-E-Mart Gougers.
Bart spielt ebenfalls Eishockey, aber in einer anderen Mannschaft (Mighty Ducks). Nun ist der Tag gekommen, an dem sich die Geschwister als Gegner gegenüberstehen. Marge ermahnt ihre Sprösslinge, sich nicht bis aufs Messer zu bekämpfen -- vergeblich. Die Aggressionen werden von Spielern und Zuschauern geschürt. Auf dem Höhepunkt der Auseinandersetzung erinnern sie sich schließlich doch noch daran, dass sie Geschwister sind.

\notiz{
\begin{itemize}
	\item Das Logo auf Dolphs Computer (ein \glqq Newton\index{Newton}\grqq ) ist ein Apfel, aus dem ein Wurm kommt.
	\item In Lisas Mannschaft spielen noch Nelson, Milhouse, Jimbo, Kearney und Uter.
	\item Snakes Gefangenennummer lautet: 7F20. 7F20 ist auch der Produktionscode der Simpsons-Episode \glqq Kampf dem Ehekrieg\grqq\ (siehe \ref{7F20}).
	\item Krusty singt die amerikanische Nationalhymne ähnlich falsch wie Robert Goulet, als dieser sie 1965 vor dem Kampf im Schwergewicht zwischen Muhammad Ali und Sonny Liston gesungen hat.
\end{itemize}
}
	
\subsection{Die Babysitterin und das Biest}
Homer und Marge besuchen die Süßigkeitenmesse. Babysitterin Ashley Grant\index{Grant!Ashley} soll auf die Kinder aufpassen. Später bringt Homer Ashley nach Hause. Dabei kommt es zu einem schrecklichen Missverständnis: Der Babysitterin klebt eine Süßigkeit am Po. Der galante Homer will hilfreich sein. Er entfernt das klebrige Ding und sieht sich plötzlich als Grapscher angeklagt. Die Frauenbewegung meldet sich und stempelt Homer als sexuellen Unhold ab.

\notiz{
\begin{itemize}
  \item Auf der US-Flagge, die hinter Homer bei seiner Fernsehansprache hängt, sind nur 37 Sterne enthalten.
  \item In der Richtigstellung, die \glqq Rock Bottom\index{Rock Bottom}\grqq\ mit Moderator Godfrey Jones\index{Jones!Godfrey} abgibt, ist zu lesen, dass Todd der Ältere der Flanders Kinder ist. Außerdem ist in dieser Richtigstellung zu lesen, wer diese lese, habe kein Leben\footnote{Dieser Abspann erfolgt so schnell, dass man ihn nur mithilfe des Standbildes lesen kann.}.
  \item Filmzitat: Wie Homer eine Cola mit Brause mischt und diese wirft, ist eine Anspielung auf den Film \glqq Goddfellas\grqq\ von 1990.
\end{itemize}
}

	
\subsection{Grandpa gegen sexuelles Versagen}\label{2F07}
Das Liebesleben von Homer und Marge tendiert gegen Null. Grandpa sieht sich gezwungen einzugreifen. Er überreicht seinem Sohn eine Spezialtinktur. Das Rezept hat er noch von seinem Urgroßvater. Die Tinktur verwandelt Homer in einen wahren Sexprotz. Marge ist begeistert. Sie überredet Homer, das höchst willkommene Gebräu kommerziell zu verwerten. Gesagt, getan. Ab sofort verschwinden allabendlich alle Eltern der Stadt in ihren Schlafzimmern.

\notiz{
\begin{itemize}
  \item Der Legende zufolge erfand Grandpas Urgroßvater dieses Tonikum, als er billigen Ersatz für Weihwasser erfinden wollte.
  \item Grandpa sagt, Homer sei 38 Jahre alt.
  \item Das Haus, in dem Homer aufgewachsen ist, brennt ab. Doch in der Episode \glqq Duell bei Sonnenaufgang\grqq\ (siehe \ref{AABF19}) ist es wieder intakt.
  \item Luann und Kirk Van Houten schlafen in getrennten Betten.
  \item Moe erteilt Homer ein lebenslanges Lokalverbot.
  \item Fehler: Homer fährt mit Marges Auto zur alten Farm. Als er allerdings dort ankommt, steigt er aus seinem eigenen Auto aus.
\end{itemize}
}

	
\subsection{Angst vorm Fliegen}\label{2F08}
Homer hat Freiflüge innerhalb von Amerika gewonnen, weil er sich als Pilot ausgegeben hat, um in einer Kneipe trinken zu können. Doch aus dem Urlaub wird nichts: Marge leidet unter schier unüberwindlicher Flugangst, deshalb schickt Homer seine Frau zu einer Psychiaterin. Diese findet die Ursache von Marges Flugangst in deren Kindheit. Es gelingt ihr, Marge zu heilen. Nun steht dem gemeinsamen Urlaub nichts mehr im Wege. Zufrieden und voller Vorfreude besteigen die Simpsons ein Flugzeug, das dann prompt im Meer notlanden muss.

\notiz{
\begin{itemize}
	\item Marge ist das letzte Mal -- ohne Zwischenfall -- in \glqq Einmal Washington und zurück\grqq\ (siehe \ref{8F01}) geflogen.
	\item Marges Traum ist eine Anspielung auf die TV-Serie \glqq Lost In Space\grqq\ von 1965. Marge ist die Mutter Maureen\index{Maureen}, Lisa der Roboter und Homer ist Dr. Zachery Smith\index{Smith!Dr. Zachery}.
	\item Guy Incognito sieht Homer zum Verwechseln ähnlich.
	\item Der Name der Lesbenbar lautet SHE-SHE Lounge.
	 \item Homers Lieblingslied ist \glqq It's Raining Man\grqq\ von den Weather Girls.
\end{itemize}
}

	
\subsection{Homer, der Auserwählte}\label{2F09}
Wie alle angesehenen Bürger der Stadt will auch Homer Mitglied in der exklusiven Steinmetzvereinigung\index{Steinmetze} werden. Als sich herausstellt, dass Grandpa Simpson Mitglied ist, wird auch Homer aufgenommen. Doch er benimmt sich derart daneben, dass man ihn sofort wieder auf die Straße setzt. Dabei stellen die Steinmetze allerdings fest, dass Homer das Zeichen des Auserwählten am Körper trägt. Man wirft sich ihm zu Füßen -- aber nicht für lange.

\notiz{
\begin{itemize}
  \item Homers Nummer bei den Steinmetzen: 908.
  \item Die Mitglieder der Steinmetze: Lenny (Nummer 12), Carl (Nummer 14), Homer, Jasper, Barney, Adolf Hitler\index{Hitler!Adolf}, Sideshow Mel, Chief Wiggum, Mr. Burns (Nummer 29), Smithers, Grandpa, Rektor Skinner, Moe, Herman, Dr. Hibbert, Bürgermeister Quimby, Willie, Steve Guttenberg, Krusty, Kirk Van Houten, Kent Brockman, Dewey Largo, Scott Christian, Leopold, Nummer Eins, Apu, Homer Glumplich, ein kleiner, grüner Außerirdischer und ein Typ im Eierkostüm.
  \item Im Rat der weltweiten Steinmetze: George Bush, Jack Nicholson, Orville Redenbacher und Mr. T.
  \item Als Homer mit den Steinmetzen Bowlen geht, spielt er auf Bahn Nr. 13.
  \item Ein Auto auf dem Parkplatz hat das Nummernschild \glqq 3MI ISL\grqq . (Three-Mile Island ist der Ort des bislang größten Atom-Unfalls der USA in einem Atomkraftwerk).
  \item Abe Simpson ist Präsident der Schwulen- und Lesbenvereinigung.
  \item Der Kopf von Xt'Tapalataketel\index{Xt'Tapalataketel} (das Geschenk aus \glqq Der Lebensretter\grqq ) ist immer noch im Keller der Simpsons (siehe \ref{7F22}).
\end{itemize}
}

\subsection{Und Maggie macht drei}\label{2F10}
Beim Herumblättern im Familienfotoalbum fällt Lisa auf, dass es keine Fotos von Maggie gibt. Daraufhin erzählen Homer und Marge eine Geschichte: Homer hatte seinen Job im Kernkraftwerk gekündigt. Er wollte in der Bowlingbahn arbeiten. Doch dann war plötzlich Maggie unterwegs. So sah sich Homer gezwungen, wieder im Kraftwerk anzufangen. Dort konnte er für das dritte Kind mehr Geld verdienen. Seitdem stehen Maggies Fotos an Homers Arbeitsplatz.

\notiz{
\begin{itemize}
  \item Die Simpsons schauen im Fernsehen die Serie \glqq Knight Boat\grqq\ an. Diese Serie dürfte ein Anspielung auf Knight Rider sein, da das Boot auch sprechen kann wie KITT.
  \item Als Marge Maggie gebärt, wird auch Captain McCallister Vater eines Sohnes, den er als \glqq Fang des Tages\grqq\ bezeichnet. 
  \item Bürgermeister Quimby wird ebenfalls Vater eines Kindes, allerdings ist das Kind nicht von seiner Frau sondern von einer seiner Geliebten.
  \item Die Bowlingbahn, in der Homer kurzzeitig arbeitete, gehört Barneys Onkel Al.
  \item Fernsehzitat: In einer Anspielung auf die Anfangssequenz der US-TV-Serie \glqq The Mary Tyler Moore Show\grqq\ tanzt Homer umher und singt \glqq Jetzt habe ich's endlich doch geschafft\grqq . Er endet mit einer schwungvollen Bewegung und schmeißt die Bowlingkugel in die Luft wie Mary im Original ihren Hut. Die Kugel fällt auf die Bowlingbahn und hinterlässt einen Krater.
  \item Dr. Hibberts Haare sehen in der Rückblende aus wie die von US-Komiker Arsenio Hall.
  \item In dieser Episode enthüllt: Homer hat sein Haar verloren, weil er sich jedes Mal einen Teil ausgerissen hat, als Marge schwanger war.
  \item In der Bowlingbahn ist Jacques Brunswick zu sehen, der Marge in der Episode \glqq Der schöne Jacques\grqq\ (siehe \ref{7G11}) verführen wollte.
  \item Fehler I: Ruth Powers ist auf der Babyparty zu sehen, obwohl sie die Simpsons vor Maggies Geburt noch nicht kannte.
  \item Fehler II: Man sieht hier, wie Homer reagiert hat, als er erfahren hat, dass Marge mit Bart schwanger ist. Die Szene spielt in dem Simpsons Haus (742 Evergreen Terrace), was sie aber erst nach Lisas Geburt bezogen haben.
  \item Fehler III: Marge sitzt auf der Couch und gesteht Homer, dass sie zum dritten Mal schwanger (mit Maggie) ist. An der Wand hängt allerdings bereits ein Bild mit Maggie (siehe \cite{Sel18}).
\end{itemize}
}
	
\subsection{Barts Komet}\label{2F11}
Rektor Skinner verdonnert Bart dazu, ihm bei seinen Himmelsbeobachtungen zu assistieren. Dabei entdeckt Bart einen Kometen und der rast direkt auf Springfield zu. Eine Rakete wird ins All geschickt. Sie soll den Kometen zerstören. Aber auch diese Rakete versagt und zerstört die einzige Brücke, die aus der Stadt führt. Nun flüchten alle in Flanders Atombunker. Nur Homer ist sich sicher, dass der Komet in der Umweltschmutzschicht verglühen wird und somit keinerlei Gefahr für Leib und Leben besteht.

\notiz{
\begin{itemize}
  \item Das Transparent, das Bart an Rektor Skinners Wetterballon befestigt, trägt die Aufschrift: \glqq Hi! I'm big butt Skinner\grqq\ (Hi! Ich bin Dickarsch Skinner).
  \item Rektor Skinner hat ein Auto. In der Episode \glqq Lisa will lieben\grqq\ (siehe \ref{4F01}) sagte er, er könne sich kein Auto leisten.
  \item In dieser Folge stellt Ham die Superfreunde vor: E-Mail, Kosinus, Protokoll, Database und Lisa.
  \item Ham will, dass Bart auch den Superfreunden beitritt. Barts Name lautet \glqq Kosmos\grqq .
  \item Krusty macht sich Notizen, obwohl er laut Folge \glqq Der Clown mit der Biedermaske\grqq\ (siehe \ref{7G12}) ein Analphabet ist.
  \item Kent Brockman behauptet von u.\,a. diesen Personen, dass sie schwul sind: Matt Groening, Bill Oakley, Mike Scully und Al Jean (diese sind alle an der Produktion der Simpsons beteiligt).
\end{itemize}
}


	
\subsection{Homie der Clown}\label{2F12}
Krusty hat Wettschulden. Um seine Einnahmen zu erhöhen, gründet er ein College für Clowns. Auch Homer besucht diese Lehranstalt. Und Papa Simpson macht sich ganz gut als Krusty-Double. So gut, dass die Mafia ihn für den echten Krusty hält. Homer wird entführt und dem mächtigen Boss Don Vittorio\index{Don Vittorio} vorgeführt. Wenn Homer überleben will, muss er dem Mafioso ein ziemlich schwieriges Kunststück vorführen. Nun muss sich erweisen, was er in der Schule gelernt hat.

\notiz{
\begin{itemize}
  \item Filmzitat: Homer baut beim Abendessen ein Zirkuszelt aus Kartoffelpüree, so wie Richard Dreyfuss in Steven Spielbergs Film \glqq Unheimliche Begegnungen der Dritten Art\grqq .
  \item Krusty hat eine goldene \glqq Kroan along with Krusty\grqq\ (Schmelz dahin mit Krusty) Schallplatte in seinem Büro.
  \item Das Thema von \glqq Godfather\grqq\ erklingt, als Homer mit seinem Kopf gegen die Weingläser kommt.
\end{itemize}
}
	
\subsection{Bart gegen Australien}\label{2F13}
Bart und Lisa streiten sich über eine überaus wichtige Frage: Fließt das Wasser auf der südlichen Halbkugel in entgegengesetzter Richtung ab? Bart will es genau wissen. Er ruft einen australischen Jungen per R-Gespräch\footnote{Ein R-Gespräch ist ein Telefonat, bei dem der Angerufene die Kosten des Anrufs übernimmt. Das \glqq R\grqq\ in R-Gespräch steht für Rückwärtsberechnung (englisch \glqq Reverse Charge\grqq). Der Angerufene muss vor Zustandekommen der Verbindung dieser Kostenübernahme zustimmen.} an. Die beiden Jungs fachsimpeln ein Weilchen. Weil sich die Telefonrechnung anschließend auf 900 Dollar beläuft, flippt der Vater des Jungen aus: Es kommt zu diplomatischen Verwicklungen. Die gab es aber auch schon wegen weniger wichtigen Fragen.

\notiz{
\begin{itemize}
  \item Barts R-Gespräch kostet Bruno Drundridge\index{Drundridge!Bruno} 900 Dollar -- genau dieselbe Summe, die das Springfielder Gaswerk den Simpsons in Rechnung stellt, nachdem eine von Barts Aktionen den Familientrockner abtrennt und die Hauptgasleitung entzündet (in \glqq Homer und gewisse Ängste\grqq , siehe \ref{4F11}).
  \item Die Statue zu Ehren der verbrecherischen Gründer von Australien sieht aus wie Snake.
  \item Der Schriftzug auf dem australischen Parlament lautet: \glqq Parliament-haus der Austria\grqq . \glqq Austri\grqq\ wurde dann auf \glqq Australia\grqq\ ausgebessert.
  \item Bart ruft in Argentinien an, es klingelt das Autotelefon von Adolf Hitler\index{Hitler!Adolf}. Auf dem Nummernschild seines Mercedes steht \glqq Adolf 1\grqq .
  \item Bart ruft laut Telefonrechnung an folgenden Orten an:
  \begin{itemize}
  	\item Santiago de Chile
  	\item Antarctica Naval Research Station
  	\item New Ouagadougou
  	\item Burkina Faso\footnote{Bart ruft dort zweimal an. Aber Burkina Faso liegt nicht auf der Südhalbkugel}
  	\item Unnamed Settlement
  	\item Disputed Zone 
  \end{itemize}
  \item Lisa behauptet, die Corioliskraft\index{Corioliskraft} entscheidet darüber -- abhängig von der Hemisphäre --, in welche Richtung sich Wirbel beim Ablaufen von Wasser bilden. Entgegen ihrer Annahme drehen sich Badewannenwirbel auf einer Hemisphäre nicht immer in die gleiche Richtung. Vielmehr ist es Zufall, ob das Wasser im Uhrzeigersinn oder entgegen abfließt. Grund dafür ist die Größe der Wirbel: Sie sind viel zu klein, als dass sie von der Erdrotation beeinflusst werden können \cite{wdP}.
\end{itemize}
}

\subsection{Homer gegen Patty und Selma}\label{2F14}
Homer hat in Aktien investiert und ist bald darauf völlig pleite. In seiner Verzweiflung wendet er sich an Marges Schwestern Selma und Patty. Sie leihen ihm zwar Geld, lassen sich aber einen Schuldschein ausstellen, damit können sie ihn später erpressen und erniedrigen. Doch Homer beweist trotzdem größte Anständigkeit. Er bewahrt die gemeinen Schwestern davor, dass ihre Be\-för\-der\-ung in der Führer\-schein\-stel\-le zurückgenommen wird. Und nun erlassen sie ihm gerührt seine Schulden.

\notiz{
\begin{itemize}
	\item Lisa meldet sich für Eishockey an. In der Episode \glqq Lisa auf dem Eise\grqq\ (siehe \ref{2F05}) spielte sie bereits Eishockey als Torwart und war in Barts gegnerischer Mannschaft.
	\item Homer nimmt einen Job als Chauffeur bei Classy Joe's\index{Classy Joe} an.
\end{itemize}
}

	
\subsection{Lisas Hochzeit}\label{2F15}
Auf dem Jahrmarkt sucht Lisa eine Wahrsagerin auf. Von dieser erfährt sie, dass sie einen reichen Engländer heiraten wird. Tatsächlich scheint sich die Vorhersage zu erfüllen. Dann lernt der Mann Lisas Familie kennen. Er verlangt, dass Lisa nach der Hochzeit die Beziehung zu ihrer Familie abbricht, vor allem zu ihrem Vater. Lisa gibt den Verlobungsring zurück. Sie bittet die Wahrsagerin, ihr die große einzige Liebe vorherzusagen. Doch die Seherin ist nur auf unangenehme Beziehungen spezialisiert.

\notiz{
\begin{itemize}
  \item Filmzitat: Lisa und Hugh Parkfield\index{Parkfield!Hugh} treffen sich in einer Bibliothek und streiten sich wegen eines Buches, eine Anspielung auf Ryan O'Neal und Ali McGraw in dem Film \glqq Love Story\grqq .
  \item Quimby fährt sein Taxi für die Otto Cab Corporation, gegründet 2003.
  \item Im Jahr 2010 arbeitet Kent Brockman für CNNBCBS, das wiederum Teil von ABC ist.
  \item In dieser Episode wurde der Himmel dunkler gemalt. Dies soll auf die Luftverschmutzung hinweisen.
  \item Diese Folge wurde mit einem Emmy ausgezeichnet.
  \item Kent Brockmans Liste von den verhafteten Prominenten (Auswahl) : Die Baldwin-Brüder-Gang, Dr. Brad Pitt, John John John Kennedy, George Burns, Tim Allen junior, Sideshow Ralph Wiggum und Martha Hitler.
\end{itemize}
}

	
\subsection{Wer erschoss Mr. Burns? Teil 1}\label{2F16}
Der Hausmeister der Grundschule von Springfield stößt auf Öl. Sofort bricht in der ganzen Stadt Euphorie aus. Mr. Burns sieht dadurch die Geschäfte mit seinem Kernkraftwerk gefährdet. Er versucht, der Schule das Öl wegzunehmen. Als man eines Abends auf ihn schießt, fragt sich die ganze Stadt, wer der Täter sein könnte. Die Antwort ist nicht leicht zu finden. Mr. Burns hat sich einfach zu viele Feinde gemacht.

\notiz{
\begin{itemize}
  \item Das Klassenmaskottchen der vierten Klasse, Supertyp\index{Supertyp}, stirbt und Willie soll es begraben, dabei stößt Willie auf die Ölquelle.
  \item Homer arbeitet bereits seit zehn Jahren im Atomkraftwerk.
  \item Skinner sagt, dass Mr. Burns der berühmteste 104-Jährige der Stadt sei.
  \item Mr. Burns Arme zeigen auf die Kompasspunkte \glqq S\grqq\ und \glqq W\grqq, als er auf die Sonnenuhr im Stadtzentrum fällt.
  \item Willie gibt an, dass sein Vater, als er gestorben ist, nicht begraben wurde, sondern nur ins Moor geworfen worden ist. Aber in der Episode \glqq Burns möchte geliebt werden\grqq\ (siehe \ref{AABF17}) sind sowohl Willies Vater als auch seine Mutter zu sehen.
\end{itemize}
}

	
\subsection{25 Windhundwelpen}\label{2F18}
Der Windhund der Simpsons hat endlich ein Weibchen gefunden -- das daraufhin 25 niedliche Welpen wirft. Mr. Burns schnappt sich sofort die Tiere. Er will sich aus den Fellen der Tiere einen feinen Frack schneidern lassen. Doch Lisa und Bart können dies verhindern. Sie schaffen es sogar, Mr. Burns zu einem Tierliebhaber zu machen. Leider geht der Schuss nach hinten los: Bei Hunderennen gewinnt der schwerreiche Mann noch ein paar Millionen Dollar.

\notiz{
\begin{itemize}
  \item Filmzitat: Die Szene, in der alle Welpen fernsehen und die ganze Story dieser Folge parodieren den Disney-Film \glqq 101 Dalmatiner\grqq .
  \item Mr. Burns hat in seinem Badezimmer ein Telefon, einen Magazinständer und ein Bidet, aber offensichtlich kein Toilettenpapier.
  \item Es sind 26 anstatt 25 Welpen, die in der Küche aus den Fressnäpfen fressen.
\end{itemize}
}	
	
\subsection{Der Lehrerstreik}\label{2F19}
Die Lehrer von Springfield streiken: Sie fordern eine Gehaltserhöhung und Verbesserungen der Lehrmittel. Zunächst sind alle Kinder froh über den Streik. Doch anstatt, dass der Unterricht ausfällt, werden Laienlehrer engagiert, um den Schulbetrieb aufrecht zu erhalten. Der Unterricht wird zum Albtraum. Und so kommt es zu einer einmaligen Aktion: Die beiden Schüler Bart und Milhouse leiern eine Schlichtungsaktion zwischen den streitenden Parteien an.

\notiz{
\begin{itemize}
  \item In dieser Episode tritt erstmals Jimbos Mutter auf.
  \item In dieser Folge ist Jasper mit Sandalen zu sehen. Seine beiden Füße sehen normal aus. In der Episode \glqq Wer erschoss Mr. Burns? Teil 1\grqq\ (siehe \ref{2F16}) hat er aber ein Holzbein.
  \item Als Schulbuch steht u.\,a. das Buch \glqq Die satanischen Verse\grqq\ zur Verfügung.
  \item Aushilfslehrer, die in Barts Klasse unterrichtet haben: Clancy Wiggum, Barney Gumble, Lionel Hutz, Gabe Kaplan\index{Kaplan!Gabe}\footnote{Gabe Kaplan (geboren am 31. März 1945 in Brooklyn, New York) ist ein amerikanischer Filmschauspieler und professioneller Pokerspieler \cite{Kaplan}.}, Moe Szyslak und Marge Simpson.
  \item Snakes Gefangenennummer ist die 7F20. Der Produktionscode der Folge \glqq Kampf dem Ehekrieg\grqq\ (siehe \ref{7F20}) lautet ebenfalls 7F20.
\end{itemize}
}
	
\subsection{Die Springfield Connection}\label{2F21}
Marge Simpson wird Polizistin. Nach Ausbildung und bestandener Prüfung steht sie im Außendienst ihre Frau. Auch gegen Ehemann Homer ist sie unerbittlich -- zum Beispiel, wenn er falsch parkt oder Pokerpartys veranstaltet. Anschließend kommt sie einem illegalen Jeanshandel auf die Spur. Und darin sind auch korrupte Polizisten ihres Reviers involviert. Entrüstet gibt Marge ihre Marke zurück.

\notiz{
\begin{itemize}
  \item Burns nimmt das Bestechungsgeld in Höhe von 300 Dollar, das Apu eigentlich Marge geben wollte.
  \item Neben Homer gehören noch Barney, Carl, Lenny, Herman und Moe seiner Pokerrunde an.
\end{itemize}
}

	
\subsection{Auf zum Zitronenbaum}\label{2F22}
Grandpa erzählt eine Geschichte: Sie handelt davon, wie der alte Jebediah Springfield und sein Partner Shelbyville Manhattan\index{Shelbyville!Manhattan} das Land in Besitz genommen haben. Daraus sind schließlich die beiden Städte Springfield und Shelbyville entstanden. Eines Tages kam es zwischen den beiden Städten zum Streit um einen Zitronenbaum. Aus dessen Früchten war nämlich köstliche Limonade gemacht worden. Dies wollte sich keine Stadt entgehen lassen.

\notiz{
\begin{itemize}
  \item Prof. Frink erfindet ein fliegendes Motorrad.
  \item Neben Bart und Milhouse begeben sich noch Todd (hier wieder der Kleinere der beiden Flanders Kinder), Nelson, Martin und Database nach Shelbyville, um den Zitronenbaum zurückzuholen.
  \item Milhouse trägt rote Schuhe zu seiner Tarnkleidung.
  \item Die Shelbyviller trinken Fudd\index{Fudd} Bier (vorgestellt in \glqq Homer auf Abwegen\grqq , \ref{8F19}), kaufen im Speed-E-Mart ein und trinken in der Kneipe Joe's, die genau wie Moe's aussieht. Ihre Grundschule wird gepflegt von einer schottischen Hausmeisterin, die Willie sehr ähnlich sieht.
  \item Bart sprüht seine Füße grün an. Eine Szene später ist die Farbe verschwunden, noch eine Szene später allerdings wieder da.
  \item Luann Van Houten wurde in Shelbyville geboren.
\end{itemize}
}
	
\subsection{Springfield Film-Festival}\label{2F31}
In Springfield wird ein Film-Festival organisiert. Dazu kann jeder einzelne Bür\-ger Film\-bei\-trä\-ge bei\-steu\-ern. Verbissen setzt der unsympathische Mr. Burns alles daran, als Sieger aus dem Festival hervorzugehen. Zur Jury gehört nicht nur der berühmte New Yorker Kritiker Jay Sherman\index{Sherman!Jay}. Neben ihm haben auch Marge und Homer Simpson Richtergewalt -- und das mindert die Chancen von Mr. Burns beträchtlich. Der Mann hat ihnen einfach schon zu viel angetan.

\notiz{
\begin{itemize}
  \item Zwei der Autoren von Burns Film heißen Lowell Burns\index{Burns!Lowell} und Babaloo Smithers\index{Smithers!Babaloo}.
  \item Film I: \glqq Ein Burns zu jeder Jahreszeit\grqq\ (A Burns for All Seasons): Burns Propagandafilm über sich selbst.
  \item In Burns Film gibt es Anspielungen auf die Filme \glqq Viva Zapata!\grqq , \glqq E. T.\grqq\ und \glqq Ben Hur\grqq .
  \item Film II: \glqq Der Rinderwahnsinn am helllichten Tag\grqq\ (Bright Lights, Beef Jerky): Apus Beitrag zum Springfielder Film-Festival.
  \item Film III: \glqq Vom Fußball getroffen\grqq : Hans Maulwurfs Film, in dem er von einem Football in die Leistengegend getroffen wird und umfällt. (Homers Liebling, bezeichnet in als \glqq Der Ball ins Zentralmassiv\grqq ).
  \item Film IV: \glqq Kotz 1001 (Pukahontas)\grqq : Barney Gumbles einfühlsame Dokumentation über sein Alkoholproblem.
  \item Barney gibt an, dass er 40 Jahre alt ist.
  \item Dies ist die einzige Folge, in der Matt Groenings Name nicht im Abspann auftaucht. Das ist darauf zurückzuführen, dass er den Eindruck hatte, dass die gesamte Episode eine einzige Werbeveranstaltung für die Serie \glqq The Critic\grqq\ war \cite{Simpsons001}. Die Serie \glqq The Critic\grqq\ stammt von den beiden Simpsons Produzenten Al Jean und Mike Reiss.
\end{itemize}
}


\subsection{Zu Ehren von Murphy}\label{2F23}
In einer Schachtel \glqq Krusty's Cornflakes\grqq\ war ein Metallring und diesen hat der arme Bart verschluckt. Niemand glaubt ihm, dass er Schmerzen hat -- außer Schwester Lisa. Schließlich muss ihm der Blinddarm herausgenommen werden. Vom eingeklagten Schadensersatz (100.000 Dollar) bleiben ihm 500 Dollar. Und genau diesen Betrag brauchte Lisa, um die einzige Platte des Jazzmusikers Murphy (\glqq Sax On The Beach\grqq ) zu kaufen.

\notiz{
\begin{itemize}
	\item Die Anwälte Robert Chaporo\index{Chaporo!Robert} und Albert Dershman\index{Dershman!Albert} (der drei Billardkugeln auf einmal in seinen Mund nehmen kann) sind Parodien auf die amerikanischen Staranwälte Robert Shapiro und Alan Dershowitz, die u.\,a. auch den ehemaligen Footballspieler O. J. Simpson verteidigt haben.
	\item Krusty ist wegen sexueller Belästigung angeklagt und ferner gesteht er seine Aspirinsucht.
	\item Aufgrund von Budgetkürzungen muss Hausmeister Willie Französisch unterrichten.
	\item Homer trägt auf seinem linken Oberarm eine Tätowierung der \glqq Starland Vocal Band\grqq .
	\item Lisas Katze Schneeball I wurde überfahren, dies steht allerdings im Widerspruch zur der Folge \glqq Es weihnachtet schwer\grqq\ (siehe \ref{7G08}), in welcher Schneeball I bereits überfahren wurde.
	\item Der Ausspruch \grqq cheeseeating surrender monkeys\grqq\ (\glqq käsefressende Kapitulationsaffen\grqq ) von Hausmeister Willie hat es es in das \emph{Oxford Dicitionary of Modern Quotations} geschafft (siehe \cite{Reiss19}).
\end{itemize}
}
	
\subsection{Romantik ist überall}\label{2F33}
Marge möchte ihren Kindern beibringen, was Liebe ist. Sie versucht, alle Familienmitglieder zu aktivieren: Jeder soll einen individuellen Beitrag zum Thema \glqq Liebesgeschichte\grqq\ leisten. Doch Marge scheitert auf der ganzen Linie: Weder die Kinder noch die Eltern noch die Großeltern haben zum Thema Happy End etwas zu sagen. Bis Marge dann ihre wahre Liebesgeschichte mit Homer erzählt. Und die endet ja bekanntlich wirklich mit einem Happy End.

\notiz{
\begin{itemize}
  \item Diese Folge besteht aus Ausschnitten aus 28 vorangegangen Episoden.
  \item Marge liest das Buch \glqq The Bridges Of Madison County\grqq\ von Robert James Waller. Dieses Buch wurde 1995 von Clint Eastwood verfilmt. Der deutsche Filmtitel lautet \glqq Die Brücken am Fluss\grqq .
\end{itemize}
}

\section{Staffel 7}

\subsection{Filmstar wider Willen}\label{2F17}
Hollywood plant, den Kassenschlager \glqq Radioactive Man\grqq\ neu zu verfilmen. Als Drehort wird Springfield ausgewählt. Der Partner des \glqq Radioactive Man\grqq\ ist der \glqq Fallout Boy\grqq\index{Fallout Boy}. Diese Rolle soll von Barts Freund und Klassenkameraden Milhouse gespielt werden. Doch der hat eigentlich gar keine Lust. Selbst Hollywoodstar Micky Rooney\index{Rooney!Micky} kann den Jungen nicht zum Weitermachen überreden. Auch Bart hatte sich um die Rolle beworben, da er aber um zwei Zentimeter zu klein war, wurde er abgelehnt.

\notiz{
\begin{itemize}
	\item Moe erzählt Barney, dass er einer der original \glqq Kleinen Strolche\grqq\ war, bis er einen der Strolche, Alfalfa\index{Alfalfa}, umgebracht hat, weil der seine Auspuff-ins-Gesicht-Nummer geklaut hatte.
	\item Bart und Milhouse haben alle 814 Hefte des Radioactive Man gelesen.
	\item Die Szene, in der Radioactive Man und Fallout Boy in einem Käfig gefangen sind, ist eine Anspielung auf den Kevin Costner Film \glqq Waterworld\grqq .
\end{itemize}
}
	
\subsection{Wer erschoss Mr. Burns? Teil 2}\label{2F20}
Die Polizei sucht fieberhaft nach dem Mann, der Mr. Burns niedergeschossen hat. Mr. Burns ist mit viel Glück mit dem Leben davon gekommen. Er erwähnt immer wieder einen Namen: Homer Simpson. In höchster Not flieht Homer. Er sucht das Krankenhaus auf, um mit Mr. Burns das schreckliche Missverständnis zu klären. Homer mag Mr. Burns zwar auch nicht -- wie so viele, doch deshalb schießt er keinen Menschen nieder. Schließlich stellt sich heraus, dass Maggie Mr. Burns niedergeschossen hat.

\notiz{
\begin{itemize}
  \item Filmzitat I: Hausmeister Willies Verhör, währenddessen er einen Kilt trägt, erinnert an Sharon Stones berühmte Szene in \glqq Basic Instinct\grqq .
  \item Filmzitat II: Homers Flucht im Krusty-Burger Drive-Thru parodiert \glqq Auf der Flucht\grqq .
  \item Fernsehzitat I: Der durch den Genuss verdorbener Sahne ausgelöste Traum von Chief Wiggum, in dem er rückwärts spricht, ist eine Parodie auf \glqq Twin Peaks\grqq .
  \item Fernsehzitat II: An \glqq Dallas\grqq\ wird zweimal erinnert: An die Folge \glqq Wer schoss auf J.R.?\grqq\ aus dem Jahr 1979 und an die Duschszene von 1986, welche die ganze vorangegangene Staffel als Traum abtat.
  \item Mr. Burns Zimmernummer im Krankenhaus ist 2F20, wie der Produktionscode dieser Episode.
  \item Wiggum lässt fälschlicherweise Dr. Colossus frei.
  \item Tito Puente spielt im Restaurant \glqq Chez Guevara\grqq\index{Chez Guevara}.
  \item Jasper wohnt im Altersheim auf Zimmer Nummer 26.
  \item Synchronisationsfehler: In Teil 1 fragt Burns noch: \glqq Na, worüber lachen Sie denn so?\grqq\ Hingegen fragt er in Teil 2 in der Auflösung: \glqq Na, worüber lachst du denn so?\grqq
  \item In dieser Episode wird erstmals der Nachname von Moe genannt und zwar heißt Moe vollständig Moe Szyslak\index{Szyslak!Moe}.
  \item Homer gibt in der späteren Folge \glqq Rache ist dreimal süß\grqq\ (siehe \ref{JABF05}) an, dass er auf Mr. Burns schoss und es auf Maggie schob.
\end{itemize}
}

	
\subsection{Bei Simpsons stimmt was nicht!}\label{3F01}
In der Schule kommt etwas Unangenehmes ans Licht: Bart hat Läuse! Das Jugendamt wird alarmiert. Nun werden die Simpson-Kinder von Amts wegen zu den Nachbarn Flanders in Pflege gegeben. Homer und Marge müssen einen Elternkurs absolvieren. Da stellt Flanders fest, dass seine neuen Schützlinge nicht getauft sind. Sofort schreitet er zur Tat. Doch Homer kann verhindern, dass seine Kinder getauft werden.

\notiz{
\begin{itemize}
	\item Vor dem Gericht steht eine Statue eines Pferdes mit einem Reiter, welche mit Swartzwelder beschriftet ist. John Swartzwelder\index{Swartzwelder!John} ist einer der Autoren und Produzenten der Simpsons. Diese Episode stammt beispielsweise aus der Feder von John Swartzwelder
	\item Bei dem Unterricht, um bessere Eltern zu werden, sind neben Marge und Homer u.\,a. noch Agnes Skinner und Brandine und Cletus Del Roy.
	\item Die Szene, in der Maggie den Kopf um 180 Grad dreht, ist eine Anspielung auf den Film \glqq Exorzist\grqq\ von 1973.
	\item In dieser Episode wird Maggies richtiger Name enthüllt: Margaret.
\end{itemize}
}

	
\subsection{Bart verkauft seine Seele}\label{3F02}
Bart und Milhouse diskutieren über das Thema Seele. Bart erklärt, dass die Seele nicht existiert. Da bietet ihm Milhouse fünf Dollar für seine Seele an. Bart steigt darauf ein. Er schreibt auf einen Zettel das Wort Seele und verkauft diesen. Wider Erwarten hat dies Konsequenzen: Bart kann nicht mehr lachen und er ist zu keinen Empfindungen mehr fähig. Verzweifelt versucht er, den Zettel zurückzubekommen. Milhouse verlangt fünfzig Dollar dafür.

\notiz{
\begin{itemize}
  \item Moe eröffnet \glqq Onkel Moes Familienschlemmerecke\grqq\ (Uncle Moe's Family Feedbag). Wenn Moe beim Servieren der Rechnung nicht lächelt, gehen die Kosten auf Moe.
  \item Rod Flanders feiert seinen zehnten Geburtstags in Moes Familienschlemmerecke. Ebenso feiern am selben Tag die Zwillinge Terri und Sherry in Moes Familienrestaurant ihren Geburtstag.
  \item Fehler: Bei der Großaufnahme in der Kirche ist Lou gelb. Außerdem ist Carl kurz mit gelber Hautfarbe zu sehen.
\end{itemize}
}
	
\subsection{Lisa als Vegetarierin}\label{3F03}
Ein schwerer Schlag für die ganze Familie Simpson: Töchterchen Lisa ist entschlossen, ihre Familie zu verlassen und zwar aus vegetarischen Gründen. Das Mädchen ist überzeugte Vegetarierin. Ihrer Meinung nach wird dies in ihrer Familie nicht richtig gewürdigt. Deshalb greift sie zur letzten Konsequenz. Doch dann wird sie eines Besseren belehrt -- ausgerechnet von einem Herrn namens Paul McCartney.

\notiz{
\begin{itemize}
  \item Der Film über die Fleischproduktion ist die Nummer 3F03 in der \glqq Resistance Is Useless\footnote{Widerstand ist zwecklos}\grqq -Serie (3F03 ist außerdem der Produktionscode dieser Episode).
  \item Der einzige, der beim Barbecue Gazpacho isst, ist Knecht Ruprecht.
  \item Linda und Paul McCartney waren erst einverstanden in der Episode aufzutreten, als die Produzenten ihnen versprachen, dass Lisa auch in späteren Folgen Vegetarierin bleibt.
  \item Apu ist ebenfalls Vegetarier. Er isst keine Produkte, die von Tieren stammen, wie beispielsweise Käse oder Eier. Aber dennoch ist er auf Homers Barbecue-Party zu sehen.
\end{itemize}
}
	
\subsection{Die Panik-Amok-Horror-Show}\label{3F04}
\begin{itemize}
	\item \textbf{Angriff der 15-Meter Reklamefiguren}\\ Als Homer Simpson einer Werbestatue einen riesigen Donut stiehlt, erwachen plötzlich alle Werbefiguren zum Leben und beginnen, das Städt\-chen Springfield zu zerstören. Das einzig wirksame Mittel gegen diese ist völliges Ignorieren.
	\item \textbf{Alptraum in Evergreen Terrace}\\  Nach einem Unfall mit dem Heizungskessel schwört Hausmeister Willie sterbend, sich an den Springfielder Kindern zu rächen und zwar in ihren Träumen.
	\item \textbf{Homer$^3$}\\ Homer entdeckt die dritte Dimension beim Versuch, sich vor einem Besuch von Patty und Selma zu verstecken. Schließlich stürzt er in ein \glqq Schwarzes Loch\grqq\ und landet dann in Beverly Hills in einer Mülltonne.
\end{itemize}

\notiz{
\begin{itemize}
  \item Professor Peanut ist eine Parodie auf das Maskottchen der Erdnussfirma Planters, Mr. Peanut.
  \item Viele Elemente von Barts ersten Traum -- der Hintergrund, die Musik, das Verhalten der Personen -- parodieren die Cartoons von Tex Avery.
  \item Fernsehzitat: Die ganze Geschichte basiert auf der Episode \glqq Little Girl Lost\grqq\ aus \glqq The Twilight Zone\grqq .
  \item Homer meint mit \glqq diesem Rollstuhltypen\grqq\ Prof. Dr. Steven Hawking, den an einen Rollstuhl gefesselten Wissenschaftler und Autor von \glqq Eine kurze Geschichte der Zeit\grqq\ (1988).
  \item Professor Frink bezeichnet den Würfel als \glqq Frinkahydron\index{Frinkahydron}\grqq\ nach seinem Entdecker.
  \item In der dritten Episode ist die Gleichung $1782^{12} + 1841^{12} = 1922^{12}$ zu sehen. Diese Gleichung ist gemäß dem großen Satz von Fermat\index{Fermat} falsch. David X. Cohen benutzte ein C-Programm um die Basen zu bestimmen (\cite{DavidXCohenC}). Das Programm ist in Abschnitt \ref{CProgramme} abgedruckt.
  \item Wenn die hexadezimale Zeichenfolge \glqq 46 72 69 6E 6B 20 72 75 6C 65 73 21\grqq\ als ASCII-Code interpretiert wird, steht diese für \glqq Frink rules!\grqq\ (siehe \cite{FrinkRules}).
  \item Die Teilepisode Homer$^3$ wurde im Rahmen einer Sendung über 3D-Grafik im März 1996 von Premiere im O-Ton mit Untertiteln ausgestrahlt.
\end{itemize}
}

	
\subsection{Der behinderte Homer}\label{3F05}
Damit Homer im Kernkraftwerk nicht an den Gymnastikübungen teilnehmen muss, versucht er, 300 Pfund auf die Waage zu bringen, um als behindert zu gelten. Dann darf er außerdem zu Hause arbeiten. Homer futtert sich also das benötigte Übergewicht an. Fortan bedient er das Kernkraftwerk von zu Hause aus, wenn er nicht gerade im Kino sitzt. Und genau in einem solchen Augenblick passiert es: Es kommt zu einem Zwischenfall -- der Reaktor droht zu explodieren.

\notiz{
\begin{itemize}
  \item Homer bringt 315 Pfund auf die Waage. Zu Beginn der Episode wog er noch 239 Pfund.
  \item Als es zum Zwischenfall im Kraftwerk kommt, ruft Homer: \glqq Marge, Lisa, Flanders\grqq\ (jeweils mit einer kurzen Pause zwischen den Namen).
\end{itemize}
}

	
\subsection{Wer ist Mona Simpson\index{Simpson!Mona}?}\label{3F06}
Homer täuscht seinen Tod vor, um Samstags nicht für Mr. Burns arbeiten zu müssen. Deshalb besucht Homers für tot geglaubte Mutter sein Grab. Dort begegnen sich beide. Sie ist in Wirklichkeit gar nicht tot, sondern nur auf der Flucht vor Mr. Burns. Sie hat in den Sechziger Jahren zur Hippie-Bewegung gehört und hatte mit anderen Mr. Burns biologisches Waffensystem zerstört. Schon bald muss sie sich wieder auf die Flucht begeben.

\notiz{
\begin{itemize}
	\item Aufschrift auf Homers Grabstein: \glqq Homer J. Simpson. We Are Richer for Having Lost Him\grqq\ (Unser Leben wurde bereichert, weil wir ihn verloren haben).
	\item Bart entdeckt bei seiner Großmutter folgende Ausweise (in Klammern ist der Bundesstaat angegeben, in dem der Ausweis ausgestellt worden ist bzw. ausgestellt worden sein soll):
	\begin{itemize}
		\item Mona Simpson, geb. 15.03.29 (Wisconsin)
		\item Mona Stevens\index{Stevens!Mona}, geb. 05.05.31 (Missouri)
		\item Martha Stewart\index{Stewart!Martha}, geb. 18.10.33 (Alaska)
		\item Penelope Olson\index{Olson!Penelope}, geb 18.07.33 (Ohio)
		\item Muddy Mae Suggins\index{Suggins!Muddy Mae}, geb. 27.02.29 (Tennessee) 
	\end{itemize}
\end{itemize}
}

	
\subsection{Das schwarze Schaf}\label{3F07}
Das neue Videospiel \glqq Bonestorm\index{Bonestorm}\grqq\ ist auf den Markt gekommen. Aber da es 60 Dollar kostet, kann Bart es sich nicht leisten. Seine Eltern weigern sich strikt, das Spiel zu kaufen. Bart versucht, das Spiel im Supermarkt zu stehlen und wird dabei vom Hausdetektiv erwischt. Daraufhin erhält er Hausverbot in dem Supermarkt. Als Marge ein neues Familienfoto schießen lassen will, wird Bart vom Hausdetektiv entdeckt und er zeigt der ungläubigen Familie das Überwachungsvideo, das Bart des Ladendiebstahls überführt. Es gibt Ärger und Bart fühlt sich nicht mehr geliebt. Aber zum Weihnachtsfest versöhnen sie sich wieder und Bart erhält als Weihnachtsgeschenk das Golf-Videospiel von Lee Carvallo\index{Carvallo!Lee}.

\notiz{
\begin{itemize}
  \item Auf den Socken der Kinder in der Erziehungsanstalt stehen Nummern anstatt Namen.
  \item Detektiv Don Brodka\index{Brodka!Don} hat eine Tätowierung des USMC (United States Marine Corps) auf seinem linken Arm.
  \item Detektiv Brodka sagt, dass Bart gegen \glqq das 11. Gebot -- \glq Du sollst nicht stehlen\grq \grqq\ verstoßen hat. In Wahrheit ist es das 8. Gebot.
  \item Ein Schild in der Erziehungsanstalt lautet: \glqq Proud Home of the Soap Bar Beating\grqq\ (Stolze Heimat der Seifenprügel).
  \item Diese Folge wurde in den Vereinigten Staaten zum sechsten Geburtstag der Simpsons ausgestrahlt.
  \item In der Videoabteilung des Supermarktes (Try-N-Save\index{Try-N-Save}) gibt es u.\,a. das Spiel \glqq Save Hitlers Brain\grqq\ (Rette Hitlers Gehirn).
  \item Obwohl die Episode im Winter spielt, trägt Bart seine kurze Hose, aber auch eine dicke Winterjacke.
\end{itemize}
}

	
\subsection{Tingeltangel-Bobs Rache}\label{3F08}
Tingeltangel-Bob erpresst die Bürger von Springfield mit einer Atombombe. Er will, dass alle Fernsehsender abgeschaltet werden. Bart und Lisa treten in Aktion und versuchen, zu retten, was zu retten ist. Glücklicherweise handelt es sich um eine nicht mehr funktionierende Atombombe. Tingeltangel-Bob nimmt Bart als Geisel. Er will mit einem alten Flugzeug fliehen. Wunderbarerweise rettet ein Absturz den beiden das Leben.

\notiz{
\begin{itemize}
  \item Filmzitat I: Colonel Hapablap\index{Hapablap} fragt Sideshow Bob im Original \glqq What's your major malfunction?\grqq\ (\glqq Was ist Ihre wesentliche Fehlfunktion\grqq ) -- genau wie der Drill Sergeant Hartman (gespielt von R. Lee Ermey, der in dieser Folge den Colonel spricht) in \glqq Full Metal Jacket\grqq .
  \item Filmzitat II: Der Krisenraum im Untergrund, in dem Quimby die Entscheidungen fällt, ist eine Anspielung auf \glqq Dr. Strangelove\grqq\ (Dr. Seltsam oder wie ich lernte, die Bombe zu lieben) von Stanley Kubrick (Kubrick war auch der Regisseur von \glqq Full Metal Jacket\grqq ).
  \item Tingeltangel-Bobs Gefangenennummer lautet: A 113.
  \item Im Hanger 18 ist ein Außerirdischer zu sehen.
  \item Die Flugshow findet am 25. November statt.
\end{itemize}
}

	
\subsection{Die bösen Nachbarn}\label{3F09}
Den Simpsons direkt gegenüber steht eine Villa, die unbewohnt ist. Eines Tages zieht der ehemalige Präsident George Bush sen. mit seiner Frau Barbara dort ein. Alle fühlen sich geehrt. Nur Homer ist empört: Er fühlt sich gestört. Als der Präsident Bart verprügelt, weil der im Hause Bush ein totales Chaos angerichtet hat und George Bushs Memoiren vernichtet hat, ist Homers Geduld zu Ende. Zwischen den beiden Streithähnen kommt es zu einer wüsten Schlägerei.

\notiz{
\begin{itemize}
	\item Beim Straßenverkauf verkaufen die Simpsons alte \glqq I Didn't Do It\grqq -T-Shirts, zuletzt gesehen in \glqq Bart wird berühmt\grqq\ (siehe \ref{1F11}); das \glqq Mary Worth\grqq -Telefon aus  \glqq Liebhaber der Lady B.\grqq\ (siehe \ref{1F21}); das \glqq Simpson und Sohn\grqq -Revitalisierungselixier aus \glqq Grandpa gegen sexuelles Versagen\grqq\ (siehe \ref{2F07}); eine Kopie des Albums \glqq Bigger Than Jesus\grqq\ von den Überspitzten aus \glqq Homer und die Sangesbrüder\grqq\ (siehe \ref{9F21}); den olmekischen Indianerkopf aus \glqq Der Lebensretter\grqq\ (siehe \ref{7F22}) und ein Gemälde von Ringo Starr aus \glqq Marges Meisterwerk\grqq\ (siehe \ref{7F18}).
	\item Mit George Bush joggen Dr. Hibbert, Lenny und Reverend Lovejoy.
	\item Michail Gorbatschow bringt George Bush eine Kaffeemaschine als Geschenk.
	\item Erster Auftritt von Disco Stu\index{Disco Stu}.
	\item Das Autokennzeichen von Gerald Fords Auto lautet \grqq MR DUH\grqq .
	\item Die Simpsons begegneten bereits in der Folge \glqq Einmal Washington und zurück\grqq\ (siehe \ref{8F01}) Barbara Bush im Weißen Haus.
	\item Diese Episode scheint eine Abrechnung mit George Bush sen. zu sein, da dieser sagte, dass die Simpsons ein schlechtes Beispiel für Familien seien und das sie sich eher an den Waltons orientieren sollten.
\end{itemize}
}

	
\subsection{Homers Bowling-Mannschaft}\label{3F10}
Homer will eine Bowling-Mannschaft gründen. Dazu braucht er 500 Dollar. Mr. Burns gibt sie ihm -- im Rausch. Später bereut er dies. Er tritt der erfolgreichen Mannschaft bei und siehe da, die Zeichen stehen auf Abstieg. Burns verliert jedes Spiel. Durch einen glücklichen Umstand macht er bei einem entscheidenden Spiel jedoch Punkte und die Mannschaft gewinnt den Meisterschaftstitel.

\notiz{
\begin{itemize}
  \item Barneys Onkel Al, zuletzt gesehen in \glqq Und Maggie macht drei\grqq\ (siehe \ref{2F10}), taucht auch in dieser Episode auf.
  \item Homer hat den Oscar von Dr. Haing S. Ngor gestohlen, der ihn 1984 für die beste Nebenrolle in \glqq The Killing Fields\grqq\ (Schreiendes Land) bekam. Er hat Dr. Ngors Namen durchgestrichen und seinen eigenen hineingeschrieben.
  \item In liebevoller Erinnerung an Doris Grau: Diese Episode wurde Doris Grau, der Sprecherin von Küchenhilfe Doris, der Chefin der Essensausgabe, gewidmet, die im Dezember 1995 verstorben ist.
  \item Moe behauptet, Carl, Lenny und Barney hätten Freundinnen und kämen deshalb nicht mehr so oft in sein Lokal.
  \item Die Bowlingmannschaft des Kanal 6 besteht aus Kent Brockman, Krusty, Arnie Pye und dem Bienenmann. Die Mannschaft \glqq Holy Rollers\grqq\ besteht aus Ned Flanders, Maude Flanders, Helen Lovejoy und Timothy Lovejoy. Der Polizeimannschaft gehören neben Chief Wiggum, Lou und Eddie noch Snake an. Snake nutzt seine Chance und flieht. Der Mannschaft der \glqq Homewreckers\grqq\ gehören Lurleen Lumpkin, Jacques, Mindy Simmons und Prinzessin Kashmir an.
  \item Fehler: Ralph Wiggum ist in Barts Klasse zu sehen.
\end{itemize}
}

	
\subsection{Eine Klasse für sich}
Grandpa hat den Fernseher ruiniert. Als man im Einkaufszentrum einen neuen kauft, ersteht Marge bei der Gelegenheit ein Kostüm von Chanel. Im Kwik-E-Mart trifft sie ihre alte Freundin Evelyn\index{Evelyn} wieder. Evelyn ist Mitglied im vornehmen Country-Club. Marge ist begeistert und will unbedingt auch Mitglied werden. Auf Homer und die Familie kommen ungeahnte Probleme zu. Von den immensen Kosten ganz zu schweigen.

\notiz{
Historische Anspielung: Marges Mädchenname ist ja bekanntlich Bouvier, genauso wie der von Jacqueline Kennedy. Am Tag des Attentats auf John F. Kennedy Jr. in Dallas trug Jacqueline Kennedy Bouvier ein rosa Chanel Kostüm, genau wie Marge in dieser Episode.
}

	
\subsection{Bart ist an allem Schuld}\label{3F12}
Bart Simpson will von Clown Krusty unbedingt ein Autogramm bekommen. Bart steckt ihm einen Scheck über 25 Cent in die Hosentasche. Dadurch kommt an das Tageslicht, dass Krusty die Finanzbehörden prellt und Steuern hinterzieht. Ein Skandal ohnegleichen. Um sich aus der Affäre zu ziehen, täuscht Krusty einen tödlichen Flugzeugabsturz vor. Dies wiederum ruft Bart auf den Plan, der den verschollenen Krusty suchen geht.

\notiz{
\begin{itemize}
  \item In dieser Episode enthüllt: Jimbo Jones\index{Jones!Jimbo} richtiger Vorname ist \glqq Corky\grqq\index{Corky}. Der Vorname von Rektor Skinners Mutter ist Agnes.
  \item Unter den Tausenden bei Krustys Beerdigung sind Don King (oder seine Simpsons-Parodie Lucius Sweet), David Crosby, Kermit der Frosch und Rainier Wolfcastle.
  \item Der Text auf einem Trauerkranz lautet: \glqq Krusty -- You can never be replaced. Laffs, 369-3084\grqq\ (Krusty -- Du bist nicht zu ersetzten. Witze, 369-3084).
  \item Nachdem Krusty seinen Tod vorgetäuscht hat, benutzt er den Namen Rory B. Bellows\index{Bellows!Rory B.}.
\end{itemize}
}

	
\subsection{Das geheime Bekenntnis}\label{3F13}
Lisa recherchiert für einen Aufsatz. Dabei findet sie Ungeheuerliches heraus: Stadtgründer Jebediah Springfield\index{Springfield!Jebediah} war in Wirklichkeit der bösartige Pirat Hans Sprungfeld\index{Sprungfeld!Hans}. Lisa fühlt sich der Wahrheit verpflichtet und schreibt dies in ihrem Aufsatz. Ein Sturm der Entrüstung bricht los. Lisa versucht, die Wogen zu glätten und kommt noch dem betrügerischen Museumsdirektor Hollis Hurlbutt\index{Hurlbutt!Hollis} auf die Schliche. Die Stadt Springfield sollte ihr dankbar sein.

\notiz{
\begin{itemize}
  \item Jebediah Springfields Geständnis: Geheime Aufzeichnungen von Jebediah Springfield. Wer das hier liest, weiß mehr über mich als sämtliche Historiker. Zunächst mal habe ich den legendären Büffel gar nicht gezähmt, er war schon zahm, ich habe ihn lediglich erschossen. Außerdem hieß ich nicht von Anfang an Jebediah Springfield, bis 1796 hieß ich Hans Sprungfeld und war ein mörderischer Pirat. Aber diese Halbgebildeten hier werden die Wahrheit nie erfahren. Ich schreibe dieses Bekenntnis, damit meine Niedertracht noch weiterlebt, wenn mein Körper schon längst der infektiösen Diphtherie erlegen ist.
  \item Bei einer Spelunkenschlägerei wurde Jebediah Spring\-field von einem Tür\-ken die Zunge abgebissen. Seitdem hatte er eine kostbare Zungenprothese aus Silber.
  \item In dieser Episode gibt es den ersten Hinweis auf Kearneys relativ hohes Alter, als er von seinen Erinnerungen an die Feier '76 erzählt. In späteren Episoden wird gezeigt, dass er einen Sohn hat (\glqq Scheide sich, wer kann\grqq , \ref{4F01}), ein Auto besitzt (\glqq Lisa will lieben\grqq , \ref{4F01}) und alt genug ist, um Alkohol zu trinken (\glqq Homers merkwürdiger Chili-Trip\grqq , \ref{3F24}).
  \item Die Stadt Springfield wird 200 Jahre alt.
\end{itemize}
}

	
\subsection{Butler bei Burns}\label{3F14}
Mr. Burns schickt Smithers in den wohlverdienten Urlaub. Als dessen Vertreter engagiert er Homer Simpson -- doch der kann Mr. Burns nichts rechtmachen. Als er die ewigen Nörgeleien seines Chefs nicht mehr aushält, reißt Homer der Geduldsfaden: Er schlägt Mr. Burns nieder. Darauf muss dieser alleine zurechtkommen, was dann auch so gut klappt, dass er Smithers entlässt. Nun will Homer helfen, damit der Arbeitslose seinen Posten zurückbekommt.

\notiz{
\begin{itemize}
	\item Smithers benutzt einen Macintosh Computer.
	\item Homer behauptet, Mr. Burns sei 104 Jahre alt.
	\item Waylon Smithers sagt, Mr. Burns Mutter sei 122 Jahre alt. Diese hatte eine Affäre mit dem US-Präsidenten William Howard Taft\footnote{William Howard Taft (geboren am 15. September 1857 in Cincinnati, Ohio; gestorben am 8. März 1930 in Washington D. C.) war 27. Präsident der Vereinigten Staaten von Amerika vom 4. März 1909 bis 3. März 1913. Er gehörte der Republikanischen Partei an \cite{Taft}.}.
	\item Die Telefonnummer von Moes Taverne lautet 636-76484377.
	\item Die Telefonnummer von Mr. Burns Büro ist die 636-5555246.
	\item Nachdem Waylon Smithers entlassen worden ist, arbeitet er vorübergehend beim Klaviertransporteur \glqq Neat \& Tidy\grqq \index{Neat \& Tidy}.
	\item Homer schlägt Mr. Burns genauso K. o. wie Jake LaMotta den Boxer Janiro in \glqq Wie ein wilder Stier\grqq .
\end{itemize}
}
	
\subsection{Selma heiratet Hollywoodstar}\label{3F15}
Selma hat den Hollywoodstar Troy McClure kennen gelernt und sich auf der Stelle in ihn verliebt. McClure heiratet auf Anraten seines Agenten Arthur McParker\index{McParker!Arthur} schließlich Selma. Seine Gründe für die Hochzeit sind allerdings nicht ganz uneigennützig: Er hofft, damit seine Karriere wiederzubeleben. Sein Agent ist zudem der Meinung, dass die beiden ein Kind bekommen sollten. Dabei versagt der Schauspieler kläglich. Enttäuscht gibt ihm Selma den Laufpass.

\notiz{
\begin{itemize}
	\item Bei den Prominentenfotos an der Wand des Pimento Grove\index{Pimento Grove} (Olivenhain) hängen auch Ex-Simpsons-Autor Conan O'Brien\index{O'Brien!Conan} und Birch Barlow\index{Barlow!Birch} (beide zu sehen in \glqq Filmstar wider Willen\grqq , siehe \ref{2F17}).
	\item Selma und Troy gehen auch in das Restaurant Ugli\index{Ugli}.
\end{itemize}
 }

	
\subsection{Wer erfand Itchy und Scratchy?}\label{3F16}
Der alte Chester J. Lampwick\index{Lampwick!Chester J.} fordert von Produzent Roger Myers jun.\index{Myers!Roger jun.} 800 Millionen Dollar Ent\-schä\-di\-gung: Er sei der wahre Erfinder der beliebten Zeichentrickfiguren Itchy und Scratchy. Mit Barts Hilfe und einer Zeichnung kann dies auch bewiesen werden. Das bereut Bart schon bald: Weil der Produzent nun nicht weitermachen kann, müssen Bart und Lisa auf ihre Lieblingssendung verzichten. Das ist natürlich kein Dauerzustand.

\notiz{
\begin{itemize}
  \item Zur Bevölkerung des Pennerviertels gehört auch Charlie Chaplin.
  \item Roger Myers wohnt im Worst Western Hotel -- eine Anspielung auf die Hotelkette \glqq Best Western\grqq . 
  \item Lester\index{Lester} und Eliza\index{Eliza} sehen aus wie Bart und Lisa in den alten Simpsons-Shorts in der \glqq Tracey Ullman Show\grqq\ von 1987 - 1989.
  \item Krusty sagt, Lester und Eliza hätten ihn mit seiner durchgebrannten Frau wieder zusammengebracht.
  \item Der olmekische Indianerkopf Xt'Tapalataketel\index{Xt'Tapalataketel}, den Mr. Burns den Simpsons in \glqq Der Lebensretter\grqq\ (siehe \ref{7F22}) geschenkt hat, steht immer noch im Keller.
  \item Roger Myers sen.\index{Myers!Roger sen.} gründetet seine Firma 1921.
  \item Der Film von Chester, in dem Itchy zu sehen ist, stammt aus dem Jahr 1919.
  \item Kirk Douglas, welcher Chester J. Lampwick seine Stimme lieh, bestand darauf, alle Szenen nur einmal zu sprechen.
\end{itemize}
}

	
\subsection{Die Reise nach Knoxville}\label{3F17}
Bart erstellt sich in der Führerscheinstelle einen Führerschein. Martin begleitet seinen Vater an die Börse und macht bei Spekulationen 600 Dollar Gewinn. So ausgestattet mieten die Freunde Bart, Martin, Nelson und Milhouse einen Leihwagen. Sie wollen zusammen nach Knoxville fahren, um dort die Sonnenkugel auf der Weltausstellung zu besichtigen. Klar, dass der Trip nicht ganz ohne Komplikationen verläuft.

\notiz{
\begin{itemize}
  \item In Barts Führerschein steht unter Geburtstag 11.02.1970, als Größe 1,22 m, als Gewicht 39 kg und unter Augenfarbe \glqq BL\grqq\ (wahr\-schein\-lich \glqq blau\grqq ).
  \item Das Lied, das Bart und die Jungs im Auto hören, ist \glqq Radar Love\grqq\ von Golden Earring aus dem Jahr 1975.
  \item Die Grundschule in Springfield liegt in der Plympton Street 19.
\end{itemize}
}

	
\subsection{22 Kurzfilme über Springfield}\label{3F18}
Springfield ist eine überaus ereignisreiche Stadt. Dort kann wirklich jedermann ganz besondere Erlebnisse haben. Dass dem so ist, beweisen 22 Kurzfilme: Apu zum Beispiel kann nur fünf Minuten seinen Laden schließen, um auf eine Party zu gehen, Homer sperrt Maggie im Zeitungskasten ein und Dr. Nick Riviera werden falsche Behandlungsmethoden vorgeworfen. Das ist längst nicht alles, was in Springfield passiert.

\notiz{
\begin{itemize}
  \item Filmzitat I: Der Episodentitel \glqq 22 Short Films about Springfield\grqq\ ist eine Anspielung auf \glqq Thirty-Two Short Films about Glenn Gould\grqq .
  \item Filmzitat II: Die Diskussion über McDonald's und die Szene mit Herman parodieren Quentin Tarantinos \glqq Pulp Fiction\grqq .
  \item In dieser Episode hat die Capital-City-Knalltüte ihren dritten Auftritt. Zuletzt sahen wir sie in \glqq Krusty, der TV-Star\grqq\ (siehe \ref{9F19}).
  \item Es gibt mehr als 2000 McDonalds in dem Staat, in dem Springfield liegt, aber nicht einen direkt in Springfield.
  \item Der Chihuahua in dieser Episode war auch schon in \glqq Barts Komet\grqq\ (siehe \ref{2F11}) und in \glqq Filmstar wider Willen\grqq\ (siehe \ref{2F17}) zu sehen.
  \item Uter gibt an, dass sein Vater bei den Kaugummiwerken in Basel beschäftigt ist.
  \item Dieses ist eine von nur vier Simpsons-Episoden, bei denen im Original der Titel eingeblendet wurde. Die anderen waren \glqq Bart köpft Ober-Haupt\grqq\ (siehe \ref{7G07}), \glqq Bart kommt unter die Räder\grqq\ (siehe \ref{7F10}) und \glqq Die 138. Episode, eine Sondervorstellung\grqq\ (siehe \ref{3F31}).
  \item Snake gibt nach dem Überfall in Moes Taverne an, mit der Beute endlich sein Studentendarlehen zurückzahlen zu können. In der späteren Episode \glqq Die scheinbar unendliche Geschichte\grqq\ (siehe \ref{HABF06}) wird enthüllt, dass er Archäologe ist.
  \item Oberschulrat Chalmers stammt aus Utica\index{Utica}.
  \item Im Vorspann taucht der Name von Matt Groening nicht, da er das Crossover zu The Critic nicht wollte. Er fand, dass die Folge \glqq die Integrität der Simpsons beeinträchtigt\grq\ (siehe \cite{Reiss19}).
\end{itemize}
}

	
\subsection{Simpson und sein Enkel in \glqq Die Schatzsuche\grqq}\label{3F19}
Abe Simpson war im Zweiten Weltkrieg Sergeant einer Sondereinheit (\glqq Fighting Hellfish\grqq\index{Fighting Hellfish}), zu der auch Mr. Burns zwangsversetzt wurde. Sie haben gegen Kriegsende eine wertvolle Ge\-mäl\-de\-samm\-lung gefunden. In einer wasserdichten Truhe hatte man die Sammlung im Meer versenkt, um sie später zu bergen. Nun sind bis auf diese beiden alle Mitglieder der Einheit verstorben. Mr. Burns will den Schatz heben und zwar für sich. Die Simpsons sind dagegen und heben den Schatz selbst. Mr. Burns will ihnen die Gemälde abjagen, aber schließlich schaltet sich das amerikanische Außenministerium ein und gibt die Bilder an einen Nachfahren der ursprünglichen Besitzer zurück.

\notiz{
\begin{itemize}
	\item Obwohl Grandpas Einheit \glqq Die fliegenden Höllenfische\grqq\ genannt wurde, stehen auf dem Höllenfisch-Papier und dem Denkmal auf dem Friedhof \glqq Fighting Hellfish\grqq\ (Kämpfende Höllenfische).
	\item Nelsons Großvater ist Richter und verhängte bisher 47 Todesurteile.
	\item Neben Abe Simpsons und Charles Montgomery Burns gehörten noch Sheldon Skinner\index{Skinner!Sheldon}, Arnie Gumble\index{Gumble!Arnie}, Asa Phelps\index{Phelps!Asa}, Iggy Wiggum\index{Wiggum!Iggy}, Milton \glqq Ox\grqq\ Haas\index{Haas!Milton}, Etch Westgrin\index{Westgrin!Etch} und Griff McDonald\index{McDonald!Griff} den \glqq Fighting Hellfish\grqq\ an.
	\item Mr. Burns heuert den Killer Fernando Vidal\index{Vidal!Fernando} an, um Abe Simpsons zu töten.
\end{itemize}
}

	
\subsection{Volksabstimmung in Springfield}\label{3F20}
Als sich ein Bär in die Evergreen Terrace verirrt, wird der Ruf nach einer Bärenpatrouille laut. Die Kosten dafür werden in einer fünf Dollar hohen Steuererhöhung an die Bürger weitergegeben. Die Bürger protestieren dagegen und der Bürgermeister schiebt die Schuld auf die illegalen Emigranten, deshalb sollen die illegalen Einwanderer aus dem Städtchen Springfield abgeschoben werden. Um dies durchsetzen zu können, wird eine Volksabstimmung durchgeführt. Von der Abschiebungsmaßnahme wäre auch der Inder Apu betroffen. Apu lebt nun schon seit sieben Jahren in den USA mit abgelaufenem Visum. Apu besorgt sich falsche Papiere. Doch dann hat die kleine Lisa Simpson eine wesentlich bessere Idee.

\notiz{
\begin{itemize}
  \item Der Originaltitel \glqq Much Apu about Nothing\grqq\ ist eine Anspielung auf Shakespeares \glqq Much Ado About Nothing\grqq\ (Viel Lärm um nichts).
  \item Apu schläft auf Homers geöffnetem Geschichtsbuch ein; als er aufwacht, ist es geschlossen.
  \item Schild vor dem Gebäude der Einwanderungsbehörde: \glqq The United States: 131 Years Without a Civil War\grqq\ (Die Vereinigten Staaten: Seit 131 Jahren ohne Bürgerkrieg).
  \item Apu promovierte in Philosophie.
  \item Personen, die den Einwanderungstest machen: Akira, Apu, Luigi, Bumblebee Man (Bienenmann), Dr. Nick Riviera und Moe Szyslak.
  \item Apus gefälschter Ausweis ist auf Apu Nahasapeemapetilan\index{Nahasapeemapetilan!Apu}, geboren am 09. Januar 1962 im Bundesstaat Wisconsin, ausgestellt.
  \item Nachdem Apu den Einbürgerungstest bestanden hatte, wird er zum Geschworenen berufen.
\end{itemize}
}

\subsection{Homer auf Tournee}
Anlässlich eines Musikfestivals in Springfield schwelgt Homer in Erinnerungen an seine Rock-'n'-Roll-Zeit. Schließlich wird er vom Veranstalter einer Freak-Show engagiert. Bei den Auftritten muss er sich eine Kanonenkugel auf seinen dicken Bauch schießen lassen. Dabei fällt er zwar um, überlebt aber. Mit dieser Truppe geht Homer sogar auf eine Tournee, von der er ziemlich geplättet zurückkommt.

\notiz{
\begin{itemize}
  \item Homers stolzierender Gang parodiert die berühmte Zeichnung \glqq Keep on truckin'\grqq\ des Underground-Cartoonisten Robert Crumb.
  \item In dieser Folgen wirken u.\,a. Cypress Hill, Peter Framton, The Smashing Pumpkins und Sonic Youth mit.
\end{itemize}
}


	
\subsection{Ein Sommer für Lisa}\label{3F22}
Die Nachbarn der Simpsons, die Flanders, machen dieses Jahr keinen Urlaub. Sie bieten ihr Strandhaus den Simpsons an. Erfreut nimmt die Familie an. Bart darf Milhouse mitnehmen. Daraufhin stellt Lisa fest, dass sie keine Freunde hat. Am Strand versucht sie, dies zu ändern: Sie sucht Freundschaften, indem sie sich als ausgeflippte Schülerin ausgibt. Das nimmt Bart natürlich nicht so einfach hin.

\notiz{
\begin{itemize}
  \item Das Strandhaus der Flanders liegt in Little Pwagmattassquarmsettport.
  \item Leistungen, die unter Lisas Jahrbuchbild stehen: \glqq Junior Overachievment; Record for Most Handraises in a Single Semester (763); Most Popular Student's Sister; Spelling Bee Queen; Camera Club; Tidiest Locker\grqq\ (Junior Übereifer; Rekordhalterin für die häufigsten Meldungen in einem Schuljahr (763); Schwester des beliebtesten Schülers; Buchstabierkönigin; Kamera-Club; Sauberster Schrank).
  \item In dieser Episode wurde zum zweiten mal jemandem mit der Wasserpistole an einem Jahrmarktsstand, an dem man eigentlich mit der Pistole auf Köpfe schießt und mit dem Wasser Ballons füllt, ins Gesicht gespritzt. Lisa bespritzt Bart, so wie Nelson Martin in \glqq Lisa, die Schönheitskönigin\grqq\ (siehe \ref{9F02}), nassgespritzt hat.
  \item Im Hinterzimmer des Supermarktes steht eine Bombe mit der Aufschrift \glqq Bangtime fun bomb\grqq\ (Ein bombiges Vergnügen).
  \item Lisa behauptet, Bart sei ihr kleiner Bruder. In Wirklichkeit ist er natürlich ihr großer Bruder.
  \item Christina Ricci nahm ihre Sprechrolle nicht im Studio auf sondern über Telefon.
  \item Als Bart Lisas neuen Freunden das Jahrbuch zeigt, verweist er auf einen Artikel in dem Lisa zur Gewinnerin des Grammatik-Rodeos gekürt wird. In der Episode \glqq Die Reise nach Knoxville\grqq\ (siehe \ref{3F17}) wurde aber nur ein angeblich stattfindendes Grammatik-Rodeo von Bart als Ausrede benutzt, um mit Freunden nach Knoxville fahren zu können. 
\end{itemize}
}

	
\subsection{Die 138. Episode, eine Sondervorstellung}\label{3F31}
Moderator Troy McClure stellt heute in einem Rückblick alte Simpsons-Folgen vor. Dabei stehen Szenen im Mittelpunkt, die herausgeschnitten wurden, weil die Episode zu lang war oder zu aufregend. Sechs Jahre \glqq Simpsons\grqq\ -- bei einem derartigen Rückblick erfährt man auch einiges über die Produzenten und ihre Entscheidungen.

\notiz{
\begin{itemize}
  \item Matt Groening wird als alter, kahlköpfiger Mann mit Augenklappe dargestellt. In seinem Büro hängt ein Poster seiner Cartoonreihe \glqq Life in Hell\grqq .
  \item Die Darstellung von James L. Brooks ähnelt Mr. Pennybags, dem kapitalistischen Markenzeichen der Originalversion von Monopoly.
  \item Sam Simon mit seinem grotesken Fingernägeln und hagerem Profil erinnert an den Milliardär Howard Hughes in seiner exzentrischsten Zeit.
  \item Auf dem Zettel, der aus Homers Bowlingkugel-Kopf fällt, steht \glqq IOU one brain, signed, God\grqq\ (Ich schulde dir ein Gehirn, gezeichnet, Gott).
  \item Es existiert die Behauptung, dass auf der Registrierkasse, über welche Maggie gezogen wird, \glqq NRA4EVER\grqq\ steht. Eigentlich wird der Betrag 847,63 US-Dollar angezeigt.
\end{itemize}
}


\section{Staffel 8}

\subsection{Die Akte Springfield}
Homer behauptet steif und fest, einen Außerirdischen gesehen zu haben, aber niemand glaubt ihm. Als sich schließlich bei der Suche nach dem geheimnisvollen Alien auch noch das Fernsehen einschaltet, stellt sich heraus, dass es sich dabei um Mr. Burns handelt. Mit dem dringenden Wunsch, sein Leben zu verlängern, hat er sich einer obskuren medizinischen Behandlung unterzogen und leuchtet nun.

\notiz{
\begin{itemize}
  \item Filmzitate: Milhouse steckt 40 Quarters (Vierteldollar) in das Videospiel, \glqq Kevin Cost\-ners Wa\-ter\-world\grqq, was auf die immensen Kosten des Kinoflops Waterworld anspielt. In dieser Episode gibt es außerdem Anspielungen auf \glqq The Shining\grqq , \glqq E.T.\grqq\ und \glqq Unheimliche Begegnung der dritten Art\grqq .
  \item Leonard Nimoy war zuletzt in \glqq Homer kommt in Fahrt\grqq\ (siehe \ref{9F10}) zu sehen.
  \item Moes Alkoholtester reicht von \glqq Tipsy\grqq\ (Beschwipst) über \glqq Soused\grqq\ (Angetrunken) über \glqq Stinkin'\grqq\ (Stinkbesoffen) bis zu \glqq Boris Jelzin\grqq .
  \item Schlagzeile im Springfielder Einkaufsboten über Homers UFO-Sichtung: \glqq Human Blimp sees Flying Saucer\grqq\ (Menschlicher Zeppelin sieht fliegende Untertasse).
  \item Homer und Knecht Ruprecht haben am gleichen Tag Geburtstag.
  \item Die Videokamera, die Bart zur Aufzeichnung des \glqq Außerirdischen\grqq\ benutzt, gehört Ned Flanders und wurde ihm zum Geburtstag geschenkt.
  \item Fehler: Marge präsentiert sich vor Homer mit ihrem neuen T-Shirt \glqq Homer ist ein Dummkopf\grqq . Schon in der nächsten Szene trägt sie ihre normale Kleidung.
\end{itemize}
}

	
\subsection{Das magische Kindermädchen}\label{3G03}
Marge leidet unter dem Stress im Haushalt. Ein ernstes Symptom: Marge fallen die Haare aus und das nicht zu knapp. Zu ihrer Entlastung soll ein Kindermädchen engagiert werden. So erscheint plötzlich Shary Bobbins\index{Bobbins!Shary}. Sie will den Simpsons Ordnung und Benehmen beizubringen, das ist allerdings nicht einfach. Sie versagt kläglich. Ihr wird klar, dass die Simpsons lieber so bleiben wollen, wie sie sind. Und so verschwindet Shary Bobbins ebenso schnell, wie sie gekommen ist.

\notiz{
\begin{itemize}
  \item Hinweis zum Tafelspruch\footnote{Ich darf nicht die Prozac meines Lehrers verstecken.} von Bart: \glqq Prozac\grqq\ ist ein Antidepressivum, das in den USA weit verbreitet ist.
  \item Krustys Komedy Klassics hat die Anfangsbuchstaben KKK, eine Anspielung auf den Ku-Klux-Klan\footnote{Der Ku-Klux-Klan (ist ein noch heute bestehender, rassistischer Geheimbund in den Südstaaten der USA.}.
  \item Hausmeister Willie und Shary Bobbins waren verlobt.
  \item In dieser Episode hat Präsident Gerald Ford seinen zweiten Auftritt; der erste fand in \glqq Die bösen Nachbarn\grqq\ (siehe \ref{3F09}) statt.
  \item In den Original-Simpsons-Drehbüchern wird Homers \glqq NEIN!!!\grqq\ (D'Oh!) immer als \glqq Annoyed Grunt\grqq\ (Verärgertes Grunzen) geschrieben -- daher der Originaltitel \glqq Simpsoncalifragilisticexpiala (Annoyed Grunt) Cious\grqq .
  \item Filmzitat I: Reservoir Dogs -- In der Itchy \& Scratchy-Folge \glqq Reservior Cats\grqq\ foltert Itchy den an einem Stuhl gefesselten Scratchy, indem er ihn tanzend mit Benzin übergießt und anschließend mit einem Rasiermesser das rechte Ohr abschneidet. Zudem taucht Quentin Tarantino als Regisseur auf und es wurde das für ihn typische silberne Tape verwendet.
  \item Filmzitat II: Pulp Fiction -- Itchy und Scratchy tanzen wie Vincent Vega und Mia Wallace im \glqq Jackrabbit Slim's\grqq\ Restaurant zu dem Lied \glqq Miserlou\grqq , welches auch das Er\-öff\-nungs\-the\-ma von Pulp Fiction ist. Sie sind angezogen wie Jules Winnfield und Vincent Vega. 
  \item Maggie Roswell erhielt, für ihre Darbietung der Shary Bobbins, einen Emmy.
\end{itemize}
}

	
\subsection{Das verlockende Angebot}
Homer soll im wunderschönen Cypress Creek \index{Cypress Creek} bei der Firma Globex\index{Globex} seine Atom\-kraft\-werk-Erfahrung einsetzen, um Mr. Scorpio\index{Scorpio!Hank} eine Atomkanone zu bauen. Scorpio will damit à la \glqq Goldfinger\grqq\ die Weltherrschaft gewinnen. Aber Marge, Lisa und Bart gefällt es nicht in Cypress Creek. Sie bitten Homer, mit ihnen nach Springfield zu\-rück\-zu\-zie\-hen. Der willigt ein und bekommt von Scorpio zum Dank die Football-Mannschaft \glqq Denver Broncos\grqq\ geschenkt.

\notiz{
\begin{itemize}
  \item Homers Lebenstraum: Besitzer der \glqq Dallas Cowboys\grqq .
  \item Die neue Adresse der Simpsons: 15201 Ahornanlagenstraße (Maple System Road).
  \item Der Sandstrahler und der TV-Tisch, die Homer sich von Flanders ausgeliehen hat, erschienen ursprünglich in \glqq Die rebellischen Weiber\grqq\ (siehe \ref{1F03}) und \glqq Am Anfang war das Wort\grqq (siehe \ref{9F08}).
  \item Die Internetadresse der Grundschule von Cypress Creek lautet \url{http://www.studynet.edu} (existiert nicht wirklich).
  \item Homer kauft seinen von Trainerlegende Tom Landry (Dallas Cowboys) unterschriebenen Hut in einem Laden namens \glqq The Spend Zone\grqq\ (Die Konsumzone).
  \item Homer ist unglücklich mit Scorpios Geschenk, den Denver Broncos, obwohl er John Elway (aus eben diesem Team) sein wollte, als seine Familie eine neue Identität im Rahmen eines Zeugenschutzprogramms (\glqq Am Kap der Angst\grqq , \ref{9F22}) bekommen sollte.
\end{itemize}
}
	
\subsection{Homers merkwürdiger Chili-Trip}\label{3F24}
Chili-Festival in Springfield: Wiggum hat besonders scharfe guatemaltekische Pfefferschoten, die Homer fast um den Verstand bringen. Er halluziniert: In seinen Fantasieträumen erscheint ihm ein Kojote, der ihn auffordert, seinen Seelenpartner zu finden. Diesen sieht der gute Homer eindeutig in Marge. Aber da Marge ihm immer bittere Vorwürfe macht, meint Homer, sich einen neuen suchen zu müssen.

\notiz{
\begin{itemize}
  \item Filmzitat: Die Musik aus \glqq The good, the bad and the ugly\grqq\ (Zwei glorreiche Halunken) wird in den Szenen gespielt, in denen Homer übers Chili Festival geht und Chief Wiggum an seinem Stand trifft.
  \item Fernsehzitat: Das Ersetzen des Mannes im Leuchtturm durch den Computer EARL\footnote{Electric Automatic Robotic Lighthouse}\index{EARL} parodiert eine alte Episode aus \glqq The Twilight Zone\grqq , in der sich ein Mann in einer Höhle als Computer herausstellt.
  \item Die lila Stereo-Anlage der Simpsons ist eine DisCabinetron 2000.
  \item Ein Vagabund verkauft beim Chili Kochfest Wasser für 3 Dollar pro Flasche.
  \item Neben Chief Wiggum bieten noch Frink (virtuelles Chili), Mr. Burns (im Yale-Stil), Ned Flanders, Nelson Muntz und Moe Szyslak Chili auf dem Kochfest an.
\end{itemize}
}
	
\subsection{Lisa will lieben}\label{4F01}
Lisa hat etwas übrig für Nelson. Sie glaubt, dass er tief in seinem Innern ein netter Kerl ist. Durch Liebe versucht sie, ihn zu ändern. Nelson soll sein wahres Ich anerkennen und dazu stehen. Aber die Erwartungen des Mädchens werden böse enttäuscht. Währenddessen versucht Homer, durch einen ganz besonderen Trick reich zu werden. Auch er wird schwer enttäuscht: Die Polizei legt ihm das Handwerk.

\notiz{
\begin{itemize}
  \item Oberschulrat Chalmers fährt einen 1979er Honda Accord.
  \item In dieser Episode erfahren wir folgende Telefonnummer:
  \begin{itemize}
	  \item 555-0001: Monty Burns
	  \item 555-5782: Prof. Frink
	  \item 555-8904: Ned Flanders
  \end{itemize}
  \item Der Wählautomat AT-5000\index{AT-5000} ist Prof. Frinks erstes Patent.
  \item Das Konzertprogramm des Springfielder Schulorchesters umfasst folgende Lieder:
  \begin{itemize}
	  \item Mary Had a Little Lamb
	  \item Pop Goes the Weasel
	  \item Jimmy Crack Corn
  \end{itemize}
  \item Lisa muss zur Strafe, weil sie über Nelsons Streich gelacht hat, \glqq I Will Not Be A Snickerpuss\grqq\ an die Tafel schreiben.
  \item In dieser Folge fährt Kearney einen Hundai.
  \item Nelson nennt Jimbo \glqq James\grqq .
  \item Lisa gibt Nelson ihren ersten Kuss.
\end{itemize}
}

	
\subsection{Hugo, kleine Wesen und Kang}\label{4F02}
\begin{itemize}
	\item \textbf{Das Ding und ich}\\ Bart erfährt, dass er einen Zwillingsbruder hat. Da dieser ziemlich bösartig war, mussten die beiden nach der Geburt getrennt werden.
	\item \textbf{Der Schöpfungs-Napf}\\ Lisas wissenschaftliches \glqq Zähne-in-Cola-aufgelöst\grqq -Experiment entwickelt sich zu einer hö\-her\-en Lebensform, die schließlich Lisa als Gott und Bart als den Teufel ansieht.
	\item \textbf{Citizen Kang}\\ Als die Präsidentschaftswahl näher rückt, entführen die Außerirdischen Kang und Kodos Bob Dole und Bill Clinton und nehmen ihre Plätze ein. Jetzt liegt es an Homer, ihren fürchterlichen Plan zur Erlangung der Weltherrschaft aufzudecken.
\end{itemize}

\notiz{
\begin{itemize}
  \item Auf dem Dachboden sind folgende Gegenstände zu sehen: Das Mary Worth Telefon, das Bart im Tausch für seine schlechte \glqq Itchy \& Scratchy\grqq -Trick\-film\-zeich\-nung bekommen hat (\glqq Liebhaber der Lady B.\grqq , \ref{1F21}); ein \glqq I Didn't Do It\grqq -T-Shirt (\glqq Bart wird berühmt\grqq , \ref{1F11}); eines von Marges Ringo-Starr-Bildern (\glqq Marges Meisterwerk\grqq , \ref{7F18}); mehrere Schachteln der Lisa-Löwenherz-Puppe (\glqq Lisa kontra Malibu Stacy\grqq , \ref{1F12}); eine Kiste mit Artikeln der Überspitzten (\glqq Homer und die Sangesbrüder\grqq , \ref{9F21}); der Spine Melter 2000 (\glqq Der vermisste Halbbruder\grqq , \ref{8F23}) und Barts elektrische Gitarre (\glqq Der Fahrschüler\grqq , \ref{8F21}).
  \item Homers Autobiographie heißt \glqq Homer, ich hab mich kaum gekannt\grqq .
  \item Fernsehzitat: Die Story wurde inspiriert durch eine \glqq Twilight Zone\grqq -Epi\-so\-de, in der Astronauten auf einem Asteroiden landen, um ihr Raumschiff zu reparieren und einer von ihnen entdeckt dort eine winzige Zivilisation, als deren Gott er sich aufspielt.
  \item Lisa ist am Ende der Episode barfuß, obwohl sie Slipper getragen hat, als sie in die Napfwelt gebeamt wurde.
  \item Kang und Kodos Wahlkampfdebatte kostet 5 Dollar Eintritt.
  \item Bob Dole hält einen Stift in seiner \glqq schlechten\grqq\ Hand als er vom rigelischen Raumschiff entführt wird.
  \item Die einzige, die bei der Debatte nicht jubelt und applaudiert, ist Lisa.
\end{itemize}
}
	
\subsection{Auf in den Kampf}\label{4F03}
Homers Gehirn ist in einer übergroßen Flüssigkeitsmenge eingelagert. Dies hat zur Folge, dass er Schläge auf den Kopf ertragen kann, ohne ohnmächtig zu werden. Moe ist davon begeistert. Er bildet Homer zum Boxen aus. Die Strategie: Homer lässt sich schlagen, bis der Gegner müde wird. Dann kann er ihn einfach umschubsen. Beim Kampf um die Schwergewichtsweltmeisterschaft sieht es allerdings anders aus.

\notiz{
\begin{itemize}
  \item Marge sagt, dass Homer 38 Jahre alt sei.
  \item Homer boxt mit Unterstützung der ASSBOX\index{ASSBOX} (Association of Springfield Semi-Pro Boxers).
  \item Drederick Tatum\index{Tatum!Drederick} wurde in \glqq Das achte Gebot\grqq\ (siehe \ref{7F13}) vorgestellt. Seine Gefangenennummer ist die 4F03, genau wie der Produktionscode dieser Episode.
  \item Drederick Tatum und Lucius Sweet\index{Sweet!Lucius} dürften Anspielungen auf Mike Tyson und den Boxpromoter Don King sein.
  \item Fat Tonys Name wird mit Anthony D'Amico\index{D'Amico!Anthony} angegeben. In der Episode \glqq Verbrechen lohnt sich nicht\grqq\ (siehe \ref{8F03}) hieß er allerdings noch Anthony Williams.
  \item Tatums Einmarschmusik ist \glqq Time 4 Sum Akaison\grqq\ von Redman und Homers Lied ist \glqq Why Can't We Be Friends\grqq\ von War.
\end{itemize}
}

	
\subsection{Scheide sich, wer kann}\label{4F04}
Marge Simpson hat sich Gäste geladen. Die Vorbereitungen für den Abend erfordern größte Anstrengungen. Im Laufe des Abends kommt es dann zum Streit zwischen Kirk und Luann Van Houten. Die beiden trennen sich daraufhin nach vielen Ehejahren. Aus Angst, dass ihm das auch widerfahren könnte, lässt Homer sich scheiden, um Marge erneut zu heiraten. Er möchte, dass sie mit ihm im Siebten Himmel schwebt.

\notiz{
\begin{itemize}
  \item Die Simpsons kaufen ihr Partyzubehör im \glqq Stoner's Pot Palace\index{Stoner's Pot Palace}\grqq\ ein.
  \item Marge lässt ihr Haar in \glqq The Perm Bank\index{Perm Bank}\grqq\ (Die Dauerwellen-Bank, eine Anspielung auf \glqq Sperm Bank\grqq ) schneiden.
  \item Plato's Republic Casino ist ganz in der Nähe von Shotgun Pete's Heiratskapelle.
  \item Die Hochzeitsszene stammt aus der Episode \glqq Blick zurück aufs Ehe\-glück\grqq\ (siehe \ref{8F10}). Plato's Republic Casino ist aus \glqq Die Erbschaft\grqq\ (siehe \ref{7F17}) bekannt.
  \item Kearney sagt, dass er geschieden sei und einen Sohn habe.
  \item Nelson gibt an, sein Vater habe seine Mutter wegen ihrer Abhängigkeit von Hustenbonbons verlassen. In der Episode \glqq Der Feind in meinem Bett\grqq\ (siehe \ref{FABF19}) heißt es hingegen, er wollte etwas einkaufen und habe dort einen Schokoriegel mit Erdnüssen gegessen. Da er allergisch gegen diese Erdnüsse reagiert habe, verlor er den Verstand und wurde von einem Zirkus in deren Freakshow aufgenommen.
  \item Kirk Van Houten verliert seinen Managerposten bei \glqq Southern Cracker\index{Southern Cracker}\grqq , der Kräckerfabrik seines Schwiegervaters.
  \item Luann gibt an, eine Schwester zu haben.
\end{itemize}
}

	
\subsection{Mr. Burns Sohn Larry}\label{4F05}
Bei Mr. Burns taucht plötzlich ein gewisser Larry\index{Burns!Larry} auf. Die Simpsons haben Larry als Anhalter mitgenommen. Es ist Mr. Burns Sohn. Aber Larry passt eigentlich gar nicht zu Mr. Burns. Er ist genau das Gegenteil von ihm. Und um festzustellen, ob sein Vater ihn wirklich liebt, täuscht er mit Homer eine Entführung vor. Das Manöver erfüllt seinen Zweck. Dennoch kann sich Burns nicht überwinden und Larry kehrt wieder zu seiner Frau und seinen Kindern zurück.

\notiz{
\begin{itemize}
  \item Larry ist der uneheliche Sohn von Mr. Burns und Lily Bancroft\index{Bancroft!Lily}, deren Mutter Mimsy Bancroft\index{Bancroft!Mimsy} ist -- eine unerfüllte Studienliebe von Mr. Burns.
  \item Der Kopf des olmekischen Gottes Xt'Tapalataketel\index{Xt'Tapalataketel} aus \glqq Der Lebensretter\grqq\ (siehe \ref{7F22}), steht noch immer im Keller der Simpsons.
  \item Mr. Burns war im 1914er Jahrgang in Yale.
  \item In dieser Episode hat Homer zum zweiten Mal schlecht über Mil\-house hinter seinem Rücken gesprochen. Das erste Mal war in \glqq Der Ernstfall\grqq\ (siehe \ref{8F04}), als Homer ihn als Brillenschlange mit großer Nase bezeichnet hat.
  \item Das Copyright-Symbol (\copyright ) fehlt am Ende des Abspanns.
\end{itemize}
}

	
\subsection{Der beliebte Amüsierbetrieb}\label{4F06}
Milhouses Modellflugzeug landet auf einem Hausdach. Bart klettert hinauf und dabei zerstört er einen wertvollen Wasserspeicher. Die Besitzerin Belle\index{Belle} erwartet von Homer, dass er den Jungen bestraft. Homer zwingt Bart, für Belle zu arbeiten. So erfährt Bart, dass Belle in Springfield das Amüsier-Etablissement \glqq Maison Derriere\index{Maison Derriere}\grqq\ betreibt, in dem sämtliche Männer der Stadt verkehren. Marge verlangt den Abriss des Hauses.

\notiz{
\begin{itemize}
  \item Marges Diashow mit Personen, die aus dem Maison Derrier kommen, zeigt Dr. Hibbert, Chief Wiggum (zweimal), Skinner, Patty, Cletus, Barney, Smithers und Bürgermeister Quimby.
  \item Mr. Smithers gibt als Begründung an, wieso er aus dem Etablissement komme, dass dies seine Eltern wollten. In der Episode \glqq Aus dunklen Zeiten\grqq\ (siehe \ref{CABF21}) ist allerdings zu erfahren, dass sein Vater im Kraftwerk gestorben ist, als er noch ein Baby war.
  \item Bürgermeister Quimbys Frau hat in dieser Episode ihren zweiten Auftritt. Der erste war in \glqq Bart wird berühmt\grqq\ (siehe \ref{1F11}).
  \item Prinzessin Kashmir\index{Prinzessin Kashmir}, vorgestellt in \glqq Homer als Frauenheld\grqq\ (siehe \ref{7G10}), tanzt im Maison Derriere.
  \item Marge sagt, sie lebe bereits 37 Jahre in Springfield und ihre Familie bereits in der dritten Generation.
  \item Diese Episode gewann einen Emmy für die Musik und den Text zu dem Lied \glqq We Put the Spring in Springfield\grqq .
\end{itemize}
}

	
\subsection{Der total verrückte Ned}\label{4F07}
Der Hurrikan Barbara tobt durch Springfield und dabei wird Flanders Haus zerstört. Er fühlt sich von Gott bestraft. Alle Nachbarn helfen zusammen und bauen das Haus wieder auf. Als die erste große Besichtigung stattfindet, stürzt es wieder ein. Flanders bekommt einen Wutausbruch, wie ihn die Bürger Springfields noch nicht gesehen haben. Danach begibt er sich freiwillig in ein Sanatorium für Geisteskranke.

\notiz{
\begin{itemize}
  \item Todd trägt ein T-Shirt der Band Butthole Surfers\index{Butthole Surfers}, das er in der kirchlichen Kleidersammlung gefunden hat.
  \item Dean Peterson\index{Peterson!Dean} (aus \glqq Homer an der Uni\grqq , \ref{1F02}) ist bei der Vollziehung der Todesstrafe im Gefängnis von Springfield anwesend.
  \item Ned weist sich in das \glqq Calmwood\index{Calmwood} Sanatorium für Geisteskranke\grqq\ ein. Er wird in der Zelle 107 untergebracht.
  \item Im Sanatorium ist Jay Sherman\index{Sherman!Jay} aus der Folge \glqq Springfield Film-Festival\grqq\ (siehe \ref{2F31}) zu sehen. Ebenfalls ist in dieser Anstalt die kriminelle Babysitterin Mrs. Botz\index{Botz} aus der Episode \glqq Der Babysitter ist los\grqq\ (siehe \ref{7G01}) zu sehen.
  \item Der Psychiater, der Ned bereits 30 Jahre zuvor behandelt hat, als Ned noch ein kleiner Junge war, ist Dr. Foster\index{Foster!Dr.}. In der späteren Folge \glqq Wir fahr'n nach Vegas\grqq\ (siehe \ref{AABF06}) wird allerdings behauptet, dass Ned bereits 60 Jahre alt ist. Somit kann er 30 Jahre früher kein kleiner Junge mehr gewesen sein.
  \item Krusty, der Clown, ist zu sehen, als er sich einen Witz aufschreibt, obwohl er eigentlich weder lesen noch schreiben kann.
  \item Rod mag Krusty den Clown nicht.
  \item Auf der Tafel, die vor der Kirche steht, ist zu lesen \glqq God Welcomes His Victims\grqq\ (Gott heißt seine Opfer willkommen).
\end{itemize}
}

	
\subsection{Marge und das Brezelbacken}\label{4F08}
Marge will sich endlich ihren Traum von einem eigenen Geschäft verwirklichen: Sie steigt ins mobile Brezel-Verkaufsgewerbe ein. Doch während ihre Freundinnen mit einem Pitabrot-Service viel Geld verdienen, lässt Marges Umsatz zu wünschen übrig. Homer weiß Rat: Er wendet sich an die Mafia. Marges Konkurrenz wird umgehend ausgeschaltet. So richtig glücklich werden die Simpsons aber dadurch nicht.

\notiz{
\begin{itemize}
  \item Der mexikanische Ringer, den die Investoretten (Edna Krabappel, Maude Flanders, Helen Lovejoy, Agnes Skinner, Luann Van Houten und Marge Simpsons) sponsern, heißt \glqq El Bombastico\grqq\index{El Bombastico}.
  \item Auf dem Schild am Springfield Convention Center steht: \glqq Franchise Expo, Where you can make your nonsexual dreams come true\grqq\ (Konzessions-Verkaufsmesse, wo all Ihre nicht sexuellen Wünsche wahr werden können).
  \item Disco Stus Geschäft heißt \glqq Disco Stu's Can't Stop the Learnin' Disco Academies\grqq\ (Nicht aufzuhaltende Disco Akademien).
  \item Literaturzitat: Frank Ormands\index{Ormand!Frank} Rede \glqq Wo immer eine junge Mutter nicht weiß, womit sie ihr Baby füttern soll, dort müssen Sie auftauchen. Wo immer ein mexikanischer Schnellimbiss Umsatzeinbußen verzeichnet, nix wie hin! Und wo immer ein Bayer auch nur das geringste Hungergefühl verspürt, in die Bresche springen.\grqq\ parodiert Tom Joads Rede in \glqq Früchte des Zorns\grqq .
  \item Unter den Namen seiner Kinder, die Cletus aufzählt, sind auch die Namen Among Rumor, Scout und Q-bert. Rumor und Scout sind die Namen von Bruce Willis und Demi Moores gemeinsamer Kinder. Q-bert ist der Name eines alten Videospiels. Ebenso ist Qbert\index{Qbert} der Name des Klons von Professor Farnsworth\index{Farnsworth} aus der Serie \glqq Futurama\grqq\index{Futurama}.
\end{itemize}
}

	
\subsection{Wenn der Rektor mit der Lehrerin\dots}\label{4F09}
Mrs. Krabappel und Rektor Skinner kommen sich auf Martin Princes Geburtstagsparty ge\-fühls\-mäßig näher. Bart spielt für die beiden den Liebesboten. Nun vermuten seine Freunde, dass er selbst in Mrs. Krabappel verknallt ist. Es gibt nur einen Weg, diesen grausamen Verdacht aus der Welt zu schaffen: die Wahrheit. Bart muss seinen Freunden die Gelegenheit bieten, Mrs. Krabappel und Rektor Skinner in flagranti zu überraschen. Dies führt dazu, dass beide kurzfristig aus dem Schuldienst entlassen werden. Es entstand bei den Eltern der Eindruck, beide seien in der Besenkammer der Schule intim miteinander geworden.

\notiz{
\begin{itemize}
	\item In dieser Folge sagt Chalmers, dass Rektor Skinner 44 Jahre alt sei und Skinner gesteht, noch nie mit einer Frau geschlafen zu haben.
	\item Agnes Skinner mag zwar keine Kuchen, dennoch sammelt sie Bilder von diesen seit 1941.
	\item Zu Martin Prince Party kommen u.\,a. Bart Simpson, Edna Krabappel, Seymour Skinner, Lisa Simpson, Wendell, Milhouse und Nelson.
	\item Die Kerze beim romantischen Dinner bei Mrs. Krabappel hat die Form von Charlie Brown.
	\item Edna Krabappel sammelt Streichholzschachteln berühmter Nachtclubs.
	\item Die Skinners sind die Nachbarn der Familie Prince.
\end{itemize}
}

	
\subsection{Der Berg des Wahnsinns}
Mr. Burns unternimmt mit seinen Angestellten einen Ausflug in die verschneiten Berge, um das Teamwork zu stärken. Auf verschiedenen Wegen sollen sie paarweise eine Hütte erreichen. Homer fährt mit Mr. Burns heimlich mit einem Motorschlitten nach oben. Doch kaum sind sie in der Hütte angekommen, werden sie von einer Lawine eingeschlossen. Schon bald gehen sich die beiden derart auf den Geist, dass sie aufeinander losgehen.

\notiz{
\begin{itemize}
	\item Mr. Burns feuert Lenny in dieser Episode; das letzte Mal hat er das in \glqq Burns Erbe\grqq\ (siehe \ref{1F16}) getan.
	\item Die Hütte befindet sich auf dem Mt. Useful\index{Mt. Useful}.
\end{itemize}
}

	
\subsection{Homer und gewisse Ängste}\label{4F11}
Homer stöbert in einem Trödelladen herum. Dabei lernt er den jungen und charmanten John Waters\index{Waters!John} \cite{SOTB} kennen. John gibt sich Homer gegenüber sehr aufgeschlossen, was diesem gefällt. Weniger gefällt ihm, was Marge dazu zu sagen hat: Sie weist ihn nämlich darauf hin, dass John homosexuell ist. Homer wirft John aus seinem Haus. Und dann bekommt er plötzlich Angst, dass sein Sohn Bart homosexuell sein könnte.

\notiz{
\begin{itemize}
  \item Johns Laden trägt den Namen \glqq Cockamamie's\index{Cockamamie's}\grqq , was soviel bedeutet wie \glqq Billiges Zeug\grqq , es ist allerdings auch eine sexuelle Anspielung.
  \item Als John in dieser Episode vorgestellt wird, lehnt hinter ihm ein rosa Flamingo an der Wand. Ein bekannter Film von John Waters heißt \glqq Pink Flamingo\grqq .
  \item In Johns Laden hängt eine Werbung für \glqq Fudd\index{Fudd} Beer\grqq\ (vorgestellt in \glqq Homer auf Abwegen\grqq, \ref{8F19}).
  \item Außerdem hat John Homers altes Bowlinghemd aus der Episode \glqq Homers Bowling-Mannschaft\grqq\ (siehe \ref{3F10}) aus der Altkleidersammlung. 
  \item Homer fährt mit Bart in die \glqq Ajax Steel Mill\footnote{Ajax Stahlwerk}\index{Ajax Steel Mill}\grqq .
  \item Bei keiner anderen Episode fielen so viele Szenen der Zensur des Senders (FOX) zum Opfer. Die Produzenten erhielten zwei ganze Seiten voller Zensuren von Szenen und Dialogen, die aber durch Verhandlungen dann doch reduziert werden konnten.
  \item Moe nennt sein Auto \glqq Betzi\index{Betzi}\grqq .
\end{itemize}
}
	
\subsection{Homer ist \glqq Poochie\index{Poochie} der Wunderhund\grqq }\label{4F12}
Die bisher so guten Einschaltquoten der Zeichentrickserie \glqq Itchy \& Scratchy\grqq\ sinken. Die Produzenten versuchen, die Serie mit einer neuen Figur aufzupeppen. Hund \glqq Poochie\grqq\ wird kreiert. Natürlich braucht der Hund eine Stimme. Beim Probesprechen gewinnt Homer Simpson. Er soll \glqq Poochie\grqq\ nun seine Stimme leihen. Ob es daran liegt, dass sich die neue Figur ziemlich schnell als Flop herausstellt?

\notiz{
\begin{itemize}
  \item Die Stimmen von Itchy und Scratchy stammen von June Bellamy\index{Bellamy!June}.
  \item Erster Auftritt von Lindsey Naegle\index{Naegle!Lindsey}.
  % war in der ersten Staffel bereits mit Sideshow Mel bei Luigi's zu sehen
  \item Mel sammelt Spenden für das Rock'N'Roll-Museum.
  \item Vorsprechende für die Stimme von Poochie: Lionel Hutz, Jimbo, Kearney, Miss Hoover, Hans Maulwurf, Troy McClure, Otto und Homer.
  \item Die Dialogzeile beim Vorsprechen: \glqq Wuff! Wuff! Ich bin Poochie, der clevere Hund!\grqq .
  \item Für die Mannschaft der \glqq Itchy \& Scratchy Show\grqq\ wurden Karikaturen von mehreren Zeichnern und Autoren aus der Simpsons-Crew benutzt.
  \item Benjamin\index{Benjamin}, Doug\index{Doug} und Gary\index{Gary} (vorgestellt in \glqq Homer an der Uni\grqq , \ref{1F02}) sind bei der \glqq Lernen Sie die Stimmen kennen von Itchy \& Scratchy \& Poochie\grqq -Autogrammstunde im Comicbuchladen mit dabei. Homer scheint sie nicht zu erkennen. Außerdem ist auch Database\index{Database} da.
  \item Die Animationsfolie, mit der Poochie aus dem Bild verschwindet, ist deutlich mit \glqq 4F12 SC-273\grqq\ beschriftet (4F12 ist die Nummer dieser Episode).
  \item Mit dieser Episode überholten die \glqq Simpsons\grqq\ die \glqq Familie Feuerstein\grqq\ in der Zahl der Episoden und wurden damit zur am längsten laufenden Primetime-Zeichentrickserie in den USA. Außerdem ist dies die einzige Folge, in der PRO7 eine Szene zensiert hat (in der Itchy \& Scratchy-Folge fehlen einige Sekunden).
  \item In dieser Episode ist Roy\index{Roy} bei den Simpsons zu Gast.
  \item Homer gibt an, dass er gerne die Band Deep Purple\index{Deep Purple} hört.
\end{itemize}
}

	
\subsection{Babysitten -- ein Albtraum}\label{4F13}
Lisa hat sich zu einer beliebten Babysitterin gemausert. Nur als sie zu Hause auf Bart aufpassen soll, geht alles drunter und drüber. Der Knabe bringt seine Schwester mit Leichtigkeit zur Weißglut. Doch Übermut tut selten gut: Prompt fällt Bart die Treppe hinunter. Er bleibt bewusstlos liegen. Lisa versucht, Bart ins Krankenhaus zu bringen und landet auf Umwegen bei der Eröffnungsfeier des Hafenzentrums.

\notiz{
\begin{itemize}
	\item Bart behauptet, zwei Jahre und 38 Tage älter als Lisa zu sein.
	\item Rainier Wolfcastles Restaurant \glqq Planet Hype\index{Planet Hype}\grqq\ ist eine Anspielung auf \glqq Planet Hollywood\grqq .
	\item Lisa ist Babysitterin bei Ned Flanders, Clancy Wiggum und Dr. Hibbert.
	\item Dr. Nick Rivieras Ambulanzklinik liegt in der 44 Bow Street.
\end{itemize}
}

	
\subsection{Die beiden hinterhältigen Brüder}\label{4F14}
In Springfield sollen ein Staudamm und ein Wasserkraftwerk gebaut werden. Den Auftrag dazu erhält Cecil Terwilliger\index{Terwilliger!Cecil}, der jüngere Bruder von Tingel-Tangel-Bob. Zu seiner Unterstützung holt Cecil seinen Bruder aus dem Gefängnis. Das gefällt Bart gar nicht. Tingel-Tangel-Bob hat ihm früher nach dem Leben getrachtet. Doch wie es aussieht, hat sich Bob im Gefängnis wider Erwarten zu einem guten Menschen gewandelt. Cecil will aus Frust darüber, dass sein Bruder ihm die Rolle des Assistenten von Krusty unfreiwillig weggeschnappt hat, ihm seine Betrügereien beim Bau des Staudamms in die Schuhe schieben.

\notiz{
\begin{itemize}
	\item Hinter Bobs Tisch im Pimento Grove\index{Pimento Grove} hängen Bilder von US-Nach\-ri\-chten\-mo\-der\-a\-tor Tom Brolaw und Birch Barlow.
	\item Zu den Bauarbeitern gehört neben Cletus auch sein Cousin Merl\index{Merl}.
	\item Bob gibt an, in Amerika geboren worden zu sein und fünfmal vorbestraft zu sein.
\end{itemize}
}
	
\subsection{Der mysteriöse Bier-Baron}
Nachdem Bart beim St. Patricks Day betrunken durch die Straßen Springfields torkelt, gräbt die Stadtverwaltung ein uraltes Gesetz aus, wonach in Springfield seit gut 200 Jahren striktes Alkoholverbot herrscht. Zur Überwachung des Verbots wird der FBI-Beamte Rex Banner\index{Banner!Rex} engagiert. Aber Banner hat nicht mit Homer gerechnet. Homer holt sich das Bier von der Müllkippe und nachdem dieses aufgebraucht ist, braut er heimlich Bier, sodass Moes Kneipe weiterhin regen Zulauf hat. Dann aber stellt sich heraus, dass dieses Verbot seit knapp 199 Jahren ungültig ist.

\notiz{
\begin{itemize}
  \item Name von Moes neuer, illegaler Kneipe: Moe's Pet Shop.
  \item Schilder, die von einer Gruppe wütender Frauen in der Stadthalle hoch gehalten werden: \glqq Draft Men, Not Beer\grqq\ (Zieht Männer ein, nicht Bier!), \glqq Prohibition Now!\grqq\ (Prohibition jetzt) und \glqq Say No to Drunks\grqq\ (Sagt Nein zu Säufern).
  \item Zeitdauer bis die Duff-Brauerei nach Einführung von \glqq Duff Zero\grqq\ pleite war: 30 Minuten.
  \item In Moe's Pet Show ist Prinzessin Kashmir beim Tanzen mit Polizeichef Wiggum zu sehen.
\end{itemize}
}

	
\subsection{Der tollste Hund der Welt}\label{4F16}
Bart bestellt sich eine Kreditkarte. Er kauft damit u.\,a. einen Collie namens Laddie. Natürlich hat Bart nicht bedacht, dass auch Käufe per Karte irgendwann bezahlt werden müssen. Weil er das nicht kann, soll der Hund gepfändet werden. Bart gibt dem Gerichtsvollzieher seinen Hund Knecht Ruprecht mit, um Laddie\index{Laddie} behalten zu können. Bald wird Laddie zum Liebling der Familie. Bart aber bekommt Sehnsucht nach Knecht Ruprecht. Er stellt fest, dass der Hund mittlerweile an einen Blinden weitergeben wurde. Bart bricht in das Haus des Blinden ein, um Knecht Ruprecht zurückzuholen.

\notiz{
\begin{itemize}
  \item In dieser Episode enthüllt: Der Name des Babys mit nur einer Augenbraue ist Gerald\index{Gerald}.
  \item Barts Kreditkartennummer ist 4123 0412 3456 7890. Sie ist gültig bis 12/03/99. Auf der Karte ist eine Art Taube abgebildet und lautet auf den Namen \glqq Santos L. Halper\index{Halper!Santos L.}\grqq .
  \item Knecht Ruprechts neuer Name bei seinem neuen Besitzer Mitchell\index{Mitchell}: Sprüh\-re\-gen\index{Sprühregen}.
  \item Ein Ausschnitt aus dieser Folge ist im Film \glqq Garfield\grqq\ von 2004 zu sehen.
\end{itemize}
}

	
\subsection{Der alte Mann und Lisa}\label{4F17}
Mr. Burns hat die wirtschaftliche Entwicklung verschlafen. Er macht pleite und verliert sein Atomkraftwerk und seine Villa an die Banken. Er trifft Lisa. Und die macht ihm auch noch Vorwürfe, weil er nichts für die Umwelt getan hat. Auf der Stelle gründet Mr. Burns reumütig eine Recycling-Fabrik. Lisa ist glücklich. Sie glaubt, Mr. Burns endlich bekehrt zu haben. Ihre Hoffnung wird grausam enttäuscht. Mr. Burns verkauft schließlich die Recycling-Firma für \$ 120.000.000 und bietet Lisa als seine Teilhaberin 10 \% der Summe an, diese lehnt jedoch ab. Daraufhin erleidet Homer einen vierfachen Herzinfarkt, weil er davon ausgegangen ist, dass 12.000 Dollar 10 \% von 120.000.000 sind.

\notiz{
\begin{itemize}
  \item Eine der vielen deutlich veralteten Aktien in Mr. Burns Depot ist von der \glqq Konföderations Sklaven AG\grqq .
  \item Fernsehzitat: Mr. Burns jagt hinter Lisa her, um ihre Unterstützung zu bekommen. Eine Szene erinnert an den Vorspann der klassischen US-TV-Serie \glqq That Girl\grqq .
  \item Filmzitat: Lisa rennt durch Springfield und verkündet die Wahrheit über Burns Recycling Center wie Charles Heston in \glqq Soylent Green\grqq\ (Jahr 2022\dots die überleben wollen).
  \item Homer gibt Lisa ein Buch von Leon Uris\index{Uris!Leon} zum Recyceln, Uris hat unter anderem \glqq Exodus\grqq\ geschrieben.
  \item Burns hat seine Autobiografie \glqq Will There Ever Be a Rainbow\grqq\ (Wird es jemals einen Regenbogen geben) in der Episode \glqq Der Lebensretter\grqq\ (siehe \ref{7F22}) geschrieben.
  \item In dieser Episode besitzt Rektor Skinner ein Auto. In der Folge \glqq Lisa will lieben\grqq\ (siehe \ref{4F01}) träumte er noch davon, irgendwann einmal Autobesitzer zu sein.
  \item Lenny leitet vorübergehend das Kraftwerk.
  \item Waylon Smithers wurde 1954 (25 Jahre nach dem großen Börsencrash) geboren.
\end{itemize}
}
	
\subsection{Marge als Seelsorgerin}\label{4F18}
Reverend Timothy Lovejoy erlaubt Marge, sich als Seelsorgerin zu betätigen. Marge ist mit Feuereifer bei der Sache. Zunächst erzielt sie auch schöne Erfolge. Doch dann erteilt sie Ned Flanders einen Rat, der diesen in große Schwierigkeiten bringt: Als Ned von jugendlichen Rowdys verfolgt wird, flüchtet er in den Tierpark -- ins Paviangehege.

\notiz{
\begin{itemize}
  \item In der Schlange für Marge stehen hinter Lenny: Kirk Van Houten, Ruth Powers, die Skinners, Dr. Nick Riviera, Miss Hoover und Larry aus Moe's Bar.
  \item Homer, Bart und Lisa gehen ins \glqq The Happy Sumo\grqq\ (vorgestellt in \glqq Die 24-Stunden-Frist\grqq , \ref{7F11}), um Akira nach der Box zu fragen, die sie gefunden haben.
  \item Die Werbefigur \glqq Meister Glanz\index{Meister Glanz}\grqq\ ist eine Anspielung auf \glqq Meister Proper\grqq .
  \item Lenny gesteht Marge, nicht verheiratet zu sein, obwohl er gegenüber Carl dies immer behauptet.
\end{itemize}
}

	
\subsection{Homer hatte einen Feind}\label{4F19}
Frank Grimes\index{Grimes!Frank} ist neu im Kernkraftwerk. Er hat sich mühselig emporgearbeitet. Grimes verachtet Homer, der seiner Meinung nach nur faul herumsitzt, nichts richtig macht und trotzdem relativ wohlhabend ist. Als Homer bei einem Kinderwettbewerb auch noch den ersten Preis gewinnt, dreht Grimes durch: Er wird wahnsinnig und fasst zwei Starkstromkabel an. Die Beerdigung seines Kollegen verschläft Homer.

\notiz{
\begin{itemize}
  \item Bei der Zwangsversteigerung ist auch Mr. Largo, der Musiklehrer, dabei.
  \item Bart erwirbt bei dieser Zwangsversteigerung eine alte Fabrik für einen Dollar. Diese Fabrik liegt in der Straße 35 Industry Way (in dieser Episode genau gegenüber von Moes Taverne).
  \item Am schwarzen Brett des Kraftwerks hängt ein Formular zum Eintragen für das Softball-Team.
  \item Frank Grimes war 35 Jahre alt, als er starb.
  \item In dieser Episode erfährt man, dass Lenny und Carl einen Magister in Atomphysik haben und dass Lisas IQ bei 156 liegt.
\end{itemize}
}

	
\subsection{Ihre Lieblings-Fernsehfamilie}\label{4F20}
Troy McClure stellt Spin-off-Serien von der Kult-Zeichentrickserie \glqq Die Simpsons\grqq\ vor: 
\begin{itemize}
	\item \textbf{Chief Wiggum, P.I.}\\ Chief Wiggum wird Detektiv und zieht nach New Orleans. Sein Assistent: Rektor Skinner. Dort bekommen sie es mit dem Gangster Big Daddy\index{Big Daddy} zu tun, der Ralph entführt.
	\item \textbf{The Love-Matic Grampa}\\ Opa Simpsons Seele gerät in Moes Liebestester gefangen und Moe bemüht sich mit Grandpas Hilfe darum, die schöne Betty\index{Betty} für sich zu gewinnen.
	\item \textbf{The Simpsons Family Smile-Time Variety Hour}\\ Die Simpson-Familie ist Gastgeber einer 70er-Jahre Unterhaltungssendung.
\end{itemize}

\notiz{
\begin{itemize}
  \item Das Bild von \glqq Rhoda\grqq\ zeigt Valerie Harper und Julie Kavner, die auch Marge im Original spricht.
  \item Als Kent Brockman die \glqq Simpsons Family Smile-Time Variety Hour\grqq\ ankündigt, behauptet er, die Sendung würde live ausgestrahlt. Die Uhr hinter ihm zeigt \glqq 8:20\grqq\ an, ziemlich genau die Zeit, zu der dieser Teil der Folge an der amerikanischen Westküste zu sehen war.
\end{itemize}
}

	
\subsection{Lisas geheimer Krieg}
Bart hat mit mehreren hintereinander geschalteten Megaphonen in der Polizeistation ein Chaos ausgelöst. Zur Strafe wird er in die Kadettenanstalt Rommelwood\index{Rommelwood} geschickt. Dort sollen dem Jungen Manieren beigebracht werden. Und da es Lisa so gut gefällt, bleibt sie ebenfalls dort. Bart gelingt es schnell, Freunde zu finden. Mädchen haben es in einer solchen Anstalt natürlich schwerer. Doch Lisa wird mit den Jungs schon fertig.

\notiz{
\begin{itemize}
  \item Lisa ist das erste Mädchen, das in der 185-jährigen Tradition von Rommelwood aufgenommen wird.
	\item Das Motto an der Eingangstür von Rommelwood lautet: \glqq A Tradition of Heritage\grqq\ (Eine Tradition der Vererbung).
	\item Chief Wiggum gibt an, dass er einen Bruder hat, der eine Höhle besitzt und in dieser wohnt.
	\item Fehler: Als Lisa beim Üben am Eliminator runterfällt, ist sie an einem Seil befestigt. Vor dem Fall war das Seil aber noch nicht da.
\end{itemize}
}


\section{Staffel 9}

\subsection{Die Saxophon-Geschichte}\label{3G02}
Lisa spielt begeistert auf ihrem heiß geliebten Saxophon und nervt damit den Rest der Familie beim Fernsehen. Bart wirft das Saxophon seiner Schwester kurzerhand aus dem Fenster. Auf der Straße wird es von einem Auto kaputt gefahren. Alle hoffen, dass nun Ruhe einkehrt. Lisa ist untröstlich, weiß sich aber schnell zu helfen. Homer muss schließlich wieder einmal auf eine Klimaanlage verzichten, um Lisa ein neues Saxophon kaufen zu können.

\notiz{
\begin{itemize}
  \item In dieser Folge wird Sheriff Lobo zum dritten Mal in der Serie erwähnt. Das erste Mal fiel sein Name in \glqq Marge wird verhaftet\grqq\ (siehe \ref{9F20}) und zuletzt in \glqq Kampf um Bobo\grqq\ (siehe \ref{1F01}).
  \item Hans Maulwurf fährt den Lastwagen, der Lisas Saxophon platt walzt.
  \item Lisa baut in dieser Folge das Wort STARS (Sterne) mit Spielklötzen. Nachdem sie diese umgeworfen hat, ist das Wort RATS (Ratten) zu lesen.
  \item In dieser Folge tritt zum ersten Mal seit \glqq Homer kommt in Fahrt\grqq\ (siehe \ref{9F10}), Dr. J. Loren Pryor\index{Pryor!Dr. J. Loren} wieder auf.
  \item Der Schulpsychologe hatte fälschlicherweise zuerst die Akte von Milhouse auf dem Schreibtisch liegen und machte Andeutung auf eine mögliche homophile Neigung dessen.
  \item Diesmal verdeckt Michelangelos David seine Blöße mit einem Feigenblatt -- in \glqq Das Fernsehen ist an allem Schuld\grqq\ (siehe \ref{7F09}) war er ohne zu sehen.
  \item Homer rieb sich zuletzt in \glqq Ein grausiger Verdacht\grqq\ (siehe \ref{1F22}), mit Tiefkühlkost ab, um sich kühl zu halten.
  \item Die weiße Katze im Wohnzimmer der Simpsons ist Schneeball I.
  \item Der Laden, in dem Homer die Klimaanlage kaufen wollte, heißt \glqq It Blows\grqq\ (Es bläst).
  \item An der Wand des King Toot's Music Store hängt eine rot weiß gestreifte Gitarre im Eddie-Van-Halen-Stil der 1984er Ära.
  \item Das Saxophon kostet 200 Dollar genauso viel wie die Klimaanlage.
  \item Der von Warner Brothers produzierte Fernsehfilm mit Fyvush Finkel in der Hauptrolle heißt: \glqq Die Story von Krusty, dem Clown: Alkohol, Drogen, Waffen, Lügen, Erpressung und Gelächter.\grqq .
  \item Dr. Hibbert sieht wie Mr. T aus dem A-Team\index{A-Team} aus.
  \item Das Lied, welches Lisa am Ende der Episode auf ihren Saxophon spielt, ist der Theme von Gerry Rafferty's \glqq Baker Street\grqq .
\end{itemize}
}

	
\subsection{Homer geht zur Marine}\label{3G04}
Homer Simpson ist dermaßen übermüdet, dass er während seiner Arbeit im Atomkraftwerk zum wiederholten Mal einschläft und fast die Kaffeepause verschläft. Da nur noch ein Donut übrig ist, will er diesen im Kernreaktor vergrößern. Dies führt zu einem Zwischenfall und ist für Mr. Burns Grund genug, Homer fristlos zu entlassen. In seiner Verzweiflung meldet der sich zur Marinereserve, aber auch da geht natürlich nicht alles glatt: Während einer Übung gerät Homer mit seinem U-Boot in russische Hoheitsgewässer und löst einen diplomatischen Konflikt aus. Das Ergebnis: Homer wird unehrenhaft aus der Marine entlassen.

\notiz{
\begin{itemize}
  \item Filmzitat I: Homers Traum vom Planeten der Donut ist eine Referenz an \glqq Planet der Affen\grqq . In den Episoden \glqq Homer der Weltraumheld\grqq\ (siehe \ref{1F13}) und \glqq Selma heiratet Hollywoodstar\grqq\ (siehe \ref{3F13}) gab es ebenfalls Anspielungen auf diesen Film.
  \item Filmzitat II: Die Szene beim Haareschneiden und der anschließende Stubendurchgang ist eine Anspielung auf den Film \glqq Full Metal Jacket\grqq .
  \item Homers Kollege mit der großen Hand trat erstmals in \glqq Homer liebt Mindy\grqq\ (siehe \ref{1F07}) auf.
  \item Als Homer seine Columbo-Parodie zum Besten gibt, verdreht er nur eine Pupille.
  \item Vor der Offiziersmesse ist Barneys Mutter zu sehen.
  \item Ein Schild an der früheren Berliner Mauer trägt die Aufschrift \glqq Berlin reunited and it feels so good\grqq\ (Berlin wiedervereinigt -- ein herrliches Gefühl).
  \item In dieser Folge hat Moes Katze Mr. Snookums\index{Snookums} ihren ersten Auftritt.
  \item Homer verrät, dass er sich mit zehn Jahren ebenfalls ein Ohr hat piercen lassen.
  \item Als das U-Boot ausläuft, singen die Seemänner \glqq In The Navy\grqq\ von den Village People.
  \item Neben Homer treten noch Apu, Barney und Moe der Marinereserve bei.
\end{itemize}
}
	
\subsection{Homer und New York}\label{4F22}
Barney hat sich mit Homers Wagen abgesetzt. Er fährt nach New York und lässt das Auto dort einfach stehen. Einige Zeit später erhält Homer einen Strafzettel per Post. Weil er den Wagen innerhalb von drei Tagen entfernen muss, reist Familie Simpson nach New York. Während sich Marge, Lisa und Bart in der Stadt prächtig amüsieren und den unerwarteten Urlaub genießen, muss Homer einen einsamen Kampf gegen die Verkehrspolizei führen. Das Chaos nimmt seinen Lauf.

\notiz{
\begin{itemize}
  \item Filmzitat: Im Central Park liefert sich Homer eine Wettfahrt mit einer Pferdedroschke -- eine Anspielung auf das Wagenrennen in Ben Hur. Dieselbe Szene wurde auch in \glqq Das Seifenkistenrennen\grqq\ (siehe \ref{8F07}) parodiert.
  \item Woody Allen war zuletzt auf einer rotierenden Zeitung in der \glqq King Homer\grqq -Geschichte in \glqq Bösartige Spiele\grqq\ (siehe \ref{9F04}) zu sehen.
  \item Der große Mann, der im Bus hinter Homer sitzt, war zuletzt in \glqq 22 Kurzfilme über Springfield\grqq\ (siehe \ref {3F18}) dabei.
  \item Einer der Einwanderer auf dem Schiff, das an der Freiheitsstatue vorbeifährt, sieht aus wie Adil Hoxha aus \glqq Tauschgeschäfte und Spione\grqq\ (siehe \ref{7G13}).
  \item Aus Pietätsgründen wurde von Fox empfohlen, diese Episode nach dem 11.9.2001 nicht mehr auszustrahlen.
  \item Als Homer Hunger bekommt, kauft er sich \glqq Khlav-Kalash\index{Khlav-Kalash}\grqq\ und trinkt dazu Krabbensaft.
\end{itemize}
}
	
\subsection{Alles Schwindel}\label{4F23}
Rektor Skinners 20-jähriges Dienstjubiläum steht kurz bevor. Plötzlich taucht ein fremder Sergeant auf und behauptet, Seymour Skinner zu sein. Über\-ra\-schen\-der\-wei\-se gibt Rektor Skinner auf der Stelle zu, Armin Tamzarian\index{Tamzerian!Armin} zu heißen und erklärt seinen sofortigen Rücktritt. Doch niemand ist mit dem neuen Rektor Skinner zufrieden. Daher holt man den falschen Skinner schon bald wieder zurück und schickt den richtigen Skinner wieder fort.

\notiz{
\begin{itemize}
  \item Armin Tamzarian war ein Waisenkind und wuchs in Capitol City auf und war dort als Draufgänger bekannt. Er wurde dazu verurteilt, in der Armee zu dienen und nach Vietnam zu gehen. Nach der Rückkehr aus Vietnam nahm er die Identität von Seymour Skinner an, weil er glaubte, der richtige Seymour Skinner sei gefallen und lebte 26 Jahre bei Agnes Skinner, die gewusst hatte, dass es sich bei ihm nicht um ihren Sohn handelt.
  \item Agnes Satz \glqq I have no son\grqq\ (Ich habe keinen Sohn) war schon mehrmals zu hören, so z.\,B. von Rabbi Krustofski und zwar in der Folge \glqq Der Vater eines Clowns\grqq\ (siehe \ref{8F05}). Küchenhilfe Doris sagte ihn außerdem in \glqq Homers Bowling-Mannschaft\grqq\ (siehe \ref{3F10}), zu dem quietschstimmigen Teenager und Homer bekam ihn in \glqq Die Erbschaft\grqq\ (siehe \ref{7F17}) von seinem Vater zu hören.
  \item In Springfield ist ein Schild zu sehen, auf dem steht, dass Capitol City 30 Meilen entfernt ist.
  \item In dieser Episode hat Richter Snyder gelbe Hautfarbe.
  \item Hausmeister Willie feiert auch sein 20-jähriges Dienstjubiläum an der Grundschule.
\end{itemize}
}
	
\subsection{Vertrottelt Lisa?}\label{4F24}
Als Lisa eine Denksportaufgabe nicht lösen kann, für die ihre Klassenkameraden nur Augenblicke benötigen, macht sie sich ernsthafte Sorgen um ihre Intelligenz. Sie sucht Rat bei Opa Simpson. Der erzählt ihr, dass auch Homer und Bart schlaue Kerlchen waren: Bis zum neunten Lebensjahr. Dann wurde das Simpson-Gen aktiv und machte sie zu den bekannt faulen Burschen. Der neunte Geburtstag bekommt eine ganz neue Bedeutung. Lisa will die letzte Zeit ihrer Intelligenz genießen.

\notiz{
\begin{itemize}
  \item Der Name, den Apu dem eingefrorenen Jasper gibt, lautet \glqq Frostillicus\grqq\index{Frostillicus}.
  \item Milhouse ist nicht der Erste in seiner Klasse, der Läuse hat. Bart hat es in \glqq Bei Simpsons stimmt was nicht\grqq\ (siehe \ref{3F01}) erwischt.
  \item Nachdem Apu den Namen seines Ladens in \glqq Freak-E-Mart\index{Freak-E-Mart}\grqq\ geändert hat, hängt er Schilder ins Schaufenster, auf denen steht: \glqq See the incredible siamese hotdog\grqq\ (Sehen Sie das unglaubliche siamesische Hotdog) und \glqq Astonishing rubber check from mysterious unknown bank\grqq\ (Erstaunlicher Gummischeck von geheimnisvoller unbekannter Bank). Eigentlich ist rubber check eine geläufige Umschreibung für einen geplatzten Scheck.
  \item Nachdem Apu den Namen seines Ladens in \glqq Nude-E-Mart\index{Nude-E-Mart}\grqq\ geändert hat, hängt er ein Schild mit der Aufschrift \glqq Topless Dancers! Bottomless Coffee!\grqq\ (Oben-ohne-Tänzerinnen! Bodenloser Kaffee!) auf.
  \item Als Homer und Bart sich im Fernsehen die einstürzenden Gebäude ansehen, ist bei einem Haus \glqq The House Of Usher\grqq\ zu lesen. \glqq The Fall Of The House Of Usher\grqq\ ist eine Kurzgeschichte von Edgar Allan Poe.
\end{itemize}
}

	
\subsection{Homer und der Revolver}\label{5F01}
Während eines Fußballspiels zwischen Mexiko und Portugal kommt es in Springfield zu einem Aufruhr. Daraufhin legt sich Homer einen Revolver zu. Schließlich muss er seine Familie beschützen. Marge und die Kinder ziehen lieber in ein Motel. Es kommt zu einem Überfall, den Homer trotz des Revolvers nicht verhindern kann. Homer gesteht Marge, dass ihm die Waffe ein ungeheures Machtgefühl vermittelt. Er bittet sie, den Revolver wegzuwerfen. Kaum hat Marge diesen in der Hand, empfindet sie ebenfalls dieses Machtgefühl und steckt ihn in ihre Handtasche.

\notiz{
\begin{itemize}
  \item Bei den Fußballausschreitungen wird u.\,a. Dr. Riviera von Dr. Hibbert gewürgt.
  \item Die Mitglieder der Springfielder Ortsgruppe der NRA\index{NRA} sind Agnes Skinner, Moe, Krusty, Lenny, Dr. Hibbert, Ruth Powers, Louie (einer von Fat Tonys Schlägern) und Cletus.
  \item Homers Schild für die Versammlung: \glqq Gun Warming Tonight Nachos Rifles Alcohol\grqq\ (Heute Abend Waffen-Einweihung Nachos Gewehre Alkohol).
  \item Das Lied, das gespielt wird, während Homer auf seinen Revolver wartet, heißt \glqq The Waiting\grqq\ und ist von Tom Petty and The Heartbreakers.
  \item Patty und Selma wohnen im Zimmer 1599 in den \glqq Spinster City Apartments\grqq .
  \item Homer kauft seinen Revolver bei \glqq Bloodbath \& Beyond Gun Shop\grqq .
\end{itemize}
}

	
\subsection{Neutronenkrieg und Halloween}
\begin{itemize}
	\item \textbf{The Homega Man}\\ Bürgermeister Quimby hat die Franzosen als aufgeblasene Frösche bezeichnet. Daraufhin bestraft deren Präsident Springfield mit einer Neutronenbombe. Alle Einwohner werden zu Mutanten -- mit Ausnahme der Simpsons.
	\item \textbf{Fliege gegen Fliege}\\ Homer kauft einen Teleporter von Professor Frink, aber das Schnäppchen des Lebens verwandelt sich in einen Fluch, als Bart versucht, seine DNA mit der einer Fliege zu mischen.
	\item \textbf{Die flotte Hexenbraterei}\\ Im Sprynge-Fielde von 1649 ist es Zeit für die Hexenjagd. Ohne, dass sie es wissen, führen Marge und ihre Schwestern dabei die Halloween-Tradition \glqq Trick or Treat\grqq\ ein. Denn für ihr Hexensüppchen brauchen sie kleine Kinder. Bei den Flanders bekommen sie stattdessen Lebkuchenkinder.
\end{itemize}

\notiz{
\begin{itemize}
  \item Auf der Neutronenbombe klebt ein Etikett mit dem \glqq Intel Inside\grqq -Logo.
  \item Nachdem Springfield von der Neutronenbombe getroffen wurde, erlischt die seit 1989 brennende Reifenhalde.
  \item Die Mutanten, die Homer töten wollen: Mr. Burns, Moe, Dr. Hibbert, Lenny, Sideshow Mel, Ned Flanders und Chief Wiggum.
  \item Eines der Angebote auf Professor Frinks Flohmarkt ist ein Ra\-keten-Mo\-tor\-rad im Stil des berühmten Stuntpiloten Evel Knievel\index{Knievel!Evel}.
  \item Sprynge-Fielde\index{Sprynge-Fielde}: Der frühere Name von Springfield. 
  \item In der ersten Szene sieht man Snake am Pranger der Stadt und im Hintergrund \glqq Moe's Inn\grqq\ (Kneipe). Es sieht auch so aus, als würden Luann Van Houten, Agnes Skinner und Miss Hoover auf dem Scheiterhaufen verbrannt. Später sieht man sie allerdings während Marges Verwandlung und als Engel verkleidet beim \glqq Trick or Treat\grqq\ wieder.
  \item Gevatterin Krabappel trägt ein scharlachrotes \glqq A\grqq\ auf der Brust, eine Anspielung auf den oft verfilmten Roman \glqq Der scharlachrote Buchstabe\grqq\ aus dem Jahre 1850 von Nathaniel Howthorne, der in der gleichen Zeit spielt wie diese Geschichte.
\end{itemize}
}

	
\subsection{Bart ist mein Superstar}\label{5F03}
Homer trainiert die Schüler-Footballmannschaft, nachdem er Ned Flanders als Trainer raus gemobbt hat. Unter allen Umständen will er seinen Sohn Bart als Quarterback groß herausbringen. Der versagt jedoch zum Missfallen seiner Mannschaftskameraden auf der ganzen Linie. Kein Wunder: Bart ist gegen seinen Willen in der Mannschaft. Ein Star werden will er erst recht nicht. Erst als Bart den Schwerverletzten mimt, stellt Homer ihn nicht mehr auf. Ohne Bart gewinnt die Schülermannschaft dann auch tatsächlich die Meisterschaft.

\notiz{
\begin{itemize}
  \item Rod Flanders trägt im Team die Nummer 66, sein Bruder Todd die 6. Als Homer ver\-kün\-det, dass Bart der Quarterback der Wildcats\index{Wildcats} ist, stehen die Flanders-Jungs nebeneinander und so ergibt sich 666, die Zahl des Teufels.
  \item Im Spiel um die Meisterschaft tritt Springfield gegen Capitol City an. Die größte Stadt in der Nähe von Springfield schreibt sich aber normalerweise Capital City.
  \item In der Rückblende erfährt man, dass Homer früher Bodenturner war. Bei einem Wettkampf war auch Marge unter den Zuschauern (sie sitzt neben Lenny).
  \item In dieser Episode ist Nelsons Vater zu sehen, als er mit einem Motorrad seinen Sohn vom Training abholt.
  \item Unter den Zuschauern eines Spiels ist die Familie Hill\index{Hill} aus \glqq King Of The Hill\grqq\ zu sehen.
  \item Die Footballmannschaft spielt u.\,a. gegen die Ogdenville Wildcats, gegen Victory City, gegen Waynesport und gegen Arlen.
\end{itemize}
}

	
\subsection{Hochzeit auf indisch}\label{5F04}
Apu soll heiraten. Seine Frau wurde ihm schon in der Kindheit von den Eltern bestimmt. Zwanzig Jahre hat er sie nicht mehr gesehen. Apu erhält Rat von Homer: Apu solle so tun, als sei er bereits verheiratet und er bietet ihm Marge als Ehefrau an. Apus Mutter taucht auf, der Schwindel wird entlarvt. Dann zwingt sie ihren Sohn, die ihm versprochene Frau Manjula zu heiraten. Manjula trifft in Springfield ein und Apu ist sprachlos: Seine Zukünftige ist eine echte Schönheit.

\notiz{
\begin{itemize}
  \item Hier sagt Apu, dass er in Computerwissenschaften seinen Doktor gemacht hat. In der Episode \glqq Volksabstimmung in Springfield\grqq\ (siehe \ref{3F20}) sagte er, er habe in Philosophie promoviert.
  \item Manjulas Mitgift: Zehn Ziegen, ein elektrischer Ventilator und eine Textilfabrik.
  \item Bei der Auktion sitzen Stacy Lovell\index{Lovell!Stacy} aus \glqq Lisa kontra Malibu Stacy\grqq\ (siehe \ref{1F12}) und Waylon Smithers hinter den Simpsons.
  \item Die abgelehnten Junggesellen: Der Comicbuchverkäufer, Otto, Captain McCallister, Professor Frink, Disco Stu, Kirk Van Houten, Hans Maulwurf, Moe und Barney.
  \item Dem Stammbaum der Familie Nahasapeemapetilon zufolge hat Apu zwei jüngere Brüder; einer davon ist Sanjay\index{Sanjay}\index{Nahasapeemapetilon!Sanjay}.
  \item Apu lässt sich bei Hairy Shearers\index{Hairy Shearers} (haarige Scherer) die Haare schneiden (einer der Originalsprecher heißt Harry Shearer).
  \item Das Logo der Air India zeigt eine Stewardess, die einem Ochsen einen Drink serviert. Der Slogan lautet: \glqq We Treat You Like Cattle\grqq\ (Wir behandeln Sie wie Rindvieh).
  \item Am Flughafen sind Ernst\index{Ernst} und Gunter\index{Gunter} zu sehen. Die Tigerbändiger aus Mr. Burns Casino traten erstmals in \glqq Vom Teufel besessen\grqq\ (siehe \ref{1F08}) auf.
  \item Die Lehrerin Miss Hoover wohnt neben Luann Van Houten.
  \item Krusty liest bei der Junggesellenversteigerung vom Teleprompter ab, obwohl in früheren Episoden behauptet wird, er sei Analphabet.
\end{itemize}
}

	
\subsection{Der Tag der Abrechnung}\label{5F05}
In Springfield wird ein Einkaufszentrum gebaut. Auf dem zukünftigen Parkplatz fanden einmal Ausgrabungen statt. Deshalb sorgt Lisa dafür, dass vor Beginn der Bauarbeiten nochmals gesucht wird. Sie entdeckt ein Skelett. Die Einwohner halten das Ding für einen Engel. Homer baut in seiner Garage einen Devotionalienschrein und kassiert die Besucher ab. Aber Lisa glaubt nicht an Engel. Deshalb beauftragt sie den Wissenschaftler Dr. Stephen Jay Gould\index{Gould!Stephen Jay}, der allerdings keine Ergebnisse liefern kann. Wie sich später herausstellt, hatte er keine Untersuchung vorgenommen. Am Ende wird klar, dass es sich bei dem Engel nur um einen Werbetrick der neu eröffneten \glqq Heavenly Hills Mall\grqq\ gehandelt hat.

\notiz{
\begin{itemize}
  \item Filmzitat I: Die Szene, in der sich die Silhouetten der Schüler an der Ausgrabungsstätte gegen den Sonnenuntergang abzeichnen, ist eine visuelle Verbeugung vor \glqq Jäger des verlorenen Schatzes\grqq .
  \item Filmzitat II: Als Lisa in Rektor Skinners Büro stürmt, sagt sie etwas von einem \glqq Gefallen\grqq\ einlösen. Als Skinner das hört, dreht er seinen Stuhl und die Jalousie wirft ein etwas seltsames Licht auf ihn. Diese Szene ähnelt sehr stark einer Szene aus \glqq Der Pate\grqq . Am Anfang befindet sich Don Corleone in einem ähnlichen Zimmer, auch mit heruntergelassenen Jalousien und sie reden auch über einen Gefallen.
  \item Unter den Verdächtigen, die im Polizeirevier von Springfield zusammen getrieben wurden, ist auch \glqq Jimmy the Scumbag\grqq\ (Jimmy, der Schleimbeutel), der zuletzt in \glqq Lisa will lieben\grqq\ (siehe \ref{4F01}) dabei war.
  \item Homer muss für 235 Parkvergehen eine Strafe von 175 Dollar zahlen.
  \item In Homers \glqq Safe Deposit Closet\grqq\ (Sicherheitsaufbewahrungsschrank) befinden sich folgende Gegenstände: Der antike Dreispitz der Flanders und die Nacht\-wäch\-ter\-glocke (aus \glqq Das geheime Bekenntnis\grqq , \ref{3F13}), ein Sechserpack Billy-Bier (aus \glqq Der Fahrschüler\grqq\ , \ref{8F21}), Homers Boxhandschuhe (aus \glqq Auf in den Kampf\grqq , \ref{4F03}), Homers Grammy (aus \glqq Homer und die Sangesbrüder\grqq , \ref{9F21}), eine Tüte Farmer Homers XX Zucker (aus \glqq Lisas Rivalin\grqq , \ref{1F17}), Homers weißer Cowboyhut (aus \glqq Homer auf Abwegen\grqq , \ref{8F19}), eine Packung Meister Glanz (aus \glqq Marge als Seelsorgerin\grqq , \ref{4F18}), die animatronischen Köpfe von Itchy \& Scratchy (aus \glqq Der unheimliche Vergnügungspark\grqq , \ref{2F01}), der Helm von Homers Raumanzug (aus \glqq Homer der Weltraumheld\grqq , \ref{1F13}), seine \glqq Mr. Plow\grqq -Jacke (aus \glqq Einmal als Schnee\-könig\grqq , \ref{9F07}), das Hemd aus seiner Zeit als Dancin' Homer (aus \glqq Das Maskottchen\grqq , \ref{7F05}) und eine Bowling-Trophäe (aus \glqq Homers Bowling-Mannschaft\grqq , \ref{3F10}).
  \item Unter den Leuten, die sich bei Sonnenuntergang um den Engel scharen, sieht man eine gelbe Version von Dr. Hibberts Gattin.
  \item In dieser Episode küsst Smithers Mr. Burns.
  \item Lisa ist bereits zum 13. Mal Gast in Kent Brockmans Sendung \glqq Smartline\grqq .
\end{itemize}
}
	
\subsection{Todesfalle zu verkaufen}\label{5F06}
Marge macht einen Kurs als Immobilienmaklerin. Anschließend muss sie in der ersten Woche ein Haus verkaufen, um nicht gefeuert zu werden. Den Flanders dreht sie ein Anwesen an, in dem Menschen umgebracht worden sind. Mit schlechtem Gewissen erklärt sie ihnen hinterher, dass es sich um ein Mordhaus handelt. Die Flanders sind begeistert. Homer hat Ärger mit Snake, der sein Auto wieder haben will. Homer erwarb das Auto bei einer Polizeiauktion. Mit dem Wagen rast Homer in das Haus. Es stürzt in sich zusammen. Marge gibt den Flanders ihren Anzahlungsscheck zurück und wird wegen Redlichkeit entlassen.

\notiz{
\begin{itemize}
  \item Snakes Autokennzeichen lautet \glqq GR8 68\grqq .
  \item Snakes Gefangenennummer ist die 7F20 (7F20 ist der Produktionscode der Folge \glqq Kampf dem Ehekrieg\grqq\ (siehe \ref{7F20}), in der Snake seinen ersten Auftritt hatte).
  \item Snakes Klavierdraht trennt Kirk Van Houtens Arm ab. Auf dem Ar\-beits\-amt sieht man ihn später mit einem Gipsarm. In der Schlange stehen außer ihm noch eine Frau, die an eine ungepflegte Lurleen Lumpkin\index{Lumpkin!Lurleen} erinnert, Larry Burns\index{Burns!Larry} (erstmals dabei in \glqq Mr. Burns Sohn Larry\grqq , \ref{4F05}), Jimbo Jones, Simpsons-Autor George Meyer, sowie George Bush sen\index{Bush!George}.
  \item In dieser Folge sagt Homer: \glqq Versuchen ist der erste Schritt zum Versagen.\grqq
  \item Erster Auftritt von Gil und Cookie Kwan.
  \item Gil gibt an, 42 Jahre das Makler-Büro geleitet zu haben.
  \item Ned Flanders wohnt in der 744 Evergreen Terrace.
\end{itemize}
}
	
\subsection{Die Lieblings-Unglücksfamilie}\label{5F07}
Bart ist neugierig auf die Weihnachtsgeschenke. Er zündet den Weihnachtsbaum an und vernichtet Baum und Geschenke. Bart vergräbt die Überreste im Garten. Er behauptet, ein Einbrecher habe das Haus ausgeraubt. Mitleidig spenden die Bürger Geld. Homer kauft sich davon sofort ein teures Auto, das kurz darauf in einem See versinkt und explodiert. Als der Schnee schmilzt, kommen die Reste zum Vorschein und Barts Schwindel fliegt auf. Nach anfänglicher Wut auf die Simpsons, verzeihen die Spender den Simpsons und als Entschädigung nehmen sie die Einrichtungsgegenstände aus dem Haus mit.

\notiz{
\begin{itemize}
  \item Filmzitat: Marges Satz im Original \glqq \dots you won't believe what's happened! It's a miracle!\grqq\ hat Maude Flanders in \glqq Ein Fluch auf Flanders\grqq\ (siehe \ref{7F23}), auch schon zu Ned gesagt. Das Zitat und folgende Szene, in der die Stadt sich zusammentut, um den Simpsons zu helfen, ist eine Reminiszenz an \glqq It's A Wonderful Life!\grqq\ (Ist das Leben nicht schön?).
  \item Das ist die dritte Weihnachtsfolge der Simpsons. Die anderen beiden sind \glqq Es weihnachtet schwer\grqq\ (siehe \ref{7G08}) und \glqq Das schwarze Schaf\grqq\ (siehe \ref{3F07}).
  \item Das älteste Kind der Hibberts (zuletzt dabei in \glqq Bart verkauft seine Seele\grqq , \ref{3F02}) verbringt Weihnachten nicht mit seiner Familie.
  \item Viele der tanzenden Senioren im Springfielder Altenheim kopieren Bewegungen aus dem Peanuts-TV-Special \glqq A Charlie Brown Christmas\grqq\ (Fröh\-li\-che Weih\-nachten, Charlie Brown).
  \item Barney fährt seinen Plow King-Truck (ein Schneepflug), der zuletzt in \glqq Einmal als Schnee\-kö\-nig\grqq\ (siehe \ref{9F07}), auftauchte.
  \item Unter den Stofftieren, die Wiggum aus dem Haus der Simpsons stiehlt, sind ein Hase à la \glqq Life in Hell\grqq\ und ein Affe mit Fez.
\end{itemize}
}

\subsection{Die armen Vagabunden}\label{5F08}
Bart hat auf einem Jahrmarkt gewütet. Nun muss er seine Schulden abarbeiten. Vor lauter Begeisterung lässt sich Homer sogleich mit engagieren. Sie vertreten Cooder\index{Cooder} und Spud\index{Spud} in deren Wurfbude. Da taucht Polizeichef Wiggum auf und bezichtigt Homer und Bart des Betruges. Er beschlagnahmt die Bude. Nun stehen Cooder und Spud auf der Straße. Hilfsbereit nimmt Homer die beiden mit nach Hause. Die Gäste übernehmen das Haus und sperren die Simpsons aus. Homer hilft nur noch ein Trick.

\notiz{
\begin{itemize}
  \item Enthüllung: In dieser Folge macht Spud eine Bemerkung über Lisas blaue Augen.
  \item In dieser Folge wird zum dritten Mal jemandem mit einer Wasserpistole ins Gesicht geschossen, weil er für eine Jahrmarktsfigur gehalten wird, in deren Mund man treffen muss, um einen Ballon aufzublasen. In \glqq Ein Sommer für Lisa\grqq\ (siehe \ref{3F22}), schoss Bart auf Krusty und Martin bekam in \glqq Lisa, die Schönheitskönigin\grqq\ (siehe \ref{9F02}) eine Ladung ab.
  \item Auf dem Glasbodenboot steht Karl\index{Karl}, erstmals dabei in \glqq Karriere mit Köpf\-chen\grqq\ (siehe \ref{7F02}) neben Lisa. Er taucht in dieser Staffel später noch einmal auf: In \glqq Die Trillion-Dollar-Note\grqq\ (siehe \ref{5F14}) steht er im Postamt in der Schlange.
  \item Auf der Fahrt mit dem Glasbodenboot kann man folgende Gegenstände sehen: Einen Einkaufswagen, einen Auspufftopf, ein Fass \glqq Li'l Lisa Slurry\grqq\ (Klein Lisas patentiertes Tier\-pü\-ree) aus \glqq Der alte Mann und Lisa\grqq\ (siehe \ref{4F17}) ein Radio, Reifen und eine Tonne mit radioaktivem Abfall.
  \item Obwohl die Cooders angeblich sämtliche Fenster der Simpsons zugenagelt haben, sieht man, kurz bevor Homer aus dem Baumhaus fällt, die hinteren Fenster des Hauses ohne Bretter.
  \item Marge weißt Homer auf seinen vierfachen Bypass am Herzen hin, bevor er mit der Achterbahn fährt.
\end{itemize}
}

	
\subsection{Die sich im Dreck wälzen}
Homer ist es schon seit einiger Zeit leid, immer die Mülltonne auf die Straße stellen zu müssen, damit sie geleert werden kann. Er beschließt kurzerhand, die Müllentsorgung selbst in die Hand zu nehmen. Er lässt sich zur Wahl des Referenten der Müllabfuhr aufstellen und wird prompt gewählt. Leider kommt er mit seinem Budget nicht so ganz zurecht. Schon nach einem Monat hat er das ganze Geld verschleudert. Er kann zwar Geld auftreiben, doch dies führt dazu, dass aus allen Ecken in Springfield der Müll hervorquillt, deshalb muss die komplette Stadt umziehen.

\notiz{
\begin{itemize}
  \item Die Warteschlange der meldepflichtigen Sexualstraftäter besteht aus Jimmy, dem Schleimbeutel, Patty, Selma, Freddy Quimby und Moe.
  \item Als Homer von der Security hinter der Bühne verprügelt wird, spielen U2 \glqq In The Name Of Love\grqq .
  \item Fernsehzitat: Während des Liedes der Müllmänner sieht man \glqq Oskar in der Tonne\grqq\ aus der Sesamstraße.
  \item Fehler: Bei der Szene, als Homer und Bart den Müll aus dem Fenster werfen, sitzt Bart erst rechts von Lisa, danach links.
  \item 1998 gewann diese Episode in Amerika einen Emmy.
  \item Diese Episode wurde Linda McCartney gewidmet. Sie hatte Lisa in \glqq Lisa als Vegetarierin\grqq\ (siehe \ref{3F03}) den fleischlosen Lebensstil beigebracht.
\end{itemize}
}
	
\subsection{Krustys letzte Versuchung}\label{5F10}
Clown Krusty ist zutiefst verzweifelt: Was unmöglich schien, ist schließlich doch wahr geworden -- Krusty hat bei seinen Auftritten keinen Erfolg mehr. Weil auch die Ratschläge seines berühmten Kollegen Jay Leno\index{Leno!Jay} nichts helfen, will sich Krusty sowohl aus der Werbung, als auch aus dem Berufsleben zurückziehen. Bei seiner Abschiedsansprache sagt er einfach mal die Wahrheit und reißt damit das Publikum zu Lachsalven hin. Daraufhin ist klar: Krusty muss einfach weitermachen.

\notiz{
\begin{itemize}
  \item Gil\index{Gil} arbeitet als Verkäufer bei \glqq Goody New Shoes\grqq\ (Prima neue Schuhe).
  \item Als Krusty nach durchzechter Nacht im Studio erscheint, wird er von Kent Brockman vertreten.
  \item In dieser Folge ist erstmals eine \glqq Simpsons\grqq -Figur mit blutunterlaufenen Augen zu sehen.
  \item Als Krusty ein Bad nimmt, sind seine Herzschrittmacher-Narbe und seine überzählige dritte Brustwarze zu sehen.
  \item Krusty und Jay Leno trinken im \glqq Java the Hutt\grqq\ Kaffee, eine Verballhornung der Figur \glqq Jabba the Hutt\grqq\ aus \glqq Star Wars\grqq .
  \item In dieser Episode liest Krusty aus der Zeitung vor und liest von seinen Notizblöcken ab.
\end{itemize}
}
	
\subsection{Der blöde UNO-Club}\label{5F11}
Auf der Fahrt zum Uno-Modell-Treffen stürzt der Bus mit den Kindern von Springfield ins Meer. Sie stranden auf einer menschenleeren Insel. Ihre Essensvorräte werden bald von einem Wildschwein aufgefressen. Aus Rache verzehren sie eben das Wildschwein. Nach und nach bauen sich die Kids eine kleine Gemeinde auf. Währenddessen will Homer übers Internet eine eigene Internetfirma aufbauen (\glqq Compuglobal Hypermeganet\index{Compuglobal Hypermeganet}\grqq ). Kaum floriert das Geschäft, wird er von Bill Gates aufgekauft.

\notiz{
\begin{itemize}
  \item Historisches Zitat: Rektor Skinner ruft zur Ordnung, indem er mit seinem Schuh auf den Tisch hämmert -- wie Nikita Chruschtschow\index{Chruschtschow!Nikita} in seiner berüchtigten \glqq We Will Bury You\grqq\ (Wir werden euch begraben) Rede vor den Vereinten Nationen.
  \item Filmzitat: Nachdem Milhouse sich an einer Liane über den Abgrund geschwungen hat, bittet Bart ihn, sie zurückzuwerfen, doch Milhouse weigert sich -- so in etwa erging es Indiana Jones in \glqq Jäger des verlorenen Schatzes\grqq . Die ganze Episode basiert auf dem Buch/Film \glqq Der Herr der Fliegen\grqq .
  \item Auf Homers Schreibtisch stehen der trinkende Vogel, der in \glqq Der vermisste Halbbruder\grqq\ (siehe \ref{8F23}) erstmals zu sehen war und eine Nagelskulptur mit einem Abdruck seines Gesichtes.
  \item Ralph bemalt sein Gesicht wie Peter Criss\index{Criss!Peter} von der Rockgruppe KISS\index{KISS}.
  \item Als Milhouse beschuldigt wird, die Essensvorräte aufgegessen zu haben, wird er von Lisa verteidigt.
  \item Ned Flanders besitzt auch die Firma \glqq Flancrest Enterprises\grqq\index{Flancrest Enterprises}, die religiöse Häkeldeckchen über das Internet verkauft. Die Adresse der Firma lautet 740 Evergreen Terrace. In der Episode \glqq Todesfalle zu verkaufen\grqq\ (siehe \ref{5F06}) lautet die Adresse der Flanders 744 Evergreen Terrace.
\end{itemize}
}
	
\subsection{Eine Frau für Moe}\label{5F12}
Moe hat sich in Renée\index{Renée} verliebt. Er glaubt, ihr imponieren zu müssen. Das geht ins Geld. Moe bittet Homer, sein Auto zu stehlen. Er will von der Versicherung Geld kassieren. Homer wird dabei erwischt und landet im Gefängnis. Moe denkt nicht daran, ihm zu helfen. Renée hält Moe inzwischen für verrückt und verschwindet. Verzweifelt setzt Moe aus Versehen seine Kneipe in Brand. Mit dem inzwischen aus dem Gefängnis entflohenen Homer richtet er im Haus der Simpsons eine Kneipe ein.

\notiz{
\begin{itemize}
  \item Beim Tanzen in Stu's Disco\index{Stu's Disco} sieht es aus, als hätte Luann Van Houten braune Haare.
  \item Moe, Renée, Homer und Marge gehen zusammen ins \glqq The Gilded Truffle\grqq\ (Zum goldenen Trüffel), der erstmals in \glqq Alpträume\grqq\ (siehe \ref{8F02}) vorkam.
  \item Zu den Teilnehmern, die an der von der Polizei veranstalteten Mit\-ter\-nachts-Kreuz\-fahrt teilnehmen, zählen auch Rektor Skinner und Mrs. Krabappel.
  \item Homers Gefangenennummer lautet 5F124670 (der Produktionscode dieser Episode lautet 5F12).
  \item Barney rettet Moe und Homer aus der brennenden Kneipe.
  \item In dieser Episode sind folgende Lieder zu hören: \glqq One Bourbon, One Scotch, One Beer\grqq\ von George Thorogood \& the Destroyers, \glqq Brick House\grqq\ von den Commodores, und \glqq I'm a Believer\grqq\ von den Monkees.
  \item Moe gibt an, dass er früher viel geboxt hat.
\end{itemize}
}
	
\subsection{Der merkwürdige Schlüssel}\label{5F13}
Marge bringt Bart und Ralph Wiggum als Spielkameraden zusammen. Wahrscheinlich hat sie sich erhofft, damit Ralphs Popularität und Selbstbewusstsein zu steigern, aber das hilft sicherlich nicht Barts Image. Das gilt natürlich nur so lange, bis dieser in den Besitz von Chief Wiggums Universalschlüssel gelangt. Unglücklicherweise vergessen sie bei ihrer Entdeckungsreise durch die Stadt, den Sicherheitshebel eines verlassenen elektrischen Stuhls wieder umzulegen, was Bürgermeister Quimbys Leben in Gefahr bringen könnte.

\notiz{
\begin{itemize}
  \item Im Knowledgeum\index{Knowledgeum} sehen wir Database\index{Database}, Ham\index{Ham} und Protokoll\index{Protokoll}, alles Mitglieder der Super-Freunde. Sie (und ihre Organisation) wurden in \glqq Barts Komet\grqq\ (siehe \ref{2F11}) eingeführt.
  \item Unter Ralphs Spielsachen befinden sich ein Hase à la \glqq Life in Hell\grqq\ und ein Affe mit Fez. Beides war zuletzt in \glqq Die Lieblings-Unglücksfamilie\grqq\ (siehe \ref{5F07}) zu sehen, als Chief Wiggum die Sachen aus dem Haus der Simpsons gestohlen hat.
  \item Auf den Akten in Chief Wiggums \glqq Forbidden Closet of Mystery\grqq\ (Verbotener Schrank der Rätsel) steht \glqq Confidential -- Do Not Remove from Police Station\grqq\ (Vertraulich -- Im Polizeirevier belassen). Bart sieht die Akten von \glqq Riviera, Dr. N.\grqq , \glqq Krustofsky, H.\grqq\ (alias Krusty der Clown), \glqq Terwilliger, R.\grqq\ (alias Sideshow Bob) und \glqq Simpson, Homer\grqq . In Homers Akte sieht er hinein und stellt fest, dass dieser bereits sechs mal im Gefängnis war. Er merkt an, Marge war nur zweimal im Gefängnis.
  \item Der Spielwarenladen, in den Ralph und Bart nachts einsteigen, trägt den Namen \glqq J.R.R. Toyskin's\grqq . Der Name ist eine Anspielung auf den Schriftsteller John Ronald Reuel Tolkien, der vor allem wegen seines Romans \glqq The Lord of the Rings\grqq\ (Herr der Ringe) bekannt ist.
\end{itemize}
}
	
\subsection{Die Trillion-Dollar-Note}\label{5F14}
Homer hat Steuerschulden. Er wird von den Behörden gezwungen, für das FBI als Spitzel zu arbeiten. Unter anderem soll er die Trillion-Dollar-Note ausfindig machen, die Mr. Burns vor Jahrzehnten unterschlagen hat und eigentlich nach Europa zur Finanzierung des Aufbaus nach dem Zweiten Weltkrieg hätte bringen sollen. Mr. Burns war damals der reichste Bürger Amerikas. Deshalb wurde vermutet, er würde diese Banknote nicht unterschlagen. Homer entdeckt die Banknote. Aber statt sie den Behörden zu übergeben, flüchtet er mit Mr. Burns und Smithers nach Kuba. Hier kassiert Fidel Castro den Trillion-Dollar-Schein und setzt die Drei anschließend auf einem Floß im Meer aus.

\notiz{
\begin{itemize}
  \item Lucius Sweet\index{Sweet!Lucius}, der zuletzt in \glqq Auf in den Kampf\grqq\ (siehe \ref{4F03}) dabei war, wischt sich im Warteraum der IRS (Amerikanische Steuerbehörde) die Stirn mit einem Hundert-Dollar-Schein ab.
  \item In Smithers Wohnung sind auf einem Regal mehrere Malibu-Stacy-Puppen zu sehen. Seine Leidenschaft für diese Puppen wurde in \glqq Lisa kontra Malibu Stacy\grqq\ (siehe \ref{1F12}) enthüllt.
  \item Das \glqq El Duffo O Muerte\grqq -Schild (Trink Duff Bier oder du bist tot) in Kuba zeigt ein Bild des kommunistischen Revolutionärs Che Guevara. Die letzte Anspielung auf ihn gab es in \glqq Wer erschoss Mr. Burns? -- Teil 2\grqq\ (siehe \ref{2F20}). Tito Puente\index{Puente!Tito} spielte in einem Club namens Chez Guevara.
  \item Marge gibt an, das Bild, das über der Coach im Wohnzimmer hängt und ein Schiff zeigt, selbst gemalt zu haben.
\end{itemize}
}
	
\subsection{Die neusten Kindernachrichten}\label{5F15}
Lisa ist zur Moderatorin der Kindernachrichten-Sendung avanciert. Bald macht ihr Bart den Rang streitig: Er rührt die Zuschauer mit sentimentalen Beiträgen. Mithilfe von Hausmeister Willie will sich Lisa an ihrem Bruder rächen. Bart hatte Willies Hütte zerstört, nachdem ihm dieser sein Skateboard abgenommen hatte. Willie nimmt diese Aufgabe zu ernst und Bart gerät in ernsthafte Gefahr. Lisa kann ihn in letzter Sekunde retten. Sie beschließt, fortan mit ihm zusammenzuarbeiten, dennoch werden die Kindernachrichten abgesetzt.

\notiz{
\begin{itemize}
	\item Die Schwester von Kent Brockman, die ebenfalls Journalistin ist, arbeitet in New York bei CNN.
	\item Erster Auftritt der Katzenlady\index{Katzenlady}.
	\item Homers Hilfsaffe heißt Mojo\index{Mojo}.
	\item Zu Lisas Kindernachrichtenteam gehören außer Bart noch Nelson und Milhouse.
\end{itemize}
}

	
\subsection{König der Berge}\label{5F16}
Weil Homers Fettleibigkeit Bart ziemlich peinlich wird, entscheidet er sich, dass ein bisschen Training nicht schaden könnte. Mithilfe von Rainier \glqq McBain\grqq\ Wolfcastle und einer reinen Apfelmus-Riegel-Diät entsteht ein neuer Homer, der jede Herausforderung annimmt, die Brad\index{Brad} und Neil\index{Neil} von der Apfelmus-Firma ihm aufschwatzen. Sie überreden ihn, das \glqq Mörderhorn\index{Mörderhorn}\grqq\ zu besteigen. Der Berg gilt als unbezwingbar und birgt manches Geheimnis. Zwei Sherpas werden ihn im Auftrag der Firma begleiten und sicher hinaufbringen. Homer bemerkt dies und schickt die Sherpas weg. Ganz allein erklimmt er den Berg unter den merkwürdigsten Umständen. Bart hat daraufhin größte Hochachtung vor seinem Vater, der dies nur für seinen Sohn getan hat.

\notiz{
\begin{itemize}
	\item Der höchste Berg in Springfield: Das Mörderhorn. Dies ist eine Anspielung auf das Matterhorn in den Alpen.
	\item Homers Vater versuchte 1928 mit C. W. McAllister\index{McAllister!C. W.} das Mörderhorn zu besteigen. Diese Unternehmung scheiterte und McAllister kam dabei um.
	\item In dieser Folge ist Rod der größere der Flanders Söhne.
	\item Der Originaltitel dieser Episode \glqq King Of The Hill\grqq\ ist eine Anspielung auf die gleichnamige Zeichentrickserie von Mike Judge\index{Judge!Mike} (\glqq Beavis und Butthead\grqq) und Greg Daniels (\glqq Simpsons\grqq).
	\item Der Powerriegel heißt \glqq Powersauce\index{Powersauce}\grqq .
\end{itemize}
}
	
	
\subsection{Die Kugel der Isis}\label{5F17}
Lisa will unbedingt ins Museum. Dort gibt es zur Zeit eine Isis-Ausstellung\index{Isis}. Da niemand für sie Zeit hat, fährt sie allein mit dem Bus. Der fährt jedoch statt ins Museum ganz woanders hin. Aber sie findet wieder in die Stadt zurück, wo Homer sie bereits sucht. Schließlich steigen die beiden nachts heimlich in die Isis-Ausstellung ein. Dann machen die beiden eine sensationelle Entdeckung: Homer und Lisa kommen dahinter, dass die Kugel der Isis eine antike Musikbox ist.

\notiz{
\begin{itemize}
	\item Auf dem russischen Markt ist der Khlav-Kalash-Verkäufer\index{Khlav-Kalash} aus \glqq Homer und New York\grqq\ (siehe \ref{4F22}) zu sehen.
	\item In Homers Schublade im Kraftwerk befindet sich neben dem Alleskleber unter anderem noch ein Revolver und ein Brennstab.
	\item Moe fährt mit dem Taxi in die Klinik für Geschlechtskrankheiten.
	\item Die Kugel der Isis ist in der Episode \glqq Bibelstunde, einmal anders\grqq\ (siehe \ref{AABF14}) in der Pyramide zu sehen.
\end{itemize}
}
	
\subsection{Die Gefahr, erwischt zu werden}\label{5F18}
Homer und Marge entdecken, dass ihr Liebesleben neuen Auftrieb erhält, wenn sie fürchten, dabei erwischt zu werden. Also suchen sie sich für ihre Liebesspiele die gewagtesten Orte aus. Auf einem Minigolfplatz werden sie ertappt. Nackt ergreifen sie die Flucht. Sie landen mit einem Ballon in einem Fotoballstadion voller Zuschauer. All dies stört die beiden aber überhaupt nicht. Ganz im Gegenteil: Es verleiht ihrem Liebesleben den gewünschten Auftrieb.

\notiz{
\begin{itemize}
  \item Barts Tafelspruch\footnote{Ich war nicht der Denkanstoß für \glqq Kramer\grqq.} ist eine Anspielung auf die Sitcom \glqq Seinfeld\index{Seinfeld}\grqq.
  \item In seinem Sakko findet Homer ein Programm von Frank \glqq Grimey\grqq\ Grimes\index{Grimes!Frank} Beerdigung. Grimes wurde in \glqq Homer hatte einen Feind\grqq\ (siehe \ref{4F19}) vorgestellt (und kurz darauf beerdigt).
  \item Homers Auto hat immer noch eine Trackstar-8-Track-Stereoanlage, ein veraltetes Kassettensystem.
  \item Homer und Marge stehlen sich in die Windmühle, um zu \glqq kuscheln\grqq . Sie nennen sie \glqq ihr altes Liebesnest\grqq\ und sagen, dass sie darin Bart gezeugt haben. In \glqq Blick zurück aufs Eheglück\grqq\ (siehe \ref{8F10}) wurde Bart allerdings im Schloss des Golfplatzes empfangen.
  \item Moes Klage \glqq Won't somebody please think of the children?\grqq\ (Könnte vielleicht ausnahmsweise mal jemand an die Kinder denken?) stammt ursprünglich von Helen Lovejoy.
  \item Gil arbeitet jetzt als Gebrauchtwagenhändler.
  \item Auf der Ausgabe des Springfield Shopper mit der Schlagzeile \glqq Local Couple Bares All!\grqq\ (Einheimisches Paar enthüllt alles!) ist eine weitere Schlagzeile zu lesen: \glqq Police Dog Clings to Life\grqq\ (Polizeihund ringt um sein Leben). Der Hund war zuvor in der Folge zu sehen, wie er winselnd davonlief, nachdem er an Homers Unterwäsche geschnuppert hatte.
  \item Während des Abspanns läuft \glqq Rock the Casbah\grqq\ von The Clash.
  \item Marge und Homer feiern ihren elften Hochzeitstag.
  \item In dieser Folge ist Rod wieder der Größere der beiden Flanders Kinder.
\end{itemize}
}
	
\subsection{In den Fängen einer Sekte}\label{5F23}
Homer schließt sich einer Sekte an. Den Anhängern wird die absolute Glück\-se\-lig\-keit versprochen -- auf einem anderen Planeten. Auch seine Familie wird gezwungen, in die Sekte einzutreten. Doch Marge setzt alles daran, ihre Freiheit wieder zu erlangen. Mithilfe von Reverend Lovejoy, Nachbar Flanders und Hausmeister Willie gelingt dies Marge auch: Sie kann Homer und die Kinder von dem Gelände der Sekte entführen. Mit Bier bringt sie ihren Homer wieder zur Vernunft.

\notiz{
\begin{itemize}
  \item Die Frage \glqq Will there be beer?\grqq\ (Gibt's da auch Bier?) stellt Homer auch in \glqq Der merk\-wür\-di\-ge Schlüssel\grqq\ (siehe \ref{5F13}).
  \item Das \glqq K\grqq , das Mr. Burns als Zeichen verwenden will, gehört der Firma Kelloggs.
  \item In dieser Episode rennt Homer mitten in einem Gespräch davon, um einem Vogel nachzujagen. In \glqq Homer an der Uni\grqq\ (siehe \ref{1F02}) haut Homer vor einer Prüfung ab, um ein Eichhörnchen zu jagen, und in \glqq Angst vorm Fliegen\grqq\ (siehe \ref{2F08}) setzt er einem Hund mit buschigem Schwanz nach.
  \item Fernsehzitat: Bei ihrer Flucht aus dem Lager wird Marge von einer Riesenblase verfolgt, die aus dem Fluss aufsteigt und schließlich Hans Maulwurf umschließt -- eine Referenz an die Kult-Fernsehserie \glqq The Prisoner\grqq\ (Nummer Sechs).
  \item Filmzitat: Willie macht Marge und Reverend Lovejoy auf sich aufmerksam, indem er mit den Fingernägeln über eine schmutzige Fensterscheibe kratzt. In \glqq Homer an der Uni\grqq\ (siehe \ref{1F02}) tat Bart dasselbe mit einer Tafel, um die Aufmerksamkeit von Homer, Benjamin, Doug und Gary auf sich zu lenken. Beide Fälle sind eine direkte Referenz an Robert Shaws aufmerksamkeitsheischende Methode als Haifischjäger Quint in dem Film \glqq Jaws\grqq\ (Der weiße Hai).
\end{itemize}
}
	
\subsection{Und nun alle singen und tanzen}
Die Simpsons einmal ganz anders: Homer, Marge, Bart und Lisa lassen in der Form einer Musical-Passage die letzten Jahre ihres aufregenden Lebens in dem beschaulichen Städtchen Springfield Revue passieren. Natürlich geht es auch bei der Schließung eines Nachtclubs, der Auflösung der Freimaurerloge und dem lustigen Leben in der Bürgerversammlung recht turbulent zu. Eines aber machen die Simpsons ganz klar: Mit Musik kommt man auch mit den Problemen besser zurecht.

\notiz{
\begin{itemize}
  \item Die Musikausschnitte dieser Folge stammen aus den folgenden Episoden:
  \begin{itemize}
    \item \glqq Homer und die Sangesbrüder\grqq\ (siehe \ref{9F21}),
    \item \glqq Der beliebte Amüsierbetrieb\grqq\ (siehe \ref{4F06}),
    \item \glqq Auf Wildwasserfahrt\grqq\ (siehe \ref{1F06}),
    \item \glqq Apu der Inder\grqq\ (siehe \ref{1F10}),
    \item \glqq Krusty, der TV-Star\grqq\ (siehe \ref{9F19}),
    \item \glqq 25 Windhundwelpen\grqq\ (siehe \ref{2F18}),
    \item \glqq Homer kommt in Fahrt\grqq\ (siehe \ref{9F10}),
    \item \glqq Bart verkauft seine Seele\grqq\ (siehe \ref{3F02}) und 
    \item \glqq Homer der Auserwählte\grqq\ (siehe \ref{1F09}).
  \end{itemize}
  \item Die anderen \glqq Simpsons Clip Shows\grqq , auf die sich das Transparent bezieht, sind: \glqq Nur ein Aprilscherz\grqq\ (siehe \ref{9F17}), \glqq Romantik ist überall!\grqq\ (siehe \ref{2F33}) und \glqq Die 138. Episode, eine Sondervorstellung\grqq\ (siehe \ref{3F31}).
\end{itemize}
}



\section{Staffel 10}

\subsection{Kennst du berühmte Stars?}\label{5F19}
Durch einen Zufall landet Homer bei einem Parasailing-Ausflug im Haus von Kim Basinger und Alec Baldwin. Das Schauspieler-Paar verbringt den Sommer in Springfield, möchte jedoch unerkannt bleiben. Homer schwört ihnen, sie nicht zu verraten und arbeitet für die Schauspieler als Mädchen-für-alles. Natürlich kann Homer seinen Mund nicht halten und verrät die Neuigkeit seinen Freunden. Kurze Zeit später weiß die ganze Stadt Bescheid.

\notiz{
\begin{itemize}
  \item Unter den Ausflüglern am See sind Lewis, Apu, Ralph, Todd und Ned Flanders, Rektor Skinner, Mrs. Krabappel, Cletus und Wendell.
  \item Während Marge im Keller die Wäsche macht, sieht man im Hintergrund den olmekischen Indianerkopf aus \glqq Der Lebensretter\grqq\ (siehe \ref{7F22}).
  \item Homer gesteht, selbst nicht genau zu wissen, wo Springfield liegt.
  \item Waylon Smithers hat Schuhgröße 45.
  \item Der Produzent, dem Ron Howard am Ende der Folge Ideen erzählt, ist Brian Grazer, sein Partner in der gemeinsamen Produktionsfirma \glqq Imagine Films\grqq .
\end{itemize}
}
	
\subsection{Ein jeder kriegt sein Fett}\label{5F20}
Homer hat beschlossen, durch den Verkauf von Fett reich zu werden. Doch das Geschäft mit altem Fett will nicht so recht anlaufen. Zusammen mit Sohn Bart wird er dann zu allem Überfluss auch noch von Hausmeister Willie erwischt, als sie in der Schulküche altes Fett klauen wollen. Tochter Lisa hat ganz andere Sorgen. Die neue Mitschülerin Alex Whitney\index{Whitney!Alex}, die unbedingt einen Tanzabend veranstalten möchte, macht ihr zu schaffen.

\notiz{
\begin{itemize}
	\item Homer nennt sein Fettunternehmen \glqq Simpsons \& Son Grease Co.\grqq .
	\item Für vier Pfund Fett bekommt Homer 63 Cent.
	\item In dieser Episode bietet Lisa Milhouse an, mit ihr auf den Tanzabend zu gehen. Milhouse lehnt jedoch ab, da er bereits eine Tanzpartnerin hat.
	\item Willie ist in North Kilt-Town in Schottland aufgewachsen.
\end{itemize}
}

	
\subsection{Im Schatten des Genies}\label{5F21}
Homer will unter die Erfinder gehen, doch das erweist sich als gar nicht so einfach: Was er sich auch ausdenkt, alles gibt es bereits! Selbst der Stuhl mit sechs Beinen, mit dem man nicht umfallen kann, wurde bereits von Thomas Edison erfunden. Um nun eben diesen Stuhl zu zerstören, nimmt Homer seinen selbstgebastelten Elektrohammer und macht sich auf zum Edison-Museum. Doch der Besuch dort nimmt eine ungewöhnliche Wende.

\notiz{
\begin{itemize}
  \item Der Text der Digitalanzeige am Tor von Homers potentiellem Friedhof lautet \glqq The Graveyard of the Future\grqq\ (Der Friedhof der Zukunft). Auf einem der Gräber steht \glqq Tamzarian\grqq , Rektor Skinners tatsächlicher Nachname.
  \item Bei Homers Beerdigung hält Barney vier Oscars. Sein Talent als Filmemacher offenbarte er in \glqq Springfield-Film-Festival\grqq\ (siehe \ref{2F31}).
  \item Bei Homers Beerdigung sind außerdem zugegen: ein Roboter aus der TV-Serie \glqq Lost in Space\grqq\ (Verschollen zwischen fremden Welten), Heckle und Jeckle (zwei Raben aus einer alten amerikanischen Zeichentrickserie), Lenny (der eine Präsidentenschärpe trägt) und Ned Flanders (gekleidet wie ein Kardinal).
  \item Eines der Bücher, die Homer in der Bibliothek liest, ist \glqq A Child's Garden of Edison\grqq\ (Eine Anspielung auf den Gedichtband \glqq A child's garden of verses\grqq\ von Robert Louis Stevenson).
  \item Hier sagt Homer, er sei 38 Jahre alt, woraufhin ihn Marge berichtigt und sagt, er sei bereits 39 Jahre alt.
  \item Homers vier Erfindungen:
  \begin{itemize}
	  \item Elektrischer Allzweckhammer,
	  \item Alles-OK-Alarmanzeiger,
	  \item Make-Up-Flinte und
	  \item Toilettensessel. 
  \end{itemize}
  \item Im Edison-Museum ist im Souvenirladen ein AC/DC-T-Shirt zu sehen.
	\item Homer hat an der Tafel die Gleichung $3987^{12} + 4365^{12} = 4472^{12}$ stehen, die nach Fermats\index{Fermat} großem Satz nicht richtig sein kann. David X. Cohen, benutzte ein C-Programm zur Berechnung dieser Gleichung (\cite{DavidXCohenC}). Das Programm ist in Abschnitt \ref{CProgramme} abgedruckt.
\end{itemize}
}
	
\subsection{Bart brütet etwas aus}\label{5F22}
Bart hat aus Versehen einen Vogel erschossen. Um sein schlechtes Gewissen zu beruhigen, versucht er, wenigstens dessen Eier zu retten -- sprich auszubrüten. Doch statt der erwarteten Piepsvögel schlüpfen kleine Eidechsen aus den Eiern, die zu allem Überfluss auch noch einer gefährlichen Art angehören! Doch letztlich sollen sich die vögelfressenden bolivianischen Baum\-ei\-dech\-sen als Segen für die Stadt Springfield erweisen.

\notiz{
\begin{itemize}
  \item Reverend Lovejoy ist in der Bibliothek, um sich die Bibel für den Sonntagsgottesdienst auszuleihen. Er leiht sich seit neun Jahren für jeden Sonntag die Bibel aus.
  \item Folgende Personen gehören den Vogelkundlern an: Seymour Skinner, Jasper, Moe Szyslak, Charles M. Burns, Waylon Smithers, Dr. Julius Hibbert, Edna Krabappel und Apu Nahasapeemapetilon.
  \item Enthüllung: Marge sagt, Barts Geburt habe 43 Stunden gedauert.
  \item Nelson schreibt seiner Mutter in einer Notiz, dass sein Vater angerufen habe und sie die Kaution zahlen solle. In der Episode \glqq Der Feind in meinem Bett\grqq\ (siehe \ref{FABF19}) wird Nelsons Vater Abwesenheit damit begründet, dass er eine Jahrmarktattraktion war.
  \item Die bolivianische Baumeidechse hat angeblich folgende Vogelarten ausgerottet:
\begin{itemize}
	\item Der Dodo ist tatsächlich ausgestorben, daran ist aber der Mensch Schuld.
	\item Der Kuckuck ist nicht ausgestorben, legt aber seine Eier in fremde Nester, wie es die Baumeidechse angeblich tut.
	\item Die N\mbox{$\bar{e}$}n\mbox{$\bar{e}$} (Hawaiigans) gibt es noch, war aber in den 1950ern durch eingeführte Raubtiere fast ausgerottet.
\end{itemize}
  \item Letzter Auftritt von Phil Hartman, der Stimme u.\,a. von Lionel Hutz und Troy McClure. 
\end{itemize}
}

	
\subsection{Das unheimliche Mord-Transplantat}
\begin{itemize}
	\item \textbf{Hell Toupée}\\ Homer lässt sich die Haarpracht des exekutierten Mörders Snake transplantieren und wird dadurch selbst zum Mörder von Apu und Moe.
  \item \textbf{The Terror of Tiny Toon}\\ Bart und Lisa werden in den Fernseher gesaugt und spielen Katz' und Maus mit Itchy und Scratchy.
  \item \textbf{Starship Poopers}\\ Maggie ist in Wirklichkeit ein Alien-Baby, denn Maggies wahrer Vater ist Kang, was den Simpsons einen Auftritt in der Jerry-Springer-Show einbringt.
\end{itemize}

\notiz{
\begin{itemize}
  \item Die Wurfeier von Bart sind beschriftet mit \glqq Lisa\grqq , \glqq Skinner\grqq , \glqq Flan\-ders\grqq , \glqq Dad\grqq , \glqq Dad\grqq\ und \glqq Dad\grqq.
  \item Auf Homers Werkzeugkasten (engl. toolbox) steht \glqq Homer's Tulebox\grqq .
  \item Poochie\index{Poochie}, wieder mit der Stimme von Homer, hat einen kleinen Auftritt (siehe \ref{4F12}).
\end{itemize}
}

	
\subsection{Homer ist ein toller Hippie}\label{AABF02}
Als Homer Nachforschungen bezüglich seines zweiten Vornamens, Jay, anstellt, findet er heraus, dass seine Mutter wollte, dass er ein Hippie wird. Prompt beschließt Homer, es seiner Mutter gleichzutun. Gesagt, getan. Mit Seth\index{Seth} und Munchie\index{Munchie}, zwei ehemaligen Freunden seiner Mutter, stellt er sich gegen die Polizei. Doch die Aktion von Hippie-Homer hat einen eher fragwürdigen Erfolg.

\notiz{
\begin{itemize}
  \item Fernsehzitat: Nachdem sie den speziellen \glqq Garden Blast\index{Garden Blast}\grqq -Saft getrunken haben, sitzen Grampa und Jasper da und kichern im Original wie Beavis und Butthead.
  \item Filmplakat-Zitat: Die erste Einstellung von Grampa und Homers Mutter in Woodstock, in der das Paar gemeinsam im Vordergrund steht, ist eine Referenz an das Plakat des legendären Rockfestival 1969 in Woodstock.
  \item Als Smithers Mr. Burns vorschlägt, Essen vom Chinesen zu bestellen, meint Mr. Burns, das wolle er nicht, da ihm die Chinesen zu unterentwickelt seien.
  \item Am Ende von Burns Rekrutierungsfilm steht \glqq An Alan Smithee Film\grqq\ (Ein Film von Alan Smithee). Alan Smithee\index{Smithee!Alan} ist ein falscher Name, der oft für Filme verwendet wird, von denen sich die tatsächlichen Regisseure distanzieren.
  \item Homer sieht sich eine alte Bob-Hope-Sendung an, die sich auf Dean Rusk bezieht, den Staatssekretär der Kennedy- und Johnson-Regierung. Er spielte eine große Rolle in der Gestaltung der Vietnam-Politik der USA und wurde deshalb als Hippie bezeichnet.
  \item In dieser Episode sind folgende Songs zu hören: \glqq Incense and Peppermints\grqq\ von Strawberry Alarm Clock, \glqq Uptown Girl\grqq\ von Billy Joel, die Titelmusik des Musicals \glqq Hair\grqq , die Jimi Hendrix Version von \glqq The Star-Spangled Banner\grqq , \glqq Time of the Seasons\grqq\ von den Zombies und \glqq White Rabbit\grqq\ von Jefferson Airplane. Die Abspannmusik wurde von Yo La Tengo eingespielt.
  \item Homer wird als Kleinkind in Woodstock gezeigt. Eigentlich war er damals bereits 14 Jahre alt.
\end{itemize}
}

	
\subsection{Die große Betrügerin}\label{AABF03}
Lisa Simpson hat sich hinreißen lassen und bei einer Schularbeit geschummelt und das mit großem Erfolg. Nach der anfänglichen Euphorie bekommt das Mädchen dann doch ein schlechtes Gewissen. Als sie schließlich die Wahrheit gestehen will, wird sie ausgerechnet von Rektor Skinner daran gehindert: Es ist nämlich so, dass die Schule dank Lisas guter Note mehr Fördergelder bekommen hat. Dass der Schuldirektor fürs Schummeln ist, muss die kleine Lisa erst einmal verdauen.

\notiz{
\begin{itemize}
  \item Videospiel-Zitat: Dash Dingo\index{Dash Dingo} ist eine ironische Anspielung auf die beliebte Videospiel-Serie \glqq Crash Bandicoot\grqq\ von Sony.
  \item In einer Szene am Anfang läuft Wendell\index{Wendell} mit braunen Haaren über den Flur.
  \item Gil\index{Gil} verkauft der Schule einen Coleco-Computer. Coleco\index{Coleco} war in den frühen Achtzigern ein Hersteller von Videospielen.
  \item Die Schule erhält eine Förderung in Höhe von 250.000 Dollar.
  \item Oberschulrat Chalmers behauptet, das Gebäude der Grundschule befand sich im US-Bun\-des\-staat Missouri, doch aufgrund ihres schlechten Zustandes wurde sie Stein-für-Stein nach Springfield verlegt.
\end{itemize}
}

\subsection{Grandpas Nieren explodieren}\label{AABF04}
Homer macht mit seiner Familie einen Ausflug. Weil Grandpa während des Ausflugs nicht auf die Toilette gehen darf, platzen ihm die Nieren. Es gibt nur eine Möglichkeit: Eine Niere von Homer kann sein Leben noch retten. Doch dieser hat Angst vor der Operation und läuft davon. Schließlich hat Homer einen Unfall und er muss ins Krankenhaus eingeliefert werden. Dort wird er nicht nur wieder zusammengeflickt, es wird ihm auch eine Niere entfernt und seinem Vater transplantiert.

\notiz{
\begin{itemize}
  \item Homers liebste Gorilla-Filme: \glqq Gorillas im Anmarsch\grqq , \glqq Die Gorilla Insel 4\grqq\ und \glqq Die Affen drehen durch\grqq .
  \item In \glqq Bloodbath Gulch\index{Bloodbath Gulch}\grqq\ besuchen die Simpsons den \glqq Ye Olde Animatronic Saloon\grqq\ (Der gute alte animatronische Saloon). Dort jagt ein animatronischer Cowboy ein Saloongirl rund um einen Balkon -- eine Anspielung auf eine mechanische Attraktion in Disneyland: \glqq Pirates of the Caribbean\grqq .
  \item Über die Lautsprecheranlage des Krankenhauses hört man die Ansage \glqq Dr. Spreizfuß bitte in die Pediatrie\grqq . Im Original wird hier \glqq Doc Martens\grqq\ aufgerufen -- eine populäre englische Schuhmarke. Später hört man eine Durchsage für \glqq Dr. Bombay\grqq , eine Figur aus der 60er-Jahre-Sitcom \glqq Bewitched\grqq\ (Verliebt in eine Hexe).
  \item Das Schiff der Verdammten heißt \glqq Honeybunch\index{Honeybunch}\grqq .
	\item In der Episode \glqq Die Erbschaft\grqq\ (siehe \ref{7F17}) gibt Abe allerdings an, dass er nur noch eine funktionierende Niere habe.
\end{itemize}
}

	
\subsection{Der unerschrockene Leibwächter}\label{AABF05}
Homer ist zu Bürgermeister Quimbys neuem Leibwächter ernannt worden. Als solcher genießt Homer einige angenehme Vorteile: Er wird hofiert und auch bestochen. Als er entdeckt, dass der Gangster Fat Tony den Kindern in der Schule Rattenmilch verkauft, kommt es zum Krieg zwischen Fat Tony und Bürgermeister Quimby. Als der Gangster während einer The\-a\-ter\-auf\-führ\-ung ei\-nen Mord\-an\-schlag auf Quimby verübt, kann Homer die Attacke abwehren -- allerdings nur mit Mark Hamills\index{Hamill!Mark} Hilfe.

\notiz{
\begin{itemize}
  \item Auf der Convention trägt Uter\index{Uter} ein \glqq Futurama\index{Futurama}\grqq -Shirt. Unter den Wesen, die Autogramme geben, sind Gort aus \glqq The Day The Earth Stood Stil\grqq\ (Der Tag, an dem die Erde still stand), Doctor Who aus der gleichnamigen BBC-Fernsehserie und Godzilla.
  \item Für die Convention verkleidet sich Mrs. Krabappel als Barbarella. Rektor Skinner, Benjamin, Doug und Gary verkleiden sich als Mr. Spock. Weitere Kostüme auf der Con sind Chewbacca, Xena, der Terminator, ein Borg, der Unsichtbare und Lieutenant Commander Geordi LaForge.
  \item Im Hintergrund der Convention ist ein Stand für \glqq Rosswell, Little Green Man\grqq\ zu sehen.
  \item Der Comicbuchverkäufer gibt an, 45 Jahre alt zu sein, noch Jungfrau zu sein und noch bei seinen Eltern zu wohnen.
  \item Die Ausbildung zum Bodyguard absolvieren neben Homer noch Kirk van Houten, Gil Gunderson, Ruth Powers und Jeremy Peterson. 
\end{itemize}
}
	
\subsection{Wir fahr'n nach\dots Vegas}\label{AABF06}
Homer Simpsons etwas wunderlicher Nachbar Ned Flanders hat sein verkrampftes und spießiges Leben endgültig satt, er beschließt, eine grundsätzliche Än\-der\-ung vor\-zunehmen: Ned Flanders will von Homer in die lustige Vergnügungswelt Amerikas eingeführt werden. Kurzentschlossen fährt Homer mit seinem Nachbarn nach Las Vegas, wo die beiden in trunkenem Zustand sofort zwei Kellnerinnen heiraten. Nur mit großer Mühe gelingt es ihnen, aus der heiklen Situation wieder heil herauszukommen.

\notiz{
\begin{itemize}
  \item Fernsehzitat: Das surreale Duo im Cabrio, das aus Las Vegas zu\-rück\-kommt, erinnert an die Illustration von Hunter S. Thompson und seinen Anwalt aus \glqq Fear and Loathing in Las Vegas\grqq\ (Angst und Schrecken in Las Vegas).
  \item Unter den Gästen im Casino sind Gil\index{Gil}, Ernst\index{Ernst}, Gunter\index{Gunter} und Drederick Tatum\index{Tatum!Drederick}.
  \item In dieser Folge wird behauptet, dass Ned Flanders 60 Jahre alt ist. In der Episode \glqq Der total verrückte Ned\grqq\ (siehe \ref{4F07}) hat es allerdings den Anschein, dass Ned Flanders höchstens um die 40 Jahre alt sein kann.
  \item Der Name des Stuntman ist Lance Murdock\index{Murdock!Lance}.
  \item Barney hat am 8. Januar Geburtstag.
  \item Mr. Burns betreibt auch eine Pflegeheimkette.
  \item Ned Flanders behauptet, laut Deuteronomium\index{Deuteronomium} Kapitel 7 sei Glücksspiel eine Sünde; es lassen sich allerdings hier keine Verweise auf Glücksspiele finden.
\end{itemize}
}


	
\subsection{Allgemeine Ausgangssperre}\label{AABF07}
Obwohl Homer und seine Freunde im Rausch die Schule demoliert haben, werden die Kinder dafür verantwortlich gemacht. Als die Kinder mit einer Ausgangssperre bestraft werden, beschließen sie, sich zu wehren. Über einen selbst gebastelten Radiosender plaudern sie Geheimnisse ihrer erwachsenen Mitbürger aus. Als sie schließlich gefasst werden, schalten sich die Senioren ein und verhängen eine allgemeine Ausgangssperre für alle unter siebzig Jahre alte Mitbürger.

\notiz{
\begin{itemize}
  \item Broadway-Zitat: Die Musiknummer im Autokino ist eine Umarbeitung des Songs \glqq Kids\grqq\ aus \glqq Bye Bye Birdie\grqq .
  \item Filmzitat: Die Geschichte von \glqq The Bloodening\grqq\ (Das Blutgemetzel) erinnert an den Film \glqq Village of the Damned\grqq\ (Das Dorf der Verdammten).
  \item Das Lied \glqq Let's All Go to the Lobby\grqq\ (Lasst uns alle ins Foyer gehen), das im Springfielder Autokino zu hören ist, sang Mr. Burns in \glqq Burns Erbe\grqq\ (siehe \ref{1F16}). Das Lied wurde in amerikanischen Kinos früher zwischen Werbung und Hauptfilm gespielt, damit die Kinobesucher Eis und Süßigkeiten kaufen gehen.
  \item In dieser Folge sind Professor Frinks Augen deutlich zu erkennen. Normalerweise sind sie hinter seinen dicken Brillengläser nicht zu sehen.
  \item Die Rowdys, welche die Ausgangssperre verursachen, sind Homer, Barney, Carl und Lenny.
  \item Als Bart Milhouse anruft, schaut dieser im Fernsehen gerade die Teletubbies. Später sieht man Milhouse mit einer zerrissenen Short, unter welcher eine Teletubbies-Unterhose hervorscheint.
  \item Als Seymour mit Edna das Autokino besucht, sind auf dem Rücksitz Oberschulrat Chalmers und Agnes Skinner Arm in Arm zu sehen.
  \item Die Isotopen gewinnen die Baseball-Meisterschaft.
\end{itemize}
}

	
\subsection{Nur für Spieler und Prominente}\label{AABF08}
Als Homer Simpson den Reiseveranstalter Wally Kogen\index{Kogen!Wally} kennenlernt, handeln sie einen Deal aus: Wenn Homer seine Freunde zu einer Busreise zum Super Bowl nach Miami überredet, kann er umsonst mitfahren. Natürlich tut Homer sein bestes. Doch bei der Ankunft in Miami entdecken die Freunde, dass die Eintrittskarten zum Endspiel gefälscht sind. Doch dafür lernen sie den Medienmogul Rupert Murdoch\index{Murdoch!Rupert} kennen und landen schließlich in der Umkleidekabine der Siegermannschaft.

\notiz{
\begin{itemize}
  \item Filmzitat: Rudy\index{Rudy}, der kleinwüchsige junge Mann, den die Gruppe nicht in den Super-Bowl-Bus lässt, stammt aus dem Film \glqq Rudy\grqq\ von 1993.
  \item Ralphs Sammlung unzustellbarer Briefe, die den Drogenhund zum Bellen animieren, sind an Otto, den Busfahrer, adressiert.
  \item Zu der Gruppe, die an der Fahrt zum Super Bowl teilnehmen, gehören: Der Comicbuchverkäufer, der Hummelmann, Ned Flanders, Lenny, Carl, Apu, Barney, der quietschstimmige Teenager, Reverend Lovejoy, Moe, Krusty, Dr. Nick Riviera, Sideshow Mel, Jasper, Kirk Van Houten, Dr. Hibbert, Captain McCallister, Mr. Burns Anwalt, Charlie und Chief Wiggum.
  \item Wally Kogens Name erinnert sehr an das Simpsons-Autorenteam Wallace Wolodarsky und Jay Kogen.
  \item Auf die Frage von Lenny, was Carl mit den Super-Bowl-Ring mache, antwortet dieser, den schenke er seiner Frau, da sie heute Hochzeitstag hätten.
  \item Moes Lieblingsmannschaft sind die Atlanta Falcons.
  \item Obwohl Homer im Mai Geburtstag hat (siehe Folge \glqq Karriere mit Köpfchen\grqq, \ref{7F02}), schenkt Bart ihm vor dem Super Bowl das Gutscheinheft zum Geburtstag. Der Super Bowl findet üblicherweise Anfang Februar statt.
\end{itemize}
}

	
\subsection{Namen machen Leute}\label{AABF09}
Als in einer neuen Fernsehserie ein Kriminalkommissar namens Homer Simpson auftaucht, fühlt sich Homer zunächst sehr geehrt. Es gefällt ihm, auf diese Weise \glqq irgendwie prominent\grqq\ zu sein. Doch zu Homers Leidwesen entwickelt sich die Figur mehr und mehr zu einem Trottel. Homer will dagegen etwas unternehmen -- vergeblich. So beschließt er, einen neuen Namen anzunehmen. Von nun an wird er zu den Partys der Honoratioren eingeladen, die mit etwas wunderlichen Aktionen enden.

\notiz{
\begin{itemize}
  \item Homers Liste potentieller neuer Namen: Hercules Rockefeller\index{Rockefeller!Hercules}, Rembrandt Q. Einstein\index{Einstein!Rembrandt Q.}, Schönling B. Wunderbar und Max Power\index{Power!Max}.
  \item An Moes Bar hängt ein Transparent mit der Aufschrift \glqq TV Sensation Homer Simpson Drinks Here!\grqq\ (Hier trinkt TV-Sensation Homer Simpson!).
  \item Als Homer zum Gericht geht, um seinen Namen zu ändern, sieht man Agnes, Snake, Herman, Otto und Hans Maulwurf auf der Galerie sitzen.
  \item Auf Trent Steels\index{Steel!Trent} Party sieht man Woody Harrelson in Hosen aus Hanf.
  \item Als Homer Max Power heißt, kennt Mr. Burns seinen neuen Namen auf Anhieb.
  \item Marge gibt an, eine Tätowierung zu haben.
\end{itemize}
}


	
\subsection{Marge Simpson im Anmarsch}\label{AABF10}
Homer hat sich einen neuen Wagen angeschafft: Das neue Auto ist ein Super-Van, der Canyonero\index{Canyonero}, in der Frauenausstattung. Als Homer erfährt, dass der Wagen die Frauenvariante ist, weigert sich Homer, mit dem Wagen zu fahren. Er befürchtet, immer verhöhnt zu werden, wenn er mit der Familienkutsche unterwegs ist. Als bei einem Zoobesuch die Rhinozerosse ausbrechen und Homer verfolgen, ändert er seine Meinung: Denn Marge gelingt es, mit dem Wagen die riesigen Tiere abzulenken und Homer zu retten.

\notiz{
\begin{itemize}
  \item Gil\index{Gil} ist wieder als Autoverkäufer zu sehen.
  \item Gil ist verheiratet.
  \item Evergreen Terrace hat eine eigene Ausfahrt vom Freeway.
  \item Außer Marge nehmen auch Oberschulrat Chalmers, Agnes Skinner, Moe, Kearney, Apu und Krusty am Verkehrsunterricht teil.
  \item Nach dem Unterricht fährt Kearney in einem roten VW-Käfer davon.
  \item Filmzitat: Die Szene, in welcher die Nashörner aus dem Zoo laufen und Homer \glqq Jumanji\grqq\ ruft, ist eine Anspielung auf den gleichnamigen Film.
  \item Fehler: In der Auffahrt der Simpsons sind keine Autos zu sehen. In der nächsten Szene sind beim Blick aus dem Wohnzimmer der Simpsons ihre beiden Autos in der Auffahrt zu erkennen.
\end{itemize}
}
	
\subsection{Apu und Amor}\label{AABF11}
Ausgerechnet vor dem Valentinstag bekommt Apu Streit mit seiner Frau Manjula, die sich vernachlässigt fühlt, da er nur arbeitet. Um ihre Liebe wiederzugewinnen, beginnt Apu, sie täglich mit Überraschungen zu verwöhnen -- sehr zum Missfallen der anderen Männer, die es ihm auf keinen Fall gleichtun wollen. Als Apu eine Liebeserklärung an den Himmel schreiben lässt, verhindert Homer die Ausschreibung des Namens Manjula, sodass jede Frau glaubt, es könnte sie gemeint sein.

\notiz{
\begin{itemize}
  \item Bei seiner Dinner-Party legt Apu ein Album mit dem Titel \glqq Concert Against Bangladesh\grqq\ (Konzert gegen Bangladesch) auf, eine Parodie auf das Konzert für Bangladesch, das 1971 von George Harrison zugunsten der Kriegsopfer veranstaltet wurde.
  \item Die Folge endet mit einer herzförmigen Iris-Blende nach Art der US-Fernsehserie \glqq Love, American Style\grqq\ (1969 bis 1974).
  \item Als Apu Captain McCallister die Pornohefte auf das Boot bringt, meint dieser, die Hefte halten seine Männer zehn Minuten von der Homosexualität ab. Daraufhin sagt einer der Seeleute, das sage genau der Richtige.
  \item Elton John spielt ebenfalls in dem Garten, der sich auf dem Kwik-E-Mart befindet, wie in der Folge \glqq Lisa als Vegetarierin\grqq\ (siehe \ref{3F03}) Paul McCartney.
  \item Sideshow Mels Freundin bzw. Frau heißt offenbar Barbara.
  \item Der Kwik-E-Mart hat die Hausnummer 8306.
\end{itemize}
}

	
\subsection{Es tut uns leid, Lisa}
Weil Homer die \glqq Bill of Rights\grqq\ beschädigt hat und den Schaden nicht bezahlen kann, hat Homer eine Handy-Antenne der Firma Omnitouch\index{Omnitouch} auf sein Hausdach setzen lassen. Dass die technische Ausrüstung dazu in Lisas Schlafzimmer installiert wird, missfällt dieser sehr. Um sie zu trösten, geht Homer mit seiner Tochter in einen sogenannten New-Age-Laden. Dort lassen sie sich zur Bewusstseinssteigerung in Behälter einschließen. Lisa hat tatsächlich Halluzinationen, während Homer, ohne es zu merken, eine außergewöhnliche Reise macht.

\notiz{
\begin{itemize}
  \item Maggie spielt in ihrem Kinderbett mit einer Bongo-Puppe, dem einohrigen Hasen aus Matt Groenings \glqq Life in Hell\grqq .
  \item Marges Notizen über die Handy-Anrufe, die sie mithört, lauten: \glqq Otto -- drugs?\grqq\ (Drogen?), \glqq Mayor Quimby -- Interns?\grqq\ (Praktikanten?), \glqq Burns -- Greedy?\grqq\ (habgierig?), \glqq Krusty -- Gay?\grqq\ (schwul?).
  \item Der New-Age-Laden, in den Homer mit Lisa geht, heißt \glqq Karma Ceu\-ti\-cals\index{Karma Ceuticals}\grqq , ein Wortspiel mit \glqq Pharmaceuticals\grqq\ (Medikamente).
  \item Unter den Artikeln im \glqq Karma Ceuticals\grqq\ befindet sich ein Symbol der Steinmetze. In \glqq Homer der Auserwählte\grqq\ (siehe \ref{2F09}) schloss sich Homer der als Steinmetze bekannten geheimen Bruderschaft an. Außerdem ist in dem Laden eine kleine Skulptur des Hindu-Gottes Ganesh zu sehen.
\end{itemize}
}

	
\subsection{Das Geheimnis der Lastwagenfahrer}\label{AABF13}
Zwischen Homer und dem Lastwagenfahrer Red Barclay\index{Barclay!Red} kommt es zu einem Wettessen und dabei fällt Red plötzlich tot um. Daraufhin übernehmen Homer und Bart Reds letzte Lieferung nach Atlanta. Immer wenn Homer unterwegs am Steuer einschläft, übernimmt der Autopilot. Zwar soll dies ein Geheimnis bleiben, doch Homer plaudert es aus. Die anderen Trucker wollen sich nun an Homer rächen, aber der meistert alle kritischen Situationen und bringt die Lieferung pünktlich ans Ziel.

\notiz{
\begin{itemize}
  \item Homers Truck-Musik: \glqq Wannabe\grqq\ von den Spice Girls.
  \item Die Fahrerkabine von Reds Truck, dem Red Rascal\index{Red Rascal} (Der rote Strolch), zieren aufgemalte Cartoon-Figuren, die eine Reminiszenz an Tex Averys \glqq Wolf \& Red\grqq\ darstellen.
  \item Red Barclay und Tony Randall\index{Randall!Tony} waren die einzigen Personen, die das 16-pfündige Steak essen konnten.
  \item Gil ist Verkäufer bei Se$\tilde{\mbox{n}}$or Ding-Dong.
  \item Dr. Hibbert ist an dem Restaurant (Das Schlachthaus) mit 12 Prozent beteiligt, in dem Red Barclay stirbt.
  \item Fehler: Der Preis an Homers Mütze ändert sich von 8,99 Dollar zu 9,99 Dollar.
\end{itemize}
}
	
\subsection{Bibelstunde, einmal anders}\label{AABF14}
In der Kirche liest Reverend Lovejoy derart spannend aus der Bibel, dass die ganze Familie Simpson in den Bänken einschläft. Sie schlafen tief und jede Generation hat ihren ganz speziellen \glqq biblischen\grqq\ Traum: Während Homer und Marge als Adam und Eva aus dem Paradies verjagt werden, schreiten Lisa und Milhouse wie Moses durch das Rote Meer. Homer schlichtet als König Salomon den Streit um einen Kuchen und Bart ist David, der den Riesen Goliath besiegt. Nicht ganz bibelgerecht enden die Träume schließlich in einem Weltuntergang.

\notiz{
\begin{itemize}
  \item Fernsehzitat: Als Knecht Ruprecht mit Bart (König David) spricht, klingt er wie der Hund Goliath aus der Fernsehsendung \glqq Davey and Goliath\grqq .
  \item Die Kugel der Isis, zuletzt zu sehen in \glqq Die Kugel der Isis\grqq\ (siehe \ref{5F17}), steht auf einem Podest in der Pyramide, in die Milhouse (Moses) und Lisa geworfen werden.
  \item Während Milhouse (Moses) das Schofar (Widderhorn) bläst, meißelt Bart einen Spruch in die Tafel: Ein Auge, einen Brunnen, einen Knoten, \glqq D+\grqq\ und einen Pharaonenkopf. Übersetzt bedeutet das: \glqq I will not deface\grqq\ (Ich werde nichts verunstalten).
  \item Die Folge endet mit \glqq Highway to Hell\grqq\ von AC/DC.
\end{itemize}
}
	
\subsection{Überraschung für Springfield}
Als Homer einen Gartengrill aufzubauen versucht, kann er das Gerät anschließend nur noch auf die Müllkippe bringen. Er zieht den Grill hinter seinem Auto her. Als der Strick reißt, stößt der Grill dem Wagen der Galeriebesitzerin Astrid Weller\index{Weller!Astrid} zusammen. Als die Frau mit Kunstverstand das sieht, was einmal ein Grill war, ist sie von dem \glqq Kunstwerk\grqq\ restlos begeistert. Sie fordert Homer auf, weitere \glqq Kunstwerke\grqq\ herzustellen. Schließlich kommt der \glqq Künstler\grqq\ auf die Idee, ein neues Venedig zu schaffen. Dazu muss er nur Springfield unter Wasser setzen.

\notiz{
\begin{itemize}
  \item Die Notiz, die Homer für sich selbst an die Garage malt, lautet: \glqq Start Here Tomorrow 7/17/95\grqq\ (Morgen fange ich an, 17.07.1995).
  \item Homer hat einen Autoaufkleber, auf dem \glqq Single 'N' Sassy\grqq\ (Alleinstehend und kess) steht. Sein Autokennzeichen lautet \glqq 3FJP24\grqq .
  \item In der Kunstgalerie ist Barneys Freundin zu sehen, die Yoko Ono nachempfunden ist.
  \item Barneys Serviettengemälde ist eine Reproduktion von \glqq Ein Sonntag auf der Grande Jatte\grqq\ von Georges Seurats.
  \item In dem Museum, in das Marge und Homer gehen, hängt ein Bild von Matt Groening\index{Groening!Matt}.
\end{itemize}
}

	
\subsection{Seid nett zu alten Leuten!}\label{AABF16}
Nicht Springfield, sondern Shelbyville wird Olympia-Stadt und daran ist nicht zuletzt Bart schuld: Sein Benehmen hat dem Olympischen Komitee nicht gefallen. Nun muss er zur Strafe gemeinnützige Arbeit im Altersheim leisten. Dort überredet Bart die Senioren zu einer Schifffahrt. Der Ausflug verläuft nicht ganz problemfrei, denn der Dampfer wird gerammt und geht beinahe unter. Dass das Schiff samt den Leuten wieder hochgespült wird, ist auch Homer zu verdanken, da er die von ihm entworfenen Olympiamaskottchen die Toilette hinuntergespült hat.

\notiz{
\begin{itemize}
  \item Filmzitat I: Die Szene, in der die Senioren mit Bart fliehen und in Springfield herumrennen, spiegelt die \glqq Can't Buy Me Love\grqq -Sequenz aus \glqq A Hard Day's Night\grqq\ wieder. Das Lied wird hier von der amerikanischen R'n'B-Band NRBQ interpretiert.
  \item Filmzitat II: Auf Mr. Burns Jacht fertigt Smithers eine Zeichnung von Mr. Burns an, eine Parodie einer Szene aus \glqq Titanic\grqq .
  \item Filmzitat III: \glqq Einer flog übers Kuckucksnest\grqq\ wurde auch in den Folgen \glqq Die Geburtstagsüberraschung\grqq\ (siehe \ref{7F24}) und \glqq Nur ein Aprilscherz\grqq\ (siehe \ref{9F17}) parodiert.
  \item Smithers war bei der Marine und wurde unehrenhaft entlassen.
  \item Auf dem Schild des Internationalen Olympischen Komitees steht \glqq Now with Myanmar!\grqq\ (Jetzt mit Myanmar!). Außerdem steht das olympische Symbol auf dem Kopf und die Ringe sind nicht miteinander verbunden.
  \item Auf dem Schild der Springfielder Reifenhalde steht \glqq Est. 1989\grqq\ (Gegründet 1989). 1989 war das Jahr, in dem die Simpsons im US-Fernsehen Premiere als eigene Serie feierten.
  \item Das Lied \glqq The Children Are Our Future\grqq\ (Die Kinder sind unsere Zukunft) benutzt die bekannte \glqq rolling down the river\grqq -Tonfolge aus dem Song \glqq Proud Mary\grqq\ von Creedance Clearwater Revival. Die Sequenz wurde von George Meyer choreographiert.
    \item Mr. Burns Boot heißt \glqq Gone Fission II\grqq\index{Gone Fission II}. Auf den Segeln ist das Symbol für Atomenergie zu sehen.
\end{itemize}
}

	
\subsection{Burns möchte geliebt werden}\label{AABF17}
Kaufhauskönig Arthur Fortune\index{Fortune!Arthur} wird von den Einwohnern Springfields enthusiastisch gefeiert. Mr. Burns beschließt, alles zu unternehmen, damit er von den Menschen auch so geliebt wird. Burns Plan: Mit der Hilfe von Homer, Prof. Frink und Willie möchte er das Ungeheuer von Loch Ness aufspüren und nach Springfield bringen. Als der Plan gelingt, jubeln die Leute ihm begeistert zu. Doch als dann das Ungeheuer präsentiert werden soll, kommt es zu einem folgenschweren Chaos, das Burns seine Beliebtheit wieder kostet.

\notiz{
\begin{itemize}
  \item Barts Tafelspruch\footnote{I have neither been there nor done that} am Anfang der Episode ist eine Anspielung auf die amerikanische Redewendung \glqq Been there, done that\grqq .
  \item Filmzitat: Das Ende von Mr. Burns Pressekonferenz, als alle Lichter ver\-lö\-schen, ist eine Reminiszenz an \glqq King Kong\grqq .
  \item Arthur Fortune und sein Megastore sind eine Anspielung auf den Milliardär Richard Branson, den Gründer der Virgin-Kette.
  \item Mr. Burns ausgestopftes Ungeheuer von Lech Ness trägt ein Hemd mit dem Aufdruck \glqq Macarena Monster\grqq .
  \item Als Homer sich in seinem Kilt dreht, sieht man, dass er keine Unterwäsche trägt.
  \item In der Folge \glqq Ralph liebt Lisa\grqq\ (siehe \ref{9F13}) sagte Willie, sein Vater wurde wegen Schweinediebstahls gehängt, trotzdem waren sein Vater und seine Mutter in der Episode zu sehen \cite{snpp}. Willies Eltern betreiben eine Taverne.
  \item Mr. Smithers sagt zu Mr. Burns: \glqq Ich liebe Sie, Sir!\grqq
  \item In dieser Episode ist Mr. Burns Milliardär.
  \item Mr. Burns spendet dem Krankenhaus in Springfield 200.000 \$.
\end{itemize}
}

	
\subsection{Die Stadt der primitiven Langweiler}\label{AABF18}
Als Lisa in den Club der Intellektuellen aufgenommen wird, regt sie ein neues Projekt an: Zusammen mit Dr. Hibbert, Professor Frink, Rektor Skinner, Lindsey Naegle und dem Comicbuchverkäufer will sie der Bevölkerung Bildung und Kultur beibringen. Die Bevölkerung ist davon jedoch weniger begeistert. Doch das ändert sich, als der hoch intellektuelle Wissenschaftler Stephen Hawking\index{Hawking!Stephan} auftaucht und deutlich macht, dass es sehr viele verschiedene Vorstellungen von Perfektion gibt. Unterdessen lässt Homer erotische Fotos von sich machen, nachdem der beim \glqq Wie tief kannst Du sinken?\grqq -Wettbewerb den zweiten Preis hat mitgehen lassen.

\notiz{
\begin{itemize}
  \item Mitglieder der Springfielder Mensa-Gruppe: Professor Frink, Dr. Hibbert, Rektor Skinner, Lindsey Naegle und der Comicbuchverkäufer.
  \item Der Aufdruck auf dem T-Shirt des Comicbuchverkäufers lautet \glqq C:/DOS, C:/DOS/RUN, RUN/DOS/RUN\grqq .
  \item Jimmy, der Schleimbeutel, tritt in dem Werbespot des Wettbewerbs auf.
  \item Die amerikanische Außenministerin unter Präsident Clinton, Madelein Albright, sitzt beim \glqq Wie tief kannst du sinken?\grqq -Wettbewerb in der Jury.
  \item Obwohl er in einer vorherigen Szene die Stadt verlassen hat, ist auf der Hunderennbahn ein Mann in Begleitung von zwei Frauen zu sehen, der auffallend an Bürgermeister Quimby erinnert.
  \item Barney erreicht bei dem Wettbewerb 22 Punkte, Homer 12 Punkte und Moe 0 Punkte.
  \item Andere Städte auf der Liste von Amerikas lebenswertesten Städten: \glqq Flint, Michigan (\#296)\grqq , \glqq Ebola, Rhode Island (\#297)\grqq\ und \glqq Dawson's Creek, North Carolina (\#298)\grqq .
  \item Dr. Hibbert wohnte einmal in Alabama.
  \item Prof. Frink hat einen IQ von 199, Dr. Hibbert von 155, der Comicbuchverkäufer von 170 und Prof. Hawking von 280.
  \item Obwohl Martin Prince einen IQ von 216 hat, ist er nicht Mitglied im Mensa Club.
  \item Carl hat ab dieser Episode offenbar Diabetes.
\end{itemize}
}

	
\subsection{Die japanische Horror-Spiel-Show}\label{AABF20}
Als Homer bei einem Raubüberfall sein gesamtes Vermögen verliert, muss die Familie ihre Urlaubspläne ändern: Es bleibt ihnen nichts anderes mehr übrig, als eine Billigreise nach Japan zu buchen. Allerdings haben sie dabei nicht bedacht, dass das Leben in Japan sehr teuer ist. Im Nu ist das Urlaubsgeld weg und so versuchen die Simpsons schließlich, das Geld für den Rückflug in einer Fernseh-Spielshow zu gewinnen. Dabei durchleben sie einen wahren Horrortrip.

\notiz{
\begin{itemize}
  \item Bart, Lisa und Homer besuchen das Geschäft \glqq Java Server\grqq , eine Parodie auf die \glqq Java Hut\grqq -Coffeeshop-Kette.
  \item Der ältere Herr mit Zylinder, der bei dem Seminar neben Mr. Burns sitzt, ist Mr. Pennybags, die Symbolfigur von Monopoly.
  \item Nachdem Bart, Lisa und Marge durch das Fernsehen einen epileptischen Anfall hatten, sieht man einen Ausschnitt des Meister-Glanz-Werbespots mit einer zweiköpfigen Kuh. Der Spot wurde schon in \glqq Marge als Seelsorgerin\grqq\ (siehe \ref{4F18}) gezeigt.
  \item Während Homer mit seiner Familie in Japan im Urlaub ist, trinkt Barney bei Moe als Homer verkleidet auf dessen Rechnung.
  \item Beim Online-Banking ist zu sehen, dass Marge einen zweiten Vornamen hat, der mit dem Buchstaben B beginnt.
\end{itemize}
}


\section{Staffel 11}
 	
\subsection{Duell bei Sonnenaufgang}\label{AABF19}
Nachdem Homer einen \glqq Zorro\grqq -Film gesehen hat, fordert er Snake\index{Snake} zum Duell heraus -- er möchte Marges Ehre verteidigen. Snake macht sich aus dem Staub, was Homer eine enorme Steigerung seines Selbstbewusstseins verschafft. So gestärkt, fordert Homer nun jeden zum Duell heraus, dem er begegnet. Niemand traut sich, gegen Homer anzutreten, bis sich eines Tages doch ein mutiger Gegner findet. Nun bekommt Homer Panik und flieht mit seiner Familie auf die abgelegene Farm, auf der aufgewachsen ist und schlägt sich als Landwirt durch.

\notiz{Homers neue Frucht heißt ToMacco\index{ToMacco}. Er verkauft sie zu je einem Dollar.}

	
\subsection{Homer als Restaurantkritiker}\label{AABF21}
Homer bekommt einen Job als Restaurantkritiker bei der Zeitung \glqq Springfield Shopper\grqq . Dazu braucht er natürlich Lisas Unterstützung, die ihm als \glqq Ghostwriter\grqq\ hilft. Homer ist voll des Lobes über alle Restaurants, was ihm Probleme mit seinem Verleger einbringt. Also schwenkt er um und lässt an den Gastronomen der Stadt kein gutes Haar mehr. Dies treibt die Restaurantbesitzer von Springfield dazu, ein letztes, ganz spezielles Menü für Homer Simpsons zu kreieren.

\notiz{
\begin{itemize}
	\item Die Inhaber des Planet Springfield sind Rainier Wolfcastle, Chuck Norris, Johnny Carsons dritte Frau und die russische Mafia.
	\item Die Zeitung \glqq Springfield Shopper\grqq\ wurde 1883 gegründet.
\end{itemize}
}

	
\subsection{Ist alles hin, nimm Focusin}
Als Bart in der Schule ausrastet, wird er unter Drogen gesetzt. Diese Drogen bewirken zum einen, dass er sich voll und ganz auf seine Schularbeiten konzentrieren kann, haben jedoch auch Nebenwirkungen. Bald stellt sich heraus, dass diese unerwünschten Wirkungen bei Bart ungewöhnlich stark zutage treten: Er entwickelt schier übersinnliche Fähigkeiten und schafft es, eine Verschwörung innerhalb der Major League Baseball aufzudecken.

\notiz{
\glqq Focusin\grqq\ \index{Focusin} ist eine Anspielung auf das Medikament \glqq Ritalin\grqq .
}
	
\subsection{Mit Mel Gibson in Hollywood}\label{AABF23}
Weil Homer den \glqq Elec-Taurus\index{Elec-Taurus}\grqq , ein Elektroauto getestet hat, bekommt der zwei Eintrittskarten für eine Testvorführung von Mel Gibsons Remake des Klassikers \glqq Mr. Smith Goes to Washington\grqq . Alle sind von dem Streifen begeistert -- außer Homer, dem die Action dabei viel zu kurz kommt. Mel Gibson sucht daraufhin den Rat von Homer und bittet ihn um Mithilfe bei der Überarbeitung des Films. Als die Studiobosse die neue Homer-Version des Streifens ablehnen, beginnt für Homer und Mel ein leidenschaftlicher Kampf für ihr Werk.

\notiz{
\begin{itemize}
  \item Springfielder bei der Probevorführung: Homer und Marge Simpson, Hans Maulwurf, Edna Krabappel, Agnes Skinner, Seymour Skinner, Jasper, Ned und Maude Flanders, Krus\-ty der Clown, Luann Van Houten, Apu, Manjula, Sideshow Mel, Miss Hoover, Prof. Frink, Dr. Nick Riviera, Chief Wiggum, Lou, Eddie, Mr. Burns, Mr. Smithers, Bürgermeister Quimby, Willie, Kirk Van Houten, Fat Tony, Reverend Lovejoy, Helen Lovejoy, Pédro, Lenny, Carl, Sanjay, Moe, Barney und Captain McCallister.
  \item Der Grund, warum John Travolta Mel Gibson nach Springfield geflogen hat: Mel hilft John am Wochenende beim Umzug.
  \item Filmzitat: Das Zeigen des nackten Hintern von Mel und Homer ist eine Anspielung auf \glqq Braveheart\grqq .
  \item Erstmals wurden die Simpsons-Episoden auch im deutschsprachigen Raum zur Haupt\-sen\-de\-zeit aus\-gestrahlt.
  \item Fehler I: In einer Szene hat Homer seine Hand auf Barts Schulter, dann nimmt er die Hand weg, aber die Finger sind immer noch auf seiner Schulter zu sehen.
  \item Fehler II: Im Kino ist ein Filmplakat eines McBain Films zu sehen. Auf dem Plakat steht allerdings \glqq McBane\index{McBane}\grqq .
\end{itemize}
}

	
\subsection{Ich weiß, was du getudel-tan hast}
\begin{itemize}
	\item \textbf{Ich weiß, was du getudel-tan hast}\\ Ned Flanders kehrt aus dem Grab zurück und sucht Rache, nachdem Marge ihn versehentlich überfahren hat und die Familie seine Leiche verschwinden ließ.
   \item \textbf{Verzweifelt auf der Suche nach Xena}\\ Der Comicbuchverkäufer hat die Xena Darstellerin Lucy Lawless entführt. Aber Bart und Lisa, die über radioaktive Superkräfte verfügen, eilen zu Hilfe. Gemeinsam können sie den Entführer erledigen.
   \item \textbf{Das Leben ist eine Rutschbahn und dann stirbt man}\\ Eine Jahr-2000-Kernschmelze ereignet sich, nachdem Homer vergessen hatte, die Uhr des Kraftwerks umzustellen.
\end{itemize}

\notiz{Filmzitat: Die ganze erste Geschichte parodiert den Film \glqq Ich weiß was du letzten Sommer getan hast\grqq }

	
\subsection{Die Kurzzeit-Berühmtheit}
Homer hat Riesenglück beim Bowling und wird so zu einer lokalen Berühmtheit. Da er jedoch keine neue Leistung nachlegen kann, verblasst sein Ruhm sehr schnell. Doch damit kann Homer gar nicht umgehen: Er verfällt in eine tiefe Depression und beschließt, sich vom Springfield State Building in den Tod zu stürzen. Zum Glück wird er aber gerade noch von Bungee-Jumper Otto gerettet. Nun sucht Homer nach einer wirklich sinnvollen Beschäftigung. Er will sich voll und ganz seinen Kindern widmen.

\notiz{
\begin{itemize}
  \item Homer verschläft 26 Stunden.
  \item Marge stickt ein Decken, auf dem steht \glqq Get well Lenny\grqq\ (Gute Besserung Lenny). Kurz darauf sieht man Marge noch ein weiteres Decken sticken, auf dem \glqq We love you, Lenny\grqq\ (Wir lieben dich, Lenny) steht.
  \item Unter der Zuschauermenge bei Homers Bowlingerfolg ist auch Lurleen Lumpkin.
  \item Homers Liste \glqq Before I Die I Want To\grqq\ (Bevor ich sterbe, möchte ich noch): \glqq End Crime \& Injustice\grqq\ (Kriminalität und Ungerechtigkeit beenden), \glqq Bowl a Perfect Game\grqq\ (ein perfektes Spiel machen) und \glqq See Stevie Nicks Naked\grqq\ (Stevie Nicks nackt sehen). Der dritte Eintrag hat bereits drei Hacken.
  \item Bart spielt auf einer Sony Playstation I.
\end{itemize}
}

	
\subsection{Schon mal an Kinder gedacht?}\label{BABF03}
Apu ist der Meinung, dass nun die Zeit reif ist für Kinder. Doch Manjula und er haben kein Glück, bis Homer ihnen einen besonderen Tipp gibt -- neun Monate später bringt Manjula Achtlinge zur Welt! Zuerst erhält die Familie viele Spenden, doch dann bekommt eine Frau aus Shelbyville Neunlinge -- die Spenden bei Apus Familie bleiben nun aus. Da macht der Besitzer des Zoos ein Angebot: Er will die große Familie unterstützen, dafür sollen jedoch die Babys im Zoo ausgestellt werden.

\notiz{
\begin{itemize}
  \item Firmen für die Apu und Manjula Werbung machen: \glqq Pepsi B -- For Export Only\grqq\ und \glqq Sony\grqq .
  \item Apus Wohnung hat die Türnummer 8.
  \item Gil arbeitet im Kwik-E-Mart.
  \item Das große Möbelhaus Sh{\o}p ist eine Anspielung auf IKEA.
  \item Homer ist mittlerweile unfruchtbar, da er im Atomkraftwerk arbeitet.
\end{itemize}
}

	
\subsection{Lisa und ihre Jungs}
Da Marge mit einer Beinverletzung im Krankenhaus liegt, muss Lisa den Haushalt schmeißen und erhält dabei von Homer und Bart aber keinerlei Unter\-stütz\-ung. Aus Rache beschmiert sie die Gesichter der beiden nachts mit Farbe und Haferflocken und redet ihnen am nächsten Morgen erfolgreich ein, sie hätten Lepra. Flanders schickt Homer und Bart in ein Krankenhaus auf Hawaii. Obwohl die beiden inzwischen gemerkt haben, dass sie reingelegt wurden, unterziehen sie sich trotzdem einer schmerzhaften Hautbehandlung, genießen aber andererseits auch einen herrlichen Urlaub am Strand von Hawaii.

\notiz{
\begin{itemize}
  \item In der Itchy \& Scratchy Show kommt auch Poochie\index{Poochie} wieder vor.
  \item Als Lisas Plan funktioniert und Homer und Bart glauben, dass sie Lepra haben, sagt Lisa \glqq Ausgezeichnet\grqq\ und reibt sich die Hände wie Mr. Burns.
  \item Fehler: Obwohl Marge die Uhr auf das linke Bein fällt, ist das rechte Bein gebrochen.
\end{itemize}
}


	
\subsection{Der Kampf um Marge}\label{BABF05}
Homer und Marge gewinnen bei einem Tanzwettbewerb eine Harley Davidson. Grund genug für Homer, eine eigene Motorradgang (Hell's Satans\index{Hell's Satans}) zu gründen, der er aber ausgerechnet den Namen einer bereits bestehenden, berüchtigten Gang gibt. Deren Anführer taucht kurz darauf bei den Simpsons auf und ent\-führt Marge, um Homer für den Diebstahl des Clubnamens zu bestrafen.

\notiz{
\begin{itemize}
  \item Der Tafelgag\footnote{Ich kann keine Toten sehen.} ist eine Anspielung auf den Film \glqq The 6th Sense\grqq .
  \item Mitglieder von Homers Gang: Lenny, Homer, Moe, Carl und Ned.
  \item Hauptquartier von Homers Gang: Neds Hobbyraum im Keller.
  \item Die anderen Höllen-Teufel kommen aus Bakersfield.
  \item Parzellen-Nummer, auf der die Biker übernachten: Sektion K, Parzelle 2/17.
\end{itemize}
}

	
\subsection{Bart hat die Kraft}\label{BABF06}
Als Bart einem Wanderprediger zuhört, der ihm einredet, dass er über besondere Kräfte verfüge, zumal er Homer einen auf den Kopf gestülpten Eimer auf wundersame Weise entfernt, beschließt er selbst, eine Sekte zu gründen. Er wird von allen nur noch als der Wunderheiler angesehen und all seine Versuche, die Leute davon abzubringen, schlagen fehl. Erst als der Footballstar Anton Luvchenko\index{Luvchenko!Anton} sein Bein verliert und Bart nichts machen kann, werden die Leute endlich skeptisch.

\notiz{
\begin{itemize}
	\item Der Originaltitel dieser Episode \glqq Faith Off\grqq\ ist eine Anspielung auf den Film \glqq Face Off\grqq\ von 1997.
	\item Der Dekan Dean Peterson\index{Peterson!Dean} ist nicht der gleiche Dean Peterson wie aus Episode \glqq Homer an der Uni\grqq\ (siehe \ref{1F02}), den Homer mit dem Auto überfahren hat.
	\item Fehler: Als Homer dem Dekan den Streich spielen will, trägt er eine Mütze. Doch als Bart den Eimer von Homers Kopf entfernt, ist die Mütze nirgends zu sehen.
	\item Lenny und Carl haben auf der Springfielder A\&M-Universität studiert.
\end{itemize}
}
	
\subsection{Die böse Puppe Lustikus\index{Lustikus}}\label{BABF07}
Durch den Bau von Zufahrtsrampen für Behinderte gerät die Schule in finanzielle Schwierigkeiten und wird geschlossen. Daraufhin wird die Schule von einer Spielzeugfirma übernommen, die mit den Kindern insgeheim Marktforschung betreibt, um das perfekte Spielzeug zu entwickeln: eine Puppe namens Lustikus\index{Lustikus}, die sofort vermarktet wird. Als Lisa dahinterkommt, dass sie und ihre Mitschüler zu Marktforschungszwecken missbraucht wurden, ist sie empört und stellt Nachforschungen an. Sie und Bart finden heraus, dass die Puppe darauf programmiert ist, die Spielzeuge der Konkurrenz zu zerstören. Lisa versucht vergeblich, die Kunden vor dem Kauf der Puppe zu warnen -- Lustikus ist ein echter Verkaufsschlager geworden. Also bittet sie Homer, sämtliche Puppen dieser Art zu stehlen und zu vernichten -- mit Erfolg. 

\notiz{
\begin{itemize}
  \item Der Tafelspruch \glqq Ich werde meine Niere nicht über eBay verkaufen\grqq\ ist eine Anspielung auf eine Versteigerung bei eBay im Jahre 1999. Dort wollte tatsächlich jemand eine Niere versteigern. Startpreis: 25.000 Dollar. eBay hat die Auktion aber gesperrt. Kurz vor der Sperre lag das Höchstgebot bei 5,7 Millionen Dollar.
  \item Als Bausubstanz für die Rampen in der Springfielder Grundschule wurden Brotreste mit Farbe und Schellack verwendet. Baukosten: 200.000 Dollar.
  \item Milhouses Spielzeug hat ein empfohlenes Alter von zwei bis vier Jahre.
  \item Bart und Lisa vergleichen die Puppe Lustikus, auf Grund ihres Ehrgeizes jede Konkurrenz auszuschalten, mit Microsoft.
\end{itemize}
}

	
\subsection{Wenn ich einmal reich wär!}\label{BABF08}
Da Mr. Burns zu einer Generaluntersuchung in die Klinik geht, sollen die Simpsons auf sein Haus aufpassen. Homer genießt das Leben als \glqq Milliardär\grqq\ und veranstaltet eine Riesenparty auf Mr. Burns Privatyacht, die dann von Piraten überfallen wird. Nach einigen Turbulenzen können sich Homer und seine Freunde aber ans Ufer retten. Marge und Lisa haben inzwischen das Haus geputzt, was ihnen Mr. Burns Lob einbringt.

\notiz{
\begin{itemize}
  \item Marge bekommt bei der Verleihung einen Preis, weil sie am meisten Blut gespendet hat.
  \item Patienten in der Mayo-Klinik sind unter anderem Papst Johannes Paul II. und Fidel Castro.
  \item Name von Mr. Burns Boot: Gone Fission\index{Gone Fission} (vergleiche mit \ref{AABF16}).
  \item Als bei der Preisverleihung nach dem Tod von Cornelius Chapman\index{Chapman!Cornelius} (108 Jahre) nach dem nun ältesten Springfielder Mitbürger gesucht wird, stellt sich heraus, dass Mr. Burns 100 Jahre oder älter sein muss und Ned Flanders 60 Jahre oder älter ist. 
  \item In Burns Flur hängt das Bild von ihm, das Marge in \glqq Marges Meisterwerk\grqq\ (siehe \ref{7F18})  gemalt hat. Erneut werden seine Genitalien von Blumen verdeckt.
\end{itemize}
}

	
\subsection{Ein Pferd für die Familie}\label{BABF09}
Das Wunderpferd Duncan\index{Duncan} tritt auf dem Jahrmarkt als Turmspringer auf. Als Wiggum das Tier ins Schlachthaus bringen will, nehmen die Simpsons es bei sich auf. Homer entscheidet, es als Rennpferd antreten zu lassen. Duncan entpuppt sich als Superrennpferd und gewinnt einen Preis nach dem anderen. Den gegnerischen Jockeys gefällt das gar nicht, weshalb sie Homer zu erpressen versuchen. Aber der lässt sich nicht beirren und schließlich wird Duncan zum Deckhengst.

\notiz{
\begin{itemize}
  \item Die Aufschrift auf dem T-Shirt des Comicbuchverkäufers lautet: \glqq Worst Episode ever\grqq\ (Schlechteste Folge überhaupt).
  \item Homer zeigt Duncan Fotos von Pferden. Die Namen der Pferde lauten \glqq Queenie\grqq , \glqq Dolly\grqq\ und \glqq Gertie\grqq .
  \item Auf der Trabrennbahn können auch Wunder\-lich\-keits\-wet\-ten abgegeben werden. Vor diesem Schalter sind Reverend Lovejoy, Ned Flanders, Rektor Skinner und Hans Maulwurf zu sehen.
  \item Es ist das zweite Mal, dass die Simpsons ein Pferd besitzen. Das erste Mal war in der Episode \glqq Lisas Pony\grqq\ (siehe \ref{8F06}).
\end{itemize}
}

	
\subsection{Ned Flanders: Wieder allein}\label{BABF10}
Ned Flanders ist über den Tod seiner Frau sehr bekümmert und zieht sich immer mehr zurück. Um ihm zu helfen, produziert Homer ein Video über Ned. Mit diesem Film soll sich Ned bei einem Heiratsvermittlungsinstitut melden, um eine neue Frau zu finden. Ned gerät jedoch in eine religiöse Krise und sucht Trost bei Gott. In der Kirche lernt er die außerordentlich attraktive und christlich gesinnte Rocksängerin Rachel Jordan der Band Kovenant\index{Kovenant} kennen. Die beiden finden Gefallen aneinander.

\notiz{
\begin{itemize}
  \item Folgende Grabsteine sind auf dem Springfielder Friedhof zu sehen: \glqq Bleed\-ing Gums Mur\-phy\grqq\ (Zahn\-fleisch\-blu\-ter Murphy), \glqq Dr. Marvin Monroe\grqq , \glqq Beatrice Simmons -- Grampa's Girlfriend\grqq\ (Beatrice Simmons -- Grampa's Freundin) und \glqq Frank \grq Grimey\grq\ Grimes\grqq\ (Frank Grimes, der Schleimer).
  \item Ned hat ein Foto von Gott auf dem Nachttisch stehen.
  \item Wir erfahren hier, dass Cletus und Brandine Geschwister sind.
  \item Moe wurde aus der Kirche ausgeschlossen.
  \item Der PIN von Neds Kreditkarte lautet: 5316
\end{itemize}
}

	
\subsection{Der beste Missionar aller Zeiten}\label{BABF11}
Großspurig hat Homer 10.000 Dollar gespendet, um eine TV-Sendung zu sehen, ohne allerdings das Geld zu haben. Damit er der Strafverfolgung entgeht, schickt ihn Reverend Lovejoy als Missionar auf eine Südseeinsel. Dort krempelt Homer die gesamte Missionsarbeit um und eröffnet mit den Eingeborenen das Spielcasino \glqq The Lucky Savage Casino\grqq . Dies führt zu einigem Durcheinander.

\notiz{
\begin{itemize}
  \item Rufnummer von PBS: 262-7007.
  \item Unter Homers Verfolgern sind auch die Teletubbies.
  \item Homer hat in den letzten zehn Dienstjahren 17 Kernschmelzungen verursacht und waffenfähiges Plutonium an die Iraker verkauft.
  \item Am Spendentelefon von PBS sitzen auch Bender\footnote{Bei Bender handelt es sich um einen Roboter aus der Serie \glqq Futurama\grqq , die ebenfalls von Matt Groening geschaffen wurde.}\index{Bender} aus Futurama und Rupert Murdoch. Bender ist ebenfalls in \glqq Klassenkampf\grqq\ (siehe \ref{DABF20}) und \glqq Future-Drama\grqq\ (siehe \ref{GABF12}) zu sehen.
	\item Moe gibt an, aus der Kirche ausgeschlossen worden zu sein.
\end{itemize}
}
	
\subsection{Moe mit den zwei Gesichtern}\label{BABF12}
Moe unterzieht sich einer Schönheitsoperation. Er verwandelt sich in einen gut aussehenden Mann und wird als Schauspieler für die Serie \glqq It Never Ends\grqq\ entdeckt. Eine Rolle, für die er früher abgelehnt wurde. Als er jedoch glaubt, aus der Serie rausgeschrieben zu werden, inszeniert er mit Homer einen Eklat und fliegt prompt raus. Beim Verlassen des Studios fällt ihm dann eine Kulisse aufs Gesicht und er sieht aus wie früher.

\notiz{
\begin{itemize}
  \item Moe hat seine Alkoholausschanklizenz selbst ausgestellt und diese ist nur auf Rhode Island gültig. Die Lizenz ist 1973 abgelaufen.
  \item Folgende Aufkleber verdecken Moes Gesicht im Duff-Kalender: \glqq Drink Duff\grqq , \glqq Duff Enuff\grqq , \glqq Viva La Duff\grqq , \glqq Kiss Me I'm Duff\grqq , \glqq Ich bin ein Duff\grqq\ und \glqq I'mo Duff you up!\grqq .
  \item In dieser Episode wird Carls Familienname enthüllt: Carlson.
  \item Fehler: Als die Simpsons aus dem Auto aussteigen, sind sie plötzlich angezogen, als sie eingestiegen sind, hatten sie noch ihre Nachtbekleidung an.
\end{itemize}
}
	
\subsection{Barts Blick in die Zukunft}\label{BABF13}
Ein Indianer gewährt Bart in einem Indiandercasino einen Blick in die Zukunft: Lisa wird Präsidentin der Vereinigten Staaten, während Bart als Versager ihr Regierungskonzept ge\-fähr\-det. Homer sucht im Weißen Haus nach Lincolns Gold und Marge weiß nicht mehr, was sie machen soll. Als ausländische Gläubiger auftauchen, gelingt es Bart dann, diese zu beruhigen und Lisas Sympathie wiederzugewinnen.

\notiz{
\begin{itemize}
  \item Der Name des Indianercasinos lautet \glqq Cesar's Pow-Wow\grqq .
  \item Der Originaltitel \glqq Bart to the Future\grqq\ ist eine Anspielung auf den Film \glqq Back To The Future\grqq\ (Zurück in die Zukunft).
  \item Barts Band heißt \glqq Captain Bart and the Tequila Mockingbirds\grqq .
  \item Nelson ist in der Zukunft Besitzer einer Bar mit dem Namen \glqq Nelsons Crab Shack\grqq . Seine Gäste sind unter anderem Barney, Carl und Lenny. Auch Hausmeister Willie arbeitet in der Bar.
  \item Nachbarhäuser des Weißen Hauses: \glqq Drive-Thru Liquors\grqq\ und \glqq Hustler Superstore\grqq .
  \item In der Zukunft ist Milhouse Staatssekretär.
  \item Kearney ist Lisas Bodyguard.
  \item Personen in Camp David sind unter anderem Bart, Krusty, Ralph, Nelson und Otto.
\end{itemize}
}
	
\subsection{Barneys Hubschrauber-Flugstunde}\label{BABF14}
Barney will ein neues Leben beginnen, er entsagt dem Alkohol und nimmt Hubschrauber-Flugstunden, weil er feststellt, dass er sich bei seinem letzten Geburtstag unmöglich aufgeführt hat. Homer hat das Gefühl, dass Barney sich nun für etwas besseres hält und will nichts mehr mit ihm zu tun haben. Doch als Bart und Lisa einen Waldbrand verursachen und vom Feuer eingeschlossen werden, rettet Barney die Kinder mit dem Hubschrauber und die alte Freundschaft zwischen Homer und Barney ist wieder hergestellt.

\notiz{
\begin{itemize}
  \item Moes Bar steht im \glqq Schwulenführer durch Springfield\grqq .
  \item Personen bei den Anonymen Alkoholikern: Lindsey Naegle, Kirk Van Houten, Mrs. Hibbert, Gil und Kent Brockman.
  \item Lisa und Bart sehen die Teletubbies im Fernsehen.
  \item Die Adresse, an die das Foto für das Springfielder Telefonbuch gesendet werden sollte, lautet: Springfield Bell; Springfield, USA; P.O. Box 2153-1264; ROOM RR-312-BB5; Attn: Photo Submissions Supervisor.
  \item Fehler: Als Homer in die Bar kommt und Moe sagt, dass er sich nicht auf den Hocker setzen sollte, sondern auf den nebenan, stehen zwei leere Hocker dort. Kurze Zeit später, als Homer tanzt, sieht man aber nur mehr einen Hocker.
  \item Co-Autor dieser Episode ist der Sprecher von Homer Dan Castellaneta.
\end{itemize}
}
	
\subsection{Sie wollte schon immer Tänzerin werden}
Lisa möchte unbedingt Tanzen lernen und nimmt Unterricht bei der ehemals berühmten Tänzerin Vicky\index{Vicky}. Da sie nicht gut genug ist, um bei einer Tanzvorstellung aufzutreten, erfindet Professor Frink für Lisa ein paar elektrische Steppschuhe, mit denen sie Vicky beim Auftritt die Show stiehlt. Aber der Schwindel fliegt auf und Lisa sieht ein, dass sie nie eine berühmte Tänzerin wird.

\notiz{
\begin{itemize}
	\item Das Geschäft, in dem Marge die Camping-Ausrüstung für Bart kauft, heißt \glqq Tommy Hillclimber\grqq . Dies dürfte eine Anspielung auf den Designer \glqq Tommy Hillfiger\grqq\ sein.
	\item Der Cyborganizer ist eine Anspielung auf die \glqq Robocop\grqq\ Filme.
	\item Der Optiker, bei dem sich Homer eine Brille kaufen will und sich schließlich die Sehkraft mit einer Laseroperation wiederherstellen lässt, heißt \glqq Eye Caramba\grqq .
\end{itemize}
}

	
\subsection{Kill den Alligator und dann\dots}
Homer erleidet einen Nervenzusammenbruch und wird zur Kur nach Florida geschickt. Dort ist nicht alles so, wie Familie Simpson es sich vorstellt. Statt sich zu erholen, stürzt Homer sich ins Vergnügen. Dabei richtet er etliches Unheil an, bis die ganze Familie schließlich im Gefängnis landet. Mit der Auflage, sich nie wieder in Florida blicken zu lassen, schickt man die Simpsons schließlich nach Hause zurück.

\notiz{
\begin{itemize}
  \item Homer stellt in einem Test fest, dass seine Lebenserwartung nur 42 Jahre beträgt. Er sagt daraufhin, er habe nur noch drei Jahre zu leben.
  \item Homer gibt an, dass er seit acht Jahren Nichtraucher ist.
  \item Banner auf Homers Party-Boot: \glqq Spring Break 4-Ever\grqq .
  \item Nummer auf der Häftlingskleidung: Homer hat die Nummer 1028 und Marge die 4670.
  \item Die Simpsons sind noch willkommen in \glqq North Dakota\grqq\ und \glqq Arizona\grqq .
\end{itemize}
}
	
\subsection{Wird Marge verrückt gemacht?}\label{BABF18}
Eines Morgens, auf dem Weg in die Schule, nimmt Otto mit dem Bus einen kurzen Umweg, um seiner Freundin Becky\index{Becky} einen Heiratsantrag durch das Drive-In-Fenster eines Restaurants, in dem sie arbeitet, zu machen. Daraufhin überredet Bart Marge dazu, die Hochzeit in ihren Garten abzuhalten. Alles scheint perfekt, bis Otto darauf besteht Heavy-Metal-Musik, die Becky hasst, bei der Trauung zu spielen. Die Ehe ist vorüber, bevor sie eigentlich beginnt. Da Becky ihr Leben neu überdenken will, zieht sie bei den Simpsons ein. Bald wird klar, dass Becky vorteilhafte Konkurrenz für Marge im Haus ist, wenn es ums Kochen, Backen, den Kindern Karate zu lehren und ihnen bei den Hausaufgaben zu helfen geht. Es ist nur eine Angelegenheit der Zeit, bis Marge überschnappt und versucht, ihr Leben durch Gewalt zurückzuholen.


\notiz{
\begin{itemize}
  \item Ottos Freundin Becky arbeitet im \glqq Der Krazy Kraut\index{Krazy Kraut}\grqq .
  \item Auf der Schatulle, aus dem Homer das Röhrchen zum Aussaugen der Torte zieht, stehen seine Initialen \glqq HJS\grqq . Im Hintergrund läuft die Musik aus James Bond.
  \item Die Art und Weise wie Otto bei seiner Hochzeit gekleidet ist, erinnert stark an das Video zu \glqq November Rain\grqq\ von \glqq Guns N' Roses\grqq .
  \item Auf dem Banner, welches die Musikkapelle herumträgt, steht \glqq Springfield Mental Asylum Marching Band\grqq\ (Marschkapelle des Asyls für Geisteskranke).
\end{itemize}
}
	
\subsection{Hinter den Lachern}\label{BABF19}
Die Simpsons erzählen in Rückblicken die Entstehungsgeschichte ihrer Serie, indem sie sämtliche Höhen und Tiefen ansprechen, mit Interview-Ausschnitten, in der Hoffnung, dass die Serie fortgesetzt wird. Dabei erfährt die Öffentlichkeit, dass in Wahrheit Homer die Drehbücher zu der Produktion schreibt. Er wollte immer schon eine Familienserie, die das wahre Leben zeigt und Marge meinte: \glqq Dann mach's doch!\grqq

\notiz{
\begin{itemize}
  \item Die Familie Simpson kommt angeblich aus Kentucky. Dies kann allerdings nicht stimmen, vergleiche dazu \glqq Gibt es Springfield wirklich?\grqq\ (siehe \ref{ExistiertSpringfield}).
  \item Für diese Folge gewannen die Simpsons den Emmy für die beste halbstündige Animationsserie.
\end{itemize}
}


\section{Staffel 12}

\subsection{O mein Clown Papa}\label{BABF17}
Ganz unerwartet taucht eine Tochter (Sophie\index{Sophie}) von Krusty dem Clown auf. Zunächst will er nichts mit ihr zu tun haben, doch dann freundet er sich mit ihr an. Als er beim Pokern ihre Geige an Fat Tony verliert, ist sie von ihrem \glqq Daddy\grqq\ zutiefst enttäuscht. Doch mit Homers Hilfe bricht er beim Gangsterboss Fat Tony ein und holt sich die Geige zurück. Sophie ist nun von ihrem Daddy restlos begeistert.

\notiz{
\begin{itemize}
  \item So wie es scheint, handelt es sich bei Sophie bereits um das zweite uneheliche Kind von Krusty. In der Episode \glqq Homer kommt in Fahrt\grqq\ (siehe \ref{9F10}) ist eine Junge zu sehen, der ihm sehr ähnlich sieht und dessen Mutter Krusty fragt, warum er sich nicht um seinen Sohn kümmere.
  \item Fertigstellungstermin der neuen Hundehütte: Januar 2007.
  \item Bart überschreibt die Tafel: \glqq The future of reading\grqq\ (Die Zukunft des Lesens), Bart korrigiert das Wort \glqq reading\grqq\ zu \glqq breading\grqq\ (panierens).
  \item In der Pokerrunde sitzen Homer, Snake, Moe, Krusty und Fat Tony.
  \item Nummer der Wohnungstür von Sophie ist 107.
  \item Die Simpsons sehen im Fernsehen die Serie \glqq Dawson's Creek\grqq , als Austauschstudent spielt der Hummelmann mit.
  \item Autokennzeichen vor Fat Tonys Villa: Bagman2, MOB MOM und KILLER.
  \item Name von Fat Tonys Website: \url{crime.org} (Soll so viel wie organisiertes Verbrechen bedeuten; org = organisation).
\end{itemize}
}

	
\subsection{Die Geschichte der zwei Springfields}\label{BABF20}
Da Springfield in zwei verschiedene Vorwahlnummernbereiche aufgeteilt wird, kommt es zum Streit zwischen den beiden Stadtteilen. Homer ruft sich zum Bürgermeister von Neu Springfield aus und lässt die Stadt durch eine Mauer trennen. Als er die Band \glqq The Who\grqq\ dazu bringt, in Neu Springfield statt in Alt Springfield aufzutreten, kommt es zu einem offenen Kampf. Bis \glqq The Who\grqq\ die Unsinnigkeit des Streits erfahren und zum Anlass nehmen, so laut zu spielen, dass die Mauer in sich zusammenfällt.

\notiz{
\begin{itemize}
  \item Der Specht von Rod und Todd \glqq lacht\grqq\ wie Woody Woodpecker\index{Woodpecker!Woody}, nachdem er Bart und dem Dachs entkommen ist.
  \item Die neuen Vorwahlen in Springfield: 636 und 939.
  \item Homer schreibt sich die neue Vorwahl auf die linke Hand, auf der rechten Hand hat er stehen: \glqq Lenny = white; Carl = black\grqq\ (Lenny = weiß; Carl = schwarz).
  \item Die Simpsons haben die Telefonnummer: 555-0113.
  \item Als die vom Radiosender die Telefonnummer 555-0113 wählen, geht Mr. Burns an das Telefon, da dieser im 636er Vorwahlbereich wohnt.
\end{itemize}
}

\subsection{D-D-Der G-G-Geister D-D-Dad}\label{BABF21}
\begin{itemize}
	\item \textbf{D-D-Der G-G-Geister D-D-Dad}\\ Homer stirbt und macht als \glqq Geister-Dad\grqq\ Springfield unsicher. Er wird von Petrus gezwungen, eine gute Tat zu vollbringen, um in den Himmel eingelassen zu werden. Dies gelingt ihm erst in letzter Sekunde, indem er ein Baby rettet, doch diese Tat wird nicht beachtet.
	\item \textbf{Schreckliche Märchen können wahr werden}\\ Homer verliert seinen Job und kann seine Familie nicht mehr ernähren. So müssen Bart und Lisa im Wald ausgesetzt werden und geraten schließlich zu einem Pfefferkuchenhaus, wo sie auf eine Hexe treffen.
	\item \textbf{Die Nacht des Delphins}\\ Delphine behaupten, dass die Menschen sie einst ins Wasser getrieben hätten. Nun wollen sie an Land und den Spieß umdrehen. 
\end{itemize}

\notiz{
\begin{itemize}
  \item Homers Sternzeichen ist Stier. In seinem Horoskop steht, dass er heute sterben wird und ein Kompliment von einem Mitarbeiter bekommen wird. In Marges Horoskop steht, dass heute ihr Mann sterben wird.
  \item Der Name der Hexe ist Susan\index{Susan} und ihr Freund heißt George Kochkessel\index{Kochkessel!George}.
  \item Filmzitat: Die Szene, in der Lisa den Delphin \glqq Snorky\grqq\ befreit, parodiert den Film \glqq Free Willy\grqq .
\end{itemize}
}


	
\subsection{Der berüchtigte Kleinhirn-Malstift}\label{BABF22}
In seiner Kindheit hat Homer sich immer Malstifte in die Nase geschoben, von denen schließlich einer in seiner Nase blieb, sodass er ein tumber Mensch wurde. Als man ihm den Stift aus dem Gehirn entfernt, wird er plötzlich hochintelligent, was dazu führt, dass er seine Freunde verliert. Schließlich bittet er Moe, ihm erneut einen Malstift durch die Nase ins Gehirn zu hämmern, damit er wieder wie früher wird. Lisa, die so stolz auf ihren Vater war, muss nun wieder mit dem alten Homer vorliebnehmen, verzeiht ihm aber.

\notiz{
\begin{itemize}
  \item Diese Episode stellt einen Widerspruch zu \glqq Vertrottelt Lisa?\grqq\ (siehe \ref{4F24}) dar, weil dort behauptet wurde, die Dummheit der männlichen Simpsons kommt von den Genen.
  \item Homer hat normalerweise einen IQ von 55. Als man ihm den Malstift entfernt hat, hat er einen IQ von 105.
  \item Aufschrift auf dem T-Shirt des Comicbuchverkäufers: \glqq Worst Convention Ever!\grqq .
  \item Moes Visitenkarte: \glqq That's Right. I'm a Surgeon. -- Moe Szyslak (800) 555-0000\grqq\ (Ganz recht. Ich bin Chirurg).
  \item Für diese Folge gewannen die Simpsons den Emmy für beste halbstündige Animationsserie.
  \item Carl behauptet, dass er ohne den Job im Atomkraftwerk seine Familie nicht ernähren kann.
\end{itemize}
}


\subsection{Lisa als Baumliebhaberin}
Da die Stadt Springfield einen Sequoiabaum rodeln lassen will, schließt sich Lisa den Umweltschützern an. Als Baumbesetzerin nistet sie sich auf dem ältesten Sequoiabaum ein. Als sie ihn einmal für kurze Zeit verlässt, wird er vom Blitz getroffen und stürzt um. Zunächst hält man Lisa für tot. Aber als ein reicher Texaner auf dem Gelände einen Vergnügungspark errichten will, erscheint sie wieder in der Öffentlichkeit. Das für sie errichtete Denkmal wird umgestürzt und zerstört seine Firma Omni-Pave\index{Omni-Pave}.

\notiz{
\begin{itemize}
  \item Krusty der Clown wurde präsentiert von der neuen \glqq Gamestation 256\grqq .
  \item Filmzitat: Als Bart die Speisekarten austrägt, sind seine Bewegungsabläufe dem Film \glqq Matrix\grqq\ nachempfunden.
  \item Jesse Grass\index{Grass!Jesse} ist Veganer Stufe 5, er isst nichts, was einen Schatten wirft.
  \item Aufkleber auf Homers Auto: \glqq My child is a Dead Honor Student\grqq\ (Mein Kind ist eine tote Ehrenschülerin).
  \item Autokennzeichen des reichen Texaners \glqq No Shame\grqq (Keine Schande).
  \item Mr. Burns ist laut Kent Brockman der älteste Einwohner Springfields. Da Mr. Burns Mutter noch lebt (\glqq Butler bei Burns\grqq , \ref{3F14}), wohnt sie anscheinend nicht in Springfield.
\end{itemize}
}

\subsection{Mr. X und der Website-Schund}\label{CABF02}
Homer hat sich einen Computer zugelegt und deckt auf einer eigenen Website geheime Verschwörungen und Missetaten auf, was ihm zwar den Pulitzer Preis einbringt, aber auch das Missfallen seiner Freunde. Eines Tages wird er auf eine Insel entführt, auf der Leute gefangengehalten werden, die zu viel wissen. Die dortigen Herrscher sind empört, dass Homer durch eine erfundene Geschichte ihren Plan durchkreuzt hat, die Weltherrschaft zu übernehmen. Homer gelingt es zwar zu entkommen und nach Hause zurückzukehren, aber schon nach kurzer Zeit befindet er sich mit der gesamten Familie wieder auf der Insel.

\notiz{
\begin{itemize}
  \item Homer nimmt auf Grund seines neuen 5.000 Dollar Computers eine fünfte Hypothek auf das Haus auf.
  \item Schild am Waldesrand: \glqq Springfield Forest -- Witch-Free Since 1998\grqq\ (Hexenfrei seit 1998).
\end{itemize}
}

\subsection{Das große Geldabzocken}
Bart und Homer versuchen ihre Mitmenschen mit kleinen Betrügereien reinzulegen und ihnen Geld abzuzocken. Das gelingt ihnen zunächst ganz gut, aber als ein vermeintlicher FBI-Mann sie festnimmt und der Polizei ausliefert, sind sie die Dummen, denn er stiehlt ihren Wagen. Statt Marge alles zu gestehen, tischen sie ihr eine Lügengeschichte auf, mit dem Resultat, dass Hausmeister Willie als Täter festgenommen und vor Gericht gestellt wird. Nach seiner Verurteilung schießt er Rektor Skinner nieder, woraufhin Homer endlich mit der Wahrheit herausrückt. Nun stellt sich heraus, dass die ganze Geschichte von Marge und Lisa inszeniert wurde, um Homer eine Lektion zu erteilen.

\notiz{
\begin{itemize}
  \item Kennzeichen auf Homers Auto: I GRIFT (Ich ergaunere).
  \item Als auch Bart Handschellen angelegt bekommt, sagt er \glqq Nein!\grqq\ wie Homer.
  \item Auf der vermeintlichen FBI-Dienstmarke steht \glqq Colgate Cavity Patrol\grqq\ (Colgate Zahnloch-Polizei).
\end{itemize}
}

\subsection{Homer und das Geschenk der Würde}\label{CABF04}
Zur Aufbesserung seines Gehalts verdingt sich Homer bei Mr. Burns als so genanntes \glqq Spaß-Äffchen\grqq , um ihn zu unterhalten. Dies funktioniert auch eine Weile ganz gut, bis Homer als Pandabär verkleidet in arge Bedrängnis gerät und Lisa erkennt, was ihr Vater so nebenbei alles macht. Er fühlt sich zutiefst in seiner Würde verletzt und versucht, sie um jeden Preis wiederzuerlangen, aber ohne Lisas Hilfe geht es nicht.

\notiz{
\begin{itemize}
  \item Bart schreibt seine erste Eins (\glqq A\grqq) in Astronomie, was die Familie auch in einem Restaurant feiert.
  \item Die Tastenkombination für den Safe, in dem das erste Spiderman-Heft aufbewahrt wird, lautet: 007.
  \item Lindsey Naegle arbeitet als Finanzplanerin.
  \item Waylon Smithers schreibt ein Musical über die Malibu-Stacy-Puppe.
\end{itemize}
}


\subsection{Jack und der Rückgratzylinder}
Marge lernt bei einem Gefängnisbesuch den talentierten Maler Jack Crowley\index{Crowley!Jack} kennen und sorgt dafür, dass er auf Bewährung entlassen wird. Außerdem verschafft sie ihm den Auftrag, eine Schulwand zu bemalen. Aber seine Vorstellungen unterscheiden sich gewaltig von Rektor Skinners Ideen. So kommt es zum Eklat: Jack zündet sein Gemälde und Skinners Auto an und wandert wieder ins Gefängnis.

\notiz{
\begin{itemize}
  \item Aufschrift auf Lisas Schürze: \glqq Barbecue is Murder\grqq\ (Grillen ist Mord).
  \item Homer hält Lisa beim Gefängnisrodeo des Waterville Prisons\index{Waterville Gefängnis} über die Bande. Dies dürfte eine Anspielung auf Michael Jackson sein, der eines seiner Kinder in einem Berliner Hotel über die Balkonbrüstung hielt.
  \item Homers Chiropraktiker heißt \glqq Dr. Steve\index{Steve!Dr.}\grqq .
  \item Jack saß u.\,a. im Gefängnis, weil er auf Apu geschossen hatte.
  \item Fehler: Als Jack im Streifenwagen sitzt, sieht man im Hintergrund das gelöschte Auto von Rektor Skinner. Obwohl es gebrannt hat, ist es weiß.
\end{itemize}
}

\subsection{Rektor Skinners Gespür für Schnee}\label{CABF06}
Die Grundschule von Springfield wird total eingeschneit, sodass keiner mehr das Gebäude verlassen kann. Vergeblich versucht Rektor Skinner Ruhe und Ordnung zu bewahren. Aber die Kinder überrumpeln ihn und übernehmen selbst die Kontrolle, indem sie alles auf den Kopf stellen. Inzwischen haben Homer und Ned sich auf den Weg gemacht, um die Kinder zu retten, bleiben aber selbst im Schnee stecken. Als sie schließlich gegen ein riesiges Salzsilo fahren und dieses umstürzt, fängt der Schnee an zu schmelzen und die Eingeschlossenen kommen frei.

\notiz{
\begin{itemize}
  \item Die Simpsons besuchen den \glqq Cirque de Pur$\acute{e}$e\grqq .
  \item Schulen, die wegen des Schnees geschlossen bleiben: Shelbyville, Ogdenville, Ogdenville technisches College und die Springfield Dr. Watson Detective School.
  \item Die Kinder sehen sich in der Schule Rektor Skinners Lieblingsfilm \glqq The Christmas That Almost Wasn't But Then Was\grqq\ (Das Weihnachtsfest das fast nicht da war aber dann doch) aus dem Jahre 1938 an.
  \item Skinners Tafelstrafe: \glqq I Ain't not a dorkus\grqq\ (Ich bin nicht kein Unfähikus).
  \item Rektor Skinner hat ein Jahresgehalt von 25.000 Dollar.
  \item Kearneys Sohn wurde im Dezember geboren.
  \item Homer trägt bei der Rettung der Schulkinder seine \glqq Mr. Plow\grqq -Jacke aus der Episode \glqq Einmal als Schneekönig\grqq\ (siehe \ref{9F07}) .
  \item Nelson behauptet, Halbeskimo zu sein.
\end{itemize}
}

\subsection{Tennis mit Venus}
Die Simpsons legen sich einen eigenen Tennisplatz zu. Alle wollen bei ihnen spielen und schließlich meldet Homer sich und Marge zu einem Turnier an. Aber weil Homer nicht gut genug ist, ersetzt Marge ihn durch Bart. Homer ist empört und nimmt sich Lisa als Doppelpartnerin. Beim Turnier wechselt er sie allerdings gegen Venus Williams aus, die zufällig anwesend ist. Im Gegenzug holt sich Marge deren Schwester Serena. Die Williams-Schwestern wiederum holen sich Pete Sampras und Andre Agassi, sodass schließlich kein Simpson mehr auf dem Platz ist. Untereinander wieder versöhnt, sehen sie sich ein Weltklasse-Tennisspiel an.

\notiz{
\begin{itemize}
  \item Grandpas Beerdigung würde 17.000 Dollar kosten.
  \item Folgende Personen spielen bei den Simpsons Tennis: Lenny und Carl, Kent Brockman und die Wettermaus Stephanie, Chief Wiggum und Lou, Mr. Burns und Mr. Smithers, Barney und Moe, Krusty und Sideshow Mel, Dr. und Mrs. Hibbert.
  \item Fehler beim Tennisspiel mit Kent Brockman: Obwohl der Ball im Aus war, gewinnen Kent und Stephanie das Spiel.
\end{itemize}
}

\subsection{Die schlechteste Episode überhaupt}\label{CABF08}
Da Bart und Milhouse dem Comichändler ein Geschäft vermasseln, dürfen sie den Laden nicht mehr betreten. Dies ist umso schlimmer, weil dort kurz darauf der berühmte Special Effects Mann Tom Savini\index{Savini!Tom} auftritt. Bei der Vorstellung erleidet der Comichändler einen Herzinfarkt. Bart ruft den Notarzt und rettet ihm damit das Leben. Aus Dankbarkeit überlässt er Bart und Milhouse vorübergehend seinen Laden, während er sich in seiner Reha-Zeit mit Mrs. Skinner anfreundet. Bart und Milhouse entdecken in dem Laden einen geheimen Raum, in dem verbotene Videos und Raubkopien aufbewahrt werden. Als die Polizei davon erfährt, wird der ganze Laden hochgenommen und der Comichändler verhaftet.

\notiz{
\begin{itemize}
  \item Außer Bart und Milhouse sind noch folgende Personen auf Lebenszeit im Comicbuchladen gesperrt: Sideshow Bob, Nelson Muntz und Matt Groening.
  \item Milhouse kauft 2000 Exemplare des Comics \glqq Biclops\grqq .
  \item Aufschrift auf dem T-Shirt des Comicbuchverkäufers bei seiner Verabredung mit Agnes Skinner: \glqq My other T-Shirt is clean\grqq\ (Mein anderes T-Shirt ist sauber).
  \item Ned Flanders ist geheimer Informant bei der Polizei. Er hat dort die Informanten-Nummer 2381.
  \item Milhouse ist nach eigener Aussage drei Monate jünger als Bart.
\end{itemize}
}

\subsection{Der hungrige, hungrige Homer}\label{CABF09}
Homer findet zufällig heraus, dass die Duff Brauerei, der Springfields Baseballteam gehört, die Mannschaft nach Albuquerque\index{Albuquerque} verkaufen will. Vergeblich versucht Homer, die Öffentlichkeit über den geplanten Deal zu informieren: Keiner glaubt ihm. Aus Protest tritt er in einen 12-tägigen Hungerstreik. Schließlich stellt sich heraus, dass Homer recht hatte. Der Verkauf der Baseballmannschaft kann noch rechtzeitig verhindert werden.

\notiz{
\begin{itemize}
  \item Homer und Lisa fahren das Boot mit der Nummer 28.
  \item Lisa gibt Homer das Buch \glqq My core Beliefs\grqq\ von Mike Farrell.
  \item Milhouse nennt Kirk Van Houten \glqq Wochenenddad\grqq .
  \item Mit dieser Episode überholten die Simpsons \glqq Eine schrecklich nette Familie\grqq\ als Comedy-Serie mit den meisten Episoden auf Fox.
  \item Der Vorname vom Duffman scheint Sid zu sein. Doch in der Folge \glqq Moe mit den zwei Gesichtern\grqq\ (siehe \ref{BABF12}) heißt er Larry und in der Folge \glqq Warten auf Duffman\grqq\ (siehe \ref{TABF10}) heißt er Barry.
\end{itemize}
}

\subsection{Hallo, du kleiner Hypnose-Mörder}\label{CABF10}
Tingel-Tangel Bob wird aus dem Gefängnis entlassen und will sich an Krusty rächen, der ihn ins Gefängnis gebracht hat. Er hypnotisiert Bart und schnallt ihm eine Bombe um. Dann soll er Krusty bei einem Fernsehauftritt umarmen und ihn so in die Luft sprengen. Doch als Krusty sich während der Sendung bei Tingel-Tangel Bob entschuldigt, zeigt er Reue und will das Attentat verhindern. Dem Affen Mr. Teeny gelingt es gerade noch, Bart die Bombe zu entreißen und in die Garderobe zu werfen. Trotzdem erwartet Bob die Todesstrafe.

\notiz{
\begin{itemize}
  \item Moe spielt in der Quizshow \glqq Me Wanted!\grqq\ (Ich werde gesucht; bei uns \glqq Wer wird Millionär\grqq ) mit. Die 500.000 Dollar Frage lautet: \glqq Which of the following is not a subatomic Particle? A: Proton; B: Neutron; C: Bonbon; D: Electron\grqq\ (Welches der folgenden Elemente ist kein atomares Partikel? A: Protonen; B: Neutronen; C: Bonbons; D: Elektronen).
  \item Moe sagt, dass er in Indiana geboren wurde. Allerdings wird in der Episode \glqq Volksabstimmung in Springfield\grqq\ (siehe \ref{3F20}) enthüllt, dass er ein illegaler Immigrant ist. In der späteren Folge \glqq Geächtet\grqq\ (siehe \ref{FABF17}) gibt Moe selbst an, aus Holland zu stammen.
  \item Krusty ist bereits seit 61 Jahren im Showgeschäft tätig.
  \item Laut Ansage bei der Krusty Show geht Krusty zum fünften und letzten Mal in den Ruhestand.
  \item Tingel-Tangel Bob tritt auf eine Harke wie in der Folge \glqq Am Kap der Angst\grqq\ (siehe \ref{9F22}). 
\end{itemize}
}

\subsection{Lisa knackt den Rowdy-Code}\label{CABF11}
Francine\index{Francine}, eine neue Mitschülerin, fällt vor allem durch ihre extreme Aggressivität auf. Lisa findet heraus, dass dieses Verhalten eine Reaktion auf die Schweißausdünstungen von intelligenteren Menschen ist. Sie findet ein Gegenmittel, für welches sie größtes Lob erhält.

\notiz{
\begin{itemize}
  \item Drederick Tatum\index{Tatum!Drederick} ist zu Gast in der Schule.
  \item Die Sporthalle der Schule ist jetzt nach Dr. Marvin Monroe benannt.
  \item Im Bus hat ein Kind mit Zahnspange ein T-Shirt an, auf dem \glqq Frankie says relax\grqq\ steht. Dies ist eine Anspielung auf das Lied \glqq Relax\grqq\ von Frankie Goes To Hollywood.
  \item Lisa gewinnt den Schweiß für ihr Experiment von Milhouse, Database und Martin.
\end{itemize}
}

\subsection{Die sensationelle Pop-Gruppe}\label{CABF12}
L.T. Smash\index{Smash!L.T.} gründet mit Bart, Milhouse, Ralph und Nelson eine junge aufstrebende Band namens \glqq Party Posse\grqq \index{Party Posse}. Aber schon bald merkt Lisa, dass in den Musik-Videos Schleichwerbung für die Marine gemacht wird. Die Jungs stört das wenig, schließlich haben sie Erfolg. Als die Regierung L.T. Smashs Projekt einstellt, dreht er durch und fährt mit dem Flugzeugträger einen Angriff auf das New Yorker Redaktionsgebäude der Satire-Zeitschrift \glqq Mad\grqq . Am Ende wird Smash in polizeilichen Gewahrsam genommen.

\notiz{
\begin{itemize}
  \item Homer behauptet in dieser Folge, 38 Jahre alt zu sein.
  \item Teilnehmer am Marathon (in Klammer die Startnummer): Homer Simpson (74), Apu Nahasapeemapetilon (33), Ned Flanders (16), Edna Krabappel, Moe Szyslak (79), Waylon Smithers (628), Charles Montgomery Burns\footnote{wird von Smithers auf einer Rikscha gezogen} (108), Comicbuchverkäufer, Melvin Van Horne (Sideshow Mel) (26), Otto Mann (10), Dr. Julius Hibbert (8), Carl Carlson (134), Lenny Leonard, Kapitän Horatio McCallister und Rektor Seymour Skinner (18).
  \item Video der Party Truppe: Party Posse, \glqq Drop Da Bomb\grqq , Regie: Ang Lee\index{Lee!Ang}.
\end{itemize}
}
	
\subsection{Die Sippe auf Safari}
Die Simpsons haben eine Reise nach Afrika gewonnen. Dort angekommen, erleben sie die wunderlichsten Abenteuer. Als sie sich verirren, lernen sie die berühmte Affenforscherin Doktor Joan Bushwell\index{Bushwell!Joan} kennen. Diese entpuppt sich jedoch als Ausbeuterin ihrer Schützlinge, die sie in einer tief unter der Erde gelegenen Diamantengrube arbeiten lässt. Mithilfe von Greenpeace wird die zwielichtige Forscherin schließlich entlarvt. Daraufhin versucht sie, sich mit Diamanten freizukaufen, was den Simpsons natürlich gerade recht ist.

\notiz{
\begin{itemize}
	\item Joan Bushwell ist eine Verulkung von Jane Goodall\footnote{Dr. Jane Goodall (geboren am 3. April 1934 in London) ist eine britische Verhaltensforscherin. Sie studiert seit 1960 das Verhalten von Schimpansen im Gombe Stream National Park in Tansania \cite{Goodall}.}
	\item Der Fremdenführer der Simpsons, Kitenge\index{Kitenge}, wird Nachfolger Muntus\index{Muntu} als Präsident von Tansania. Muntu hingegen arbeitet, nachdem Kitenge an die Macht gekommen ist, als Flugbegleiter.
	\item Fehler: Als die Simpsons bei Dr. Bushwell essen, ist Barts Zunge für kurze Zeit grün. Ein Kolorationsfehler.
\end{itemize}
}

	
\subsection{Trilogie derselben Geschichte}\label{CABF14}
Ein Tag im Leben der Simpsons wird erst aus Homers, dann aus Lisas und schließlich aus Barts Blickwinkel erzählt. Marge hat Homer aus Versehen einen Daumen abgeschnitten, als er nach einem Stück Kuchen greifen wollte. Sofort rasen die beiden ins Krankenhaus, wo der Daumen wieder angenäht werden soll. Wegen dieses Unfalls kommt Lisa fast zu spät zur Schule, wo sie ein Forschungsprojekt vorstellen will. Unterdessen hat Bart den Schmugglerring um Fat Tony ausgehoben, dessen Gehilfe, Lex\index{Simpsons}, schließlich Homers Daumen wieder annäht.

\notiz{
\begin{itemize}
  \item Die Anfangssequenz mit der Müllabfuhr und dem Zeitungsjungen ist in allen drei Geschichten gleich.
  \item Marge gibt Chief Wiggum die Adresse \glqq 123 Fake Street\grqq\ (Schwin\-del\-stra\-ße 123) am Telefon durch.
  \item Ned Flanders verbrennt ein Harry Potter Buch.
  \item Filmzitat: Die Szene, in der Lisa durch die Straßen läuft, parodiert den Film \glqq Lola rennt\grqq . Auch die Hintergrundmusik ist die gleiche.
  \item Als sich Homer in Richtung Shelbyville zu Fuß auf den Weg macht, ist ein Schild zu sehen, auf dem steht, dass Shelbyville 20 Meilen entfernt ist.
\end{itemize}
}

\subsection{Wunder gibt es immer wieder}\label{CABF15}
Zwischen Ned und einer alten Bekannten, der Sängerin Rachel Jordan\index{Jordan!Rachel} (\glqq Ned Flanders: Wieder allein\grqq , \ref{BABF10}), funkt es, doch er ist immer noch nicht über Maudes Tod hinweg. Homer und Marge bieten sich an, alle Dinge, die Ned an Maude erinnern, aus seinem Haus zu entfernen. Dabei stoßen sie in einem Skizzenblock auf Maudes Entwurf eines Vergnügungsparks namens \glqq Praiseland\grqq \index{Praiseland}. Schließlich errichten sie den Vergnügungspark, der allerdings aufgrund einer defekten Gasleitung, welche Halluzinationen bei den Besuchern hervorgerufen hat, wieder geschlossen werden muss.

\notiz{
\begin{itemize}
  \item In dieser Folge ist Todd der Kleinere der Flanders Kinder.
  \item Rachel Jordan ist auch als \glqq christliche Madonna\grqq\ bekannt.
  \item Die Augenbrauen von Milhouse sind schwarz, als er mit der \glqq King David's Wild Ride\grqq\ fährt.
  \item Ned fordert Rachel auf, in Neds Bett nicht auf der Seite von Maude zu schlafen, damit sie ihre Kuhle nicht zerstöre. Allerdings schläft er selbst in der Folge \glqq Ned Flanders: Wieder allein\grqq\ (siehe \ref{BABF10}) auf ihrer Seite.
\end{itemize}
}

	
\subsection{Der Schwindler und seine Kinder}
Da Homer sich am Knie verletzt hat und arbeitsunfähig ist, macht er eine Kindertagesstätte auf. Allerdings kümmert er sich nur noch um die fremden Kinder und nicht mehr um seine eigenen. Bart und Lisa sind empört. Als ihm auch noch ein Ehrenpreis verliehen werden soll, reicht es ihnen. In einen Film über Homers gute Taten schneiden sie kurzerhand einige Szenen aus seinem Privatleben, u.\,a. eine, in der er Weihnachten betrunken vor dem Weihnachtsbaum am Boden liegt. Alle Eltern sind empört und wollen ihre Kinder nicht länger in Homers Obhut geben. Homer versucht, mit den Kindern zu fliehen. Die Flucht endet allerdings am Gefängniszaun. Geläutert will Homer sich jetzt nur noch um seine eigenen Kinder kümmern.

\notiz{
\begin{itemize}
	\item In Homers Morphium-Trip begegnet er Vater Jetson aus der Serie \glqq Die Jetsons\grqq .
	\item In der Basketballmannschaft der über 85-jährigen spielen nur Abraham Simpson und Mr. Burns.
	\item Obwohl zum Schluss Ralph Wiggum hinten im Polizeitransporter zu sehen ist, als Homer die Türen schließt, ist er in der nächsten Szene auf dem Beifahrersitz zu sehen.
	\item Ned Flanders besucht ein Konzert von Chris Rock.
	  \item Der Originaltitel \glqq Children Of A Lesser Clod\grqq\ ist eine Anspielung auf den Marlee Maltin Film \glqq Children Of A Lesser God\grqq\ von 1986.
	  \item Der Balletttrainer Lugash ist zu sehen, bei dem Lisa in der Folge \glqq L.S. -- Meisterin des Doppellebens\grqq\ (siehe \ref{DABF15}) Ballettunterricht nimmt.
	  \item Marge hat einen Onkel namens Lou.
\end{itemize}
}

\subsection{Drei unglaubliche Geschichten}
Die Simpsons haben bei einem Preisausschreiben mal wieder eine Reise gewonnen, aber da Homer die Flughafengebühr (fünf Dollar) nicht bezahlen will, reisen sie mit dem Güterzug nach Delaware. In dem Waggon treffen sie auf einen Landstreicher, der ihnen drei unglaubliche Geschichten erzählt. Obwohl diese zwar zum Teil Jahrhunderte zurückliegen, kommen die Simpsons in allen drei Geschichten vor. So wird ihnen die doch recht ungemütliche Zugfahrt versüßt.

\notiz{
\begin{itemize}
  \item Maggie verreist nicht mit den Simpsons.
  \item Anstatt des Flugzeuges nehmen die Simpsons einen Zug der \glqq Union Pacific\grqq .
\end{itemize}
}

\section{Staffel 13}
 	
\subsection{Ich bin bei dir, mein Sohn}\label{CABF22}
Nach einer Spritztour mit Chief Wiggums Wagen müssen Bart und Milhouse vor das Jugendgericht. Milhouse wird freigesprochen. Kurz bevor Bart freigesprochen wird, geht Richter Snyder in Urlaub und die neue Richterin Constance Harm übernimmt den Fall. Sie erkennt, dass Homer seine Aufsichtspflicht vernachlässigt hat. Zur Strafe werden er und Bart mit einem Seil aneinandergefesselt, damit dies so schnell nicht wieder vorkommt. Als Marge schließlich das Seil durchschneidet, werden Marge und Homer mit Zwangsblöcken bestraft.

\notiz{
\begin{itemize}
  \item Bill Clinton ist der neue Briefträger der Simpsons.
  \item Laut Moe dürfen Kinder nicht in seine Bar, weil einmal die Bush-Töchter dort waren.
  \item Während der ganzen Folge wechseln Bart, Homer und später Marge die Kleidung, obwohl dies durch ihre Bestrafung eigentlich nicht möglich sein sollte.
  \item Die Richterin lebt auf einem Hausboot und hat die Adresse \glqq 1 Ocean View Drive\grqq .
  \item Bart landet in dieser Folge seinen ersten Homerun.
\end{itemize}
}

\subsection{Homer und Moe St. Cool}
Moe ist deprimiert, weil seine Gäste nur langweile Lügengeschichten erzählen und deshalb entscheidet er sich, wieder an seine alte Universität zurückzukehren. Während seiner Abwesenheit führt Homer die Bar. Moe erhält von seinem alten Professor den Rat, seine Kneipe einer Generalsanierung zu unterziehen. Tatsächlich macht Moe aus dem heruntergekommenen Schuppen eine exklusive Bar. Homer und seine Freunde fühlen sich dort nicht länger wohl und verlegen ihre Trinkgelage in Homers Garage.

\notiz{
\begin{itemize}
  \item Homer darf Moes Bar führen, weil er beim Pinkelwettbewerb gegen Lenny, Carl und Barney gewonnen hat.
  \item Der Name der Universität lautet \glqq Swingmore University\index{Swingmore Universität}\grqq .
  \item Das Ziffernblatt der alten College-Uhr besteht nur aus Fünfen. Auf der Uhr steht \glqq No Drinking before 5:00\grqq\ (Nicht trinken vor fünf).
  \item Die Frage an die Collegestudenten lautet: \glqq Wie viel Grenadin gehört in einen Cosmopoliten?\grqq\ Die Antwort lautet: \glqq Keiner, weil da Preiselbeersaft hinein kommt.\grqq
  \item Der Name des Türstehers bei Moes neuer Kneipe lautet Cecil\index{Cecil}.
  \item R.E.M. spielen das Lied \glqq It's The End Of The World As We Know It\grqq\ in Homers Garage.
  \item Filmzitat: Die Szene, in der Homer, Barney, Carl und Lenny auf der Theke tanzen, ist eine Anspielung auf den Film \glqq Coyote Ugly\grqq .
\end{itemize}
}

\subsection{Gloria, die wahre Liebe}\label{CABF18}
Homer hat einen neuen Job: Er schreibt Zukunftsprognosen auf Zettelchen, die in Glückskeksen versteckt werden. Mr. Burns lässt sich von einer solchen Vorhersage überzeugen, dass ihm die große Liebe bevorsteht. Mit Gloria\index{Gloria} hat er seine neue Herzdame gefunden. Dabei hat er die Rechnung ohne Snake -- Glorias soeben aus dem Gefängnis entflohenem Ex-Freund -- gemacht.

\notiz{
\begin{itemize}
  \item Anspielung: In Chinatown gibt es ein Geschäft, das \glqq Toys \grq L\grq\ Us\grqq\ heißt, eine Anspielung auf die Spielwarenkette \glqq Toys \grq R\grq\ Us\grqq .
  \item Gloria sagt, Mr. Burns sei 104 Jahre alt.
  \item Auf Snakes Postkasten steht \glqq Snake -- aka Jailbird\grqq\ (Snake -- auch bekannt als Knastbruder).
  \item Auf der Treppe zur Polizeistation ist ein El Barto Graffiti zu lesen.
  \item Mr. Pennybags (Figur aus dem Monopoly-Spiel) verschwindet mit der Frau, mit der Mr. Burns geflirtet hat. Mr. Pennybags war bereits in der Folge \glqq Die japanische Horror-Spiel-Show\grqq\ (siehe \ref{AABF20}) zu sehen.
  \item Chief Wiggum gibt an, dass er eine Schwester hat.
  \item Diese Episode ist dem Ex-Beatles George Harrison gewidmet.
\end{itemize}
}

\subsection{Aus dunklen Zeiten}\label{CABF21}
Als sich Homer in einem Restaurant hypnotisieren lässt, setzt dies eine traumatische Kindheitserinnerung frei. Das traumatische Ereignis muss verarbeitet werden, da ist sich die Familie einig. Und so kommt es, dass die Simpsons an den Ort des Verbrechens zurückkehren, um dort mithilfe von Chief Wiggum nach Spuren zu suchen. Ein Abwasserrohr birgt schließlich den entscheidenden Hinweis.

\notiz{
\begin{itemize}
  \item Die Marke der Papierhandtücher lautet Burly\index{Burly}.
  \item Der Hypnotiseur heißt Mesmerino\index{Mesmerino}.
  \item Carls Definition des Internets: Das ist ein Stoff, den man in eine Badehose einnäht.
  \item Der Vater von Smithers ist bei einem Zwischenfall im Atomkraftwerk ums Leben gekommen. Mr. Burns erzählte Waylon junior, dass sein Vater von einem Stamm wilder Frauen im Amazonas getötet wurde.
  \item Homer erinnert sich an den Sprung über die Teufelsschlucht aus der Episode \glqq Der Teufelssprung\grqq\ (siehe \ref{7F06}).
  \item Fehler I: Im Pimento-House schwenkt während Homers Hypnose die Kamera kurz zum Tisch der Simpsons. Homers Sessel ist verschwunden, kurz darauf ist er aber wieder da.
  \item Fehler II: Als Moe in der Rückblende auf dem Stein steht, hat er Pickel, als er vom Stein herunter springt, sind sie aber weg.
\end{itemize}
}

\subsection{Allein ihr fehlt der Glaube}\label{DABF02}
Bart und Homer zerstören mit einer selbst gebauten Rakete Springfields Kirche. Burns bietet finanzielle Hilfe beim Wiederaufbau an, wenn er die Kirche anschließend wie ein Gewinn bringendes Unternehmen führen kann. Lisa ist schockiert. Offen für neue religiöse Richtungen trifft sie auf Richard Gere, der ihr Interesse für den Buddhismus weckt.

\notiz{
\begin{itemize}
  \item Homers hat die 57843654343410709 als Kreditkartennummer in dieser Episode.
  \item Personen im Kirchenvorstand: Reverend Lovejoy, Helen Lovejoy, Jasper, Ned Flanders, Marge Simpson, Agnes Skinner und Kearney.
	\item Carl und Lenny sind Buddhisten.
\end{itemize}
}

\subsection{Familienkrawall -- Maggie verhaftet}
Ned und Homer werden von ihrer Vergangenheit eingeholt. Die zwei Frauen, die sie in der Episode \glqq Wir fahr'n nach\dots Vegas\grqq\ (siehe \ref{AABF06}) im angeheiterten Zustand während eines Las Vegas Ausflugs geheiratet hatten, stehen plötzlich vor ihrer Tür. Homer geht vor Gericht und bemüht sich um eine Annullierung der Ehe. Sein Antrag wird abgelehnt. Er muss sich nun auch um seine Zweitfrau kümmern. Um das neue Familienmitglied wieder loszuwerden, lässt sich Marge etwas einfallen. Sie macht die Frau betrunken und verheiratet sie mit Grandpa.

\notiz{
\begin{itemize}
  \item Mitglieder der republikanischen Partei Springfields: Mr. Burns, Mr. Smithers, Krusty, Dracula, Rainier Wolfcastle, Ralph Nader\index{Nader!Ralph} und der reiche Texaner.
  \item Der Sozialarbeiter heißt Gabriel.
  \item Als sich die Familie das Video von Homer und Neds Hochzeit ansieht, fällt Maggie der Schnuller aus dem Mund.
\end{itemize}
}

\subsection{Sein Kiefer ist verdrahtet}
Homer bricht sich seinen Kiefer. Zum Stillschweigen verdammt, wird aus ihm ein guter Zuhörer und respektiertes Mitglied der Gesellschaft. Dies ändert sich nicht einmal, als Homers Verletzung ausgeheilt ist. Zunächst begrüßt Marge diese Entwicklung. Bald jedoch vermisst sie ihren rüpelhaften Mann. Aus lauter Frust nimmt Marge mit dem Familienwagen an einem Crash-Car-Rennen teil.

\notiz{
\begin{itemize}
  \item Homers Lieblingszeitschrift ist \glqq Pie Pride\index{Pie Pride}\grqq\ (Kuchenstolz).
  \item Banner bei der Parade: \glqq Gay Pride Parade -- Formerly Springfield Heritage Day\grqq\ (Stolze Schwulenparade -- ehemals Springfielder Kulturtag).
  \item Homer und Marge treten in der Talkshow \glqq Afternoon YAK\index{Afternoon YAK}\grqq\ auf. Nach ihnen kommen Milhouse und seine Mutter auf die Bühne.
  \item Fernsehzitat: Als Bart die Bierdose für Homer ausdrückt, ertönt die Melodie aus Popeye. Homer trinkt das Bier, wie Popeye seinen Spinat isst. Marge schreit aus dem Auto um Hilfe, wie sonst Olivia um ihren Popeye schreit.
\end{itemize}
}

\subsection{Die süßsaure Marge}\label{DABF03}
Homer will eigentlich mit der höchsten Menschenpyramide in das Duff Buch der Rekorde, stattdessen erhält Springfield als \glqq dickste Stadt der Welt\grqq\ einen Eintrag in diesem Buch. Marge erwirkt daraufhin ein allgemeines Zuckerverbot vor Gericht. Die entsprechenden Lebensmittel werden aus den Regalen der Supermärkte verbannt. Homer und einige seiner Freunde beschließen, sich als Schmugglerbande zu betätigen und Zucker von der Insel San Glucose\index{San Glucose} nach Springfield einzuführen.

\notiz{
\begin{itemize}
  \item Filmzitat: Die Szene, in der Homer die Menschenpyramide nach oben klettert, erinnert an \glqq Mission Impossible\grqq .
  \item Das Gesamtgewicht der Stadtbewohner beläuft sich auf 64.152 Pfund.
  \item Cletus unterschreibt mit dem Nachnamen \glqq Spuckler\grqq , obwohl sein eigentlicher Nachname \glqq Del Roy\grqq\ lautet.
  \item Marge engagiert Gil als Anwalt.
  \item Der Name des Schiffes lautet \glqq Gone Fission\index{Gone Fission}\grqq .
  \item Mitglieder der Zuckerschmuggler: Mr. Burns, Graf Dracula, Apu, Garth Motherloving\index{Motherloving!Garth}, Homer und Bart.
  \item Diese Folge ist dem Andenken an Ron Taylor gewidmet. Dieser sprach im englischen Original die Rolle des Zahnfleischbluter Murphy \cite{WikipediaRonTaylor}.
  \item Es hat den Anschein, als grenzt der Bundesstaat, in dem Springfield liegt, im Süden an Tennessee.
\end{itemize}
}

\subsection{Ein halbanständiges Angebot}
Patty und Selma erfahren, dass Marges Jugendliebe Artie Ziff\index{Ziff!Artie} der fünftreichste Mann Amerikas ist. Sie nehmen Kontakt mit ihm auf. Kurz darauf erscheint Artie in Springfield und macht Marge ein unmoralisches Angebot. Er bietet ihr eine Million Dollar für ein gemeinsames Wochenende. Da Homer dringend eine kostspielige Nasenoperation braucht, um von seinem Schnarchen kuriert zu werden, nimmt Marge an.

\notiz{
\begin{itemize}
  \item Filmzitat: Die ganze Episode ist eine Anspielung auf den Film \glqq Ein unmoralisches Angebot\grqq .
  \item Der Comicbuchverkäufer schläft mit einer Puppe von Jaja Bings aus Star Wars.
  \item Patty und Selma wohnen in den \glqq Spinster City Apartments\grqq\ im Appartement mit der Nummer 1599.
  \item Patty, Selma und Marge schauen die Serie \glqq Sex in New York\grqq , eine Anspielung auf \glqq Sex and the City\grqq\ an.
  \item Als Patty das E-Mail versendet, geht es über Leitungen zu einem Kasten auf dem \glqq Cisco Systems\grqq\ steht. Cisco ist der weltweit größte Hersteller von Routern für das Internet.
  \item Auf Arties Abschlussball sind unter anderem folgende Personen: Professor Frink, Chief Wiggum, Hausmeister Willie, Disco Stu, Rektor Skinner, Kirk Van Houten, Edna Krabappel, Sarah Wiggum, Otto, Ned Flanders und Luann Van Houten.
  \item Homer ist im Video mit zwei Stofftieren zu sehen, Puppe Lustikus\index{Lustikus} (zuletzt zu sehen in  \glqq Die böse Puppe Lustikus\grqq , \ref{BABF07}) und einem Dinosaurier.
  \item In West-Springfield befindet sich der Mt. Carlmore\index{Mt. Carlmore}.
  \item Laut Lisa ist West Springfield dreimal so groß wie Texas.
  \item Die Taxifahrt von Arties Anwesen zu den Simpsons kostet 912 \$.
\end{itemize}
}

\subsection{Nach Kanada der Liebe wegen}\label{DABF06}
Bart lernt Greta\index{Greta}, die Tochter von Rainier Wolfcastle, kennen. Obwohl sich Bart als schwer von Begriff erweist, findet Greta Gefallen an ihm. Die beiden verbringen viel Zeit miteinander. Als sich Bart von ihr trennt, geht Greta aus Rache mit Milhouse aus. Bart ist gekränkt. Um sie zurückzugewinnen, folgt Bart Greta nach Kanada, wo der neue Film von Rainier Wolfcastle gedreht wird.

\notiz{
\begin{itemize}
  \item Auf der Periodentafel in der Privatschule sind 250 Elemente aufgeführt, auf der in der Springfielder Grundschule lediglich 16 (nur die Latanide).
  \item Anspielung: Marge verrät Wolfgang Puck wie sie ihre Puffreisschnitten macht. Als dieser dann abhaut, läuft er zu seinem \glqq Puckmobil\grqq . Im Hintergrund ertönt die Melodie von Batman.
  \item Maggie spielt mit dem Hasen Bongo\index{Bongo} aus Matt Groenings Comic-Strip \glqq Life In Hell\grqq .
  \item Gäste bei Rektor Skinners Auftritt: Bart, Milhouse, Timothy und Helen Lovejoy, Mr. Smithers, Dr. Nick Riviera, Dr. Hibbert und Agnes Skinner. Der Moderator ist Krusty. Vor Rektor Skinner tritt Captain McCallister auf.
  \item Rainier Wolfcastles neuer Film heißt \glqq Undercover Nerd\grqq\ (Geheimdiensttrottel).
  \item Milhouse ist Barts bester Freund aus \glqq geographischer Bequemlichkeit\grqq .
\end{itemize}
}


\subsection{Bart und sein Westernheld}
Bart trifft Buck McCoy\index{McCoy!Buck}, einen alternden Helden zahlreicher Wild-West-Filme. Die beiden freunden sich an und Bart kann Buck sogar zu einem Auftritt bei Krusty dem Clown überreden. Nach Jahren der Zurückgezogenheit scheut der Schauspieler den neuerlichen Schritt ins Rampenlicht. Betrunken erscheint er zur Show und blamiert sich bis auf die Knochen. Als eine Bande Räuber eine Bank überfällt, sieht Homer eine Möglichkeit, dem Idol seines Sohnes zu neuerlichem Ruhm zu verhelfen.

\notiz{
\begin{itemize}
  \item Buck McCoy ist 76 Jahre alt.
  \item Buck McCoy machte für \glqq Drunken Cowboy\grqq -Whiskey Werbung.
  \item Vor der Bank steht ein grüner VW Beetle.
\end{itemize}
}


\subsection{Hex and the City}\label{CABF19}
\begin{itemize}
	\item \textbf{Hex and the City}\\ Die erste Geschichte bringt die Familie nach \glqq Ethnictown\grqq , wo Homer eine Zigeunerin ärgert, die dann einen Fluch ausspricht, der allen in seiner Umgebung Unglück bringt, außer ihm selbst. Marge wächst ein Bart, Lisa verwandelt sich in ein halbes Pferd und Barts Hals wird gummiartig. Um den Fluch zu brechen, nimmt Homer einen Gnom gefangen, aber der Gnom verliebt sich in die Zigeunerin und heiratet sie schlussendlich.
	\item \textbf{House of Whacks (Haus der Hiebe)}\\ In der zweiten unheimlichen Geschichte beschließt Marge das Ultrahouse 3000 zu kaufen. Ein futuristisches Haus (mit der Stimme von Pierce Brosnan) mit gemeingefährlichen Tendenzen, das jeden Aspekt der Familie kontrolliert. Als sich das Haus in Marge verliebt, versucht es Homer zu töten, um seine Stelle einzunehmen.
	\item \textbf{Wiz Kids (Zauberlehrlinge)}\\ In der letzten Halloween-Geschichte finden sich die Simpsons in einer Harry Potter Existenz wieder, in der die Kinder auf einer Zauberschule die Hexerei erlernen. Montymort\index{Montymort} und sein Helfer Slithers\index{Slithers}, versuchen Lisas Zauberkraft zu stehlen, aber sie haben nicht mit Bart gerechnet.
\end{itemize}
  
  
\notiz{
\begin{itemize}
  \item Kang und Kodos sind Gäste bei der Hochzeit. Der Pfarrer ist Yoda aus Star Wars.
  \item Folgende Stimmen stehen beim Ultrahouse zur Verfügung: \glqq Standard\grqq , \glqq Matthew Perry\grqq , \glqq Dennis Miller\grqq\ und \glqq Pierce Brosnan\grqq .
  \item Als Bart sagt, dass Haus habe auch die Stimme von 007, glaubt Marge, es sei George Lazenby.
  \item Filmzitat: Die Geschichte \glqq Zauberlehrlinge\grqq\ ist eine Anspielung auf die \glqq Har\-ry Pot\-ter\grqq\ Bücher bzw. Filme.
  \item Die Springfielder Grundschule heißt hier \glqq Springwart's -- School of Magicry\grqq\ (Springwarts -- Schule für Zauberei). Eine weitere Anspielung auf Harry Potter.
\end{itemize}
}

\subsection{Drei uralte Geschichten}
Homer liest drei Geschichten aus einem alten Kinderbuch vor. In der ersten ist er selbst der Held Odysseus, der dem König ein trojanisches Pferd liefert. In der zweiten Geschichte führt Lisa als Jeanne D'Arc die Franzosen in den Krieg gegen England. In der letzten gibt Bart den Hamlet und soll den Tod seines Vaters Homer rächen.

\notiz{
\begin{itemize}
  \item Darsteller der ersten Geschichte: Odysseus = Homer; Priamos (Kö\-nig von Troja) = Ned Flanders; Griechische Soldaten = Lenny, Carl, Moe, Apu und Professor Frink; Penelope = Marge; Sohn von Odysseus = Bart; Zeus = Quimby; Poseidon = Kapitän McCallister; Dionysos (griechischer Gott des Weines, des Vergnügens und des Theaters) = Barney; die Sirenen = Patty und Selma; mögliche Nachfolger von Odysseus = Willie, Krusty, Kent Brockman, Rektor Skinner, Disco Stu, Kirk Van Houten, Mr. Burns und Sideshow Mel; Helena von Troja = Agnes Skinner.
  \item Darsteller der zweiten Geschichte: Jeanne D'Arc = Lisa; Vater von Jeanne D'Arc = Homer; Mutter von Jeanne D'Arc = Marge; Bartro = Bart; Führer der französischen Armee = Wiggum; Adjutant des französischen Königs = Quimby; Französischer König = Milhouse; Untertane = Martin; Hofnarr = Krusty; englischer Soldat = Willie; englischer Richter = Reverend Lovejoy; Adjutant des Richters = Ned Flanders; Lukenöffner = Hans Maulwurf; Geschworener = Lenny.
  \item Darsteller der dritten Geschichte: Hamlet = Bart; Hamlets Vater (König von Dänemark) = Homer; Mutter von Hamlet (Königin von Dänemark) = Marge; Onkel Claudius = Moe; Hofnarr = Krusty; Adjutanten des Hofnarren = Mr. Teeny und Sideshow Mel; Orphelia = Lisa; Tulonius = Wiggum; Laertes = Ralph; Adjutanten von Moe = Lenny und Carl.
  \item Filmzitat: Als Homer als Geist durch die Wand des Schlosses fliegt, hinterlässt er grünen Schleim auf der Wand, wie Slimer in \glqq Ghostbusters\grqq .
  \item Aufschrift auf der Flasche Ohrengift: \glqq Ear Poison -- Do not get in eyes\grqq\ (Ohrengift -- Nicht in die Augen bringen).
  \item Am Ende der Episode tanzen die Simpsons zur Melodie von Ghostbusters (Ray Parker, Jr.).
\end{itemize}
}

\subsection{Abraham und Zelda}\label{DABF09}
Abraham Simpson verliebt sich in eine Frau namens Zelda\index{Zelda}. Um sie zu beeindrucken, macht er den Führerschein, borgt sich Homers Wagen und führt seine neue Flamme aus. Bald kommt es zu einem Konflikt mit einigen Rentnern. Der Zwist wird in einem mörderischen Autorennen geklärt. Obwohl Grandpa gewinnt, trifft sich Zelda mit seinem Rivalen zum Essen.

\notiz{
\begin{itemize}
  \item Abes Führerscheindaten: Nr.: S6890145; Abraham Simpson; Springfield Retirement Castle; Springfield, DO (oder DD), USA.
  \item Als Grampa und Bart mit Marges Auto nach Branson fahren, läuft im Hintergrund das Lied \glqq Born To Be Wild\grqq\ von Steppenwolf.
  \item Laut Aussage von Lisa ist Branson über 1000 Meilen von Springfield entfernt.
  \item In der Show in Branson\index{Branson} tritt auch Mr. T auf.
\end{itemize}
}

\subsection{Das ist alles nur Lisas Schuld}\label{DABF10}
Eine Telefonrechnung über 400 Dollar flattert bei den Simpsons ins Haus. Lisa gesteht, einen Anruf nach Brasilien getätigt zu haben. Sie ist auf der Suche nach dem Waisenkind Ronaldo\index{Ronaldo}, zu dem sie vor einiger Zeit den Kontakt verloren hat. Die Simpsons entschließen sich zu einer Reise nach Südamerika, um Lisa bei ihrer Suche zu helfen. Dabei gerät Homer in die Gewalt von Kidnappern, die für seine Freilassung 50.000 Dollar fordern.

\notiz{
\begin{itemize}
  \item Homer hat sich nach Brasilien das Buch \glqq How To Loot Brazil\grqq\ (Wie man Brasilien ausbeutet) mitgenommen.
  \item Moe braucht ebenfalls 50.000 Dollar.
  \item Gar nicht komisch fand man diese Episode in Brasilien. Stein des Anstoßes war nicht etwa Homers Entführung, sondern die Ratten und Affen, welche die Zeichner durchs Bild huschen lassen. Produzent James L. Brooks entschuldigte sich kurz nach der Erstausstrahlung offiziell.
\end{itemize}}



\subsection{Homer einmal ganz woanders}
Bei einer Vogelattacke erleidet Homer schmerzhafte Augenverletzungen. Sein Arzt Dr. Hibbert verschreibt ihm zur Linderung seiner Beschwerden Marihuana. Homer wandelt nun -- völlig legal -- high durchs Leben. In diesem Zustand findet er sogar Mr. Burns Witze lustig. Dieser honoriert Homers Verhalten und ernennt ihn zum Vizechef seiner Firma. Als Marihuana auf Rezept verboten werden soll, sieht Homer auch seine Karriere den Bach runtergehen.

\notiz{
\begin{itemize}
  \item Bevor Homer den ersten Joint raucht, ist das Lied \glqq Incense and Peppermints\grqq\ von Strawberry Alarm Clock zu hören. Zuletzt gehört in der Folge \glqq Homer ist ein toller Hippie\grqq\ (siehe \ref{AABF02}).
  \item Mr. Burns muss 60 Millionen Dollar auftreiben, um das Kraftwerk zu retten.
  \item Homers neues Auto ist ein schwarzes Mercedes-Cabrio.
  \item Beim Konzert von Phish sieht man Seth\index{Seth} und Munchie\index{Munchie} bei Otto stehen. Die beiden stammen ebenfalls aus der Folge \glqq Homer ist ein toller Hippie\grqq\ (siehe \ref{AABF02}).
  \item Das finale Lied von Phish ist die Titelmelodie der Simpsons. 
  \item In Australien wurde diese Episode erst nach 22:00 Uhr gesendet; sogar nach den \glqq Osbournes\grqq .
  \item Marge kocht Gengemüse von Union Carbide\index{Union Carbide}, einem US-Chemie-Konzern der 1984 wegen einer der größten Katastrophen in der chemischen Industrie in die Schlagzeilen geriet, bei dem in Indien tausende Menschen starben und hunderttausende verletzt wurden.
  \item Marge dekoriert die Vogelscheuche mit Kleidung aus alten Episoden: Lisas Shirt aus \glqq Lisa auf dem Eise\grqq\ (siehe \ref{2F05}), Barts Jockeyhose aus \glqq Ein Pferd für die Familie\grqq\ (siehe \ref{BABF09}), die Laterne aus den Horrorsendungen III, IX und XII und Grampas Hut aus \glqq Wer hat Grampas Hut erschossen\grqq\ (diese Folge gibt es nicht).
  \item Bill Clintons Gage für den Auftritt bei der Aktionärsversammlung beträgt 200.000 US Dollar.
  \item Mr. Burns Scherz \glqq What's better, hard working or hardly working?\grqq\ geht in der deutschen Übersetzung \glqq Was ist besser, hart arbeiten oder kaum arbeiten?\grqq\ verloren.
  \item Fehler: Am Ende der Episode gibt Homer sich und Mr. Smithers Ohrfeigen mit der an einer Schnur befindlichen rechten Hand von Mr. Burns. Als Mr. Burns sich selbst ohrfeigen will, benutzt er die linke Hand, die Homer allerdings nicht steuert, da er nur ein Seil in der Hand hat. 
\end{itemize}
}

\subsection{Forrest Plump und die Clip Show}
Im Stil von Forrest Gump sitzt Homer auf einer Parkbank bereit, seine Lebensgeschichte Revue passieren zu lassen. Doch Homer ist nicht der einzige, der etwas zu erzählen hat. Auch Bart, Lisa, Reverend Lovejoy und Ned Flanders geben Anekdoten zum Besten.

\notiz{
\begin{itemize}
  \item Filmzitate: Der Anfang der Episode ist eine Nachahmung des Films \glqq Forrest Gump\grqq . Moe ahmt \glqq Austin Powers\grqq\ nach und Dr. Hibbert ist als \glqq Darth Vader\grqq\ aus \glqq Star Wars\grqq\ zu sehen.
  \item Gäste bei der Veranstaltung zu Ehren von Homer: Krusty, Dr. Hibbert und seine Frau Beatrice, Reverend Lovejoy und Helen Lovejoy, Clancy und Sarah Wiggum, Mrs. Krabappel, Rektor Skinner, Agnes Skinner, Sideshow Mel, Mr. Burns, Jasper, Hans Maulwurf, Nelson, Kent Brockman, Lenny, Carl, Mr. Smithers, Fat Tony, Apu, Manjula, Bürgermeister Quimby, Ned Flanders, Moe und Grandpa.
  \item Running Gag in dieser Episode: Zuerst der Lacher von Dr. Hibbert, dann Nelsons \glqq Haha\grqq\ und zuletzt Mr. Burns \glqq Ausgezeichnet\grqq.
  \item Als Mr. Burns zum Podium tritt, ertönt die Melodie des Imperators aus \glqq Star Wars\grqq .
\end{itemize}
}

\subsection{Der rasende Wüterich}\label{DABF13}
An der Springfielder Grundschule finden sogenannte Karrieretage statt. An einem dieser Tage spricht der Schöpfer von \glqq Danger Dog\index{Danger Dog}\grqq . Daraufhin beschließt Bart, Comic-Zeichner zu werden und erfindet seinen eigenen Zeichentrickcharakter: \glqq Angry Dad\grqq . Vorlage zu dieser Figur ist Homer, der Bart mit seinen ständigen Wutausbrüchen genügend Stoff für immer neue Geschichten liefert. Der Cartoon wird zum durchschlagenden Erfolg im Internet und Homer Opfer permanenter Provokation.

\notiz{
\begin{itemize}
  \item Barts Comics werden bei \glqq BetterThanTv.com\grqq\ (BesserAlsFernsehen.com) als Internetzeichentrickserie produziert.
  \item Bart besitzt zum Schluss 52 Millionen Aktien von \glqq BetterThanTv.com\grqq .
  \item Filmzitat: Die Szene, in der Homer wütend und komplett in grün durch Spring\-field rennt, ist eine Anspielung auf \glqq Hulk\grqq\index{Hulk}.
  \item Stan Lee steckt ein X-Man-Comic vor ein Superman-Comic. Stan Lees Commics erscheinen bei Marvel, während hingegen die Superman-Comics beim Erzrivalen DC erscheinen.
  \item Fehler: Als die Simpsons im Krankenhaus sind, ist Maggies Schleife erst rot und dann blau.
\end{itemize}
}

\subsection{Die Apu und Manjula Krise}\label{DABF14}
Nach einer Bürgerkriegsaufführung will Homer ein leeres und verbeultes Bierfass zurückgeben und erwischt Apu dabei, wie er sich mit der Squishee-Lieferantin vergnügt. Er stellt seinen Freund zur Rede. Kurze Zeit später weiß auch Manjula Bescheid und trennt sich von ihrem Mann. Apu bittet um Vergebung, stattdessen erhält er aber die Scheidungspapiere. Als die Simpsons davon erfahren, wollen sie den Freunden helfen, ihre Ehe zu retten. Zu diesem Zweck setzt Marge zusammen mit der betrogenen Ehefrau eine Liste an Forderungen auf, die Apu erfüllen muss, damit es zu einer Versöhnung kommt.

\notiz{
\begin{itemize}
  \item Filmzitat: Prof. Frink mit seiner Riesenspinne ist eine Anspielung auf den Schurken im Film \glqq Wild, Wild West\grqq .
  \item Apus Tafel der Reinkarnation: Tiger -- Schlange -- Tölpel -- Ziege mit einem Hut -- Apu -- Bandwurm -- Assistent von Lorne Michaels\index{Michaels!Lorne}.
  \item Apu wohnt im gleichen Wohnheim, in dem auch Kirk Van Houten untergekommen ist. Seine Appartement hat die Nummer 5B.
\end{itemize}
}
	
\subsection{L.S. -- Meisterin des Doppellebens}\label{DABF15}
Lisa muss ihre Leistungen im Turnen verbessern und erhält Stunden bei dem Privatlehrer Lugash\index{Lugash}. Dort trifft sie auf Schülerinnen, die wesentlich älter sind als sie. Lisa verschweigt, dass sie noch in die Grundschule geht und wird sogleich freundlich im Kreise der College-Studenten aufgenommen. Doch es kommt, wie es kommen muss. Lisas Doppelleben bleibt nicht lange geheim. Bart wird inzwischen von einer chinesischen Mücke gestochen. Die Zeit der Quarantäne muss er in einer Luftblase verbringen.

\notiz{
\begin{itemize}
  \item Beim Turnunterricht ist im Hintergrund Francine\index{Francine} aus der Folge \glqq Lisa knackt den Rowdy-Code\grqq\ (siehe \ref{CABF11}) zu sehen.
  \item Bart isst bei Krusty Burger ein \glqq lachendes Mahl\grqq , eine Anspielung auf das \glqq Happy Meal\grqq\ von McDonalds. Grandpa isst ein \glqq Nostalgie Mahl\grqq .
  \item Auf Skinner Parkplakette steht \glqq Seymour Skinner; 1953 -- 2010\grqq .
\end{itemize}
}

\subsection{Am Anfang war die Schreiraupe}\label{DABF16}
Homer muss 200 Stunden Sozialdienst leisten, weil er eine Schreiraupe verletzt hat. Während er \glqq Essen auf Rädern\grqq\ ausliefert, lernt er eine betagte, nette Dame kennen. Als sie eines unnatürlichen Todes stirbt und Homer und Marge ihr Vermögen vererbt, geraten die beiden sofort unter Mordverdacht. Sie werden verhaftet. Als die Lage aussichtslos scheint, nimmt der Wahnsinn eine unvermutete Wende. Denn sie waren nur ahnungslose Marionetten in einer Reality Show.

\notiz{
\begin{itemize}
  \item Homer wird wegen \glqq versuchten Inseketenmordes\grqq\ und \glqq Raupenvergewaltigung\grqq\ verurteilt.
  \item Bart und Lisa kommen zur Familie Del Roy. Ihre neue Namen lauten \glqq Dinges Quadfour Jr.\grqq\ (Lisa) und \glqq Pamela E. Lee\grqq\ (Bart).
  \item Schlagzeile im Springfield Shopper: \glqq Ho. J. Simpson -- Trial Starts Today\grqq\ (Ho. J. Simpson -- Prozessbeginn). Eine Anspielung auf den berühmten Prozess in Amerika gegen O. J. Simpson.
  \item Der Anwalt der Simpsons ist Gil. Die Geschworenen sind unter anderem Helen Lovejoy und Ruth Powers.
  \item Homers Gefangenennummer: 5191.
  \item Marges Gefangenennummer: 6711.
  \item Filmzitat: Die Szene, als Homer sich zum elektrischen Stuhl begibt, ist eine Anspielung auf den Film \glqq The Green Mile\grqq .
  \item Zuschauer bei Homers Hinrichtung: Ned Flanders, Marge, Moe (mit Fotoapparat), Agnes Skinner, Rektor Skinner, Lenny, Patty, Selma, Grandpa, Apu, Manjula, George Bush sen. und die Schreiraupe.
  \item Als Eddie gefragt wird, wie sein Nachname und der Nachname Lous lautet, antwortet er, beide hätten -- genau wie Cher -- keinen Familiennamen.
\end{itemize}
}

\subsection{Sicherheitsdienst Springshield}\label{DABF17}
Der enorme Stromverbrauch in Folge einer Hitzewelle sorgt für einen Stromausfall in Springfield. Sofort gehen Plünderer an ihr Werk. Die örtliche Polizei ist machtlos. Zur Unterstützung der Exekutive gründet Homer einen Sicherheitsdienst. Fat Tony und seine Leute sind über diese Entwicklung gar nicht begeistert und fordern Homer auf, entweder die Stadt zu verlassen, oder sich einem Duell zu stellen. 

\notiz{
\begin{itemize}
  \item Telefonnummer des Sicherheitsdienstes: 636-555-3472. Die 636 in der Telefonnummer deutet auf den Bundesstaat Missouri hin.
  \item In der Episode \glqq Die Geschichte der zwei Springfields\grqq\ (siehe \ref{BABF20}) lag das Haus der Simpsons im Vorwahlbereich der 939 nicht der 636.
  \item Dem Sicherheitsdienst gehören neben Homer noch Lenny und Carl an.
  \item Gemäß der Ortstafel Springfields hat die Stadt 30.720 Einwohner.
  \item Krusty ist zu sehen, als er den \glqq The 99 Cent Porno Store\grqq\ verlässt.
  \item Captain McCallister hat links ein Glasauge.
\end{itemize}
}

\section{Staffel 14}

\subsection{Marge -- oben ohne}\label{DABF18}
Marge hat sich für eine Schönheits-OP entschieden, um sich das Fett absaugen zu lassen, aber stattdessen implantiert ihr der Chirurg -- zur Begeisterung Homers -- einen riesigen Busen. Schließlich arbeitet sie vorübergehend sogar als Modell.
Krusty soll indes mit einer Elefantennummer seinen Ruf bei den Kindern wiederherstellen. Doch das Kunststück läuft schief: Homer, Bart und Milhouse landen in Stampfis\index{Stampfi} Maul. Um die drei zu befreien, muss Krusty das Losungswort sagen und das fällt ihm erst ein, als er Marges entblößten Busen sieht.

\notiz{
\begin{itemize}
  \item Homer nennt sich \glqq El Homo\index{El Homo}\grqq\ in Anspielung auf Barts \glqq El Barto\grqq .
  \item Der Elefant Stampfi war bereits in der Folge \glqq Bart gewinnt Elefant!\grqq\ (siehe \ref{1F15}) zu sehen.
\end{itemize}
}

\subsection{Schickt die Klone rein}\label{DABF19}
In Springfield geht wieder einmal das Grauen um, was sich in drei schaurig schönen Geschichten manifestiert:
\begin{itemize}
	\item \textbf{Send in the Clones}\\ Als Homer von einer Zauberhängematte hundertfach geklont wird, fallen seine Duplikate wie eine Plage über die Stadt herein.
	\item \textbf{The Fright to Creep and Scare Harms}\\ Als Lisa sieht, wie viele Grabsteine von Schussopfern auf dem Springfielder Friedhof sind, beschließt sie, für ein Verbot von Waffen in Springfield zu kämpfen. Als Springfield jedoch der Bitte entspricht, ist die Stadt gegen die Horden von bewaffneten Zombies, die von den Toten zurückkehren, wehrlos.
	\item \textbf{The Island of Dr. Hibbert}\\ Die letzte Story erzählt von Dr. Hibbert, der auf einer einsamen Insel Menschen -- so auch Homer -- in Tiere verwandeln möchte.
\end{itemize}

\notiz{
\begin{itemize}
  \item Einer der Klone sieht aus wie Peter Griffin\index{Griffin!Peter} aus der Fox-Serie \glqq Family Guy\grqq \index{Family Guy}. Ein anderer Klon sieht aus, wie Homer in den ersten Staffeln der Serie gezeichnet war.
  \item Moe zeigt sein Buch \glqq My Troubled Mind\grqq .
  \item Der Bande von William H. Bonney (\glqq Billy The Kid\footnote{Henry McCarty auch William Henry Bonney oder Henry Antrim (geboren vermutlich am 23. November 1859 in New York oder Indiana; gestorben am 14. Juli 1881 in Fort Sumner, New Mexico), besser bekannt als Billy the Kid, war ein Gesetzloser und Mörder und ist heute wohl einer der bekanntesten US-amerikanischen Westernhelden. Billy the Kid wurden bis zu 21 Morde angelastet -- die wirkliche Anzahl ist möglicherweise jedoch kleiner als neun, sicher überliefert sind vier. Er ist eine der legendären Figuren in der Westerngeschichte und wurde von Sheriff Pat Garrett aus dem Hinterhalt getötet.}\grqq ) gehören noch Frank und Jesse James\footnote{Jesse Woodson James (geboren am 5. September 1847 im Clay County, Missouri; gestorben am 3. April 1882 in St. Joseph, Missouri) war ein berühmter Gesetzloser des Wilden Westens.}, Sundance Kid\footnote{Harry Alonzo Longabaugh (geboren 1867 in Pennsylvania; gestorben 1908 in San Vicente, Bolivien) war ein amerikanischer Gesetzloser, bekannt unter dem Namen Sundance Kid \cite{SundanceKid}.} und Kaiser Willhelm an.
  \item Auf einem der Transparente in der zweiten Geschichte steht zu lesen: \glqq Fire Gil Not Guns\grqq\ (\glqq Feuert Gil, nicht Waffen\grqq ).
  \item Literaturzitat: \glqq The Island of Dr. Hibbert\grqq\ basiert auf der Novelle \glqq The Island of Dr. Moreau\grqq\ von H.G. Wells, in der ein verrückter Doktor menschenähnliche Kreaturen erschafft und über diese herrscht.
\end{itemize}
}
  

\subsection{Klassenkampf}\label{DABF20}
Mit dem Kauf einer Satellitenanlage beginnt für Homer und Bart das Fernsehleben und das obwohl in der Schule der große Abschlusstest ansteht, für den bei Bart das Lernen daher ausfällt. Aufgrund miserabler Ergebnisse wird Bart in die dritte Klasse zu Mrs. McConnell\index{McConnell!Audrey} zurückversetzt, während Lisa die zweite Klasse überspringen darf und die beiden drücken somit fortan zusammen die Schulbank. Lisa kommt jedoch gar nicht damit klar, dass Bart plötzlich bessere Noten schreibt als sie und es kommt zu Streitereien. Auf einem Schulausflug nach Capital City\index{Capital City} hat die Klasse das Projekt, eine neue Staatsflagge zu gestalten. Bart sabotiert Lisas Entwurf und demütigt sie vor Gouverneur Bailey. Bart und Lisa brechen in einen Kampf aus und verpassen den Bus zurück nach Springfield.

\notiz{
\begin{itemize}
  \item Fernsehzitat: In Barts Traum ist Bender\index{Bender} aus Futurama zu sehen. Bender ist außerdem in den Folgen \glqq Der beste Missionar aller Zeiten\grqq\ (siehe \ref{BABF11}) und \glqq Future-Drama\grqq\ (siehe \ref{GABF12}) zu sehen.
  \item Der Name der Gouverneurin des Staates, in dem Springfield liegt, ist immer noch Mary Bailey\index{Bailey!Mary}, die in der Episode \glqq Frische Fische mit drei Augen\grqq\ (siehe \ref{7F01}) gegen Mr. Burns die Wahl gewann.
  \item Das Motto von Lisa für die neue Flagge lautet \glqq To Fraternal Love\grqq\ (auf die brüderliche Liebe). Bart macht daraus \glqq Learn To Fart\grqq\ (lerne zu pupsen).
  \item Der Staat, in dem Springfield liegt, gehörte im Bürgerkrieg den Nordstaaten an. 
\end{itemize}
}

\subsection{Der Videobeichtstuhl}
Als Homer durch das Atomkraftwerk spaziert, fällt ein Teil eines Rohres auf seinen Kopf und beschädigt einen Teil seines Gehirns. Mr. Burns befürchtet, dass Homer eine Klage gegen ihn einreichen könnte und bietet ihm die Firmen-VIP-Karten für ein Eishockeyspiel an. Die Simpsons genießen den Luxus, essen Sushi, nehmen heiße Bäder und Massagen. Verärgert über diese Gesellschaftsstufe, geht Lisa hinunter zur Eisfläche, um sich von dort aus das Spiel anzusehen. Sie berät einen russischen Spieler, wie man den Tormann schlagen kann. Dieser schenkt Lisa seinen Schläger, aus dem allerdings Termiten schlüpfen, die das Haus der Simpsons zerstören. Nachdem der Kammerjäger die Termiten vergiftet hat, ist das Haus für sechs Monate unbewohnbar. Da sie nicht wissen, wo sie wohnen sollen, lassen sie sich von einer Reality-Show anheuern, die ihnen ein Haus zur Verfügung stellt. Dort müssen sie allerdings leben wie im Jahre 1895. Doch die ständige Beobachtung durch die Kameras geht den Simpsons richtig auf die Nerven und gemeinsam mit ihren Vorgängern machen sie der Fernsehcrew den Garaus.

\notiz{
\begin{itemize}
  \item Der russische Eishockey-Spieler, der Lisa seinen Schläger schenkt, heißt Kozlov\index{Kozlov}.
  \item Der Name des Produzenten der Reality-Show ist Mitch Hardwell\index{Hardwell!Mitch}.
  \item Nachdem das Haus der Simpsons für sechs Monate unbewohnbar war, hätten sie bei Lenny und dem Comicbuchverkäufer einziehen können.
\end{itemize}
}

\subsection{It's only Rock'n'Roll}\label{DABF22}
Homer wird völlig berauscht heimlich für die Fernsehshow \glqq Taxicab Conversations\grqq\ in einem Taxi gefilmt, wie er sich über sein Familienleben beschwert. Marge, Bart, Lisa und Maggie wenden die Familienkrise ab, indem sie Homer in Mick Jaggers Rock'n'Roll Fantasy Camp schicken, um wieder auf klare Gedanken zu kommen. Homer lebt eine Woche den Traum vom Rockstar, gelehrt von Mick Jagger, Keith Richards, Lenny Kravitz, Elvis Costello, Tom Petty und Brian Setzer. Doch am Ende dieser fantastischen Woche will Homer nicht mehr zurück aus der Welt des Rock'n'Rolls. Daraufhin ladet Mick Jagger Homer zu einem Benefizkonzert ein. Auf der Bühne merkt Homer, dass er kein Star ist und versucht die Band mit einer eigenen musikalischen Nummer zu unterstützen. Mick, Elvis, Keith, Tom, Lenny und Brian versuchen Homer von der Bühne zu bekommen, als ein Aufruhr ausbricht.

\notiz{
\begin{itemize}
	\item Im Abspann der Folge sind die Gaststars bei den Sprechaufnahmen zu sehen.
	\item Disco Stu gesteht einem Taxifahrer, dass er eigentlich Discos hasst.
	\item Im Rock'n'Roll Lager sind neben Homer noch Otto, Barney, Apu, Wiggum, Prof. Frink, Dr. Hibbert, Kirk Van Houten, Gil und Louie\index{Louie} (einer von Fat Tonys Leuten \cite{Simpsonspedia}).
	\item Moe Syzlak erwähnt bei einem Konzert, in welchem er in der ersten Reihe sitzt, dass er das letzte Mal bei seiner eigenen Moon-Hochzeit\index{Moon-Hochzeit} in der ersten Reihe saß.
\end{itemize}
}

\subsection{Und der Mörder ist\dots}\label{EABF01}
Die Simpsons bekommen ein Gratiswochenende in \glqq Stagnant Springs Spa\grqq . Marge genießt ein Schlammbad und Homer geht ins Dampfbad. Ein geheimnisvoller Mann dreht den Regler des Thermostats von \glqq belebend\grqq\ auf \glqq Mörder\grqq\ und verriegelt die Tür mit einem Schraubenschlüssel. Homer wird gerade noch rechtzeitig von Krusty befreit. Marge und Homer erzählen Chief Wiggum von dem Mordversuch. Wiggum ist dem nicht gewachsen und er holt Sideshow Bob aus dem Gefängnis, um den Simpsons zu helfen. Obwohl der Wiederholungstäter über ausgezeichnete Kenntnisse des Verbrechermilieus verfügt, steht er Homers Fall zunächst ratlos gegenüber. Dies ändert sich erst, als das Familienoberhaupt der Simpsons zum Mardi-Gras-König gewählt wird. Bei der Parade versucht dann wieder jemand, Homer zu töten. Nach einer Verfolgungsjagd auf Stelzen, stellen Homer und Bob den Übeltäter. Es ist Homers Mechaniker, auch bekannt als Frank Grimes junior. Schlussendlich versucht auch Bob, seinen lang ersehnten Wunsch in die Tat umzusetzen.

\notiz{
\begin{itemize}
  \item Marges persönlicher Rekord beim Staubsaugen beträgt 3:40 Minuten.
  \item Personen, welche die Homerpuppe verprügeln: Moe, Patty und Selma, Willie, Reverend Lovejoy und Homer.
  \item Lenny stammt aus Chicago.
  \item Homer wird zum König des Mardi Gras\index{Mardi Gras}\footnote{Mardi Gras ist der Karneval in den Südstaaten der USA. Es findet am Tag vor Aschermittwoch statt. Mardi Gras ist übrigens offizieller Feiertag in New Orleans und Mobile \cite{MardiGras}.}\index{Mardi Gras} gewählt. Sein Vorgänger als König war Cletus.
  \item Frank Grimes jr. ist der Sohn von Frank Grimes mit einer Prostituierten.
  \item Fehler: Als Frank Grimes jr. von Homer und Tingel-Tangel-Bob verfolgt wird, sieht man Frank seinen Umhang verlieren. In der nächsten Szene hat er den Umhang wieder an.
\end{itemize}
}

\subsection{Lehrerin des Jahres}\label{EABF02}
Barts Klasse muss einen Aufsatz über den ersten Weltkrieg schreiben. Bart bittet Grampa um Hilfe, aber dieser erzählt ihm nur eine bizarre Geschichte. Bart bekommt ein \glqq F\grqq\ und muss nachsitzen. Dabei entgeht ihm nicht, dass Edna von Rektor Skinner immer wieder versetzt wird. Er bittet Lisa um Rat. Diese schlägt ihm vor, Edna zum \glqq Lehrer des Jahres\grqq\ nominieren zu lassen. Die Familie Simpson begleitet daraufhin Edna nach Orlando. Auch Rektor Skinner folgt ihr dort hin. Als sie jedoch herausfindet, dass Seymour auch seine Mutter mitgebracht hat, lässt sie ihn links liegen. Skinner will nun mit Barts Hilfe den Wettbewerb sabotieren, damit Edna nicht die Schule verlässt. Als Bart gerade versucht, den Plan in die Tat umzusetzen, schreitet Rektor Skinner ein und deckt die ganze Geschichte auf. Schließlich hält Seymour auch noch um ihre Hand an.

\notiz{
\begin{itemize}
  \item Fehler: Als der Onkel von Milhouse Bart mit dem Helikopter abholt, ist Barts Zimmer an der falschen Stelle. Denn, das Fenster aus dem Bart blickt, gehört tatsächlich zu Homers und Marges Schlafzimmer.
  \item Als Rektor Skinner mit Willies Auto Springfield verlässt, ist ein Schild zu sehen, auf dem steht, dass Orlando 2653 Meilen entfernt ist.
  \item Lehrer des Jahres wird Julio Estudiante\index{Estudiante!Julio}.
  \item Die Achterbahn \glqq Enron's Ride Of Broken Dreams\grqq\ ist eine Anspielung auf die Firmenpleite des US-Energiekonzerns Enron.
\end{itemize}
}

\subsection{Der Vater, der zu wenig wusste}\label{EABF03}
Lisa ist ziemlich wütend auf ihren Vater und wirft ihm vor, so gut wie nichts über seine Kinder zu wissen. Homer will sich derartige Anschuldigen natürlich nicht gefallen lassen und setzt den Detektiv Dexter Colt\index{Colt!Dexter} auf Lisa an, um sich mit dessen Informationen wieder bei ihr einschmeicheln zu können. Als die beiden Männer sich beim Thema Bezahlung nicht einigen können, will sich Dexter an Homer rächen, indem er Versuchstiere an den Zirkus verkauft und dies Lisa in die Schuhe schiebt.

\notiz{
\begin{itemize}
	\item Homers E-Mail-Adresse: \url{chunkylover53@aol.com}.
	\item Elliott Gould ist der Nachbar von Krusty.
	\item Das Tagebuch, das sich Lisa zum Geburtstag wünscht, heißt \glqq Turbo Diary\grqq .
	\item Die Ex-Frau des Polizisten Lou heißt Amy.
\end{itemize}
}

\subsection{Die starken Arme von Marge}
Völlig überraschend wird Marge Simpson eines Tages das Opfer eines brutalen Raubüberfalls. Nach diesem furchtbaren Erlebnis macht die dreifache Mutter eine merkwürdige Veränderung durch: Aus Angst, sich bei einem erneuten tätlichen Übergriff wieder nicht wehren zu können, trainiert sie wie verrückt und wird so zu einer wahren Kampfmaschine. Marge merkt nicht, dass sie maßlos übertreibt und durch ihr Bodybuilding-Training zur Gefahr für andere geworden ist.

\notiz{
\begin{itemize}
	\item Auf dem Nummernschild der Simpsons steht der Episodencode dieser Folge \glqq EABF04\grqq .
	\item Rainier Wolfcastle hatte innerhalb von drei Monaten drei Scheidungen.
	\item Auf dem Müllcontainer beim Kwik-E-Mart ist ein El Barto Graffiti zu lesen.
	\item Marge belegt bei dem Wettbewerb mit den Bodybuilderinnen den zweiten Platz.
\end{itemize}
}

\subsection{Bart, das Werbebaby}
Als Baby spielte Bart die Hauptrolle in einem Werbefilm gegen schlechten Baby-Atem. Als er davon erfährt und von Lisa nichts als Spott erntet, beschließt er, einen Anwalt zu nehmen und sich von seinen Eltern scheiden zu lassen. Als die Klage Erfolg hat, zieht er in ein hippes Loft und trifft zufällig auf die Skateboard-Legende Tony Hawk und die Jungs der Rockband Blink 182. Der von schlechtem Gewissen geplagte Homer vermisst seinen Sohn aber so sehr, dass er nichts unversucht lässt, Bart zurückzugewinnen.

\notiz{
\begin{itemize}
	\item Dies ist die 300. Folge der Produktionsreihenfolge.
	\item Homer hielt Bart über das Geländer eines Hotels, sowie das Michael Jackson mit einem seiner Kinder hat. Allerdings fällt Bart Homer hinunter.
	\item Vor Gericht werden die Simpsons von Gil vertreten.
	\item Um die Schulden bei Bart abzuarbeiten, macht Homer Werbung für ein Potenzmittel.
\end{itemize}
}

\subsection{Ein kleines Gebet}\label{EABF06}
Der stets gottesfürchtige Ned Flanders scheint vom Glück verfolgt zu sein. Das Geheimnis seines Erfolges vermuten die Simpsons in dessen ständigen Gebeten. In der Hoffnung, das selbe Ergebnis erzielen zu können, macht es Homer seinem Nachbarn gleich -- bis er vor der Kirche in ein Loch fällt. Als er den Klerus wegen dieses Unfalls verklagt und im Zuge dessen das Gotteshaus zugesprochen bekommt, endet dies schließlich damit, dass die Sintflut über Springfield hereinbricht.

\notiz{
\begin{itemize}
  \item Der reiche Texaner besitzt einen Frauenbasketballmannschaft.
  \item Bei den olympischen Spielen der Affen bekommt ein Affe beim Eiskunstlauf folgende Punkte: 5,1 von Frankreich, 5,2 von Italien, USA und Ungarn und 5,3 von Australien.
  \item Die Frau von Richter Synder wurde von einem Geistlichen überfahren.
  \item Als Homer in der Kirche sitzt, ist das Lied \glqq I Was Made For Loving You\grqq\ von KISS\index{KISS} zu hören.
  \item Das Foto von Reverend Lovejoy, das in den Nachrichten zu sehen ist, wurde von der Fotografin Annie Leibovitz\index{Leibovitz!Annie} aufgenommen.
\end{itemize}
}

\subsection{Buchstabe für Buchstabe}\label{EABF07}
Die ehrgeizige und vor allem schlaue Lisa Simpson nimmt an einem Buchstabierwettbewerb teil und gewinnt problemlos alle Vorentscheidungen. Vor dem großen Finale bekommt das Mädchen ein unmoralisches Angebot: Sollte sie freiwillig einen Fehler machen und den Sieg ihrem Kontrahenten überlassen, würde sie ein Stipendium für das College ihrer Wahl bekommen. Lisa, die immer für Ehrlichkeit eingetreten ist, lehnt diesen Vorschlag jedoch ab und gibt ihr Bestes, um zu siegen.

\notiz{
\begin{itemize}
  \item Nelson und Martin waren während der Ferien gemeinsam im Weltraumcamp.
  \item Der Ribwich\index{Ribwich} Burger wurde aus einem Tier hergestellt, das mittlerweile ausgerottet ist.
  \item Die Ribwich-Tournee:
  \begin{itemize}
  	\item 13.09. San Antonio
  	\item 20.09. Atlanta
  	\item 28.10. Boston
  	\item 30.10. Springfield
  	\item 06.09. San Francisco
  \end{itemize}
  \item Barney ist rückfällig geworden und liegt betrunken auf dem Bürgersteig.
  \item Beim Buchstabierwettbewerb in Calgary sitzen in der Reihe hinter den Simpsons Ruth Powers und Prof. Frink.
  \item Rektor Skinner war in den Sommerferien Empfangschef im Country Club.
\end{itemize}
}

\subsection{Ein Stern wird neu geboren}\label{EABF08}
Die Einsamkeit von Ned Flanders scheint endlich ein Ende zu haben, als die berühmte Schauspielerin Sara Loane\index{Loane!Sara} in Springfield auftaucht und mit ihm einen liebevollen Flirt beginnt. Als diese sogar mit ihm schlafen will, ist Ned zunächst verwirrt und weiß nicht, was er tun soll. Erst nach längerem Zögern willigt er ein, sein starker Glaube plagt ihn jedoch im Anschluss mit einem schlechten Gewissen.

\notiz{
\begin{itemize}
  \item Rainier Wolfcastle ist der Ex-Freund von Sara Sloane.
  \item Die Telefonnummer von Cookie Kwan lautet: 555-DO-IT.
  \item Clancy Wiggum behauptet, seine Frau Sarah kennengelernt zu haben, indem er ihr Drogen zusteckte. Dies steht im Widerspruch zur späteren Episode \glqq Die weiblichen Verdächtigen\grqq\ (siehe \ref{QABF10}).
\end{itemize}
}


\subsection{Krusty im Kongress}\label{EABF09}
Die Simpsons hatten keinen ruhigen Tag mehr, seitdem die Flugrouten geändert wurden und alle startenden Flugzeuge über ihr Haus fliegen. Um dem Desaster ein Ende zu setzen, bittet die Familie Krusty, den Clown, sich in den Kongress wählen zu lassen. Zwar wird der Star tatsächlich zum Politiker, weiß allerdings nicht im Geringsten, wie er sich gegen die anderen durchsetzen soll. Erst der Hausmeister des Kongresses gibt den entscheidenden Tipp, wie den Simpsons geholfen werden kann.

\notiz{
\begin{itemize}
  \item Es ist zu sehen, dass Springfield einen Internationalen Flughafen hat.
  \item Krusty ist der Nachfolger von Horace Wilcox\index{Wilcox!Horace} als Abgeordneter.
  \item Krustys Gegner heißt Armstrong\index{Armstrong}.
  \item Krusty gewinnt den 24sten Wahlbezirk. Nur vier Bundesstaaten haben mindestens 24 Wahlbezirke: Kalifornien, Florida, New York und Texas.
\end{itemize}
}

\subsection{Mr. Burns wird entlassen}\label{EABF10}
Es ist Valentinstag. Doch Homers Vorstellungen für diesen Tag gehen nicht in Erfüllung. Daraufhin besucht er ein Selbstverwirklichungsseminar. Mit seinem neu gewonnenen Selbstbewusstsein stellt Homer ein Liste mit Verbesserungsvorschlägen für die Sicherheit im Kraftwerk auf. Allerdings geht Mr. Burns nicht auf seine Vorschläge ein. Daraufhin verdrängt er ihn mit einer Intrige von seinem Chefsessel und steigt selbst zum Boss des Atomkraftwerkes auf. Die erste Freude über den steilen Karrieresprung ist jedoch schnell verflogen: Der dreifache Familienvater hat keine Freude mehr an der Arbeit und ärgert sich darüber, nur noch wenig Zeit für Frau und Kinder zu haben. Insofern trifft es sich gut, dass Mr. Burns alles versucht, um wieder den Chefposten zu erobern.

\notiz{
\begin{itemize}
  \item Den Kurs \glqq Strip For Your Wife\grqq\ hält Dr. Hibbert, welcher früher unter dem Künstlernamen Malcolm Sex\index{Sex!Malcolm} aufgetreten ist.
  \item Der eigentliche Besitzer des Atomkraftwerks ist der Kanarienvogel M. Burns.
  \item Mr. Burns ist bereits seit 62 Jahren im Atomkraftwerk tätig.
\end{itemize}
}

\subsection{Nacht über Springfield}\label{EABF11}
Ein britischer Filmproduzent besucht die Grundschule Springfield. Er gibt Lisa das Gefühl, dass sie nichts besonderes ist. Daraufhin beschließt sie, sich mit der Astronomie zu beschäftigen. Lisa findet jedoch keinen Platz in Springfield von dem aus die den Nachthimmel betrachten kann. Um uneingeschränkte Sicht auf den Sternenhimmel zu haben, setzt Lisa durch, dass in ganz Springfield die Lichter abgeschaltet werden. Eine Idee, an der auch Bart Gefallen findet, da er Fat Tonys Kühlerfigur klauen möchte, um damit vor seinen Kameraden zu prahlen. Zu dumm nur, dass Bürgermeister Quimby sich gegen die Simpsons durchsetzen kann und alle Lichter wieder einschalten lässt. Allerdings hat der nicht mit der cleveren Lisa gerechnet.

\notiz{
\begin{itemize}
  \item Carl gibt an, dass er seine Kindheit in Island verbracht hat.
  \item Selma und Moe sind knutschend auf einer Parkbank zu sehen.
\end{itemize}
}

\subsection{Homer auf Irrwegen}
In einem gemeinsamen Projekt bauen die Simpsons ein Puzzle zusammen. Am Ende bemerken sie jedoch, dass ein Teil fehlt. Bei der Suche nach dem fehlenden Teil findet Homer ein kleines Kästchen, das ihm Unglaubliches offenbart: Im Inneren befinden sich alte Aufzeichnungen von Marge, die besagen, dass ihn seine Frau eigentlich nie geliebt hat. Zutiefst verletzt und schwer enttäuscht verlässt Homer seine Familie und zieht erst einmal in die Wohnung von zwei homosexuellen Männern. Allerdings muss der dreifache Familienvater schon bald einsehen, dass dies nicht wirklich seine Welt ist.

\notiz{
\begin{itemize}
  \item Grady\index{Grady}, Julio\index{Julio} und Homer gehen in den Nachtclub \glqq One Night Stan's\index{One Night Stan's}\grqq .
  \item Homer und Marge waren 1989 auf einem Rolling Stones Konzert. Davon haben sie ein T-Shirt mit der Aufschrift \glqq Last Tour Ever 1989\grqq .
  \item Im Kleiderschrank ist neben der Mr. Plow Jacke aus der Folge \glqq Einmal als Schnee\-könig\grqq\ (siehe \ref{9F07}), das T-Shirt von den Pin Pals aus der Episode \glqq Homers Bowling-Mannschaft\grqq\ (siehe \ref{3F10}) zu sehen. Dieses T-Shirt gab Marge in der Folge \glqq Homer und gewisse Ängste\grqq\ (siehe \ref{4F11}) in die Kleidersammlung, von dort hatte es sich dann John geholt.
  \item Homer gibt an, dass Marge seine erste und einzige Liebe war. In der späteren Folge \glqq Die erste Liebe\grqq\ (siehe \ref{FABF13}) sagt Homer hingegen, seine erste Liebe sei nicht Marge gewesen. Es stellt sich dann aber heraus, dass es doch Marge war.
  \item In dieser Folge sollte eigentlich Homers ehemaliger Sekretär Karl\index{Karl} aus der Episode \glqq Karriere mit Köpfchen\grqq\ (siehe \ref{7F02}) einen Auftritt haben. Doch dessen damaliger Synchronsprecher, Harvey Fierstein, lehnte ab.
\end{itemize}
}

\subsection{Auf der Familienranch}
Homer schreibt neue Weihnachtslieder. Und um die nicht mehr im Radio hören zu müssen, verziehen sich die Simpsons auf eine entlegene Ranch, wo Lisa sich in den hübschen Luke\index{Stetson!Luke} verliebt. Homer und Bart reißen einen Biberdamm ein, sodass der Stausee verschwindet und die Indianer ihr Land wiederbekommen. Sie verleben eine wunderbare Zeit. Und als sie auf der Heimfahrt Moes Weihnachtslieder im Radio hören, machen sie auf der Stelle kehrt, um noch eine Woche auf dieser Ranch zu bleiben.

\notiz{
\begin{itemize}
  \item Bei Homers Lied über Flanders singen noch David Byrne, Carl und Lenny mit.
  \item Maggie ist bei einer Britney Spears Imitation zu sehen, bei welcher sie wie Britney tanzt und zum Schluss eine Dose Cola (Buzz) in der Hand hält, so wie Britney für Pepsi Cola.
\end{itemize}
}

\subsection{Die Helden von Springfield}
Homer hat die Schnauze endgültig voll: Aus Ärger über das permanent schlechte Verhalten seines Vierbeiners, setzt er Knecht Ruprecht vor die Tür. Der Zufall will es, dass just in diesem Moment der Eigentümer der Duff-Brauerei auf den Hund aufmerksam wird und ihn zu seinem neuen Werbemaskottchen macht. Als Knecht Ruprechts ehemaliger Besitzer davon Wind bekommt, macht er seine Ansprüche an dem Tier geltend und kassiert vor den Augen des entsetzten Homer kräftig ab.

\notiz{
\begin{itemize}
  \item Amish bauen Barts neues Baumhaus.
  \item Während Knecht Ruprecht Werbung für Duff machte, hieß er Suds McDuff\index{McDuff!Suds}.
\end{itemize}
}


\subsection{Stresserella über alles}
Nach einem Besuch im Springfielder Aquarium landet Bart im Krankenhaus. Da Marge Homer nicht finden kann, gibt sie ihm später ein Handy, um ihn immer erreichen zu können. Daraufhin besorgt sich Homer jede Menge Geräte fürs Auto. Als er deshalb einen Unfall hat, verliert er seinen Führerschein. Von nun an muss Marge die ganze Familie fahren. Eines Tages entdeckt Homer eine neue Leidenschaft, er geht nun alle Strecken zu Fuß. Dabei wird er plötzlich von Marge angefahren. Da Homer nun glaubt, dass sie ihn hasst, gehen sie zu einem Eheberater. Um seine Ehe zu retten, veranstaltet Homer eine Party für Marge.

\notiz{
\begin{itemize}
  \item Als Lisa in den Pinguinkäfig schaut, sieht sie die Pinguine fliegen. Bereits in der Episode \glqq Barts Komet\grqq\ (siehe \ref{2F11}) sieht man die Pinguine aus dem Springfielder Zoo fliegen.
  \item Lindsey Naegle verkaufte Homer ein Handy.
\end{itemize}
}


\subsection{Auf dem Kriegspfad}\label{EABF16}
Nachdem Bart und Milhouse in Flanders Haus eingebrochen sind und eine Spur der Verwüstung hinterlassen haben, beschließen Homer und Marge, Bart in den Club der \glqq Pre-Teen Braves\grqq\ (Vor-Teenager) zu stecken. Unter der Leitung von Homer und Marge, sollen die Kinder von dummen Gedanken abgehalten werden. In der Zwischenzeit findet sich Milhouse auch in einem solchen Club wieder, den \glqq Cavalry Kids\grqq\ (Kavallerie-Kinder) unter der Leitung von Kirk Van Houten. Zwangsläufig geraten die zwei Clubs aneinander und kämpfen darum, wer mehr für die Wohltätigkeit tut. Der Streit eskaliert, als es darum geht, wer mehr Süßigkeit verkauft und damit beim nächsten Baseballspiel der Springfield Isotopes als Ehrengast dabei sein darf. Als schließlich die Kavallerie-Kinder gewinnen, ist Homer entrüstet -- immerhin gehört sein Sohnemann den Vor-Teenagern an! Die Verlierer haben jedoch so ihre eigene Taktik, doch noch zum Gewinner zu werden: Sie verkleiden sich bei der Siegerehrung als Kavallerie-Kinder. 

\notiz{
\begin{itemize}
  \item Bart und Milhouse sehen sich im Fernsehen \glqq South Park\index{South Park}\grqq\ an.
  \item Ned Flanders ist ein großer Beatles-Fan und sagt, dass die Beatles besser als Jesus waren.
  \item Den Vor-Teenagern gehören Bart, Nelson, Database und Ralph an. Den Kavallerie-Kin\-dern gehören neben Milhouse, noch Martin, Jimbo und ein weiterer Junge an.
  \item Carl hat bei Lennys Hochzeit gesungen.
  \item Lennys Familienname lautet Leonard.
\end{itemize}
}

\subsection{Moe Baby Blues}
Moe fühlt sich alleine, keiner besucht seine Bar und er wird von öffentlichen Veranstaltungen ausgenommen, deshalb beschließt er, Selbstmord zu begehen. Doch bevor er von der Brücke springen kann, rettet er zufällig das Leben von Maggie. Daraufhin engagiert Marge Moe als Babysitter und der verliebt sich dermaßen in Maggie, dass er nicht mehr von ihr lassen will. Dies erweckt Homers Eifersucht. Als Maggie jedoch der Mafia nachspioniert und unbewusst fast in eine Schießerei gerät, holt Moe sie unbeschadet dort raus und ist bei den Simpsons wieder ein gern gesehener Freund.

\notiz{
\begin{itemize}
  \item Homer ist in der Fahrgemeinschaft von Mr. Burns und Smithers.
  \item Wiggum gesteht, dass er der schlechteste Fußballtrainer der Welt ist.
  \item Moes Telefonnummer lautet 939-15979. Das ist bereits die fünfte Telefonnummer. Diese vier Nummern hatte die Kneipe vorher: 636-5551239 (Staffel 2), 636-5556264 (Staffel 3), 636-5553543 (Staffel 5) und 636-76484377 (Staffel 7).
\end{itemize}
}

\section{Staffel 15}
 	
\subsection{Wiedersehen nach Jahren}\label{EABF18}
Über die Zeilenanfangsbuchstaben eines Zeitungsartikels findet Homer seine Mutter Mona\index{Simpson!Mona} wieder. Als Super-Radikale in den 60er Jahren war sie jahrelang vor der Polizei auf der Flucht, aber schon bald ist ihr Polizeichef Wiggum auf der Spur. Sie wird vor Gericht gestellt, aber freigesprochen. Mr. Burns kann Mona Simpson eines Verbrechens überführen. Sie hatte sich früher unter falschem Namen Zutritt zu Naturparks verschafft. Daraufhin wird sie sofort wieder zu einer Gefängnisstrafe verurteilt. Homer verhilft ihr zur Flucht und es gelingt ihr zu entkommen. Wieder versucht sie, mit Homer über Zeitungsartikel Kontakt aufzunehmen.

\notiz{
\begin{itemize}
  \item Gil vertritt Mona Simpson vor Gericht.
  \item Unter den Geschworenen sind Mrs. Glick und Sideshow Mel.
  \item Richter Snyder sagt, dass er eine kleine Schwester hat und dass er am Morgen von seiner Frau verlassen worden ist.
  \item Als Mona zum zweiten Mal in der Episode eingesperrt wird, hat sie die Gefangenennummer 2067.
\end{itemize}
}

	
\subsection{Der Dicke und der Bär}\label{EABF19}
Homer, Bart und Lisa schenken Marge zum Muttertag eine Maschine zur Herstellung von Süßigkeiten. Eines Nachts erschafft Homer eine riesige Kugel aus Zuckerwatte und Karamell, die von nun an sein ewiger Begleiter wird. Als sich Marge jedoch darüber beschwert, bringt Homer die Kugel auf die Müllkippe, wo er von einem Bären angegriffen wird. Völlig in Panik vor allem was einem Bären ähnelt, sieht Homer nur noch einen Ausweg: Er muss den Bären bekämpfen.

\notiz{
\begin{itemize}
  \item Der Originaltitel \glqq The Fat and the Furriest\grqq\ ist eine Anspielung auf den Film \glqq The Fast and the Furious\grqq .
  \item Abe Simpson arbeitet im Sprawl Mart.
  \item Die Website von Abe lautet \url{www.OldCoot.com}
\end{itemize}
}	


\subsection{Die Perlen-Präsidentin}\label{EABF20}
Martin Prince wird von seinem Posten als Schulsprecher entlassen. Lisa bewirbt sich als Nachfolger und muss sich gegen Nelson durchsetzen, was sie dank eines Liedes auch schafft. Sie engagiert sich sehr, denkt selbst und versucht, die Schule zu verbessern. Dies passt den Lehrern allerdings nicht. Durch tolle Kleider und Lügen machen sie Lisa überheblich und legen sie rein, einen Vertrag zu unterschreiben, in dem steht, dass Musik, Sport und Kunsterziehung abgeschafft werden. Daraufhin ruft Lisa zum Generalstreik auf, bis Oberschulrat Chalmers reumütig Rektor Skinner zurückpfeift. Dieser verspricht, umgehend alles wieder gut zu machen.

\notiz{
\begin{itemize}
  \item Ehemalige Schülersprecher der Springfielder Grundschule waren unter anderem Krusty, Otto und Martin Prince.
  \item Filmzitat: Die gesamte Episode ist an das Musical und den Film \glqq Evita\index{Evita}\grqq\ angelehnt.
  \item Die E-Mail-Adresse von Lisa lautet \url{klugesmaedchen63_/@yahoo.com}.
  \item Lisa sollte an die \glqq Springfield Magnet School\index{Magnet School}\grqq\ überstellt werden, was Homer allerdings in letzter Sekunde verhindert.
\end{itemize}
}

\subsection{Todesgrüße aus Springfield}
\begin{itemize}
	\item \textbf{Reaper Madness}\\ In der ersten Episode bringt Homer den Tod um und muss von nun an selbst den Tod spielen, was er aber auf die Dauer nicht übers Herz bringt, spätestens als er Marge umbringen soll. Schließlich bittet er Gott, ihn von diesem Amt zu erlösen. 
	\item \textbf{Frinkenstein}\\ In der zweiten Episode wird Professor Frink der Nobelpreis verliehen. Zu diesem Anlass erweckt er seinen verstorbenen Vater wieder zum Leben. Der läuft daraufhin Amok und klaut den Mitbürgern ihre inneren Organe. Professor Fink ist froh, als sein Vater endlich wieder stirbt.    
	\item \textbf{Stop The World, I Want To Goof Off}\\ Bart und Milhouse haben sich eine Stoppuhr bestellt, mit der sie die Zeit anhalten können. Sie treiben damit allerlei Unsinn, bis ihnen die Uhr plötzlich im Stoppzustand kaputt geht. 15 Jahre lang basteln sie daran herum, die Uhr zu reparieren, um ihre Umwelt wieder zum Leben zu erwecken. Alle sind überrascht, plötzlich einen gealterten Bart und Milhouse vorzufinden.
\end{itemize}

\notiz{
\begin{itemize}
  \item Als Homer der Tod ist, bringt er unter anderem Jasper, Kirk Van Houten und Reverend Lovejoy um.
  \item Als Bart und Milhouse die Zeit anhalten, benutzen sie Martin Prince als Sündenbock.
\end{itemize}
}


	
\subsection{Die Queen ist nicht erfreut!}\label{EABF22}
Bart hat einen Tausend-Dollar-Schein gefunden, den er in seinem Privatmuseum gegen Eintritt zur Schau stellt. Als Mr. Burns seinen Geldschein zurückfordert, hat Bart bereits 3000 Dollar an Eintrittsgeldern verdient und ist somit ein reicher Mann. Er lädt seine Familie zu einer Urlaubsreise nach England ein. Hier treibt Homer wiedermal allerlei Unfug, bis er im Gefängnis landet, aber die Queen begnadigt ihn und schickt ihn, mit Madonna im Gepäck, nach Amerika zurück. Grandpa Abe Simpson trifft in England seine alte Liebe aus den Kriegsjahren wieder. Die stellt ihm ihre Tochter Abbie\index{Abbie} vor, die Homer wie aus dem Gesicht geschnitten ist. Schleunigst ergreifen die Simpsons die Flucht.

\notiz{
\begin{itemize}
  \item Der Originaltitel \glqq The Regina Monologues\grqq\ ist eine Anspielung auf die Serie \glqq The Vagina Monologues\grqq\ aus den 90ern.
  \item Das Lachen von Bart und Lisa erinnert an  \glqq Beavis and Butthead\grqq .
  \item Das Passwort für Mr. Burns Kreditkarte ist sein Alter. Burns tippt viermal auf die Tastatur. Es ist zu vermuten, dass sein Passwort drei Ziffern lang ist und das vierte Drücken, das Bestätigen des Auszahlen-Knopfes war.
  \item Tony Blair nahm seine Sprechrolle am 11. April 2003 auf.
\end{itemize}
}

\subsection{Krustys Bar Mitzvah}
Zu seinem Entsetzen stellt Krusty fest, dass er nie eine Bar Mitzvah hatte und somit nach jüdischem Glauben nicht zum Manne geweiht wurde. Verzweifelt wendet er sich an seinen Vater, einen Rabbi, der schließlich einwilligt, ihn zu weihen. Währenddessen hat Homer Simpson Krustys Talkshow übernommen und das mit großem Erfolg, bis Lisa ihm schließlich einredet, wichtigere Themen anzuschneiden. Der Versuch schlägt fehl und Homer wird mit Schimpf und Schande entlassen.

\notiz{
\begin{itemize}
  \item Willie, Snake und Krusty nehmen jeweils eines der Hundewelpen.
  \item Auf dem jüdischen \glqq Walk Of Fame\grqq\ befinden sich die Sterne von u.\,a.: Woody Allen, Sandy Koufax, Joan Rivers, Henry Winkler und Shari Lewis.
  \item Rainier Wolfcastle ist offensichtlich antisemitisch eingestellt.
  \item Lenny war dreimal Geschworener.
  \item In Homers Talkshow sind anfangs Moe, Carl und Lenny die Talkgäste. Später wird Lenny durch Barney ersetzt. Barney wird wiederum durch Disco Stu ersetzt.
\end{itemize}
}

\subsection{Eine Simpsons-Weihnachtsgeschichte}
Homer bekommt zu Weihnachten eine wertvolle Baseballkarte geschenkt und verkauft diese für viel Geld. Davon will er der Familie eigentlich ein schönes Weihnachten machen, anstatt einen guten Christbaum zu besorgen, kauft er sich aber lieber selbst ein extrem teures und überflüssiges Geschenk. Die Familie ist enttäuscht und Homer muss auf der Couch schlafen. Dort sieht er Charles Dickens Weihnachtsgeschichte und bemerkt endlich, dass er viel zu egoistisch ist. Von nun an will er sein Leben ändern und widmet sein Leben nur noch den anderen. Ned Flanders wird eifersüchtig und will großzügiger als Homer sein, indem er Geschenke verteilt. Homer hat währenddessen den Plan, Weihnachten zum Fest der Liebe zu machen und nicht der Geschenke. Er beschließt schließlich, alle Geschenke zu klauen. Die Bevölkerung ist empört. Als dann noch Bürgermeister Quimby das öffentliche Beten verbietet, beschließen Homer und Ned, gemeinsam wieder Geschenke zu verteilen. Das Weihnachtsfest ist gerettet.

\notiz{
\begin{itemize}
  \item Carl schenkt Homer einen DVD-Player und die erste Staffel von \glqq Magnum\grqq .
  \item Mr. Burns gibt Homer für Bart die Sammlerkarte des Baseballspielers Joe Di Maggio.
  \item Rod Flanders ist eifersüchtig auf Mädchen, da diese Kleider tragen dürfen.
  \item Als die Simpsons einkaufen gehen, ist beim Stand \glqq Your Picture On A Rembrandt\grqq\ ein Bild zu sehen, auf dem sich Moe befindet.
\end{itemize}
}


\subsection{Marge gegen Singles, Senioren, kinderlose Paare, Teenager und Schwule}\label{FABF03}
Maggie ist ganz verrückt nach dem Kindersänger Roofi\index{Roofi}. Bei den Simpsons wird nur noch er angeschaut und gehört, was Lisa und Bart nicht mehr ertragen können. Als ein Konzert stattfindet, geht Marge mit Maggie dort hin. Das Konzert läuft nicht wie es soll und die Kinder fangen an zu schreien und machen einen Aufstand. Die Stadtbewohner sind nun so genervt, dass sie eine Organisation für Leute ohne Kinder gründen und fordern, dass sie kein Kindergeld mehr zahlen müssen. Marge gründet eine Organisation gegen diese Organisation, allerdings ist diese nicht sonderlich erfolgreich. Der Wahlkampf läuft auf Hochtouren, doch auf Lisas Anraten überfallen die Kinder die gegnerischen Erwachsenen mit Erkältungskrankheiten, sodass diese nicht zur Wahl gehen können. Marge trägt mit ihrer Initiative einen überwältigenden Sieg davon.

\notiz{
\begin{itemize}
  \item Lindsey Naegle\index{Naegle!Lindsey} grün\-det die Organisation \glqq SSCCATAGAPP\index{SSCCATAGAPP}\grqq , Singles, Seniors, Childless Couples and Teens, and Gays against Parasitic Parents (Singles, Senioren, kinderlose Paare und Teenager und Schwule gegen parasitäre Eltern). Marges Gegenorganisisation ist \glqq PPASSCCATAG\index{PPASSCCATAG}\grqq , Proud Parents against Singles, Seniors, Childless Couples and Teens, and Gays (stolze Eltern gegen Singles, Senioren, kinderlose Paare und Teenager und Schwule).
  \item Lindsey Naegle ist mit Sideshow Mel bei Luigi essen.
  \item Während der aufgebrachte Mob die Kindersachen zerstört, ist das Lied \glqq My Generation\grqq\ von The Who zu hören.
  \item Kabul war Partnerstadt von Springfield.
\end{itemize}
}

\subsection{Häuptling Knock-A-Homer}\label{FABF04}
Barts Fahrrad geht kaputt und Homer kauft ihm ein neues, das er selbst zusammen bauen will. Dabei versagt er kläglich, also versucht er, Bart einen Roboter zu bauen. Währenddessen stirbt Schneeball II und Lisa bekommt eine neue Katze. Doch sie scheint kein Glück zu haben. Jede neue Katze, die sie bekommt, stirbt. Auch Homer ergeht es nicht besser bei seinem Roboter, bis er die Idee hat, selbst zum Roboter zu werden. Als Roboter nimmt er mit Bart an der Show \glqq Robot Rumlbe\index{Robot Rumble}\grqq\ teil, in der Roboter miteinander kämpfen. Er kommt ziemlich weit und muss nun einen aussichtslosen Kampf bestreiten. Seine Überlebenschancen sind nicht besonders hoch und während der Pause bemerkt Bart, dass Homer in Wirklichkeit der Roboter ist. Wie durch ein Wunder wird Homer gerettet, da Professor Frink seinen Roboter so programmiert hat, Menschen nicht zu verletzen, sondern ihnen zu dienen. Überglücklich schließen sich Vater und Sohn in die Arme.

\notiz{
\begin{itemize}
  \item Dr. Hibbert fährt einen Mercedes Geländewagen.
  \item Der Roboter \glqq Killhammad Aiee\index{Killhammad Aiee}\grqq\ ist von den Frinks.
  \item Homer und Bart sehen sich zu Hause u.\,a. das Video \glqq Killhammad Aiee vs. Bender\grqq\ an.
  \item Nachdem Snowball II von Dr. Hibbert überfahren wurde, legten sich die Simpsons die Katze Snowball III zu. Snowball III ist im Aquarium ertrunken. Coltrane\index{Coltrane}, die nächste Katze, starb, weil sie aus Lisa Fenster gesprungen ist. Snowball V wurde wieder in Snowball II umgetauft und dieses Katze erhielt Lisa von der Katzenlady.
  \item Die Simpsons kämpfen u.\,a. gegen die Roboter von Clancy und Ralph Wiggum, Vater und Sohn Frink und Timothy und Jessica Lovejoy. Jessica Lovejoy war zuletzt in der Episode \glqq Barts Freundin\grqq\ (siehe \ref{2F04}) zu sehen.
\end{itemize}
}


\subsection{Fantasien einer durchgeknallten Hausfrau}\label{FABF05}
Marge beschließt, Schriftstellerin zu werden und schreibt einen Roman, in dem Homer als Walfänger auftritt und in dessen Abwesenheit sie mit Ned Flanders ein Liebesverhältnis eingeht. Beim Lesen schläft Homer schon nach wenigen Zeilen ein und so gibt er ihr die Erlaubnis, das Buch zu veröffentlichen. Damit werden beide, besonders Homer, zum Gespött der Stadt. Wutentbrannt will Homer sich an Flanders rächen, nachdem er die Hörbuchversion des Romans gehört hatte. Er stellt ihn zur Rede und bittet ihn, ihm zu helfen, ein besserer Ehemann zu werden. Versöhnt kehren alle nach Hause zurück.

\notiz{
\begin{itemize}
	\item Der Originaltitel \glqq Diatribe of a Mad Housewife\grqq\ ist eine Anspielung auf den Film \glqq Diary of a Mad Housewife\grqq\ aus den 70ern.
	\item In der Folge \glqq Die Trillion-Dollar-Note\grqq\ (siehe \ref{5F14}) gibt Marge an, das Bild über der Couch im Wohnzimmer selbst gemalt zu haben. In dieser Episode wird der Titel des Bilds verraten \glqq Szene aus Moby Dick\grqq\ (Scene from Moby Dick).
	\item In dieser Episode ist Dr. Marvin Monroe wieder zu sehen. Sein langes Fehlen erklärt er mit einer langen und schweren Krankheit.
	\item Homer arbeitet vorübergehend in \glqq Boris' Car Loft\grqq . Dort kauft er auch den Krankenwagen und wird anschließend Krankenwagenfahrer.
	\item Für diese Episode konnte der kamerascheue Schriftsteller Thomas Pynchon\index{Pynchon!Thomas} als Gast gewonnen werden. Matt Groening glaubte zunächst nicht, dass dieser zusagen würde \cite{InterviewGroening}.
\end{itemize}
}


\subsection{Geschichtsstunde mit Marge}
Bart, Lisa und Milhouse sollen Aufsätze über verschiedene historische Per\-sön\-lich\-kei\-ten schreiben. Als Marge von dieser Hausübung erfährt, weiß sie so manch schöne Geschichte zu den Aufsatzthemen zu erzählen: So berichtet sie von Heinrich VIII., der seine eigene Religion gründete und zahlreiche Frauen tötete und den Forschern Lewis und Clark. Sogar zum Wunderkind Mozart und dessen eifersüchtiger Schwester Salieri fällt ihr eine gute Story ein.

\notiz{
\begin{itemize}
  \item Darsteller der ersten Geschichte: Heinrich VIII. = Homer, Hofarzt = Dr. Nick Riviera, Bote = Moe, Eheberater = Dr. Hibbert und Scharfrichter = Chief Wiggum.
  \item Darsteller der zweiten Episode: Sacagawea\index{Sacagawea} = Lisa, Sacagaweas Mann = Milhouse, Lewis = Lenny und Clark = Carl.
  \item Darsteller der dritten Geschichte: Mozart = Bart, Salieri = Lisa, Kaiser = Mr. Burns und Beethoven = Nelson.
  \item Das \glqq Bat Out Of Salzburg\grqq\ T-Shirt ist eine Anspielung auf das Meat-Loaf-Album \glqq Bat Out Of Hell\grqq .
\end{itemize}
}

\subsection{Milhouse lebt hier nicht mehr}\label{FABF07}
Milhouse ist mutiger und frecher als sonst und benimmt sich in einem Museum daneben, wie er es normalerweise nie tun würde. Der Grund ist: Milhouse zieht um. Bart ist niedergeschlagen und kommt über den Verlust nicht hinweg. Bei einem Besuch bei Milhouse erkennt Bart Milhouse nicht mehr wieder. Als er allerdings mit Lisa das Auto waschen soll, findet Bart in Lisa einen Ersatz und die beiden werden beste Freunde, zumindest so lange, bis Milhouse wieder in Springfield auftaucht. Währenddessen geht Homer betteln, um Marge teuren Schmuck kaufen zu können.

\notiz{
\begin{itemize}
	\item Der Originaltitel \glqq Milhouse Doesn't Live Here Anymore\grqq\ ist eine Anspielung auf das Drama \glqq Alice Doesn't Live Here Anymore\grqq\ aus den 70ern.
	\item Für die Umzugsfirma, die Luann Van Houten beauftragt hat, arbeiten Mr. Largo und Miss Pommelhorst.
	\item Homer bettelt, um Marge teure Diamantohrringe zum Hochzeitstag kaufen zu können.
	\item Weil Apu und Manjula Hochzeitstag haben, geben sie in Moes Taverne Freibier für alle Anwesenden aus (Homer, Lenny und Carl).
	\item Im Fernsehmuseum steht eine Figur von Ned Flanders.
	\item Einer von Milhouse neuen Freunden ist gekleidet wie Ali G, ein anderer wie Eminem.
	\item Kirk Van Houten bekommt das Sorgerecht für Milhouse aus Mitleid.
	\item Diese Episode hatte in Amerika an dem Tag Premiere, an dem Matt Groening 50 Jahre alt wurde.
\end{itemize}
}

\subsection{Klug \& Klüger}\label{FABF09}
Da Apu und Manjula auch ihre Kinder in einen Vorkindergarten schicken, will Homer, dass auch Maggie diesen besucht. Doch sie wird nicht genommen, weil sie nicht sprechen kann. Der Schein trügt, Lisa findet raus, dass Maggie in Wirklichkeit schlau ist. Bei einem zweiten Versuch, bei dem Maggie nicht reden muss, schneidet sie dann besser ab und es kommt raus, dass sie sogar mit 167 einen um acht Punkte höheren IQ als Lisa hat. Lisa ist eifersüchtig auf Maggie und als sie merkt, dass sie voller Hass ist, haut sie von zu Hause ab und versteckt sich im Naturkundemuseum. Die Familie sucht sie und als sie Lisa finden, kommen sie in eine gefährliche Situation, aus der nur noch Maggie sie retten kann. Zum Schluss stellt die Kindergartenleiterin fest, dass Lisa Maggie unbewusst bei der Aufnahmeprüfung geholfen hat.

\notiz{
\begin{itemize}
  \item In Lisas Zimmer ist ein Foto vom Aushilfslehrer Mr. Bergstrom\index{Bergstrom} aus der Folge \glqq Der Aushilfslehrer\grqq\ (siehe \ref{7F19}) zu sehen.
  \item Homer schuldet Dr. Hibbert 14.000 Dollar.
  \item Homer und Marge melden Maggie in \glqq Wickerbottom's\grqq\ an, dabei handelt es sich um eine private Vorschule.
  \item In der Episode \glqq Homer hatte einen Feind\grqq\ (siehe \ref{4F19}) sagt Homer, Lisa habe einen IQ von 156. In dieser Folge gibt Lisa an, einen IQ von 159 zu haben.
  \item Die Telefonnummer der Simpsons: 555-5472.
\end{itemize}
}

\subsection{Rat mal, wer zum Essen kommt}\label{FABF08}
Homer geht mit Bart und Lisa in einen nicht jugendfreien Film und danach sind sie vollkommen verängstigt. Sie hören ein Gespenst auf dem Dachboden. Das Gespenst ist allerdings kein Gespenst, sondern Artie Ziff\index{Ziff!Artie}, der sein Vermögen verloren hat und sich bei den Simpsons versteckt hat. Er darf bei den Simpsons bleiben. Bei einer Poker Runde verliert er seine Firma an Homer. Allerdings soll der Besitzer der Firma verhaftet werden und Ziff bringt somit Homer hinter Gitter. Die Familie ist sauer darüber und wirft Ziff raus, der daraufhin in Moes Taverne geht und dort auf Patty und Selma trifft. Nach einer glücklichen Liebesnacht mit Selma entscheidet sich Artie, den Konkurs seiner Firma auf sich zu nehmen, damit Homer freigelassen wird.

\notiz{
\begin{itemize}
  \item Im \glqq Springfield Googolplex Kino\grqq\ laufen folgende Filme
  \begin{itemize}
	  \item Return To Ape Valley
	  \item The Fashion Of The Christ
	  \item Ghost Frat
	  \item Eating Nemo
	  \item From Justin To Kelly 4
	  \item The Unwatchable Hulk
	  \item The Pianist Goes Hawaiian
	  \item Freddy vs. Jason vs. Board Of Education 
  \end{itemize}
  \item Lenny spielt in dem Film \glqq Die Wiedertotmachung\grqq\ einen Gärtner, der von einer Puppe ermordet wird.
  \item Die Personen, die Artie in Moes Taverne als Loser bezeichnet, sind Llewellyn Sinclair\index{Sinclair!Llewellyn} aus \glqq Bühne frei für Marge\grqq\ (siehe \ref{8F18}), Professor Lombardo\index{Lombardo} aus \glqq Marges Meisterwerk\grqq\ (siehe \ref{7F18}), Mr. Seckofsky\index{Seckofsky} aus \glqq Wie alles begann\grqq\ (siehe \ref{7F12}) und Jay Sherman\index{Sherman!Jay} aus \glqq Der total verrückte Ned\grqq\ (siehe \ref{4F07}) und aus \glqq Springfield Film-Festival\grqq\ (siehe \ref{2F31}).
\end{itemize}
}

\subsection{Marge im Suff}\label{FABF10}
Der angetrunkene Homer verursacht einen Verkehrsunfall und versucht, den Vorfall Marge in die Schuhe zu schieben. Vom schlechten Gewissen getrieben besucht er Marge in der Reha-Klinik für Alkoholiker und gesteht reumütig, dass er der Schuldige ist. Bart und Lisa sind indes über den neuen Cosmic Wars\index{Cosmic Wars} Film verärgert, sodass Homer ihnen rät, dem Produzenten Randall Curtis\index{Curtis!Randall} einen Besuch abzustatten.

\notiz{
\begin{itemize}
  \item Der Film \glqq Cosmic Wars\grqq\ ist eine Anspielung auf die Star-Wars-Reihe.
  \item Auf dem Oktoberfest\index{Oktoberfest} spielt die Band \glqq Brave Combo\index{Brave Combo}\grqq .
  \item In der Reha-Klinik sind in der Gruppe für Alkoholiker: Otto, Captain McCallister, Edna Krabappel und Agnes Skinner.
  \item Carl Carlson ist in der Kirche zu sehen, obwohl er in der Episode \glqq Allein ihr fehlt der Glaube\grqq\ (siehe \ref{DABF02}) angibt, ein Buddhist zu sein.
  \item Captain McCallister hat rechts ein Holzbein.
  \item Der Duffman sagt, dass er Jude sei.
\end{itemize}
}

\subsection{Die Verurteilten}\label{FABF11}
Während eines Einkaufbummels wird Bart auf einen High-Tech-Scanner für Hochzeitspaare aufmerksam und probiert das Gerät gleich selbst aus. Er registriert sich und die imaginäre Braut Lotta Cooties\footnote{Was auf deutsch so viel heißt wie viele Läuse.}\index{Cooties!Lotta} am PC und scannt daraufhin verschiedene Artikel im Laden. Daraufhin versenden er und Milhouse Hochzeitseinladungen quer durch die Stadt. Am vermeintlichen Hochzeitstag erscheinen zahlreiche Gäste und legen die Geschenke ab: Bart verschwindet mit dem Wagen voller Geschenke und die Gäste bemerken den Betrug. Bart wird von Chief Wiggum geschnappt und landet in der Jugendhaftanstalt. Dort trifft er auf Gina Vendetti\index{Vendetti!Gina}, die mit ihm flüchtet. Beim Abschied zwischen den beiden kommt es zum romantischen Geständnis Ginas, sie werde Bart vermissen. Doch Wiggum und seine Crew schnappen die zwei Flüchtlinge und nur Gina, welche die gesamte Schuld auf sich nimmt, muss zurück in ihre Zelle.

\notiz{
\begin{itemize}
  \item Auf der Polizeimarke von Chief Wiggum steht \glqq Cash Bribes Only\footnote{Bestechung nur mit Bargeld}\grqq .
  \item Bart hat die Gefangenennummer 158735 und Gina Vendetti die Nummer 139813.
  \item Homer nimmt einen Job in dem Jugendgefängnis, in dem Bart inhaftiert ist, an.
  \item Snake hat ein Buch mit dem Titel \glqq The Ten Habits Of Highly Successful Criminals\footnote{Zehn Gewohnheiten für höchst erfolgreiche Kriminelle}\grqq\ verfasst.
\end{itemize}
}

\subsection{Hochzeit auf Klingonisch}\label{FABF12}
Rektor Skinner und Edna Krabappel wollen heiraten. Auf der Hochzeitszeremonie überkommen Edna plötzlich Zweifel und sie flüchtet vom Traualtar. Von einem Abenteuer stürzt sie sich auch schon ins nächste und lässt sich nach einem heißen Flirt mit dem Comicbuchverkäufer ein. Der am Boden zerstörte Skinner beschließt, seine Ex-Verlobte zurück zu erobern und es kommt zu einer Eifersuchtsszene zwischen Skinner und Ednas neuem Lover. Mrs. Krabappel gibt jedoch beiden einen Korb, denn sie ist auf der Suche nach einem Mann, der nicht heiraten möchte.

\notiz{
\begin{itemize}
  \item Der Comicbuchverkäufer hat eine Tätowierung des Supermanlogos auf seinem Hintern.
  \item Matt Groening signiert die Bender-Figur\index{Bender} von Milhouse.
	\item Der Originaltitel \glqq My Big Fat Geek Wedding\grqq\ ist eine Anspielung auf den Film \glqq My Big Fat Greek Wedding\grqq .
\end{itemize}
}

\subsection{Auf der Flucht}\label{FABF14}
Statt ohne die Kinder zu Großonkel Tyrones\index{Tyrone} Geburtstag zu fahren, beschließen Marge und Homer, nach Miami in die zweiten Flitterwochen zu fliegen. Aber Bart und Lisa kommen bald dahinter und verfolgen die beiden kreuz und quer durch Amerika, bis Marge und Homer sich schließlich in einer aufblasbaren Gummiburg verstecken und damit die Niagara Fälle hinunterstürzen. Voller Glück sind sie endlich mal allein und können Liebe machen. Unterdessen wird Grandpa in Miami vom homosexuellen Raoul\index{Raoul} umgarnt.

\notiz{
\begin{itemize}
  \item Lisa liest im Schulbus das Buch \glqq How To Talk To A Drunk Father\grqq\ (Wie spricht man mit einem betrunkenen Vater).
  \item Homer benutzt als Zahlungsmittel Ned Flanders Viza-Card\index{Viza-Card}, wohingegen Bart im Besitz der Viza-Card von Rod Flanders ist.
  \item In der Lackluster\index{Lackluster} Videothek ist ein Futurama-Poster zu sehen.
\end{itemize}
}


\subsection{Der Tortenmann schlägt zurück}\label{FABF15}
Homer kann die Ungerechtigkeit in seiner Umgebung nicht länger ertragen und so beschließt er, in einer Superman-Verkleidung allen Bösewichten Torten ins Gesicht zu klatschen. Schon bald wird er von allen als der \glqq Kuchenmann\index{Kuchenmann}\grqq\ bewundert. Aber niemand weiß, wer es ist, bis Lisa ihm auf die Schliche kommt. Sie nimmt Homer das Versprechen ab, damit aufzuhören. Aber auch Burns hat Homer als Kuchenmann entlarvt und erpresst ihn, allen möglichen Leuten Torten ins Gesicht zu werfen, auch dem Dalai Lama\index{Dalai Lama}. Bei dem hat Homer allerdings Hemmungen und gibt sich öffentlich als Kuchenmann zu erkennen. Die Leute glauben ihm allerdings nicht.

\notiz{
\begin{itemize}
  \item Homer ist Preisrichter auf dem Springfielder Jahrmarkt für Schweine.
  \item In Springfield werden nun Kuchen gegen Waffen eingetauscht (\glqq Guns For Pies\grqq ).
  \item Homer wird als Kuchenmann verkleidet von Lou angeschossen.
  \item Milhouse gibt an, dass seine Mutter und er zur Miete wohnen.
  \item Lisa ist nach eigener Aussage die jüngste Buddhistin in Springfield.
\end{itemize}
}

\subsection{Die erste Liebe}\label{FABF13}
Homer und Marge erzählen ihren Kindern von ihrer ersten Liebe. Die beiden hatten sich in einem Ferienlager kennen gelernt und da fiel auch ihr erster Kuss. Damals versprachen sie einander, sich wiederzusehen, doch Homer ließ das Treffen platzen und sie verloren sich aus den Augen. Marge, tief enttäuscht, verlor daraufhin jegliches Vertrauen in die Männer.

\notiz{
\begin{itemize}
  \item Milhouse behauptet, sein erster Kuss fand zwischen ihm und Homer statt, obwohl er in der Episode \glqq Liebe und Intrige\grqq\ (siehe \ref{8F22}) bereits Samantha Stanky\index{Stanky!Samantha} und in der Folge \glqq Nach Kanada der Liebe\grqq\ (siehe \ref{DABF06}) Greta\index{Greta} geküsst hat.
  \item Bart gibt zu, bereits drei verschiedene Mädchen geküsst zu haben.
  \item Homer sagt, er habe nur eine Frau geküsst. In der Folge \glqq Homer liebt Mindy\grqq\ (siehe \ref{1F07}) küsst er allerdings Mindy Simmons.
  \item Moe wurde als Teenager von seinen Eltern im Jugendlager \glqq Camp See-A-Tree\grqq\ zu\-rück\-ge\-las\-sen. Dort trifft er zwei Jahre später auf Homer, Carl und Lenny.
  \item Homer hatte einen Brieffreund namens Osama.
\end{itemize}
}

\subsection{Geächtet}\label{FABF17}
Als Bart unabsichtlich der amerikanischen Flagge den nackten Hintern zeigt und Marge im Fernsehen erklärt, dass sie Amerika hasst, wird die Familie wegen Landesverrats verhaftet und auf Alcatraz eingesperrt. Es gelingt ihnen auszubrechen und ein französisches Schiff bringt sie nach Frankreich, wo sie ungeniert über die USA lästern können. Doch bald werden Homer und Marge von Heimweh ergriffen. Als Emigranten kehren sie schließlich in die Staaten zurück.

\notiz{
\begin{itemize}
  \item Moe sagt, dass er aus Holland stammt und ein dauerhaftes Visum in Amerika besitzt.
  \item Willie gibt an, dass er seit der Boilerexplosion 1988 taub ist und das Lippenlesen gelernt hat.
  \item Im Gefängnis befinden sich neben den Simpsons noch Michael Moore\index{Moore!Michael}, die Dixie Chicks\index{Dixie Chicks}, Elmo\footnote{Elmo ist eine Figur aus der Sesamstraße. Elmo ist ein rotes Monster mit großen Augen und einer orangen Nase \cite{WikiElmo}.}\index{Elmo} und Bill Clinton\index{Clinton!Bill}.
  \item Springfield wird in \glqq Liberty-Ville\index{Liberty-Ville}\grqq\ umbenannt.
  \item Lenny hat eine Tätowierung auf der Brust, welche die amerikanische Flagge zeigt und auf der Flagge steht: \glqq Dole Kemp 96\footnote{Bob Dole war 1996 der republikanische Gegenkandidat zu Bill Clinton bei der Wahl zum amerikanischen Präsidenten. Die Wahl gewann schließlich Bill Clinton.}\grqq .
  \item Der Name des Talkshowmasters, bei dem Marge gesteht, Amerikaner zu hassen, lautet Nash Castor\index{Castor!Nash}.
  \item Gil arbeitet als Fahrlehrer und sein Fahrschüler ist der pickelige Teenager.
\end{itemize}
}

\subsection{Der große Nachrichtenschwindel}
Als Mr. Burns unter einem Geröllhaufen verschüttet wird, glauben alle, er sei tot. In den Nachrichten wird um ihn nicht getrauert, sondern zum Ausdruck gebracht, dass Springfield über seinen Tod froh sei. Mr. Burns will dies ändern, nachdem Burns dies gehört hat, kauft er sämtliche Medien in Springfield. Lisa bringt ihre eigene Zeitung heraus, um auch eine andere Meinung öffentlich kundzutun. Burns ist darüber erbost und versucht mit allen Mitteln, sie fertig zu machen. Aber mithilfe ihrer Freunde und auch mit Homers, Carls, Lennys und Barneys Unterstützung gelingt es ihr, durchzuhalten, bis Mr. Burns schließlich aufgibt.

\notiz{
\begin{itemize}
  \item Die beliebteste Touristenattraktion in Springfield ist das Felsmassiv \glqq Trottelkopf\index{Trottelkopf}\grqq\ (Geezer Rock\index{Geezer Rock}).
  \item Smithers hat die Tätowierung \glqq Boss Of My Heart\grqq\ auf seiner Brust.
  \item Lisas Zeitung trägt den Namen \glqq The Red Dress Press\grqq . Homer nennt seine Zeitung \glqq The Homer Times\grqq\ und Lenny bringt die Zeitung \glqq The Lenny Saver\grqq\ heraus. Die erste Schlagzeile von Lennys Zeitung lautet: \glqq The Truth About Carl: He's great (Die Wahrheit über Carl: Er ist großartig)\grqq .
  \item Burns gibt an, dass seine Mutter verstorben sei.
\end{itemize}
}


\section{Staffel 16}
 	
\subsection{Vier Enthauptungen und ein Todesfall}
\begin{itemize}
	\item \textbf{The Ned Zone}\\ Als Homer versucht, mit einer Bowlingkugel ein Frisbee vom Dach seines Hauses zu holen, fällt diese seinem Nachbarn Ned Flanders auf den Kopf. Seitdem hat der Visionen von Leuten, die sterben. Er hat u.\,a. die Vision, dass Homer von ihm erschossen wird.
	\item \textbf{Four Beheadings and a Funeral}\\ London, 1890: Ein Prostituierten-Mörder macht die Stadt unsicher. Der Verdacht fällt auf Homer. Er wird verhaftet und zum Tode verurteilt -- zu Unrecht, wie sich noch rechtzeitig herausstellt. 
	\item \textbf{In the Belly of the Boss} \\ Professor Frink hat eine riesige Gesundheitsvitaminkapsel erfunden -- mit fatalen Folgen. Denn er verkleinert Maggie und diese gelangt in den Körper von Mr. Burns. 
\end{itemize}

\notiz{
\begin{itemize}
	\item Die dritte Geschichte ist ein Anspielung auf den Film \glqq Innerspace\grqq\ (\glqq Die Reise ins Ich\grqq) von 1987.
	\item Aufgrund von Geldproblemen wechselten einige Synchronsprecher der amerikanischen Originalstimmen (vor allem die Sprecher lateinamerikanischer Charaktere) nach 15 Jahren.
\end{itemize}
}

\subsection{Die geheime Zutat}\label{FABF20}
Als Marge bei den Nachbarn eine funkelnagelneue Küche sieht, will sie unbedingt auch so eine haben. Voller Begeisterung nimmt sie sogleich an einem Backwettbewerb teil, um sich so ihren Wunsch erfüllen zu können und die neue Miss Ofenfrisch\index{Ofenfrisch} zu werden. Als Marge jedoch von ihren Mitstreitern böse sabotiert wird, greift sie selbst zu unlauteren Mitteln. Doch Lisa kommt ihr auf die Schliche und stellt sie zur Rede.

\notiz{
\begin{itemize}
  \item Die neue Küche der Simpsons kostet 100.000 Dollar.
  \item Eine Großmutter von Milhouse Van Houten wohnt in Omaha.
  \item Im \glqq The New Delhi Daily\grqq\ ist auf der Titelseite zu lesen, dass Steve Barnes\index{Barnes!Steve} (ist in Wirklichkeit Apu) ein Lebensmittelgeschäft eröffnet hat.
  \item Bart wird von Homer aufgeklärt. In der Episode \glqq Liebe und Intrige\grqq\ (siehe \ref{8F22}) zeigt allerdings Barts Lehrerin, Edna Krabappel, den Schülern einen Aufklärungsfilm.
  \item Teilnehmer am Backwettbewerb: Marge Simpson, Luigi Risotto, Waylon Smithers, Jasper, Ruth Powers, Manjula Nahasapeemapetilon, Hans Maulwurf,  Helen Lovejoy, Agnes Skinner, Otto Mann, Küchenhilfe Doris, Dewey Largo und Bernice Hibbert.
  \item Statt Marge wird Cletus Frau Brandine die neue Miss Ofenfrisch.
\end{itemize}
}

\subsection{Der Feind in meinem Bett}\label{FABF19}
Marge fühlt sich von ihrer Familie vernachlässigt und nimmt sich deshalb des verwahrlosten Nelson an. Als Nelson, dessen Eltern verschwunden sind, dann auch noch als neues Familienmitglied bei ihnen einzieht, wird es Lisa und Bart zu bunt. Bart macht sich auf die Suche nach Nelsons Vater, um ihn wieder loszuwerden. Mit Erfolg, denn schließlich entdeckt er ihn bei einem Zirkus und auch seine Mutter kehrt aus Hollywood zurück. Unterdessen glaubt Lisa, zu dick zu sein und etwas gegen ihr Gewicht tun zu müssen.

\notiz{
\begin{itemize}
  \item Homer gibt für Bart eine Party, weil dieser 100 Punkte in einer Schularbeit erreicht hat. Allerdings erreichte Bart bereits in der Folge \glqq Homer und das Geschenk der Würde\grqq\ (siehe \ref{CABF04}) in einem Astronomietest 100 Punkte.
  \item Nelsons Lieblingsessen ist Pfannkuchen.
  \item Im Kühlschrank der Simpsons befindet sich eine Torte, auf der \glqq Happy Labor Day Lenny\grqq\ (Fröhlichen Arbeitertag Lenny) steht.
  \item Nelsons Vater ist auf Erdnüsse allergisch.
  \item In dieser Episode ist wieder Marges Mutter zu sehen. Zuletzt war sie in der sechsten Staffel dabei.
  \item Nelson gibt an, dass er eine Schwester hat. Er weiß aber nicht, ob diese noch lebt oder schon tot ist.
\end{itemize}
}

\subsection{Marges alte Freundin}
Durch Zufall erkennt Marge in einer erfolgreichen Fernsehreporterin ihre alte Freundin Chloe Talbot\index{Talbot!Chloe} wieder. Während Marge auf ihre ehemalige Freundin eifersüchtig ist, bewundert Lisa die toughe Businessfrau. Da macht Chloe der kleinen Lisa ein Angebot: Sie soll sie zu einer UNO-Frauenkonferenz begleiten. Doch unterwegs geraten die beiden in einen Vulkanausbruch. Besorgt um Lisas Leben, begibt sich Marge sofort auf eine Rettungsmission.

\notiz{
\begin{itemize}
  \item Gegen Joe Quimby laufen 27 Vaterschaftsklagen von 27 verschiedenen Frauen.
  \item Vor dem Gerichtsgebäude ist Cookie Kwan\index{Kwan!Cookie} mit einem Baby auf dem Arm zu sehen.
  \item Während der Highschool-Zeit waren Chloe und Barney liiert. Barney machte Chloe sogar einen Heiratsantrag, doch diese lehnte ab.
  \item Als Marge auf die Highschool ging, war sie gemeinsam mit Chloe Reporterin. Beide decken den Skandal in der Schulküche auf, bei dem Moe in die Suppe gespuckt hat.
  \item Im Abspann sind Spongebob\index{Spongebob}, Jesus\index{Jesus} und Buddha\index{Buddha} zu sehen.
  \item Auf dem Fox-Truck ist Wahlwerbung für George Bush jun. und Dick Cheney im Prä\-si\-dent\-schafts\-wahl\-kampf 2004 zu sehen.
\end{itemize}
}
	
\subsection{Dicker Mann und kleiner Junge}
Homer hat seinen Job im Atomkraftwerk verloren und ist jetzt finanziell von Bart abhängig. Der hat nämlich selbst produzierte T-Shirts mit frechen Sprü\-chen an den Großhändler Goose Gladwell\index{Gladwell!Goose} verkauft. Das Geschäft boomt, doch Gladwell betrügt Bart nach Strich und Faden. Jetzt ist Homer gefragt: Mit dem Modell eines Atomkraftwerks, das er eigentlich für Lisas Physikwettbewerb gebaut hat, droht er dem fiesen Gladwell.

\notiz{
\begin{itemize}
  \item Der Titel von Lisas unvollendetem Roman: \glqq They Promised Me Ponies\grqq\ (Sie versprachen mir Ponys).
  \item Einige von Barts T-Shirt-Sprüchen:
  \begin{itemize}
	  \item Life Ends At Ten (Das Leben endet mit 10).
	  \item I've Puked More Beer Then You've Drunk (Ich kotzte mehr Bier als Du getrunken hast).
	  \item Weapon Of Ass Destruction (Waffe der Arschzerstörung).
	  \item Stop World Hunger, Eat My Shorts! (Stop den Welthunger, friss meine Hosen).
   \end{itemize}
  \item Goose Gladwell betreibt nach eigener Aussage 20 Geschäfte in 30 Staaten.
  \item Auf einem der Krusty-T-Shirts ist Poochie aus der Episode \glqq Homer ist \glq Poochie\index{Poochie} der Wunderhund\grq \grqq\ (siehe \ref{4F12}) zu sehen.
  \item Martin Prince hat für ein Schulprojekt einen Roboter namens Chum\index{Chum} konstruiert.
  \item Fehler: In einer Szene ist zu sehen, wie Bart mit links schreibt. In der nächsten Einstellung hat er den Stift in der rechten Hand.
\end{itemize}
}

\subsection{Nach Kanada der Pillen wegen}\label{FABF16}
Mr. Burns kündigt seinen Angestellten die Krankenversicherung. Nun können sich Homer und seine Kollegen ihre Medikamente nicht mehr leisten. Es müssen die billigen Arzneimittel aus Kanada her und Homer erklärt sich bereit, das Schmuggelgeschäft zu übernehmen. Als er aber von Grenzbeamten erwischt wird, müssen sie sich etwas einfallen lassen: Mit Mr. Burns altem Flugzeug wird die Schmuggelaktion fortgesetzt -- doch nicht ganz ohne Komplikationen.

\notiz{
\begin{itemize}
  \item Das hölzerne Flugzeug von Mr. Burns heißt \glqq The Plywood Pelikan\grqq\ (Der Sperrholz-Pelikan).
  \item Wie Smithers in dem Glassarg aufgebahrt ist, erinnert an den Sarg von Lenin.
  \item Homer wird von Mr. Burns zum freiberuflichen medizinischen Gutachter ernannt.
  \item Rod Flanders muss Insulin spritzen, vermutlich ist er zuckerkrank.
  \item Offenbar hat Mr. Burns außer seinem Sohn noch weitere lebende Verwandte. Denn als er aus dem Flugzeug mit dem Fallschirm abspringt, nimmt er die anderen beiden Fallschirme als Geschenk für seine zwei Neffen mit.
  \item Im Flugzeugmuseum ist das Flugzeug \glqq Spirit of Shelbyville\grqq\ zu sehen.
\end{itemize}
}


\subsection{Moes Taverne}
Als Moes Taverne vom Gesundheitsamt geschlossen wird, gibt Homer seinem Freund ein Darlehen, für das er mit seinem Haus bürgt. Marge ist empört über Homers rücksichtsloses Verhalten. Um die Investition im Auge zu behalten, steigt sie bei Moe als Partnerin ein. Die beiden verstehen sich von Tag zu Tag besser, aber als sie gemeinsam zu einem Geschäftstermin nach Aruba fliegen, wird es Homer zu bunt. Von Eifersucht geplagt, versucht er, Marge mit all seiner Liebe zurückzugewinnen.

\notiz{
\begin{itemize}
  \item Die Schwulenkneipe, in die Barney, Carl und Lenny gehen heißt \glqq League Of Extra-Horny Gentlemen\grqq\ (Liga der extrageilen Gentlemen).
  \item Nachdem aus Moes Taverne ein britisches Pub wurde, heißt es \glqq The Nag and Weasel\grqq\ (Der Gaul und das Wiesel).
  \item Richter Snyder und Lindsey Naegle flirten in Moes Pub.
  \item An der Kneipen- und Restaurantbesitzerkonferenz in Aruba nehmen neben Marge und Moe noch Krusty, Captain McCallister, Luigi Risotto und Akira teil.
  \item Marges Lieblingsessen sind Butternudeln.
  \item Lenny, Carl und Homer besuchen Itchy \& Scratchy-Land. Zuletzt war dieses in der Episode \glqq Der unheimliche Vergnügungspark\grqq\ (siehe \ref{2F01}) zu sehen.
\end{itemize}
}

\subsection{Homer und die Halbzeit-Show}\label{GABF02}
Bei einer Veranstaltung im Springfield Park zu dessen Rettung bricht Homer in einen wilden Freudentanz aus, nachdem er bei einem Spiel gegen Bart gewinnt. Flanders hat seine skurrile Performance aufgezeichnet und der Comicbuchverkäufer hat sie ins Internet gesetzt und damit das Interesse der Sportwelt auf Homer gelenkt. Schließlich bekommt er das fantastische Angebot, die Halbzeit-Show des Superbowls zu choreographieren. Homer fällt aber nichts Besseres ein, als zusammen mit Flanders ein großes Bibelspektakel aufzuziehen.

\notiz{
\begin{itemize}
  \item In dieser Folge nennt der Comicbuchverkäufer seinen richtigen Namen: Jeff Albertson\index{Albertson!Jeff}.
  \item Die Adresse, auf der Homers Video zu finden ist: \url{www.dorks-gone-wild.com}.
  \item Wie Homer mit Mülltonnen nach dem kleinen Italiener wirft, ist eine Anspielung auf das Computerspiel \glqq Super Mario Bros\grqq .
  \item Zu den Personen, die sich von Homer choreographieren lassen, gehören: Tom Brady, Deion Overstreet, Yao Ming, Warren Sapp, Michelle Kwan und Lenny.
  \item Die Super Bowl Halbzeitshow wird von Ford, Citibank und Moe's Taverne gesponsert.
  \item Der erste Film, den Ned Flanders dreht, heißt \glqq The Passion Of Cain And Abel\grqq\ (Die Passion von Kain und Abel).
  \item Der zweite Flanders Film \glqq Tales Of The Old Testament\grqq\ (Geschichten aus dem alten Testament) wird von Mr. Burns produziert, der damit Geld waschen will, das er verdient hat, indem er Wasser als Grippeimpfstoff verkauft hat.
\end{itemize}
}

\subsection{Pranksta Rap}\label{GABF03}
Bart hat eine Reklame für ein Rap-Konzert gesehen und möchte nun unbedingt hin. Als Marge es ihm verbietet, obwohl Homer es ihm vorher erlaubt hatte, wenn er die Karte für das Konzert selbst bezahlt, schleicht er sich aus dem Haus und steht kurz darauf mit der Gruppe Alcatraaaz\index{Alcatraaaz} auf der Bühne. Nach dem Konzert muss Bart wieder nach Hause. Doch wie soll er Marge und Homer sein Verschwinden erklären? Kurz entschlossen täuscht er vor, entführt worden zu sein. Er versteckt sich bei Milhouses Vater in dessen Wohnung und so wird Kirk als Entführer verhaftet. Clancy Wiggum wird für die \glqq erfolgreiche\grqq\ Aufklärung des Verbrechens zum Commissioner befördert und Lou zum Polizeichef.

\notiz{
\begin{itemize}
	\item Filmzitat I: Im Fernsehen läuft \glqq The Salad of The Christ\grqq\ (Der Salat des Christus) eine Anspielung auf den Film \glqq Die Passion Christi\grqq\ von Mel Gibson.
	\item Filmzitat II: Die Szene, in der Bart mit Alcatraaaz auf der Bühne rappt, erinnert an Eminems Film \glqq 8 Mile\grqq .
	\item Zu den Frauen, die Kirk Van Houten, attraktiv finden, als er im Gefängnis sitzt, gehören u.\,a. Agnes Skinner, Lindsey Naegle, Cookie Kwan, Miss Hoover und Ruth Powers.
	\item Im Abspann der Folge sagt Oberschulrat Chalmers, dass er verheiratet und seine Frau schwer krank sei.
	\item Das Popcorn der Marke \glqq Chinsey Pop\index{Chinsey Pop}\grqq\ kaufen im Kwik-E-Mart nur Clancy Wiggum und Kirk Van Houten.
\end{itemize}
}


\subsection{Drum prüfe, wer sich ewig bindet}\label{GABF04}
Ein negativer Fernsehbericht über Springfield sorgt dafür, dass das Tourismusgeschäft einbricht. Um das Image der Stadt wieder aufzupolieren, schlägt Lisa Bürgermeister Quimby vor, gleichgeschlechtliche Ehen zu genehmigen. Alle sind dafür, nur Reverend Lovejoy nicht. Homer, der in den Eheschließungen ein einträgliches Geschäft wittert, beschließt Priester zu werden. Auch Patty will bei ihm heiraten -- ihre Freundin Veronica\index{Veronica}. Marge ist entsetzt. Es stellt sich jedoch heraus, dass Veronica in Wirklichkeit ein Mann ist und unter diesen Umständen will Patty nicht mehr heiraten.

\notiz{
\begin{itemize}
  \item Auf dem Gerichtsgebäude steht \glqq Liberty And Justice And For Most\grqq\ (Freiheit und Gerechtigkeit und für die Meisten).
  \item Auf Homers Karte der Stars in Springfield sind Carl und Lenny aufgeführt. Aus der Karte geht hervor, dass beide Nachbarn sind.
  \item Springfield liegt an der Route 202.
  \item Springfields Web-Adresse in dieser Episode:\\ \url{http://wwww.springfieldisforgayloversoftmarriage.com}
  \item Selma war kurzzeitig mit Disco Stu verheiratet.
  \item Gäste bei Pattys Hochzeit sind u.\,a. Agnes und Seymour Skinner, Marges Mutter und Sideshow Mel.
\end{itemize}
}

\subsection{Die böse Hexe des Westens}\label{GABF05}
Da Lisa immer wieder von Bart geärgert wird, erwirkt sie eine einstweilige Verfügung gegen ihn. Das bedeutet, er darf sich ihr nicht mehr als sechs Meter nähern. Dies wird später auf 61 Meter (200 Fuß) erweitert, sodass Bart im Garten in einem Zelt schlafen muss. Auch die Schule darf er nicht mehr besuchen, da Lisa sich dort ebenfalls aufhält. Und so lebt er mit den Tieren im Garten. Als er eine riesige Statue von Lisa errichtet, lässt diese sich erweichen und verzeiht ihm, nicht ahnend, dass er diese mit seinen Freunden abfackeln will.

\notiz{
\begin{itemize}
  \item Der Park Ranger, der für den Springfield Gletscher zuständig ist, heißt Johnson\index{Johnson}.
  \item Abe Simpsons arbeitet im Sprawl Mart. Als er sich verletzt, wird er von Homer vertreten.
  \item Doktor Hibberts Haushälterin verklagt ihn wegen sexueller Belästigung.
  \item Richterin Harm gibt an, dass sie verheiratet sei.
  \item Homer hat drei Jahre lang in einem Nachtrestaurant gearbeitet.
\end{itemize}
}

\subsection{Der lächelnde Buddha}\label{GABF06}
Selma ist in den Wechseljahren und leidet sehr darunter, weil sie sich immer ein Kind gewünscht hat und keines hat. So entschließt sie sich, ein Kind zu adoptieren. Und dies soll am einfachsten in China sein. So fliegt sie mit der gesamten Familie Simpson nach Peking, wobei Homer sich als ihr Ehemann ausgeben muss. Alles verläuft nach Plan: Sie bekommt die kleine Ling\index{Ling} und ist überglücklich. Aber als die Funktionärin Wu\index{Wu} dahinterkommt, dass Homer eigentlich mit Marge verheiratet ist, nimmt sie ihr das Kind wieder weg. Selma ist am Boden zerstört. Sie versuchen, die kleine Ling zu entführen und werden dabei erwischt. Voller Mitleid überlässt ihr schließlich Madame Wu die Kleine, da sie selbst auch nur eine allein erziehende Mutter hatte.

\notiz{
\begin{itemize}
	\item Mr. Burns lässt seinen 1909 abgelaufenen Führerschein verlängern. Seine Führer\-schein\-num\-mer ist die 000044.
	\item In Peking gibt es einen Imbiss mit dem Namen \glqq Krusty Fried Chicken\grqq .
	\item Diese Folge ist bei Disney+ in Hongkong nicht verfügbar. Es ist nicht klar, ob der US-Konzern die Episode dabei aus eigener Entscheidung gesperrt hat oder von den chinesischen Behörden dazu aufgefordert wurde \cite{Hongkong}.
\end{itemize}
}

\subsection{Homer mobil}\label{GABF07}
Als Homer, vom Pech verfolgt, dauernd dem Tode nahe ist und keine Lebensversicherung hat und auch in keine Lebensversicherung mehr aufgenommen wird, kriegt Marge einen Spartick. Homer ist empört und die beiden fangen an zu streiten, bis sich Homer schließlich in seiner Wut einen Wohnwagen kauft, um von nun an darin zu wohnen. Bart und Lisa sind besorgt, dass die Eltern sich scheiden lassen wollen und beschließen, den Wohnwagen zurückzugeben. Von ihren Eltern verfolgt, landen sie schließlich auf einem türkischen Frachter. Hier wendet sich alles zum Guten und die Familie ist wieder versöhnt.

\notiz{
\begin{itemize}
	\item Als Homer, Bart und Lisa im Wohnwagen tanzen, ist \glqq Welcome to the Jungle\grqq\ von Guns N' Roses zu hören. Dieses Lied wurde bereits in \glqq Die rebellischen Weiber\grqq\ (siehe \ref{1F03}) gespielt.
	\item Marge erwähnt Homer gegenüber, dass sie auch Sideshow Mel hätte heiraten können.
	\item Homer gab im letzten Jahr 5000 Dollar für Donut aus.
	\item Am Anfang der Folge öffnet Marge ein Fotoalbum, worin Homer auf einem Bild von Mickey Maus und Pluto verprügelt wird.
	\item Homer gibt an, bis jetzt drei Schlaganfälle gehabt zu haben.
\end{itemize}
}

\subsection{Homer, die Ratte}\label{GABF08}
Als das Konzertgebäude in Springfield Konkurs macht und Mr. Burns es in ein Gefängnis umwandelt, dauert es nicht lange, bis auch Homer dort als Häftling einsitzt. Aber um ein besseres Leben führen zu können, verdingt er sich bei Mr. Burns als Gefängnisspitzel. Dies läuft auch eine Weile ganz gut, bis seine Mithäftlinge dahinterkommen. Sie wollen ihn umbringen. In letzter Sekunde erscheint Marge als rettender Engel. Homer wird verschont und die auf\-rühr\-er\-isch\-en Häftlinge werden begnadigt.

\notiz{
\begin{itemize}
  \item Neben Marge als Vorsitzender gehören dem Kulturausschuss noch Helen Lovejoy, Ned Flanders, Carl Carlson, Moe Szyslak, Dr. Julius Hibbert und Seymour Skinner an.
  \item Fat Tony hat einen Sohn namens Michael\index{D'Amico!Michael}.
  \item In dieser Folge ist gegen Ende die Gouverneurin Mary Bailey\index{Bailey!Mary} zu sehen, die in der Episode \glqq Frische Fische mit drei Augen\grqq\ (siehe \ref{7F01}) die Wahl gegen Mr. Burns gewonnen hatte.
  \item Homer wird aufgrund eines Gesetzes von 1911 inhaftiert, welches das fünfmalige Treten gegen dieselbe Dose als unerlaubte Müllentsorgung verbietet. 
  \item Filmzitat: Als Homer ins Gefängnis kommt, wird er mit einem Feuerwehrschlauch \glqq ge\-säu\-bert\grqq , ähnlich wie Silvester Stallone in Rambo I.
  \item Auf einem Schild ist zu lesen \glqq Must Know Microsoft Powerpoint, Sadism\grqq\ (Muss Microsoft Powerpoint und Sadismus kennen).
\end{itemize}
}


\subsection{Future-Drama}\label{GABF12}
Professor Frink hat eine Maschine erfunden, mit der man in die Zukunft blicken kann. Zum Erschrecken der Familie Simpson wird ihr ganzes Leben auf den Kopf gestellt. Aber um dies zu verhindern, bemüht sich jeder, dem anderen zu helfen. Bart verhindert, dass Lisa Milhouse heiratet und somit in Yale studieren kann. Lisa bewahrt Bart vor seiner schrecklichen Freundin Jenda\index{Jenda}. Alle bemühen sich, für ihr zukünftiges Leben netter miteinander umzugehen.

\notiz{
\begin{itemize}
  \item Der Originaltitel ist eine Anspielung auf Matt Groenings zweite TV-Serie namens \glqq Futurama\grqq .
  \item Wiggum, Lou und Eddie sehen aus wie Robocop.
  \item Cletus ist Vizepräsident der USA.
  \item Jenda ging vorher mit Todd Flanders.
  \item Auf dem Abschlussball spielt die \glqq The Larry Davis iPod Experience\grqq .
  \item Bender\index{Bender} aus Futurama fährt kurz mit Homer und Bart im Schwebeauto mit. Bender ist außerdem in den Folgen \glqq Der beste Missionar aller Zeiten\grqq\ (siehe \ref{BABF11}) und \glqq Klassenkampf\grqq\ (siehe \ref{DABF20}) zu sehen.
  \item Kearney ist Co-Rektor an der Springfielder Grundschule.
  \item Smithers ist mit einer Frau zu sehen. Bart spricht ihn auf seine Homosex\-ualität an, daraufhin antwortet er, wenn er sich alle zehn Minuten eine Spritze setze, sei er heterosexuell.
\end{itemize}
}

\subsection{Der eingebildete Dachdecker}\label{GABF10}
Das Dach der Simpsons hat ein Loch und Marge verlangt, dass Homer es repariert. Zufällig lernt der in der Kneipe Knockers\index{Knockers} den Dachdecker Ray Magini\index{Magini!Ray} kennen und der verspricht ihm zu helfen. Aber während der Arbeit verschwindet Ray und taucht nicht wieder auf. Alle glauben nun, dass Homer einen Dachschaden hat und sich diesen Ray nur einbildet. Dr. Hibbert behandelt Homer mit mehreren Elektroschocks, bis dieser nicht mehr an die Existenz von Ray glaubt. Als der dann plötzlich doch wieder auftaucht, sind alle anderen schockiert und zur Strafe muss Dr. Hibbert das Dach der Simpsons reparieren.

\notiz{
\begin{itemize}
	\item Der Originaltitel \glqq Don't Fear the Roofer\grqq\ ist eine Anspielung auf den 70er Jahre Hit namens \glqq Don't Fear the Reaper\grqq\ von der Rockband Blue Öyster Cult.
	\item Homer wird im \glqq Calmwood Mental Hospital\grqq\ behandelt.
	\item Lisa behauptet, Ray Magini sei ein Anagramm für \glqq Imaginary\grqq\ (Einbildung).
	\item Stephen Hawking wohnt in Springfield und ist Besitzer der Pizzeria \glqq Little Caesar's\grqq .
	\item Homers Zutaten für seine Cola: \glqq Sugar, Water, Bubbles and ?\grqq\ (Zucker, Wasser, Bläschen und ?).
	\item Dies ist die 350. Folge der Simpsons.
\end{itemize}
}

\subsection{Das große Fressen}
Da Bart zu viel Mistkram isst, wird er von seinen Eltern in ein Abmagerungsheim geschickt. Hier hat er es mit Tab Spangler\index{Spangler!Tab}, einem sehr radikalen Aufseher, zu tun. Als der ihm vorführt, wie sehr Homer und Marge darunter leiden, die Kosten für seine Umerziehung aufzubringen, schwört er, nie wieder irgendwelchen Mistkram zu essen. Denn seine Eltern müssen ihr Haus in eine Jugendherberge umwandeln und haben es dort überwiegend mit europäischen Gästen zu tun.

\notiz{
\begin{itemize}
  \item Bart erleidet einen Herzanfall.
  \item Bart ist allergisch gegen Blumenkohl.
  \item Neben Bart sind in dem Abmagerungslager noch Kent Brockman, Apu Nahasapeemapetilon und Rainier Wolfcastle.
  \item Skinner ist im Sternbild der Waage und Oberschulrat Chalmers im Sternbild des Schützen geboren.
  \item Homer singt \glqq 99 Luftballons\footnote{Dieses Lied, ist das einzige Deutschsprachige, das jemals auf Platz 1 der amerikanischen Single-Charts war.}\grqq\ von Nena im englischen Original auf deutsch.
\end{itemize}
}

\subsection{Lisa Simpson: Superstar}\label{GABF13}
Lisa will an einem Gesangswettbewerb teilnehmen und Homer schreibt die Songs für sie. Die sind so gut, dass sie problemlos ins Finale kommt. Aber zuvor streitet sie sich mit Homer und entlässt ihn. Er schreibt nun Songs für ihren Konkurrenten Cameron\index{Cameron}. So ist Lisa gezwungen, sich selbst ein Lied fürs Finale zu schreiben. Es wird eine Liebeserklärung an ihren Dad und das Publikum ist hellauf begeistert, während Homer für Cameron einen egoistischen, überheblichen Song schreibt, sodass der vom Publikum ausgebuht wird. Lisa ist überglücklich und versöhnt sich wieder mit ihrem Dad.

\notiz{
\begin{itemize}
  \item Bei der Vorausscheidung hat Ralph die Nummer 34, Milhouse die Nummer 13 und Lisa die Nummer 65.
  \item Die drei Finalisten sind Lisa Simpson, Clarissa Wellington\index{Wellington!Clarissa} und Cameron, der später den Künstlernamen Johnny Rainbow\index{Rainbow!Johnny} führt.
  \item Als Homer der Manager von Johnny Rainbow ist, trägt der den Künst\-ler\-na\-men Colonel Cool. Dies ist eine Anspielung auf Elvis Presleys Manager Colonel Tom Parker.
\end{itemize}
}

\subsection{Das Jüngste Gericht}
Als Homer einen Film über die Apokalypse sieht, schläft er ein und träumt davon, dass der Weltuntergang kurz bevorsteht. Er warnt seine Mitbürger und will damit erreichen, dass diese zusammen mit ihm in den Himmel auffahren. Aber als die Apokalypse nicht eintritt, glaubt ihm niemand mehr. Schließlich fährt er allein in den Himmel auf und muss zusehen, wie seine Familie auf der Erde umkommt. So bittet er Gott, ihn wieder auf die Erde zurückzuschicken und die Apokalypse rückgängig zu machen. Der willigt ein und Homer fühlt sich in Moes Taverne wieder wie im Himmel.

\notiz{
\begin{itemize}
  \item Die letzte Szene, als Homer mit Freunden in Moes Taverne Bier trinkt, erinnert an das Bild \glqq Das letzte Abendmahl\grqq\ von Leonardo da Vinci.
  \item Der von Homer errechnete Termin des jüngsten Gerichts ist Mittwoch, 18. Mai um 15:15 Uhr. Seine Rechnung:
  \begin{eqnarray*}
			12 - 333 \times 666 &=\\
			(a \times 2b^2) > 2 &= 7\\
			(2 + b^2) - 144,000\\
			- 0 = 3:15,05/18
	\end{eqnarray*}
	Die korrigierte Rechnung, nachdem er festgestellt hatte, dass auf dem Gemälde \glqq Das letzte Abendmahl\grqq\ 13 Personen enthalten sind:
	\begin{eqnarray*}
			13 - 333 \times 666 &=\\
			(a \times 2b^2) > 2 &= 7\\
			(2 + b^2) - 144,000\\
			- 0 = 3:15,05/19
	\end{eqnarray*}
	\item Im Zeppelin sind neben Krusty unter anderem Ron Howard, Joan Rivers, Rosie O'Donnell und Jennifer Garner zu sehen.
\end{itemize}
}

\subsection{Schau heimwärts, Flanders}\label{GABF15}
Ned hat zwei junge und hübsche Studentinnen als Untermieterinnen ins Haus genommen. Nicht ahnend, dass die mit einer Videokamera ihre freizügigen Sexspielchen ins Internet stellen. Und so wird Ned zum Gespött der Einwohner von Springfield. Empört beschließt er, mit seinen Söhnen, Rod und Todd, aus Springfield nach Humbleton\index{Humbleton} wegzuziehen. Homer ist zutiefst betroffen, weil er Ned nicht rechtzeitig aufgeklärt hat, was in seinem Hause vorgeht. Er bittet ihn um Verzeihung und überredet ihn, wieder nach Springfield zurückzukehren.

\notiz{
\begin{itemize}
  \item Die Simpsons schauen sich im Kino den Film \glqq Kosovo Autumn\grqq\ an. Im Kino ist auch der Comicbuchverkäufer zu sehen.
  \item Ned Flanders ist offensichtlich Anhänger der Republikaner.
  \item Flanders trägt in der Badewanne eine Badehose.
  \item Die Website der Studentinnen \url{www.SexySlumberParty.com}.
  \item Der Ringertrainer Clay Roberts\index{Roberts!Clay} kauft Neds Haus.
  \item Todd behauptet, er habe eine Freundin in Humbleton gefunden.
  \item Die Schlagzeile der Zeitung in Humbleton lautet \glqq Hair Führer\grqq\ mit einem Bild von Ned.
  \item Nelson gibt an, dass seine Mutter auf Entzug sei.
\end{itemize}
}

\subsection{Der Vater, der Sohn und der heilige Gaststar}\label{GABF09}
Da Bart wegen eines Streichs, den er nicht begangen hat, von der Schule verwiesen wird, beschließen Homer und Marge, ihn in ein katholisches Internat zu geben. Als Bart den Pater Sean\index{Sean} kennenlernt, beschließt er, gegen den Willen seiner Eltern, katholisch zu werden. Empört stellt Homer Pater Sean zur Rede, aber als es aus der Internatsküche nach Pfannkuchen riecht, beschließt auch Homer, katholisch zu werden. Es kommt zu einem ausgiebigen Streit zwischen Katholiken und Protestanten, bis Bart beide Parteien dazu bringt, ihre Auseinandersetzungen zu beenden, da es eigentlich mehr Gemeinsamkeiten als Streitpunkte gibt.

\notiz{
\begin{itemize}
	\item Gerade mal eine Woche nach dem Tod von Papst Johannes Paul II (gestorben am 02. April 2005) sollte in den USA diese Episode zum ersten Mal ausgestrahlt werden und zwar am 10. April 2005. Fox hat diese Episode jedoch erst mal zurückgezogen, da man sie nicht so kurz nach dem Tod des Papstes zeigen wollte. Stattdessen wurde eine Wiederholung einer anderen Folge dieser Staffel gezeigt. Die Episode wurde schließlich am 15. Mai 2005 ausgestrahlt und enthielt keine Szene mehr mit dem verstorbenen Papst.
	\item Homers Lieblingsspiel ist Bingo.
	\item Der Tanz, den die Katholiken in Marges Vorstellung aufführen, ist eine Anspielung auf Riverdance.
	\item Im Original sagt Homer: \glqq Welcome to the jungle, Kevin.\grqq\  In zwei Episoden war bereits das Lied \glqq Welcome To The Jungle\grqq\ von Guns N' Roses zu hören und zwar in \glqq Die rebellischen Weiber\grqq\ (siehe \ref{1F03}) und in \glqq Homer mobil\grqq\ (siehe \ref{GABF07}).
	\item Hausmeister Willie trägt Kontaktlinsen. 
	\item In der Episode \glqq Das Schlangennest\grqq\ (siehe \ref{9F18}) wurde Bart bereits einmal der Schule verwiesen und auf ein katholisches Internat geschickt.
\end{itemize}
}


\section{Staffel 17}

\subsection{Es lebe die Seekuh!}\label{GABF18}
Nach einem heftigen Streit mit Homer haut die wütende Marge von zu Hause ab und lernt in einem Restaurant den Seekuh-Beschützer Caleb\index{Caleb} kennen. Zum Streit mit Homer kam es, weil Homer Schulden bei der Mafia damit beglich, dass er sein Haus zum Dreh eines Pornofilms zur Verfügung stellte. Begeistert schließt sich Marge Caleb an und betreut von nun an Seekühe. Homer macht sich daraufhin mit den Kindern auf die Suche und findet schließlich seine Ehefrau. Um sie zurückzugewinnen, hilft er, Seekühe vor angriffslustigen Jetski-Fahrern zu bewahren. Die verprügeln ihn zwar, hauen aber ab -- Marge ist begeistert.

\notiz{
\begin{itemize}
	\item Lenny und Carl wirken auch bei der Produktion des Pornofilms mit.
	\item Der Pornofilms heißt \glqq Lemony Lickit: A Series of Horny Events\grqq , eine Anspielung auf das Buch und den Film \glqq Lemony Snicket's A Series of Unfortunate Events\grqq .
	\item Moes Telefonnummer ist die 3551337. In der Folge \glqq Butler bei Burns\grqq\ (siehe \ref{3F14}) hat Moe die 76484377 als Telefonnummer.
\end{itemize}
}

\subsection{Angst essen Seele auf}\label{GABF16}
Neben das Haus der Simpsons soll ein Briefmarkenmuseum gebaut werden. Nachdem die Simpson sich erfolgreich dagegen gewehrt haben, entsteht stattdessen ein Friedhof an selber Stelle. Deshalb leidet Lisa unter Schlaflosigkeit. Sie hat Wahnvorstellungen und Geisterträume. Um ihre heimlichen Ängste zu besiegen, beschließt sie, eine Nacht auf dem Friedhof zu verbringen. Hier erscheinen ihr unheimliche Geister, die sich aber bei näherem Kennenlernen als lustig und nett entpuppen. Schließlich schläft sie auf einem Grab ein.

\notiz{
\begin{itemize}
	\item Hausmeister Willies Cousin ist Totengräber Billy\index{Billy}.
	\item Homers PIN ist die 7431.
	\item Apu ist mit dem Schild \glqq Make War Not Stamps\grqq\ (Macht Krieg, keine Briefmarken) zu sehen.
	\item Das Briefmarkenmuseum wird stattdessen neben Lennys Haus gebaut.
	\item Der Polizist Lou wollte eigentlich Jura studieren, konnte es sich aber nicht leisten.
	\item Fehler: Als Lisa unter die Decke ihrer Eltern kriecht, während Homer und Marge nach Hause kommen, ist die Bettdecke blau. Als sie ins Schlafzimmer kommen, ist sie orange.
\end{itemize}
}

\subsection{Milhouse aus Sand und Nebel}\label{GABF19}
Maggie hat die Windpocken. Flanders überzeugt Homer, eine Pocken-Party zu veranstalten. Milhouses Eltern haben dort nach langer Trennung wieder zusammengefunden. Das wiederum gefällt Bart ganz und gar nicht. Um sie wieder auseinander zu bringen, versteckt er einen BH seiner Mutter Marge im Ehebett der Van Houtens. Empört läuft Luann Van Houten zu Homer, um sich über dessen Frau zu beschweren. Nun ist Homer der Eifersüchtige und er glaubt Marge nicht, dass sie ihn nicht betrogen hat. Deshalb trennt sie sich von ihm. Das wiederum passt Bart ebenfalls ganz und gar nicht.

\notiz{
\begin{itemize}
	\item Maggie steckt Homer mit den Windpocken an, die er bis dahin nicht gehabt hatte.
	\item Lisa war in der Kirche zu sehen, obwohl sie Buddhistin ist.
	\item Milhouse kann eigentlich die Windpocken nicht mehr bekommen, da er bereits in der Episode \glqq Liebhaber der Lady B.\grqq\ (siehe \ref{1F21}) die Windpocken hatte.
	\item Fehler: Als Homer den Kühlschrank leert, sind seine Windpocken schlagartig verschwunden.
\end{itemize}
}

\subsection{Mamas keiner Liebling}
Marge kauft ein Tandem und Homer eine Hantel, was dann Grund genug ist, nicht mit Marge zu radeln. Irgendwann hat Bart Mitleid mit seiner Mutter und fährt mit ihr durch die Gegend. Bart entwickelt sich so zu einem echten Muttersöhnchen. Er macht mit Marge Ausflüge auf dem Tandem und beide gehen Tee trinken. Marge baut Barts Baumhaus in ein Teehaus um. Als Marge beim Karaokewettbewerb in der Schule Rektor Skinner mit seiner Mutter auftreten sieht, will sie nicht, dass ihr Sohn einmal so endet und ermutigt ihn deshalb, wieder der \glqq alte\grqq\ Bart zu sein. Homer ist unterdessen mit Moe unterwegs, um an Wettbewerben im Armdrücken teilzunehmen.

\notiz{
Marges Ziele:
\begin{itemize}
	\item Ein Tandem kaufen.
	\item Das heilige Land besuchen und sicher wieder zurückkehren.
	\item Irgendetwas downloaden.
	\item Niemals die Frisur ändern.
	\item Weizenbrot probieren.
\end{itemize}
}

\subsection{Der Sicherheitssalamander}\label{GABF21}
Homer hat die Liebe seiner Tochter verloren. Um sie zurückzugewinnen, meldet er sich als \glqq Sicherheitssalamander\grqq , der bei allen Notfällen in der Schule einspringt. Sein Einsatz geht so weit, dass er schließlich der ganzen Stadt beisteht und in schwersten Fällen als Lebensretter auftritt.
Da Bürgermeister Quimby bei der Bevölkerung in Misskredit geraten ist, werden Neuwahlen angesetzt, zu denen sich unter anderem auch Homer als Kandidat aufstellen lässt. Doch als Marge sein Salamanderkostüm irgendwann in die Waschmaschine steckt, läuft es ein und platzt während seiner Bürgermeister-Wahlkampagne.

\notiz{
\begin{itemize}
	\item Dolph ist offensichtlich Jude, da er zum Hebräischuntericht muss.
	\item In dieser Episode wird der Name der Katzenlady genannt: Eleanor Abernathy\index{Abernathy!Eleanor}.
	\item Der erste Auftritt von Schulpsychologe Dr. Pryor seit der Folge \glqq Die Saxophon-Geschichte\grqq\ (siehe \ref{3G02}) in der neunten Staffel.
	\item Carl gesteht, dass er ein Gecko-Liebhaber ist.
\end{itemize}
}

\subsection{Marge und der Frauen-Club}\label{GABF22}
Auf der Suche nach einer neuen Freundin gerät Marge an den Damen-Club \glqq Die munteren roten Tomaten\grqq . Die haben den Plan, zur Geldversorgung für das Krankenhaus Mr. Burns Fabergé-Eier zu klauen. Und da Marge die Schlankeste der Damenrunde ist, überträgt man ihr den Auftrag. Es kommt, wie es kommen muss: Die Diebinnen werden auf frischer Tat von der Polizei ertappt. Allerdings lässt Mr. Burns sie ungeschoren davonkommen, da sie ihm die Eier wieder zurückgeben. Allerdings kann Marge doch ein Ei in ihren Haaren mitgehen lassen.

\notiz{
\begin{itemize}
	\item Milhouse voller Name lautet Milhouse Mussolini Van Houten\footnote{Benedito Mussolini war während des Zweiten Weltkriegs faschistischer Diktator in Italien.}.
	\item Der Damenclub trifft sich im Restaurant \glqq Johnny Fiestas\grqq \index{Johnny Fiestas}.
	\item In dieser Episode küsst Lisa Milhouse.
\end{itemize}
}

\subsection{Der italienische Bob}\label{HABF02}
Mr. Burns hat einen italienischen Sportwagen der Firma Lamborgotti\index{Lamborgotti} bestellt und schickt Homer samt Familie nach Italien, um das Auto dort abzuholen und ein paar Wochen Urlaub zu machen. Homer baut mit dem Wagen einen schweren Unfall und als er ihn in dem Dorf Salsiccia\index{Salsiccia} reparieren lassen will, trifft er dort auf den Bürgermeister: Tingeltangel-Bob, der Bart immer ermorden wollte. Aber er hat sich geändert und ist ein großzügiger und toleranter Mensch geworden. Unter der Bedingung, dass die Simpsons nicht verraten, dass er früher ein Gangster war, erklärt er sich bereit, den Wagen reparieren zu lassen. Beim Abschlussfest bietet er allen Wein an -- auch den Kindern. Vom Alkohol beschwipst, verrät Lisa lautstark, dass Tingeltangel-Bob ein Schwerverbrecher war.

\notiz{
\begin{itemize}
	\item Gesuchte amerikanische Verbrecher in Italien: Bürgermeister Quimby, Peter Griffin\index{Griffin!Peter} aus der Serie \glqq Family Guy\index{Family Guy}\grqq , Snake und Stan Smith\index{Smith!Stan} aus der Serie \glqq American Dad\index{American Dad}\grqq .
	\item Kelsey Grammer\index{Grammer!Kelsey}, der Sprecher von Tingeltangel-Bob, gewann für diese Folge einen Emmy.
	\item Tingeltangel-Bobs Gefangenennummer ist die HABF02, genau wie der Produktionscode dieser Folge.
	\item Tingeltangel-Bob hat eine Braut namens Francesca\index{Francesca} und einen Sohn Gino\index{Gino}.
	\item Polizeichef Wiggum hat einen Neffen namens Marc.
\end{itemize}
}


\subsection{Wer ist Homers Vater?}\label{HABF03}
Als das Eis auf dem Gipfel des Mount Springfield schmilzt und dort ein eingefrorener Postbote mit Briefen gefunden wird, erhalten viele Einwohner 40 Jahre alte Briefe. Auch Abe Simpson: Er bekommt einen Liebesbrief, der an seine Frau Mona adressiert war. Daraufhin wird Homer unsicher, ob Abe überhaupt sein echter Vater ist. Aufgrund der Unterschrift \glqq M.\grqq\ unter den Liebesschwüren gerät er an einen Mr. Mason Fairbanks\index{Fairbanks!Mason}, den er für seinen richtigen Vater hält. Der Forscher Mr. Mason ist ein reicher Mann und besitzt ein riesiges Schiff. Und nach einem DNS-Test bei Dr. Hibbert stellt sich heraus, dass Mr. Mason wohl tatsächlich Homers Vater ist. Ist er aber doch nicht, da Abe die Namensschilder auf den Röhrchen vertauscht hat.

\notiz{
\begin{itemize}
	\item Professor Frink wirkte in den 60er Jahren an der Entwicklung der Napalmbombe mit.
	\item Abe schreibt sich auf der Liste derjenigen, die einen Herzspender suchen, an die erste Stelle und zwar vor Mr. Burns, Jasper Beardsley und Jebediah Springfield IV.
	\item Die Maut beträgt 75 Cent.
	\item In einer Rückblende ist zu sehen, dass Abe bei der Hochzeit von Homer und Marge dabei war. In der Episode \glqq Kampf dem Ehekrieg\grqq\ (siehe \ref{7F20}) ist allerdings kein Verwandter bei der Hochzeitszeremonie zu sehen.
\end{itemize}
}


\subsection{B.I.: Bartifical Intelligence}
\begin{itemize}
  \item \textbf{B.I.: Bartificial Intelligence}\\ Kang und Kodos beobachten ein Baseballspiel, finden es langweilig und lassen es in der Atmosphäre verschwinden. Die Bart-Intelligenz setzt ein. Bart verschwindet und den Simpsons wird ein Roboter-Bart zugeteilt. Der Rest der Familie ist zunächst ganz begeistert von ihm, aber als Bart selbst wieder auftaucht, kommt es zwischen ihm und dem Roboter zum Streit.
  \item \textbf{Survival of the Fattest}\\ Mr. Burns macht Jagd auf seine Angestellten und andere bekannte Bewohner von Springfield. Alle haben Angst, erschossen zu werden -- meistens zurecht\dots
  \item \textbf{I've Grown a Costume on Your Face}\\ Alle haben sich zu Halloween mit lustigen Kostümen verkleidet. Als eine Hexe den ersten Preis gewinnen soll, dann aber doch abgewiesen wird, verflucht sie alle Anwesenden: Sie sollen die Identität, die sie durch ihre Verkleidung angenommen haben, von nun an auch in der Realität annehmen.
\end{itemize}

\notiz{
Kang und Kodos haben eine Sekretärin mit Namen Dorothy\index{Dorothy}.
}

\subsection{Simpsons Weihnachtsgeschichten}
Weihnachten steht vor der Tür und alle machen sich Gedanken über die Vergangenheit. Marge und Homer schlüpfen in die Rollen von Maria und Joseph und erleben die wunderlichen Heiligen Drei Könige mit einer Jesus-Geburt und einem Kampf gegen die Römer. Abe kehrt zurück in seine Kriegsabenteuer und einen Absturz mit Mr. Burns auf eine einsame Insel, bei dem sie vom Weihnachtsmann gerettet werden.

\notiz{
\begin{itemize}
	\item In der Weihnachtsgeschichte sind Prof. Frink, Skinner und Dr. Hibbert die heiligen drei Könige. König Herodes ist Mr. Burns. Die Schäfer sind Carl und Lenny.
	\item Abe hat einen älteren Bruder namens Cyrus\index{Cyrus}.
\end{itemize}
}


\subsection{Die Straße der Verdammten}
Bart und Milhouse klettern in die Dampftunnel der Schule und öffnen eines der Ventile. Es kommt zum Chaos und da sie solchen Unsinn anstellen, soll Bart in ein Erziehungslager. Hier wird er zum Holzhacken verurteilt und muss sich gut benehmen. Homer, der nach Las Vegas fahren wollte, bekommt plötzlich ein schlechtes Gewissen und holt Bart zurück. Unterdessen verkaufte Marge bei einem Gartenverkauf abgelaufene Arzneimittel von Homer und wird daraufhin von der Polizei als Drogendealerin verhaftet.

\notiz{
\begin{itemize}
	\item Homer zerstört mit seinem Auto das Geburtshaus von Matt Groening.
	\item Homer sagt, er sei 38 Jahre alt.
	\item Während der Dampf in den Musiksaal der Schule strömt, spielt das Schulorchesters das Titelthema der Serie \glqq Futurama\grqq .
	\item In Deutschland wurde diese Episode der Schauspielerin Elisabeth Volkmann gewidmet, die bis dahin u.\,a. Marge Simpsons synchronisiert hat.
\end{itemize}
}


\subsection{Ein perfekter Gentleman}\label{HABF05}
Die Sportlehrerin Miss Pommelhorst\index{Pommelhorst} macht Platz für einen neuen Sportlehrer. Dieser lässt den Tyrannen raushängen, sodass Bart sich nach Tagen der Demütigung wehren will. Wie so oft geht der Plan schief und Willie ist obdachlos. Marge springt ein und lädt Willie zu den Simpsons ein. Willie beweist musikalisches Talent und Lisa hat eine Idee für ihr wissenschaftliches Projekt. Sie wettet mit Bart, dass sie aus dem Schulhausmeister Willie einen Gentleman machen kann. Sie rasiert ihn, kleidet ihn neu ein und stellt ihn -- nach intensiver Schulung -- bei einem Naturwissenschaftsfest als vornehmen Gentleman vor. Nach erfolgreichem Auftreten besorgt sie ihm einen Job als Empfangschef in dem Restaurant \glqq Gilded Truffle\grqq . Auch hier verläuft zunächst alles nach Plan, bis Krusty pampig wird und schließlich bekommt Willie seinen alten Posten als Schulhausmeister wieder.


\notiz{
\begin{itemize}
	\item Mr. Krupt\index{Krupt} ist Barts neuer Sportlehrer, nachdem sich die Sportlehrerin Miss Pommelhorst zum Mann umoperieren lassen will.
	\item Der Chef der Firma, die Homers blaue Hosen herstellen, heißt Eli Stern VI\index{Stern!Eli}.
	\item Während Willies Abwesenheit ist Mr. Largo der Schulhausmeister.
	\item Willie gibt an, keinen Nachnamen zu haben.
\end{itemize}
}

\subsection{Die scheinbar unendliche Geschichte}\label{HABF06}
Als die Simpsons einen Ausflug in eine Höhle machen und Homer dort einbricht, beginnt Lisa dem Gefangenen eine Geschichte zu erzählen, die ihr Mr. Burns berichtet hat. Wie er sein Hab und Gut an den reichen Texaner verloren hat und dass er daher wieder bei Moe arbeiten musste, der damals eine Affäre mit Mrs. Krabappel hatte. Seinerzeit hatte er erfahren, dass Moe Snake einen Goldschatz geklaut hatte, aber Mrs. Krabappel wollte trotzdem Lehrerin in Springfield werden und schließlich kommt es zum Duell in den Kavernen.

\notiz{
\begin{itemize}
	\item \glqq Carl's Dad Caverns\grqq , welche 1956 entdeckt worden sind, sind eine Anspielung auf die \glqq Carlsbad Caverns\grqq\ in einem Nationalpark in Neu Mexico.
	\item Snake war Archäologe, bis Moe ihm einen Goldschatz gestohlen hat.
	\item Snakes Sohn heiß Jeremy.
	\item Lenny trägt gerne Babyklamotten.
	\item Moes Kneipe gegenüber liegt das \glqq Moeview Motel\index{Moeview Motel}\grqq .
	\item Der englische Originaltitel \glqq The Seemingly Never-Ending Story\grqq\ der Episode ist eine Anspielung auf den Film \glqq The Never-Ending Story\grqq\ (Die unendliche Geschichte) von Wolfgang Peterson basierend auf dem Roman von Michael Ende.
	\item Diese Folge gewann 2006 einen Emmy.
\end{itemize}
}

\subsection{Bart hat zwei Mütter}\label{HABF07}
Ned Flanders gewinnt einen Computer, welchen er Marge schenkt. Dafür bietet diese ihm an, auf seine Kinder aufzupassen. Während Marge Neds Kinder Rod und Todd babysittet, wird Bart von einer Schimpansen-Mutter gekidnappt. Homer bemüht sich erfolglos, Bart alleine zu befreien. Als Marge schließlich auftaucht, gelingt es der Schimpansin, mit Bart auf die Spitze eines Kirchturms zu flüchten. Lisa organisiert schließlich den echten Sohn des Affen, Mr. Teeny, um ihn gegen ihren Bruder Bart auszutauschen. Schließlich klettert Rod Flanders auf den Kirchturm hoch, um Mr. Teeny gegen Bart auszutauschen.

\notiz{
\begin{itemize}
	\item Der Name des Computers lautet \textit{fe}Mac\index{feMac}.
	\item Im Abspann ist Gott zu sehen, welcher fünf Finger hat.
	\item Als Ned den Garten kindersicher macht, singt er ein Lied, das ähnlich zu \glqq Welcome To The Jungle\grqq\ von Guns N' Roses, klingt. \glqq Welcome To The Jungle\grqq\ war bereits in den Episoden \glqq Die rebellischen Weiber\grqq\ (siehe \ref{1F03}) und \glqq Homer mobil\grqq\ (siehe \ref{GABF07}) zu hören.
	\item Carl hat dieselbe Blutgruppe wie Rod und Todd.
	\item Mr. Teeny, der Affe aus Krustys Show, heißt eigentlich Louis\index{Louis}.
	\item Für Ned Flanders ist das Kartenspiel UNO der Einstieg zum Pokern.
\end{itemize}
}

\subsection{Frauentausch}\label{HABF08}
Bei einem Besuch der FOX-Studios lässt Homer sich aus Geldnot auf eine Fernseh-Reality-Show ein. Denn er möchte auch einen Plasma-HDTV-Fernseher haben wie Lenny. Das bedeutet, dass er und ein gewisser Charles Heathbar\index{Heathbar!Charles} für einen Monat ihre Ehefrauen -- Marge und Verity\index{Heathbar!Verity} -- tauschen. Während Homer mit Verity so seine Probleme hat, ist Charles von Marge total begeistert. Das geht so weit, dass er seine eigene Frau nie mehr wiedersehen will. Marge aber möchte zurück zu Homer und auch Homer sehnt sich nach seiner Marge.

\notiz{
\begin{itemize}
	\item Dies ist die erste Episode, die ein Gaststar geschrieben hat.
	\item Anscheinend ist Lenny nicht oder nicht mehr verheiratet. Denn als Homer und Marge zu Lennys Party fahren und Homer sagt, Lenny habe eine Überraschung parat, fragt Marge, ob Lenny heirate.
	\item Homer schaut sich auf Lenny Plasmafernseher u.\,a. \glqq Two And A Half Men\grqq\ mit Charlie Sheen an.
	\item Homer veröffentlichte eine DVD mit dem Namen \glqq Homer Gone Wild\grqq .
	\item Am Ende der Folge will Verity Charles wegen Patty verlassen.
	\item Zu dieser Episode gibt es eine Anfangssequenz, die mit echten Schauspielern gespielt wurde. Diese wurde allerdings nicht in der deutschen Fassung übernommen.
\end{itemize}
}


\subsection{Corrida de Toro}\label{HABF09}
Die NFL möchte eine Mannschaft neu ansiedeln. Homer ist davon überzeugt, dass Springfield bestens geeignet ist und hat schnell die Bewohner der Stadt auf seiner Seite. Abe vereitelt dummerweise diesen Plan. Grandpa ist zutiefst enttäuscht und möchte Selbstmord begehen. Da wird das Stadion überraschend in eine Stierkampf-Arena umgewandelt und Grandpa entpuppt sich als grandioser Stierkämpfer -- bis Lisa ihm ins Gewissen redet.

\notiz{
\begin{itemize}
  \item Der Name der neuen Springfielder Footballmannschaft lautet \glqq Springfield Meltdowns\grqq .
  \item Das neue Stadion heißt \glqq Duff Beer Krusty Burger Buzz Cola Costington's Department Store Kwik-E-Mart Stupid Flanders Park\grqq .
  \item Dies ist die erste Episode, in der Bart und das Klassenzimmer auf den Kopf stehen, als Bart den Tafelanschrieb macht.
	\item Clancy Wiggums Schwager heißt Fred Kenicke\index{Kenicke!Fred}.
	\item Der Selbstmordapparat heißt diePOD\index{diePOD}, eine Anspielung auf Apples iPod\index{iPod}.
	\item Abe gibt an, dass er 83 Jahre alt ist.
	\item Der reiche Texaner hat einen schwulen Enkel.
	\item Der Arzt, bei dem Abe Selbstmord begehen möchte, heißt Dr. Egoyan\index{Egoyan!Dr.}
\end{itemize}
}


\subsection{Kiss, Kiss Bang Bangalore}\label{HABF10}
Das Atomkraftwerk von Mr. Burns wird nach Indien ausgelagert und Homer erkämpft sich den einzigen Arbeitsplatz, den es dort für einen Springfielder gibt. Homer bekommt schnell den Dreh raus, wie es in Indien läuft und hält sich nebenbei noch für einen Gott. Als er im Kraftwerk arbeiterfreundliche Regelungen erlässt, verlegt Mr. Burns das Kraftwerk wieder nach Springfield. Währenddessen kidnappen Selma und Patty Richard Dean Anderson, den Darsteller von McGyver der gleichnamigen Fernsehserie und es entsteht eine etwas merkwürdige Beziehung zwischen den Dreien.

\notiz{
\begin{itemize}
	\item Marge hat im Mai Geburtstag.
	\item In dieser Episode ist das von Marges Schwestern adoptierte chinesische Mädchen Ling\index{Ling} aus der Folge \glqq Der lächelnde Buddha\grqq\ (siehe \ref{GABF06}) wieder zu sehen. 
	\item Laut Moe stammt das von ihm verkaufte Bier aus Deutschland.
	\item Lenny hat sich ein Haus für drei Millionen Dollar gekauft.
\end{itemize}
}

\subsection{Drei nasse Geschichten}
Während die Simpsons im \glqq Frittierenden Holländer\footnote{The Frying Dutchman}\grqq\ auf das Abendessen waren, erzählen sie sich drei nasse Seemannsmärchen. Lisa erzählt antike Horror-Geschichten von der \glqq Mayflower\grqq , Bart die von der Meuterei auf der \glqq Bounty\grqq\ und Homer die von dem gekenterten Kreuzfahrtschiff \glqq Neptune\grqq\ auf der Reise nach Israel.

\notiz{
\begin{itemize}
	\item Fehler I: Nach dem Sideshow Mel seinen Knochen zum Öffnen der Luke verwendet hat, steckt er ihn nicht mehr in seine Haare zurück. Später ist er allerdings wieder mit dem Knochen in den Haaren zu sehen.
	\item Fehler II: Erst ist Mrs. Glicks Kleid pink, später ist es, wie üblich, hellblau.
\end{itemize}
}



\subsection{Gleichung mit einem Unbekannten}
Rektor Skinner wird suspendiert, weil er sich öffentlich despektierlich über die Intelligenz von Frauen äußert. Er behauptet, dass Jungs in allen naturwissenschaftlichen Fächern besser seien. Eine neue Schulleiterin, Mrs. Melanie Upfoot\index{Upfoot!Melanie}, wird eingesetzt und diese beschließt, derartige Vorkommnisse zu unterbinden, indem sie eine strikte Geschlechtertrennung in der Schule durchführt. Bei den Mädchen ist Mathematik so anspruchslos, dass Lisa durch Marge dazu verleitet wird, verkleidet bei den Jungs am Unterricht teilzunehmen. Nach anfänglichen Schwierigkeiten und mit Barts Hilfe, kann Lisa sich behaupten. Als Lisa den Mathematik-Wettbewerb gewinnt, lässt sie die Maskerade fallen und zeigt allen, dass auch Mädchen sich durchsetzen können.

\notiz{
\begin{itemize}
  \item In Springfield wird das Itchy \& Scratchy Musical \glqq Stab-A-Lot\grqq\ aufgeführt. Dieses Musical ist eine Anspielung auf \glqq Spam-A-Lot\grqq , ein Musical in dem die Originalstimme u.\,a. von Apu und Moe, Hank Azaria, mitwirkt.
	\item Während Otto im Schulbus den Jungenbereich aufschließt, ist das Lied \glqq Breaking The Law\grqq\ von Judas Priest zu hören.
	\item Martin Prince gewinnt den Preis als bester Flötenspieler und wird bei der Preisverleihung von Lisa mit einem Klappstuhl niedergeschlagen.
	\item Nachdem Skinner seinen Job als Rektor los ist, wird er der Gehilfe von Hausmeister Willie.
\end{itemize}
}

\subsection{Selig sind die Unwissenden}
Marge hat einen Unfall, bei dem sie ihr Gedächtnis verliert. Sie wird ins Krankenhaus eingeliefert. Aufgrund ihrer Amnesie erkennt sie zunächst niemanden aus ihrer vertrauten Umgebung wieder. Aber da die Versicherung nicht mehr zahlt, nimmt die Familie sie mit nach Hause. Nach und nach erkennt sie alle Familienmitglieder wieder. Nur einen scheint ihr Gedächtnis konsequent auszuschließen: ihren geliebten Mann Homer. Der ist fassungslos und überhaupt nicht erfreut. Als schließlich Homer das Wort \glqq Bier\grqq\ erwähnt, erkennt sie ihren Mann wieder.


\notiz{
}
\begin{itemize}
	\item Ned Flanders hat die Hausnummer 738 und Moe die Hausnummer 555.
	\item Homer bekommt die Post von Scott und Brenda Weingarten\index{Weingarten!Scott}\index{Weingarten!Brenda}, die in 74 Evergreen Terrace wohnen.
	\item Homer sagt, er und Marge seien gleich alt.
\end{itemize}

\subsection{Gott gegen Lisa Simpson}\label{HABF14}
Die Simpsons gehen ins Museum, als gerade eine Waffenausstellung neu eröffnet wird. Da die Schlange so lang ist, stellen sie sich an die erste Stelle vor Flanders. Die anderen Schlangestehenden ziehen nach, bis Flanders letzter der Schlange ist. Es bleibt nur der Besuch der Evolutionsausstellung und dort wird er mit der Evolutionslehre konfrontiert. Da sie natürlich völlig gegen das spricht, was er aus der Bibel kennt, will er dagegen vorgehen. Seine Initiative führt dazu, dass im Stadtrat beschlossen wird, die christliche Schöpfungslehre neben der Evolutionslehre als Pflichtfach in der Schule einzuführen. Als Lisa dagegen protestiert, kommt es schließlich dazu, dass die Schöpfungsgeschichte als einzige wahre Theorie zur Entstehung des Menschen in den Lehrplan aufgenommen wird. Lisa gibt nicht auf und landet vor Gericht, weil sie aus einem Buch über die Evolutionstheorie vorgelesen hat.


\notiz{
\begin{itemize}
	\item Lisas vollständiger Name lautet Lisa Marie Simpson.
	\item Barts \glqq To-Do-Liste\grqq\ für den Sommer
	\begin{itemize}
		\item Ein Baseballspiel gewinnen
		\item Ein Auftritt im Sommertheater (Grease)
		\item Den Sommerhit des Sommers im Kino sehen (Parodie auf \glqq Men in Black\grqq )
		\item Eine Sommerromanze 
	\end{itemize}
	\item Der Kirchenrat besteht aus: Sideshow Mel, Ned Flanders, Helen und Timothy Lovejoy.
	\item Lisa wird beim Prozess von der Anwältin Clarice Dremond\index{Dremond!Clarice} vertreten. Die Gegenseite vertritt Wallace Bravy\index{Bravy!Wallace}.
	\item Der Jury gehören u.\,a. Kent Brockman, Barney Gumble, Bernice Hibbert, Horatio McCallister, Lenny Leonard, Carl Carlson, Cookie Kwan und Luann Van Houten an.
	\item Bart sagt, dass er erstmalig ein Mädchen geküsst hat.
\end{itemize}
}


\subsection{Homerun für die Liebe}\label{HABF16}
Die Springfield Isotopes, eine eigentlich eher mittelmäßige Baseballmannschaft, haben einen neuen Star: Buck Mitchell\index{Mitchell!Buck}. Sogar Familie Simpsons strömt wieder ins Stadion. Doch genau an diesem Tag hat Buck eine schwere Krise, was auf Schwierigkeiten in seiner Ehe zurückzuführen ist. Da auf der Großbildleinwand im Stadion übertragen wird, wie Marge und Homer sich innig küssen, glaubt Buck, dass das Ehepaar Simpson ihm Eheberatung geben könnte. Wider Erwarten funktioniert Anfangs alles super, doch dann macht Homer einen entscheidenden Fehler.

\notiz{
\begin{itemize}
	\item Die Ex-Freundinnen von Carl und Lenny hießen Jill, die bereits gestorben ist und Kelly, die jetzt als Prostituierte arbeitet.
	\item Sideshow Mel hat eine Schwester.
	\item Abe Simpsons und Jasper spielten 1942 in der Frauenbaseballliga, um sich vor dem Kriegseinsatz zu drücken.
	\item Die Simpsons sehen sich im Fernsehen die Serie \glqq Hunch\index{Hunch}\grqq\ an, welche eine Parodie auf \glqq Monk\index{Monk}\grqq\ ist.
	\item Die Springfield Isotopes spielen in der zweiten Liga.
\end{itemize}
}


\section{Staffel 18}

\subsection{Der Koch, der Mafioso, die Frau und ihr Homer}\label{HABF15}
Lisa freundet sich mit Fat Tonys Sohn Michael\index{D'Amico!Michael} an. Er hat Talent als Koch und wird jedoch dazu gedrängt, den \glqq Müll\-be\-sei\-ti\-gungs\-be\-trieb\grqq\ seiner Familie zu über\-neh\-men. Als er bei einem Essen mit den Calabresis\index{Calabresi}, seinen Erzfeinden, erwähnt, dass das Dessert von ihm sei und Fat Tony dadurch einen schwachen Moment erleidet, wird dieser angeschossen. Daraufhin übernehmen Homer und Bart seinen Betrieb.

\notiz{
\begin{itemize}
  \item Kearney und Otto besuchten zusammen die dritte Klasse in der Grundschule.
  \item Metallica\index{Metallica} spielen auf dem Pick-up von Hans Maulwurf den Song \glqq Master Of Puppets\grqq .
  \item Otto besuchte 1997 das Metallica-Konzert in Springfield in der Reihe XX auf Platz 64.
  \item Fat Tonys Frau Anna Maria\index{D'Amico!Anna Maria} ist tot.
	\item Anspielungen auf den Film \glqq Der Pate\grqq :
	\begin{itemize}
		\item Michael ist ebenfalls der Name des Sohnes von Don Vito, dem Mafiaboss.
		\item Michael, für den eigentlich eine bürgerliche Laufbahn vorgesehen war und dadurch ebenfalls eigentlich nichts mit den Mafia-Geschäften zu tun hat, übernimmt die Führ\-ung des Clans.
		\item Michael schaltet, während sein Vater im Krankenhaus liegt, dessen Gegner aus.
		\item Die letzte Szene ist eine 1:1-Kopie der letzten Szene dieses Filmes. Die Rolle von Michaels Frau nimmt hier Lisa ein, die zusehen muss, wie sich Michael schließlich doch für das Mafiageschäft entscheidet und sie außen vor lässt.
	\end{itemize}
	\item Fehler: In einer Szene ist Lisas Perlenkette rot.
\end{itemize}
}



\subsection{Jazzy and the Pussycats}\label{HABF18}
Weil Bart auf der Beerdigung von Homers Las Vegas-Frau Amber\index{Amber} seinen Mund nicht halten kann, schicken ihn Homer und Marge zu dem Psychiater Dr. Brentano\index{Brentano!Dr.}. Dieser hat eine Lösung für Barts Problem: Schlagzeug spielen! Bart stellt sich hierbei als Naturtalent heraus und als er bei einem Jazz-Brunch im Jazzclub \glqq Jazzy Goodtime's\index{Jazzy Goodtime's}\grqq\ zusammen mit Lisa auftritt, werden auch professionelle Jazz-Musiker auf ihn aufmerksam und bieten Bart an, mit ihm als \glqq Skinny Palmer Trio\index{Skinny Palmer Trio}\grqq\ aufzutreten. Diese Tatsache wirft Lisa in eine Krise und sie beschließt, ihre Depressionen durch das Retten von Tieren zu unterdrücken. Doch der Schuss geht nach hinten los: Eines der Tiere verletzt Bart am Arm. Deshalb muss Lisa diese so schnell wie möglich wieder loswerden.

\notiz{
\begin{itemize}
	\item In der Kirche ist die Blue Man Group zu sehen.
	\item Homers Las Vegas-Frau Amber starb an einer Überdosis.
	\item Barts Jazz-Spitzname lautet \glqq Tic Tock\index{Tic Tock} Simpson\grqq .
	\item Bei der Jam-Session von Lisa und Bart wirken außerdem noch Martin (am Klavier), Uter (an der Trompete) und Database (am Bass) mit.
	\item Filmzitat: Lisa zieht mit den Tieren durch die Stadt, so wie die Gangster am Anfang des Films \glqq Reservoir Dogs\grqq . Passend dazu wird auch \glqq Little Green Bag\grqq\ gespielt.
\end{itemize}
}

\subsection{Homer, hol den Hammer raus!}
Homer kauft sich eine Heimwerker-Bücherreihe von Time Life\index{Time Life} und Marge entdeckt ihr Talent zu heimwerken. Da die Bewohner Springfields einer Frau nicht viel zutrauen, bittet sie Homer, nach Außen aufzutreten. Doch als Homer den Ruhm alleine einstreicht, ohne richtigzustellen, dass Marge die Arbeit macht, lässt Marge Homer alleine weitermachen. Dieser übernimmt sich aber dann mit dem Zoominator\index{Zoominator}, einer Achterbahn. Bart entdeckt unterdessen Rektor Skinners Allergie gegen Erdnüsse. Alles, was er also braucht, um Skinner gefügig zu machen, ist eine Erdnuss, die an einem Stab befestigt ist.

\notiz{
\begin{itemize}
	\item Barts Allergie gegen Shrimps ist paradox zu der Folge \glqq Am Anfang war die Schreiraupe\grqq\ (siehe \ref{DABF16}). Als Homer vom elektrischen Stuhl befreit wird, kommen Bart und Lisa dazu und Bart erzählt, dass er im \glqq Green-Room\grqq\ Shrimps gegessen habe.
	\item Milhouse ist u.\,a. auf Weizen, Honig und alle Milchprodukte allergisch.
	\item In der Springfield Mall wurde Joe Quimbys Vater erschossen.
	\item In der Springfield Mall ist ein \glqq e\grqq\ an einen Mülleimer gelehnt. Das \glqq e\grqq\ erinnert an das Symbol des Internet Explorers von Microsoft.
\end{itemize}
}

\subsection{Krieg der Welten}\label{HABF17}
In dieser Halloween-Folge werden folgende drei Episoden gezeigt:
\begin{itemize}
	\item \textbf{Mit dem Blob verheiratet:}\\
	Homer isst das Innere eines Meteoriten und wird zu einem riesigen menschenfressenden Monster, das die gesamte Stadt terrorisiert. Als er auch noch Dr. Phil McGraw\index{McGraw!Phil} verschlingt, der ihm eigentlich helfen wollte, weiß er, dass er etwas zu tun muss. Bürgermeister Quimby hat die Lösung: Den Obdachlosen der Stadt wird eine Fassade, hinter der der hungrige Homer-Blob sitzt, als Unterkunft präsentiert. So wurde Homers Problem doch noch zu einer positiven Veränderung der Gemeinde.
	\item \textbf{Ein Golem für alle Fälle:}\\
	Bart findet in Krustys Requisitenraum einen Golem\index{Golem}, der Befehle in Form von Zetteln, die ihm in den Mund gelegt werden, befolgt. Bart lässt ihn nachts zu seinem Haus kommen und sieht ihn als durchaus hilfreich. Als Lisa ihn jedoch zum Sprechen bringt, merken die Simpsons, wie schuldig er sich für alle von ihm vollbrachten Taten fühlt. Die Simpsons schaffen einen weiblichen Golem aus Knetmasse, damit es ihm besser geht. Prompt verliebt er sich in sie und die beiden heiraten.
	\item \textbf{Der Tag, an dem die Erde sich für dumm verkaufen ließ:}\\
	Springfield, 1938: Die gesamte Bevölkerung hält das Hörspiel \glqq Krieg der Welten\grqq , das im Radio zu hören ist, für eine echte Invasion. Panik bricht aus und alle wälzen sich nackt im Schlamm, um den Aliens Glauben zu machen, dass sie Tiere sind, da ja nur Menschen angegriffen werden. Als Lisa sie aufklärt, dass das ganze nur ein Hörspiel war, sehen Kang und Kodos das als den perfekten Zeitpunkt, die Erde wirklich anzugreifen. Da niemand an den Angriff glaubt und etwas tut, gibt es keine Hoffnung mehr. Drei Jahre später bemerken sie, dass sie nicht als Befreier gefeiert werden können, da sie alles zerstört haben. 
\end{itemize}

\notiz{
\begin{itemize}
	\item Uter ist auf dem Oktoberfest in \glqq Mit dem Blob verheiratet\grqq\ zu sehen.
	\item Barney gibt an, Ire zu sein. Als ihn darauf Moe verprügelt, behauptet er, Pole zu sein.
	\item Ein Name eines ausführenden Co-Produzenten lautet: $j\cdot\int_{t=e}^{w}\alpha r^{t} = B\mu r^{n}\varsigma$.
	\item 1938 wurde das Kraftwerk in Springfield mit Kohle befeuert.
	\item Die letzte Episode ist eine Anspielung auf die amerikanische Außenpolitik in Afghanistan und im Irak. Im amerikanischen Original ist von der \glqq Operation: Enduring Occupation\grqq\ (im Vergleich zur NATO-Mission Enduring Freedom) die Rede. Die Invasion wird mit der Herstellung von \glqq weapons of mass disintegration\grqq\ (im Irakkrieg mit Massenvernichtungswaffen) begründet.
\end{itemize}
}

\subsection{G.I. Homer}\label{HABF21}
Die Army schafft es nicht, Teenager wie Jimbo, Dolph und Kearney zu rekrutieren, deshalb besuchen die Verantwortlichen vom Rekrutierungs-Büro die Grundschule in Springfield und schließen mit den Schülern einen Vertrag: Sie verpflichten sich der Army beizutreten, sobald sie das Alter von 18 Jahren erreicht haben. Homer begibt sich zum Rekrutierungs-Büro, um den Vertrag von Bart aufzulösen, unterschreibt jedoch letztendlich selbst einen Vertrag. Als einer der dümmsten Soldaten dient er bei einer Übung als Feinddarsteller. Nachdem er dabei flieht, fällt in ganz Springfield die Army ein und versucht ihn wieder zu finden.

\notiz{
\begin{itemize}
	\item Folgende Telefonnummer werden gezeigt (alle mit der Vorwahl 555): Marge Simpson 0123, Ned Flanders: 0172, Selma: 0182, Lenny: 0154, Helen Lovejoy: 0134, Patty: 0193 und Carl: 0148.
	\item Neben Homer (Spitzname \glqq Schneeflocke\index{Schneeflocke}\grqq ) schreiben sich noch Prof. Frink (\glqq Maverick\index{Maverick}\grqq ), Gil und Cletus (\glqq Rose von England\grqq ) bei der Army ein. Ihre Grundausbildung findet im Fort Clinton\index{Fort Clinton} statt.
	\item Der Teenager mit der Quietschstimme heißt jetzt Friedman\index{Friedman}; früher hieß er Peterson (siehe \ref{JeremyPeterson}).
\end{itemize}
}

\subsection{Das literarische Duett}
Obwohl Homer Moe versprochen hat, an seinem Geburtstag mit ihm angeln zu gehen, lässt er ihn aber sitzen, weil er es vergessen hat. Stattdessen fahren sie zur Seniorenolympiade. Lisa tut es so leid für Moe, dass sie beschließt, über ihn einen Schulaufsatz zu schreiben. Dabei entdeckt sie Moes äußerst geschwollene Schreibweise. Diese \glqq Werke\grqq\ lassen sich jedoch zu Gedichten abwandeln, die Lisa dann veröffentlicht. Aufgrund dessen wird Moe zu einer Schriftstellerversammlung in Vermont\index{Vermont} eingeladen. Dort gibt er allerdings Lisas Hilfe bei seinen Werken nicht zu, worauf Lisa gekränkt ist.

\notiz{
\begin{itemize}
	\item Abe hatte 1936 bei den Olympischen Spielen in Berlin unabsichtlich ein Attentat auf Adolf Hitler\index{Hitler!Adolf} verhindert.
	\item Diese Folge wurde ursprünglich für die 16. Staffel geschrieben.
	\item Fehler: Als sich Lisa beim Abendessen entscheidet, Moes Gedicht an eine Zeitung zu verschicken, steht auf dem hinteren Tisch eine Blume. In der nächsten Szene ist die Blume verschwunden.
\end{itemize}
}

\subsection{Kunst am Stiel}\label{HABF22}
Homer wird im Atomkraftwerk gefeuert, weil er es bei einem ernsten Gespräch mit Mr. Burns vorzieht, sich lieber ein Eis zu kaufen, als die Anweisungen seines Chefs zu befolgen. Als der Fahrer Max\index{Max} stirbt, kauft sich Homer dessen Eismobil und übernimmt das Geschäft, bei dem unerwartet viel Gewinn abfällt. Darüber hinaus sorgt Marge für den Verkaufsschlager: Sie formt aus den Eisstielen Skulpturen.

\notiz{
\begin{itemize}
	\item Jimbos Vater ist bei dem Treffen der geschiedenen Väter zu sehen. Ebenfalls ist dort Rainier Wolfcastle mit seiner Tochter Greta\index{Greta} zu sehen.
	\item Otto pimpt Homers Eismobil auf.
\end{itemize}
}

\subsection{Beste Freunde}
Bart stiftet seine Freunde dazu an, Nelsons Geburtstagsparty zu boykottieren. Letztendlich ist er somit der einzige Gast, weil Marge ihn zwingt, trotzdem hinzugehen. Das führt dazu, dass Nelson Barts neuer bester Freund (und zusätzlich sein Bodyguard) wird. Im ersten Moment scheint das für Bart ein großer Vorteil zu sein, bis er bemerkt, dass Nelson unheimlich neidisch ist. Homer will Lisa abends eine \glqq Happy-End-Geschichte\grqq\ vorlesen, dabei ergibt sich nur ein Problem: das eigentliche Ende ist überraschenderweise genau das Gegenteil von einem \glqq Happy End\grqq .

\notiz{
\begin{itemize}
	\item Diese Episode war für einen Emmy nominiert.
	\item Homer liest Lisa aus dem Buch \glqq Angelica Button and The Dragon King's Trudle Bed\grqq\ von T. R. Francis\index{Francis!T. R.} vor. Dieses Buch ist eine Anspielung auf Harry Potter und ihre Schöpferin J. K. Rowling.
	\item Captain McCallister hat links ein Holzbein. In der Episode \glqq Der Lehrerstreik\grqq\ (siehe \ref{2F19}) hat er keine Holzbeine und der in Folge \glqq Marge im Suff\grqq\ (siehe \ref{FABF10}) hat er das Holzbein rechts.
\end{itemize}
}

\subsection{Kill Gil: Vol. 1 \& 2}\label{JABF01}
Pechvogel Gil Gunderson\index{Gunderson!Gil} verdingt sich in der Adventszeit als Kaufhausweihnachtsmann. Als er Lisa ein ausverkauftes Zubehörset zu ihrer Malibu-Stacy-Puppe besorgt, das eigentlich für die Tochter seines Chefs zurückgelegt war, wird er sofort gekündigt. Die Simpsons haben Mitleid mit Gil und laden ihn über Weihnachten zu sich ein. Doch als dieser nach Weihnachten nicht wieder gehen möchte, haben die Simpsons ein Problem: Wie werden sie den unliebsamen Gast wieder los? Gil zieht von selbst aus, um in Scottsdale\index{Scottsdale} einen Job als Immobilienhändler anzunehmen. Er macht dort schnell Karriere, doch Marge sorgt unfreiwillig dafür, dass er gefeuert wird. Dennoch feiern Gil und die Simpsons das Weihnachtsfest zusammen.

\notiz{
\begin{itemize}
	\item Diese Folge wurde in den USA erstmals zum 17. Geburtstag der Simpsons ausgestrahlt.
	\item Diese Episode erstreckt sich für den Zeitraum eines Jahres.
	\item Gils Nachname wird erstmals erwähnt: Gunderson.
\end{itemize}
}


\subsection{Der perfekte Sturm}
Als Marge sich ihrer Kindertage entsinnt, erinnert sie sich auch an Barnacle Bay\index{Barnacle Bay}, wo sie als Kind so manche Ferien verbracht hatte. Homer beschließt, dorthin einen Familienausflug zu unternehmen. Dabei stellt die Familie fest, dass der Ort mittlerweile ziemlich heruntergekommen ist. Voller Tatendrang will Homer wieder Alles herrichten lassen. Er verbrennt dabei jedoch ein altes Kinderkarussell, was ihm den Ärger der Ortsansässigen einbringt. Er wird gezwungen, mit den Fischern auf See zu fahren und ihnen beim Fischen zu helfen. Dabei geraten sie in einen schweren Sturm und erleiden Schiffbruch. Sie werden glücklicherweise von japanischen Fischern gerettet.

\notiz{
\begin{itemize}
	\item Das Seepferd des Karussells auf Barnacle Bay hat ein Hakenkreuz als Auge.
	\item Im ozeanografischen Forschungsinstitut ist Spongebob\index{Spongebob} zu sehen.
	\item Selma ist mit Ling\index{Ling} zu sehen, welches sie in der Folge \glqq Der lächelnde Buddha\grqq\ (siehe \ref{GABF06}) adoptiert hatte.
	\item Schiffe, welche vor Barnacle Bay liegen:
	\begin{itemize}
		\item The Wasted Bait (Der verschwendete Köder)
		\item The Empty Net (Das leere Netz)
		\item The Rotting Pelican (Der verrottende Pelikan) 
	\end{itemize}
	\item Barnacle Bay war wegen des Yum-Yum Fisches bekannt.
\end{itemize}
}


\subsection{Rache ist dreimal süß}\label{JABF05}
Als Homer bei einem Ausflug auf der Autobahn von einem reichen Texaner geschnitten wird, will er Rache nehmen. Die Familie ist skeptisch und erzählt dem wütenden Homer drei verschiedene Geschichten über Rache, um ihn so abzuschrecken. Die erste Geschichte handelt vom Graf von Monte Christo\index{Monte Christo}, die zweite Geschichte handelt davon, dass sich die Schultrottel überlegen, endlich etwas gegen die Schulschläger zu unternehmen und in der dritten Geschichte geht es um Bartman\index{Bartman}, den Rächer.

\notiz{
\begin{itemize}
	\item Eine Pappfigur von Adolf Hitler\index{Hitler!Adolf} ist zu sehen, welche das Buch \glqq Mein Kampf\grqq\ hochhält. In der Sprechblase ist zu lesen: \glqq Before I Was A Nazi Leader, I Was A Nazi Reader\grqq\ (Bevor ich eine Nazi-Führer war, war ich ein Nazi-Leser).
	\item Homer sagt, er habe in der Episode \glqq Wer erschoss Mr. Burns? Teil 1\grqq\ (siehe \ref{2F16}) auf Mr. Burns geschossen und es Maggie in die Schuhe geschoben.
	\item Der reiche Texaner sagt, er und Homer stammen aus Connecticut.
	\item Die Folge ist allen gewidmet, die in den \glqq Star Wars\grqq -Filmen umgekommen sind, u.\,a. Darth Vader und Obi Wan Kenobi.
\end{itemize}
}


\subsection{Mit gespaltener Zunge}
Lisa soll in der Schule einen Vortrag über ihre Herkunft halten. Doch wie das bei den Simpsons nicht anders zu erwarten ist, schämt sich Lisa für ihre peinlichen Vorfahren. Sie erschwindelt kurzum eine indianische Abstammung und hat damit soviel Erfolg, dass sie als Festrednerin zu Springfields multikulturellem Tag eingeladen wird. Ob Lisa ihre Doppelzüngigkeit zum Verhängnis wird? Unterdessen gerät Bart ebenfalls in eine heikle Lage. Nachdem er von den Bürgern der Stadt gefeiert wird, weil er einen Stadtbrand löschte, bekommt er von Bürgermeister Quimby einen Führerschein ausgestellt. Bart lernt auf seinen Touren mit dem Auto in North Haverbrook\index{North Haverbrook} Darcy\index{Darcy} kennen und beide wollen heiraten, da sie von einem norwegischen Austauschschüler schwanger ist.

\notiz{
\begin{itemize}
	\item Lisa behauptet, vom Stamm der Hitachi\index{Hitachi} abzustammen.
	\item Carls Mutter arbeitet im leicht entzündlichen Distrikt in Springfield.
	\item Bart fährt mit dem Auto durch die Städte Ogdenville\index{Ogdenville} und North Haverbrook. Diesen beiden Städten verkaufte Lyle Lanley\index{Lanley!Lyle} die Monorail in der Episode \glqq Homer kommt in Fahrt\grqq\ (siehe \ref{9F10}).
	\item Bart und Darcy wollen in Utah in der Kapelle \glqq Wives \grq \normalfont \reflectbox{\mbox{R}}\grq\ \emph{ Us\grqq\ heiraten.}
	\item \emph{Nelson ist beim multikulturellen Tag der Schule mit einem Hut, einem Bierkrug und einem Dackel zu sehen. Er verabschiedet sich im amerikanischen Original mit den Worten \glqq Guten Tag\grqq . Dies lässt darauf schließen, dass der deutsche Vorfahren hat.}
\end{itemize}
}


\subsection{Springfield wird erwachsen}\label{JABF07}
Der Dokumentarfilmer Declan Desmond\index{Desmond!Declan} besucht alle acht Jahre Springfield, um zu sehen, wie sich die Kleinstädter entwickeln, während sie heranwachsen. Er zeigt dabei u.\,a. wie Clancy Wiggum, Prof. Dr. John Frink und Eleanor Abernathy erwachsen werden. Er zeigt auch Homer, der als Kind schon ein Verlierertyp war, obwohl er immer behauptet hat, dass er einmal reich sein würde. Um den Filmemacher bei dessen nächstem Besuch zu täuschen, quartiert sich Homer mit seiner Familie kurzerhand in Mr. Burns leerem Sommerhaus ein und mimt den Milliardär. Doch natürlich fliegt der Schwindel auf. Declan Desmond zeigt abschließend Homer, dass er auf sein Leben durchaus stolz sein kann.

\notiz{
\begin{itemize}
  \item Eleanor Abernathy\index{Abernathy!Eleanor} hatte mit 24 Jahren bereits ein Studium der Medizin in Harvard und ein Studium der Rechtswissenschaften in Yale abgeschlossen.
  \item Disco Stu wollte eigentlich ein Schiffspatent machen.
  \item Prof. John Frink hat die Frinkskrankheit entdeckt, das Element Frinkonium\index{Frinkonium} erschaffen und die Pille für acht Monate danach entwickelt.
	\item Fehler I: Im Haus der Simpsons hängt ein Bild von Maggie, als sie noch gar nicht geboren ist.
	\item Fehler II: Eine jugendliche Edna Krabappel ist in der Schule zu sehen. Allerdings kam sie erst als Lehrerin zum ersten Mal nach Springfield, was in der Folge \glqq Die scheinbar unendliche Geschichte\grqq\ (siehe \ref{HABF06}) zu erfahren ist.
	\item Fehler III: Gemäß dieser Folge müssten Homer und Clancy Wiggum gleich alt sein. Dies steht allerdings in Widerspruch zur Episode \glqq Wer ist Mona Simpson?\grqq\ (siehe \ref{3F06}).
\end{itemize}
}

\subsection{Selig sind die Dummen}
Rektor Skinner verweigert den Kindern des Springfielder Hinterwäldlers Cletus den Zugang zu Springfields Schule. Er hat Angst davor, dass Cletus dummer Nachwuchs den Notendurchschnitt senken würde und die Schule somit keine Fördermittel mehr vom Staat erhält. Auf Vorschlag von Rektor Skinner nimmt sich Lisa der Kinder an und wird prompt als deren persönliche Lehrerin abgestellt. Ihre Förderung zeigt Wirkung und die Kinder werden durch Krusty für das Fernsehen entdeckt. Cletus versucht, aus seinen Kindern Profit zu schlagen, womit Lisa und Cletus Frau Brandine natürlich nicht einverstanden sind. Bart hingegen muss zur Psychiaterin Dr. Stacey Swanson\index{Swanson!Stacey}: Er erfindet eine Geschichte über den Schulkoch Dark Stanley\index{Stanley!Dark}, der aus Kindern Essen macht.

\notiz{
\begin{itemize}
  \item Brandine ist bei der Army und im Irak im Einsatz.
  \item Der Vorname von Oberschulrat Chalmers lautet Gary.
\end{itemize}
}

\subsection{Ein unmögliches Paar}\label{JABF08}
Homer erklärt sich vor Gericht als zahlungsunfähig, weil er meint, er könnte so seine Werte schützen. Auf Gerichtsbeschluss muss er nun überflüssige Zahlungen stoppen und beschließt, den Platz von Grandpa im Altenheim aufzugeben. Grandpa muss also ausziehen und beginnt eine Beziehung mit Selma Bouvier trotz Homers Einwänden. Schließlich heiraten beide. Bart und Lisa bestellen sich unterdessen Schiffsbaumaterial. Als die Lieferfirma jedoch herausbekommt, dass sie damit ein Fort errichten, beginnt ein Papierkrieg.

\notiz{
\begin{itemize}
  \item Der Paketdienst heißt A.S.S.\index{A.S.S.} (American Shipping Services).
  \item Selma wird in der Zulassungsstelle zur Geschäftsstellenleiterin befördert.
  \item Fernsehzitat: Homer bewegt sein Auto wie Fred Feuerstein in der Serie \glqq Familie Feuerstein\grqq .
	\item Filmzitat: Der Kampf um das Fort ist dem um Minas Tirith in \glqq Herr der Ringe: Die zwei Türme\grqq\ nachempfunden.
\end{itemize}
}


\subsection{Homerazzi}\label{JABF06}
Als Homer mit den Kerzen auf seiner Geburtstagstorte fast unglücklich das Haus in Brand steckt, beschließen die Simpsons, einen feuerfesten Safe für ihre Wertsachen, vor allem aber für das Familienalbum, zu kaufen. Als das Album aber doch irgendwie zu brennen beginnt, versuchen Homer und Marge die Fotos komplett nachzustellen. Eines der Fotos zeigt im Hintergrund einen Prominenten in einer eindeutigen Pose und Homer beschließt deshalb, es an die Presse zu verkaufen. Schnell wird Homer so zum höchstgehandelten Paparazzo in Springfield. Danach legen es andererseits die Prominenten darauf an, Homer zum meist gedemütigten Mann in Springfield zu machen, indem sie den Paparazzo Enrico Irritazio\index{Irritazio!Enrico} auf Homer ansetzen.

\notiz{
\begin{itemize}
  \item Homer verkauft seine Fotos an das Klatschmagazin \glqq The Springfield Inquisitor\index{Springfield Inquisitor}\grqq .
	\item Rainier Wolfcastles zweiter Vorname lautet \glqq Luftwaffe\grqq .
	\item Rainier Wolfcastle heiratet Maria Shriver Kennedy Quimby.
	\item Der reiche Texaner hat eine Tochter namens Paris Texan\index{Texan!Paris}.
\end{itemize}
}


\subsection{Marge online}\label{JABF10}
Bei einem Elternabend erntet Marge nur Spott, als sie gestehen muss, dass sie nicht einmal eine E-Mail-Adresse besitzt. Sie beschließt, auch online zu gehen. Dies zeigt ungeahnte Folgen, da sie zum regelrechten Internetjunkie wird. Marge droht ihr Realitätssinn abhandenzukommen, als sie in die Welt des Online-Rollenspiels \glqq Earthland Reamls\grqq\ abtaucht. Lisa entdeckt indessen den Fußball für sich. Homer wird zum Schiedsrichter in Lisas Fußballteam. Er zeigt ihr letztendlich die rote Karte, nachdem sie ihn ausgenutzt hatte, um ein Foul nach dem anderen zu schinden.

\notiz{
\begin{itemize}
  \item Marge gibt an, am selben Tag wie der Schauspieler Randy Quaid Geburtstag zu haben (1. Oktober). In der Folge \glqq Kiss, Kiss Bang Bangalore\grqq\ (siehe \ref{HABF10}) hat sie allerdings im Mai Geburtstag. 
	\item In Lisas Mädchenfußballteam ist Sophie, Krustys uneheliche Tochter aus der Folge \glqq O mein Clown Papa\grqq\ (siehe \ref{BABF17}), zu sehen.
	\item Fehler: Im Garten steht Ronaldo\index{Ronaldo} mit dem Trikot der Nationalmannschaft und als er zu Homer geht, hat er das Trikot von Real Madrid an.
\end{itemize}
}


\subsection{Ballverlust}\label{JABF11}
Nachdem Bart als Catcher seine Baseballmannschaft, die Isotots\index{Isotots}, in das Finale geführt hat, verpatzt er den Sieg durch einen peinlichen Ballverlust. Dadurch wird Bart zum Buhmann der ganzen Stadt. Dies macht ihm so sehr zu schaffen, dass schließlich Marge zu drastischen Mitteln greifen muss, um Bart aus seiner Krise zu helfen. Homer hat derweil andere Sorgen: Homer schläft in einem Bett im Kaufhaus ein und als er sich erfolgreich aus dem Schlamassel redet, verkauft er dabei eine Matratze. Daraufhin wird er als Matratzenverkäufer im Kaufhaus Costington's eingestellt. Er hat jedoch auch weitere Pläne, als Reverend Lovejoy Homers Matratze mit einer anderen tauscht, die er sich erst gekauft hatte.

\notiz{
\begin{itemize}
	\item Lenny schrieb mehrere Bestseller Thriller, u.\,a. \glqq The Murderer Did It\grqq\ (Der Mörder war es).
	\item Neben Bart gehören u.\,a. folgende Jugendliche den Isotots an: Dolph, Kearney, Jimbo, Nelson, Milhouse, Ralph, Wendell, Todd, Rod und Martin. 
	\item Die Isotots werden von Ned Flanders trainiert. Ned trainierte bereits in \glqq Bart ist mein Superstar\grqq\ (siehe \ref{5F03}) eine Jugendfootballmannschaft.
\end{itemize}
}

\subsection{Brand und Beute}\label{JABF13}
Nachdem Homer versehentlich alle Feuerwehrleute Springfields ins Krankenhaus befördert hat, sieht er sich gezwungen, Feuerwehrmann zu werden, da die Stadt sonst vollkommen unsicher wäre. Zusammen mit Apu, Moe und Skinner löschen sie mit der Zeit so einige Brände und werden auch immer entsprechend belohnt. Als Mr. Burns sie jedoch, nachdem sie ihm das Leben gerettet haben, ohne irgendeine Belohnung stehen lässt, sieht sich Moe gezwungen, ihn zu beklauen. Mit der Zeit wird es immer mehr zur Gewohnheit, aus den Häusern, in denen die Einsätze stattfinden, auch etwas mitzunehmen, was Skinner allerdings gar nicht gefällt.

\notiz{
\begin{itemize}
	\item Maggie legt mit den Bauklötzchen die Worte \glqq No Future\grqq\ (keine Zukunft).
	\item Moe, Apu und Skinner waren bereits in der Episode \glqq Ein gotteslästerliches Leben\grqq\ (siehe \ref{9F01}) als Feuerwehrmänner zu sehen.
	\item Homer nimmt das Schlafmittel Nappien\index{Nappien}.
	\item Fehler: Nachdem Homer Milhouse die Haare ausgerissen hat, sind sie in der nächsten Szene wieder da.
\end{itemize}
}


\subsection{Stop! Oder mein Hund schießt}
Als sich Homer in einem Maisfeld-Labyrinth verirrt, kommt sein Hund Knecht Ruprecht ihm zu Hilfe und geleitet ihn wieder heraus. Daraufhin schickt ihn Homer zur Polizei-Hundeschule. Als er später Bart eher beißt, als mit ihm fangen zu spielen, bekommt Bart ein neues Haustier: eine Pythonschlange. Als er sie mit zur Schule nimmt, kommt sie ihm abhanden und verursacht eine Giftgaswolke, die sich über Bart legt. Jetzt kann nur noch Knechts Ruprecht die Situation retten.

\notiz{
\begin{itemize}
  \item Cletus und Brandine wohnen in der Rural Road 27.
	\item Filmzitat I: In Barts Fantasie wird aus Knecht Ruprecht, der wie \glqq Robocop\grqq\ aussieht, sein Motorrad im Stile von \glqq Transformers\grqq .
	\item Filmzitat II: Lous Polizeiwagen hat die Kennung K-9; eine Anspielung auf den Film \glqq K-9 Mein Partner mit der kalten Schnauze\grqq .
	\item Fehler I: Im Chemielabor mischen sich die Flüssigkeiten Ethanol und Salpetersäure, daraus entsteht kein giftiges Gas sondern eine explosive Flüs\-sig\-keit.
	\item Fehler II: Im Labyrinth schiebt Homer den Mais zur Seite und es kommt ein Schild zum Vorschein: \glqq \dots electrified\grqq . Die Untertitel zeigen aber als Übersetzung \glqq elektrifiziert\grqq\ anstatt \glqq elektrisiert\grqq .
\end{itemize}
}

\subsection{24 Minuten}\label{JABF14}
Rektor Skinner eröffnet eine CTU\index{CTU} -- Counter Truancy Unit (Schwänzen-Zähl-Einheit) -- in der Springfielder Grundschule, bei der Lisa das Kommando führt. Homer soll einen Behälter mit Joghurt entsorgen, bei dem das Mindesthaltbarkeitsdatum so weit abgelaufen ist, dass er giftig sein könnte, aber der stinkende Joghurt gerät in die Hände von Dolph, Jimbo und Kearney. Im Stile von Jack Bauer\index{Bauer!Jack} und Chloe O'Brian\index{O'Brian!Chloe} müssen Bart und Lisa die Schläger davon abhalten, die ultimative Stinkbombe beim Backwarenverkauf der Schule loszulassen.

\notiz{
\begin{itemize}
  \item Der typische Simpsons-Vorspann fehlt (kein Couchgag). Der Vorspann ist ähnlich dem der Serie \glqq 24 Stunden\grqq .
	\item Willie ist zu sehen, wie er Ms. Hoover küsst.
	\item Jimbo hat eine Schwester.
	\item In dieser Episode erfahren wir die vollständigen Namen von Jimbo, Kearney und Dolph:
	\begin{itemize}
		\item Corky James Jones
		\item Kearney Zzyzwicz
		\item Dolphin Starbeam
	\end{itemize}
\end{itemize}
}


\subsection{Das böse Wort}\label{JABF15}
Auf dem Heimweg vom Zahnarzt stoppen die Simpsons für ein Eis und als Homer die einmillionste Eistüte kauft, wird er bei Kent Brockmans nächtlichem News-Programm \glqq Smartline\index{Smartline}\grqq\ interviewt. Allerdings wird das Interview zu einem Problem, als Brockman einen schockierenden Kraftausdruck benutzt, als Homer ihn mit Kaffee überkippt. Niemand bekommt das Schimpfwort mit, bis Ned Flanders sich eine Aufzeichnung der Sendung ansieht. Daraufhin wird Kent zum Wetterfrosch degradiert und für ihn übernimmt Arnie Pye\index{Pye!Arnie} die Nachrichten. Der Sender muss eine Strafe in Höhe von 10 Millionen Dollar zahlen. Schließlich wird Kent doch gefeuert und er zieht bei den Simpsons ein und will die Machenschaften seines Senders über Lisas Webcam in das Internet stellen. Der Sender weiß sich nicht anders zu helfen, als Kent wieder mit einer 50-prozentigen Gehaltserhöhung einzustellen.

\notiz{
\begin{itemize}
	\item Dies ist die 400. Folge der Simpsons.
	\item Homer ist seit 1979 Mitglied in der \glqq American Applesauce Association\grqq\ (amerikanische Gesellschaft für Apfelmus).
	\item Der Newshund, das Maskottchen der Kanal 6 Nachrichten, gewann sieben Emmys und war 15 Mal dafür nominiert.
	\item Kent Brockman wurde auch in der Episode \glqq Krusty, der TV-Star\grqq\ (siehe \ref{9F19}) entlassen.
	\item Im Haus der Simpsons hängen die Fotos von Personen, die vorübergehend bei ihnen gewohnt haben. In Klammern ist die jeweilige Folge angegeben: Artie Ziff (\glqq Rat mal, wer zum Essen kommt\grqq , \ref{FABF08}), Apu (\glqq Apu der Inder\grqq , \ref{1F10} und \glqq Hochzeit auf indisch\grqq , \ref{5F04}), Krusty (\glqq Krusty, der TV-Star\grqq , \ref{9F19}), Willie (\glqq Ein perfekter Gentleman\grqq , \ref{HABF05}), Nelson (\glqq Der Feind in meinem Bett\grqq , \ref{FABF19}), Sideshow Bob (\glqq Und der Mörder ist\dots\grqq , \ref{EABF01}), Otto (\glqq Der Fahrschüler\grqq , \ref{8F21}), Kang oder Kodos (weder Kang noch Kodos wohnten bei den Simpsons), Gil (\glqq Kill Gil: Vol. 1 \& 2\grqq , \ref{JABF01}) und Stampfi (\glqq Bart gewinnt Elefant!\grqq , \ref{1F15}).
\end{itemize}
}

\section{Staffel 19}

\subsection{Die unglaubliche Reise in einem verrückten Privatflugzeug}
Homer rettet Mr. Burns vor dem Ertrinken in einem Springbrunnen und wird zum Dank auf eine Spritztour in seinem Privatjet eingeladen. Weil ihm das Fliegen so gut gefällt, will sich Homer eine neue Arbeit suchen, bei der er von Firma zu Firma jetten kann. Er kündigt seinen alten Job, ist jedoch schon bald arbeitslos, weil er die neue Stelle nicht bekommt und verschweigt dies aber seiner Familie. Eines Tages wird er jedoch von Bart tagsüber bei Krusty-Burger entdeckt. Er will Marge seine Situation erklären und mietet dazu einen Privatjet, den er am Ende sogar selbst landen muss. 

\notiz{
\begin{itemize}
	\item Im Vorspann sind die Reste der Stadt Springfield aus dem Kinofilm zu sehen und auf der Couch sitzt Spiderschwein\index{Spiderschwein}.
	\item Der reiche Texaner, der sonst im Ölgeschäft tätig ist, besitzt in dieser Episode eine Kupferrohrfabrik.
	\item Mr. Burns Einkaufsliste umfasst folgende Artikel:
	\begin{itemize}
	  \item Laudanum (Opiumtinktur)
	  \item Cotton Gin (Entkörnungsmaschine für Baumwolle)
	  \item Spats (Gamaschen)
	  \item Cell Phone (Mobiltelefon)
	  \item Brooklyn Dodgers (Baseballmannschaft, die seit dem Umzug 1958 nach Los Angeles unter dem Namen Los Angeles Dodgers spielen)
  \end{itemize}
\end{itemize}
}

\subsection{Homerotti}\label{JABF18}
Auf der Suche nach einem günstigen Essen landen die Simpsons auf einer Trauerfeier. Als sich Homer dann noch als Sargträger versucht, stürzt er in das offene Grab und verstaucht sich den Rücken. Im Krankenhaus entdeckt Dr. Hibbert, dass Homer eine klassische Singstimme hat, wenn er auf dem Rücken liegt. Mr. Burns heuert ihn für das Opernhaus in Springfield an, wo er schnell zum Star avanciert, auch wenn er alle Rollen im Liegen absolvieren muss. Ein weiblicher Fan allerdings geht zu weit und entwickelt sich zu einer Stalkerin, derer sich Homer fortan mühsam erwehren muss. 

\notiz{
\begin{itemize}
	\item In Springfield gibt es ein Restaurant, das \glqq Luftwaffles\index{Luftwaffles}\grqq\ heißt.
	\item Mr. Burns ist der Gründer und Intendant der Springfielder Oper.
\end{itemize}
}

\subsection{Abgeschleppt!}\label{JABF21}
Homer ist verzweifelt auf der Suche nach Milch für Maggie und landet plötzlich in dem Ort Guidopolis\index{Guidopolis}. Weil er falsch geparkt hat, wird er kurzerhand abgeschleppt und lernt so den Abschleppfahrer Louie\index{Louie} kennen. Schnell stellt Homer fest, dass das eigentlich sein Traumjob wäre. Homer kauft ihm für 500 Dollar einen Abschleppwagen ab und verspricht, damit lediglich in Springfield tätig zu werden. Allerdings übertreibt er so heftig, dass sich Springfields Einwohner zusammentun und einen Plan schmieden, wie sie Homer außer Gefecht setzen können. Am Ende muss Maggie, die in dieser Folge lernt, etwas selbstständiger durchs Leben zu gehen, ihren Vater aus einer misslichen Lage befreien. 

\notiz{
\begin{itemize}
  \item Der reiche Texaner leidet an Pogonophobie\index{Pogonophobie}, d.\,h. er hat Angst vor Gesichtsbehaarung und Schnurrbärten.
  \item Agnes Skinner war dreimal mit Abschleppfahrer verheiratet.
  \item Mr. T\index{Mr. T} spielt den König der Löwen im Capital City Playhouse.
	\item Der Teddywolf von Maggie heißt \glqq Justin Timberwolf\index{Timberwolf!Justin}\grqq , eine Anspielung auf den Sänger Justin Timberlake.
	\item Lenny behauptet, seine Mutter wurde von Stalin 20 Jahre lang in ein Zwangsarbeiterlager gesteckt.
	\item Das Kennzeichen Lennys Auto lautet \glqq DUI\footnote{\glqq DUI\grqq\ steht im Englischen für \glqq driving under the influence\grqq\ (Fahren unter Alkohol- oder Drogeneinfluss).} GUY\grqq .
\end{itemize}
}

\subsection{Ich will nicht wissen, warum der gefangene Vogel singt}\label{JABF19}
Während Homer rechtzeitig in der Schule ist, um der Auszeichnung Lisas zur Schülerin des Jahrtausends beizuwohnen, wird Marge bei einem Banküberfall als Geisel genommen. Sie kann dem Geiselnehmer Dwight\index{Dwight} jedoch einreden, dass er sich stellt, falls sie ihn im Gegenzug im Gefängnis besucht. Er stellt sich, aber Marge besucht ihn nicht. Dwight bricht aus dem Gefängnis aus, entführt Marge wieder, um in dem Vergnügungspark, in dem er von seiner Mutter verlassen wurde, mit Marge einen Tag zu verbringen.

\notiz{
\begin{itemize}
	\item Gil arbeitet als Wachmann in einer Bank und wird bei einem Banküberfall getötet.
	\item Auf der Liste mit den Frauen, die Homer nach Marges Tod heiraten kann, stehen die folgenden Namen: Lindsey Naegle, Booberella, Blythe Danner und Feed Cat.
	\item Hans Maulwurf arbeitet in der \glqq First Bank of Springfield\grqq .
	\item Snake bittet seine Freundin Gloria den letzten Bearbeiter seiner Biografie in Wikipedia umzubringen.
	\item Die Nummer des Häftlings im Film ist der Produktionscode (JABF19) dieser Folge. 
	\item Laut Kent Brockman wurde die Icthy \&\ Scratchy Folge \glqq The Un-Natural\grqq\ mit einem Annie Award ausgezeichnet.
	\item Als Marge auf Homers anruft, wird Big Blue\index{Big Blue} angezeigt. Big Blue ist auch der Spitzname der Firma IBM\index{IBM}.
\end{itemize}
}

\subsection{Kleiner Waise Milhouse}\label{JABF22}
Milhouses Eltern heiraten erneut und kehren aber von ihrer Flitterwochen-Kreuzfahrt nicht mehr zurück. Nun kommt Milhouse als Waise bei den Simpsons unter. Er entschließt sich, nun zu einem echten Kerl zu werden. Auf sich allein gestellt, macht Milhouse eine erstaunliche Wandlung durch und wird dadurch zum coolsten Jungen der Schule, um das zu ändern, macht Bart Norbert Van Houten\index{Van Houten!Norbert}, einen europäischen Onkel von Milhouse, ausfindig. Derweil bekommt Homer Ärger, weil er sich nicht an Marges Augenfarbe erinnern kann.

\notiz{
\begin{itemize}
	\item Laut dieser Episode hat Luann Van Houten dänische und Kirk Van Houten niederländische Wurzeln. In der Folge \glqq Marge und der Frauen-Club\grqq\ (siehe \ref{GABF22}) ist zu erfahren, dass Milhouse eine italienische Großmutter hat. 
	\item Marges Augenfarbe ist grün-braun.
	\item Die ersten drei Biergebote auf Homers Krawatte:
  \begin{itemize}
	  \item Thou Shalt Not Spill (Du sollst nichts verschütten)
	  \item Honor Thy Lager (Ehre Dein Lager)
	  \item Thou Shalt Not Bear False I. D. 
  \end{itemize}
\end{itemize}
}


\subsection{Nach Hause Telefonieren}
In dieser Halloween-Folge werden folgende drei Episoden gezeigt:
\begin{itemize}
	\item \textbf{E.T. go home}\\ Bart findet Alien Kodos (\glqq Kodos the Destroyer\grqq ) im Butangaslagerschuppen im Garten und versteckt es. Bart und Lisa helfen Kodos ein \glqq Telefon\grqq\ zu bauen. Natürlich findet Marge und Homer heraus, dass Bart und Lisa Kodos verstecken, kurz danach kommt auch die Regierung auf den Plan. Daraufhin kommt es zu einer Verfolgungsjagd, die sehr an E.T.\index{E.T.} erinnert.
	\item \textbf{Mr. \& Mrs. Simpson}\\ Homer ist ein Profikiller, der von Mr. Burns den Auftrag bekommt, Kent Brockman zu ermorden. Doch bevor er seinen Auftrag erfüllen kann, kommt ihm Marge zuvor. Es kommt zu einer wilden Schießerei. In Folge dessen werden Abraham Simpson und Chief Wiggum ermordet.
	\item \textbf{Heck House}\\ Ned erhält von Gott die Macht, den streichversessenen Kindern Bart, Nelson, Milhouse und Lisa, die es mit den Halloween-Streichen übertreiben, eine ordentliche Lektion zu erteilen. 
\end{itemize}

\notiz{
\begin{itemize}
	\item Marge bekommt als Profikillerin für jeden Auftrag 50000 Dollar, während Homer nur den Geldbörseninhalt der Opfer behalten darf.
	\item Die PIN, mit der Homer im Bad die Waffen zugänglich macht, lautet 19537.
	\item In der Teilepisode \glqq Heck House\grqq\ ist wieder Spiderschwein\index{Spiderschwein} zu sehen.
	\item Diese Episode war 2008 für den Emmy nominiert.
\end{itemize}
}

\subsection{Szenen einer Ehe}
Der neue Comicbuchladen \glqq Coolsville\index{Coolsville}\grqq\ eröffnet in Springfield und zwingt den Comic\-buch\-ver\-käu\-fer zur Geschäftsaufgabe. Marge macht daraufhin in dessen Laden ein Fitnessstudio namens \glqq Shapes\index{Shapes}\grqq\ speziell für Frauen auf, was sie in kurzer Zeit zur erfolgreichen Geschäftsfrau werden lässt. Homer macht sich deshalb Sorgen, da Marge sich einen jüngeren und attraktiveren Mann suchen könnte und er versucht, mit den Mitteln der Schönheits\-chirurgie dagegen anzugehen. 

\notiz{
\begin{itemize}
	\item Eine Oma von Milhouse heißt Sophia.
	\item In Marges Fitnessstudio gilt: Keine Männer, keine Handys, keine Spiegel und keine Scham.
	\item Homer geht zu dem Schönheitschirurgen, bei dem Moe in der Episode \glqq Moe mit den zwei Gesichtern\grqq\ (siehe \ref{BABF12}) auch war.
\end{itemize}
}

\subsection{Begräbnis für einen Feind}
Ein Fernsehspot leitet den neuesten diabolischen Plan Sideshow Bobs ein, Bart unter die Erde zu bringen. Doch diesmal geht scheinbar einiges schief und Bob muss sich vor Gericht verantworten. Dort gelingt es ihm, die öffentliche Meinung gegen Bart zu wenden, was sich noch verstärkt, als Bob an einem Herzinfarkt stirbt, nachdem Bart eine vermeintliche Bombe aus dem Fenster wirft, die sich im Nachhinein als Bobs Medizin herausstellt. Aber natürlich ist Bob gar nicht tot, alles verläuft nach Plan und nur Lisa kann eventuell noch Barts Untergang verhindern. 

\notiz{
\begin{itemize}
	\item Eine Poochie-Figur ist auf der Parade in der Itchy- und Scratchy-Episode zu sehen.
	\item Kelsey Grammer (Sideshow Bob), David Hyde Pierce (Cecil) und John Mahoney (Vater) spielen auch in der Serie \glqq Frasier\index{Frasier}\grqq\ in der selben Familienkonstellation. 
	\item In der Werbepause sind u.\,a. der Meister-Glanz-Werbespot (siehe \glqq Marge als Seelsorgerin\grqq , \ref{4F18}) und der Werbespot Homers Sicherheitsdienstes Springshield (\glqq Sicherheitsdienst Springshield\grqq , \ref{DABF17}) zu sehen.
	\item Der Name des Restaurants \glqq Wes Doobner's World Famous Family Style Rib Huts\grqq\ ist ein Anagramm für \glqq Sideshow Bob's World Famous Family Style Return\grqq .
\end{itemize}
}


\subsection{Vergiss-Marge-Nicht}\label{KABF02}
Homer wacht im Schnee auf und kommt um 6 Uhr morgens nach Hause. Da es seelenruhig im Haus ist, will sich Homer in das Bett schleichen und sagen, er sei früher zu Hause gewesen, doch niemand ist zu Hause. Er bittet den Hund via Internet zu zeigen, wo seine Familie ist, doch Knecht Ruprecht ist überraschend aggressiv zu ihm. Er geht zu Moe und fragt ihn nach Rat. Dieser erklärt, er hat ihm einen \glqq Vergiss-es-Surprise\grqq\ gemixt, der den letzten Tag vergessen lässt. Homer dreht sich um und sieht Chief Wiggum, worauf er einen Flashback bekommt. Er erinnert sich, dass Marge ein blaues Auge hat. Zu Hause bekommt er einen weiteren Flashback über Marge und ihr blaues Auge. Er fragt seinen Vater, was er am letzten Tag gemacht hat und dieser schickt ihn zu Professor Frink, der ein Gerät entwickelt hat, dass die Erinnerungen zurückholt. Homer beginnt eine Reise durch all seine Erinnerungen und ertappt Marge, wie sie mit einem muskulösen Mann auf seiner Couch sitzt. Er bittet Lisa und Bart um Hilfe. Er sieht, wie Marge mit Duffman auf der Couch sitzt und will daraufhin von der Brücke springen. Dann entscheidet er sich um und Marges Schwestern schubsen ihn von der Brücke. Homer bekommt den berühmten Flashback, den man erhält, wenn man kurz vor dem Tod ist. Sein ganzes Leben läuft vor seinen Augen ab. In diesem Flashback erinnert er sich daran, was wirklich gestern passiert ist und landet daraufhin glücklicherweise auf einem Duff-Schiff in einer Hüpfburg. Auf dem Schiff findet eine organisierte Party von Marge für Homer statt. Da Homer gestern zu Hause mitbekommen hat, dass eine Party für ihn organisiert wurde, nahm er den \glqq Vergiss-es-Surpirse\grqq\ um sich selbst auf der Party zu überraschen. Die Party startet voll durch. Am Ende steht hinten auf dem Schiff \glqq The End\grqq . Das Schiff zieht ein kleines Boot hinter sich her, auf dem \glqq ?\grqq\ steht, was somit \glqq The End?\grqq\ ergibt. 

\notiz{
\begin{itemize}
	\item Diese Episode gewann 2008 einen Emmy.
	\item Duffman leidet an Leseschwäche.
	\item Als Homer von der Brücke fällt, zieht sein Leben an ihm vorüber. Am Ende der Sequenz wird deutlich, dass es sich um ein YouTube-Video handelt, welches \glqq Picture a Day for 39 Years\grqq\ benannt ist. 
\end{itemize}
}


\subsection{Hello, Mr. President}
Homer zerstört Springfields Fast-Food-Boulevard durch ein Feuer, weshalb auf einer Bür\-ger\-ver\-samm\-lung der Wiederaufbau durch eine Anleihe beschlossen wird. Dazu zieht Bürgermeister Quimby die Vorwahlen für die Prä\-si\-dent\-schafts\-kan\-di\-dat\-en vor. Traditionell ist in den USA New Hampshire der erste Bundesstaat, in dem Vorwahlen stattfinden. Die Änderung lockt schnell sämtliche Medienvertreter in die Stadt. Auch die Kandidaten geben alles, um unentschlossene Wähler (wie die Simpsons) auf ihre Seite zu ziehen. Um sich gegen das Spektakel zu wehren, beschließen die Bürger der Stadt, einen unmöglichen Kandidaten in das Rennen zu schicken, um den sich jedoch schon bald die großen Parteien reißen: Ralph Wiggum.
 
\notiz{
\begin{itemize}
	\item In Moes Taverne ist laut eines Schildes dienstags Frauenabend.
	\item Bei der Wahlversammlung der Demokraten sind u.\,a. Patty, Bürgermeister Quimby, der Bienenmann und Lindsey Naegle zu sehen.
\end{itemize}
}



\subsection{Die wilden 90er}\label{KABF04}
Es ist Winter und die Simpsons sitzen frierend vor dem Kamin, da Homer die Heizungsrechnung nicht bezahlt hat. Bart wirft eine Kiste mit Zeitschriften in das Feuer und findet dabei ein Abschlusszeugnis der Springfield Universität von Marge. Dieses wirft einige Fragen auf, denn Bart und Lisa wussten nicht, dass Marge nach der Highschool noch auf der Universität war. Sie wollen wissen, was ihre Eltern in dieser Zeit gemacht haben und so erzählt Homer ihnen die Geschichte über die wilden Neunziger Jahre. Marge fühlte sich zu einem Universitätsprofessor hingezogen und Homer begründete mit einer Band den Musikstil Grunge.

\notiz{
\begin{itemize}
	\item Der Name von Homers Band lautet \glqq Sadgasm\index{Sadgasm}\grqq . In der Band spielen neben Homer (Gesang und Gitarre) noch Lenny (Gitarre), Carl (Bass) und Lou, der Polizist (Schlagzeug).
	\item Im Video zu \glqq Margarine\grqq\ aus dem Album \glqq Desolation Hatchback\grqq\ führte David Mirkin (ein ehemaliger Autor der Serie) Regie.
	\item Abe Simpson betrieb das \glqq Laser Tag\index{Laser Tag}\grqq .
	\item Miss Hoover ist in der Vorlesung zur Kulturgeschichte an der Springfield Universität zu sehen.
	\item Homer ist Diabetiker.
\end{itemize}
}


\subsection{Debarted -- Unter Ratten}
Ein neuer Schüler namens Donny\index{Donny} gefährdet Barts Popularität an der Schule. Als er jedoch die Verantwortung für einen von Barts Streiche übernimmt und diesen so vor der Strafe rettet, gewinnt er Barts Vertrauen. Bart nimmt ihn in seine Bande auf und plant einen großen Coup, nicht ahnend, dass Donny ein Spitzel ist, der von Rektor Skinner beauftragt wurde, Bart auszuspionieren. Unterdessen genießt Homer mit Marge die Freuden eines kostenlos zur Verfügung gestellten Ersatzautos, während der eigene Wagen in Reparatur ist. 

\notiz{
\begin{itemize}
	\item Als Bart in der Dunkelkammer das Licht anmacht, gehen dadurch Fotos von Martin Prince kaputt, auf denen u.\,a. ein UFO, das Monster von Loch Ness und Martin, als er von Elvis umarmt wird, zu sehen sind.
	\item Homer hat eine Taxifahrerlizenz.
	\item Willie erhält den Grundschulabschluss, weil er sich als Skinners Spitzel betätigt.
\end{itemize}
}

\subsection{Bei Absturz Mord}
Bart spielt Martin im Park einen Streich mit scheinbar fatalen Folgen. Nicht nur Bart auch Lisa glaubt daraufhin, an Martins Tod schuld zu sein. Zu ihrer beider Verdruss betätigt sich Nelson als Detektiv, der versucht, die wahren Hintergründe dieses \glqq Unfalls\grqq\ aufzudecken. Währenddessen setzt Marge Homer mal wieder auf Diät. Als er aber nach einer Woche sieben Pfund zugenommen hat, lässt sie ihn heimlich vom Team der Realityshow Sneakers\index{Sneakers} ausspionieren und es stellt sich heraus, dass er schummelt.

\notiz{
\begin{itemize}
  \item Martin Prince züchtet Schmetterlinge und er hat Höhenangst.
	\item Fernsehzitat: Nelson verhält sich bei der Aufklärung des Unfalls wie Columbo\index{Columbo}.
\end{itemize}
}

\subsection{Schall und Rauch}\label{KABF08}
Marge geht mit Lisa zu einem Ballettlehrgang, wo Lisas Talent entdeckt wird -- sehr zur Freude ihrer Mutter, die früher selbst Ballerina werden wollte. Lisa muss jedoch feststellen, dass alle Tänzerinnen intensiv rauchen und dass auch ihr der Passivrauch scheinbar hilft, besser zu werden. Die Versuchung ist groß, aber Homer kommt ihr auf die Schliche und ist verständlicherweise entsetzt. Lisa muss sich zwischen dem Traum ihrer Mutter und ihrer Gesundheit entscheiden. 

\notiz{
\begin{itemize}
	\item Das letzte Angelica Button-Buch\index{Button!Angelica} heißt \glqq Angelica Button and the Teacup of Terror\grqq .
	\item Das Autokennzeichen von Marges Auto lautet WA1A2N.
\end{itemize}
}


\subsection{Die Sünden der Väter}
Lisa geht die finanziellen Aufzeichnungen der Stadt durch und entdeckt, dass Springfield Millionen an Steuergelder fehlen. Daraufhin wird von den meisten Steuerhinterziehern das Geld eingetrieben, die einzige Schuldnerin bleibt Lurleen Lumpkin\index{Lumpkin!Lurleen}. Während die Polizei versucht, sie zu finden, bleibt Lurleen bei den Simpsons. Sie wird dort mit ihrem Vater Royce\index{Lumpkin!Royce} wiedervereinigt, der dann ihre Lieder an die Dixie Chicks\index{Dixie Chicks} verkauft, während sie in Moes Taverne als Kellnerin arbeitet. Schließlich tritt sie als Vorgruppe der Dixie Chicks auf ihrer Tournee auf. 

\notiz{
\begin{itemize}
	\item Lurleen Lumpkin ist 34 Jahre alt und war dreimal verheiratet.
	\item Die Steuerschulden von Mr. Burns und Bürgermeister Quimby werden nicht eingetrieben.
	\item Lenny und Carl suchen anscheinend eine Partnerin, weil beide in Moes Taverne Lurleen anmachen.
\end{itemize}
}

\subsection{Rinderwahn}
Bart tritt der Jugendorganisation 4-H\index{4-H} bei. Dort soll er sich um ein Kalb kümmern und dieses groß ziehen. Als er merkt, dass sein Bulle gemästet und anschließend geschlachtet werden soll, retten er und Lisa ihn vor dem Gang zum Schlachthaus und schenken ihn dem Bauernmädchen Mary. Ihr Vater Cletus glaubt irrtümlich, dass das ein Zeichen zur Heirat der beiden sein soll. Homer und Marge müssen sich daraufhin einen Plan ausdenken, die Hochzeit zu verhindern und den Bullen zu retten. Apu kommt ihnen zu Hilfe und sorgt für einen Transport des Bullen nach Indien.

\notiz{
\begin{itemize}
	\item Bart nennt den Bullen Lou.
	\item Neben Bart gehören u.\,a. noch Martin, Nelson, Ralph, Sherry und Terry der Jugendorganisation 4-H an.
	\item Die Sprecherin der Tierstimmen-CD \glqq Anguished Animals III\grqq\ ist Tress McNeille, die bei den Simpsons u.\,a. Agnes Skinner im Original synchronisiert.
	\item Eines der Hochzeitsgeschenke ist Spiderschwein\index{Spiderschwein}.
	\item Filmzitat: Als sich Bart von Lou mit den Worten \glqq Ich schau Dir in die Augen Kleiner\grqq\ ist eine Anspielung auf den Film \glqq Casablanca\grqq .
\end{itemize}
}


\subsection{Die Liebe in Springfield}
Dank eines Streichs von Bart und ein wenig Gelatine stecken Marge und Homer im \glqq Tunnel of Love\grqq\ fest. Daher vertreiben sie sich die Zeit mit drei Geschichten von Liebespaaren.
\begin{itemize}
	\item Die erste Geschichte erzählt von Homer und Marge als \glqq Bonnie und Clyde\grqq .
	\item \textbf{Shady and The Vamp}\\ Diese Geschichte handelt wiederum von Marge und Homer, diesmal in \glqq Susie und Strolch\grqq .
	\item Geschichte Nummer drei zeigt Nelson und Lisa als Mitglied der Punk Band Sex Pistols\index{Sex Pistols} \glqq Sid Vicious\index{Vicious!Sid}\grqq\ und dessen Freundin \glqq Nancy Spungen\index{Spungen!Nancy}\grqq . 
\end{itemize}

\notiz{
\begin{itemize}
  \item Die Bonnie und Clyde Stunde präsentiert Thompson Maschine Gun.
	\item Die Sex Pistols bestehen aus Bart (Sänger), Jimbo (Gitarre), Nelson (Bass) und Dolph (Schlagzeug).
\end{itemize}
}

\subsection{Down by Lisa}\label{KABF11}
Lisa dreht eine Dokumentation über ihre Familie für ein Schulprojekt. Sie ist völlig begeistert, dass ihr Film anschließend beim \glqq Sundance Film Festival\grqq\ vorgeführt werden soll. An der Premiere fallen Homer, Marge und Bart aus allen Wolken, als sie merken, in welchem schrägen Licht die Familie da gezeigt wird. Die Aufregung legt sich allerdings, als ein Film von Nelson Muntz gezeigt wird. Denn anschließend ist er der Star des Festivals. 

\notiz{
\begin{itemize}
	\item Als Bart vorschlägt, noch einen Film über die Simpsons zu drehen, meint Marge, dass ein Film genug sei.
	\item Das Kino ist Rektor Skinners geheime Leidenschaft. Er schrieb schon mehrere Drehbücher, die alle von den Studios abgelehnt worden sind.
	\item Lisas und Nelsons\footnote{Nelsons Film heißt \glqq Life Blows Chunks\grqq .} Filme werden von Rektor Skinner und Oberschulrat Chalmers Produktionsfirma \glqq chalmskinn\index{chalmskinn}\grqq\ produziert.
\end{itemize}
}


\subsection{Lebwohl, Mona}\label{KABF12}
Mona Simpson muss nicht mehr vor den Behörden fliehen und kehrt in das Haus der Simpsons zurück. Homer will ihr nicht nochmal so viel Vertrauen schenken, da er ihr nicht glaubt, dass sie bleibt. Homer gibt kein Stück nach und Mona sieht ein, dass hier nur die Zeit weiterhelfen kann. Als Homer am nächsten Morgen in das Wohnzimmer kommt, muss er feststellen, dass seine Mutter verstorben ist. Um sein unnachgiebiges Verhalten etwas gut zu machen, entschließt er sich, den letzten Willen seiner Mutter umzusetzen. Dazu muss er auf den höchsten Punkt im Springfielder Nationalpark steigen und dort die Asche seiner Mutter verstreuen. Wie sich herausstellt, befindet sich in dem Berg eine geheime Raketenabschussrampe. Mit den aus dem Berg startenden Raketen entsorgt Mr. Burns den radioaktiven Giftmüll aus seinem Atomkraftwerk. 

\notiz{
\begin{itemize}
  \item Mona hinterlässt Bart ein Schweizer Armeetaschenmesser, Lisa ihren Kampfgeist und Marge eine Hanftasche.
	\item Diese Episode ist Elsie Castellaneta (Mutter von Dan Castellaneta) und Dora K. Warren (Mutter von Harry Shearer) gewidmet.
\end{itemize}
}


\subsection{Alles über Lisa}\label{KABF13}
Die gesamte Geschichte wird aus der Sicht von Sideshow Mel geschildert. Lisa stiehlt als neue Assistentin Krustys ihm die Show und bekommt ihre eigene Sendung. Doch die Honorierung mit dem Preis für den besten Entertainer des Jahres und ein dezenter Hinweis von Sideshow Mel lassen sie nachdenklich werden. Bart, der die Assistentenrolle haben wollte, geht zwischenzeitlich mit Homer einem neuen Hobby nach und zwar dem Münzsammeln.

\notiz{
\begin{itemize}
	\item Krustys Show \glqq Last Gasp\grqq\ wird u.\,a. von Nappien\index{Nappien} gesponsert. Homer nutzte in der Folge \glqq Brand und Beute\grqq\ (siehe \ref{JABF13}) diese Schlaftabletten.
	\item Der Comicbuchverkäufer ist erstmalig mit offenem Haar zu sehen.
	\item Sideshow Mel ist verheiratet und seine Frau ist schwanger.
	\item Krusty produziert seine 4000ste Show.
	\item Bei der Preisverleihung zum \glqq Entertainer des Jahres\grqq\ ist Spiderschwein\index{Spiderschwein} im Smoking zu sehen.
	\item Mr. Burns ersteigert die Münze der \glqq küssenden Lincolns\grqq\ für 10 Millionen Dollar.
\end{itemize}
}

\section{Staffel 20}

\subsection{Kuchen, Kopfgeld und Kautionen}\label{KABF17}
Homer wird nach einer Schlägerei auf der St. Patrick's Day Parade in das Gefängnis gesteckt. Nach seiner Entlassung lernt er den Kopfgeldjäger Wolve\index{Wolve} kennen, dieser überredet Homer dazu, selbst Kopfgeldjäger zu werden. Nachdem Ned Flanders Homer das Leben gerettet hatte, beschließen beide als Kopfgeldjäger zu arbeiten. Homers neue Tätigkeit bringt viel ein, aber Ned verabscheut Homers Methoden bei den Festnahmen der Gesuchten. Unterdessen fängt Marge in der erotischen Bäckerei \glqq Au Naturel\index{Au Naturel}\grqq\ von Patrick Farley\index{Farley!Patrick} an zu arbeiten. 

\notiz{
\begin{itemize}
	\item Gil Gunderson ist zu sehen, obwohl er in der Episode \glqq Ich will nicht wissen, warum der gefangene Vogel singt\grqq\ (siehe \ref{JABF19}) erschossen wurde.
	\item Homer steht auf die Hard Rock Band AC/DC\index{AC/DC}.
	\item Gloria, Snakes Freundin, ist offenbar schwanger.
	\item Homers und Neds Kopfgeldagentur heißt \glqq Good Neighbors Bounty Hunters\grqq .
	\item Homers Kaution beträgt 25.000 Dollar.
	\item Zu den Kunden im Au Naturle zählen u.\,a. Selma und Patty, Dr. Hibbert und Waylon Smithers.
	\item Snake gibt an, in Princeton\index{Princeton} studiert zu haben.
\end{itemize}
}

\subsection{Jäger des verlorenen Handys}\label{KABF15}
Bart möchte unbedingt auch ein Handy, da alle seiner Schulkameraden eines haben. Um es zu finanzieren, sammelt er Golfbälle im Golfclub. Doch Willie macht ihm einen Strich durch die Rechnung. Als Denis Leary\index{Leary!Denis} abschlagen will, klingelt sein Handy und vor lauter Wut über die dauernden Störungen, wirft er es weg. Das Mobiltelefon landet genau vor Barts Füßen. Er nimmt es mit und tätigt einige Scherzanrufe. Als Marge mitbekommt, dass Bart ein Handy hat, geht sie der Sache nach. Leary schlägt vor, Bart per GPS zu beobachten. Da dies aber gegen Lisas Überzeugung ist, verrät sie Bart, dass er bespitzelt wird. Bart befestigt das Tracking-Modul an einem Vogel und hält so seine Familie auf Trab. Der Vogel fliegt schließlich nach Machu Picchu\index{Machu Picchu}\footnote{Machu Picchu ist eine gut erhaltene Ruinenstadt der Inka, die in 2.360 m Höhe auf einer Bergspitze der Anden über dem Urubambatal in der peruanischen Region Cusco in 75 km Entfernung nordwestlich der Stadt Cusco liegt \cite{WikiMachuPicchu}.}, das Lisa unbedingt sehen wollte.

\notiz{
\begin{itemize}
	\item Diese Episode wurde Paul Newman\index{Newman!Paul} gewidmet, der im September 2008 im Alter von 83 Jahren verstorben ist.
	\item In der australischen Bar \glqq Crocodile Drunkee's\grqq\ ist ein INXS\index{INXS}-Poster zu sehen.
	\item Hausmeister Willie arbeitet an den Wochenenden und im Sommer als Greenkeeper im Country-Club.
	\item Maggie schreibt auf einen Zettel \glqq Me Sad 2\grqq .
	\item Moe wird vom FBI überwacht.
\end{itemize}
}

\subsection{Gemischtes Doppel}
Bart verhindert, dass Homer sich ein Los kauft. Als Lenny genau mit diesem Los 50.000 US\$ gewinnt, ist Homer extrem sauer auf Bart. Lenny aber teilt sein Glück mit seinen Freunden und lädt sie in das Hotel der Woosterfields\index{Woosterfield} ein. In diesem Hotel lernt Bart Simon Woosterfield kennen, der ihm sehr stark ähnelt und zufällig das Kind der reichsten Familie in Springfield ist. Bart und Simon tauschen die Rollen. So lernt Bart den Luxus und die Intrigen der Reichen und Simon bei den Simpsons das einfache Leben einer amerikanischen Durchschnittsfamilie kennen. Lisa kommt Simon auf die Schliche und er hilft ihr, Bart vor Simons Halbgeschwistern zu retten, die versuchen, ihn auf einem Skiausflug in Aspen umzubringen. 

\notiz{
\begin{itemize}
	\item Barney sagt, dass seine Mutter verstorben sei.
	\item Mr. Burns gibt an, einen Zwillingsbruder gehabt zu haben.
\end{itemize}
}

\subsection{Gefährliche Kurven}
Die Simpsons besuchen ein Stallwerk im Wald. Eine Reihe von Rückblicken stellen Homer und Marge während ihren ersten gemeinsamen Jahren dar, so wie sie das frisch verliebte Ehepaar Ned und Maude Flanders kennen lernen. Als Homer und Marge bemerken, dass einer von ihren wertvollsten Momenten auf Lügen basiert, versuchen die beiden Abstand zu halten. 

\notiz{
\begin{itemize}
	\item Der Teenager mit der Quietschstimme hat eine Freundin namens Beatrice.
	\item Das Stallwerk heißt \glqq Kozy Kabins\grqq .
\end{itemize}
}

\subsection{Das Kreuz mit den Worträtseln}
Homer stellt fest, dass er gut darin ist, Paaren beim \glqq Schlussmachen\grqq\ zu helfen. Schon bald entwickelt er daraus eine Geschäftsidee und wird von trennungswilligen Menschen engagiert, die ihren Partner loswerden wollen, so ist er u.\,a. für die Trennung von Edna und Seymour verantwortlich. Indes hat Lisa ihre Liebe für Kreuzworträtsel entdeckt und nimmt an der städtischen Kreuzworträtselmeisterschaft teil. Homer wettet sein frisch verdientes Geld auf Lisa und wird im Verlaufe des Turniers immer reicher. Beim Finale wettet er gegen und Lisa und gewinnt. Um Lisas Liebe zurückzugewinnen, bedient er sich der Hilfe des Kreuzworträtsel-Redakteurs der New York Times Will Shortz\index{Shortz!Will} und des Master-Kreuzworträtsel-Konstrukteurs Merl Reagle\index{Reagle!Merl}. 


\notiz{
\begin{itemize}
	\item Lennys aktuelle Freundin heißt Doreen. Sie ist auch in Moes Bar zu sehen.
	\item Bei der Kreuzworträtselmeisterschaft nehmen neben Lisa u.\,a. noch Dr. Hibbert, McCallister, Database, Arnie Pye, Prof. Frink, Richter Synder, Miss Hoover, Hans Maulwurf und Gil Gunderson teil. Gil gewinnt schließlich das Finale gegen Lisa.
\end{itemize}
}

\subsection{Bin runterladen}\label{KABF20}
Lisa ist begeistert: Im Einkaufszentrum von Springfield gibt es endlich einen Mapple Store\index{Mapple}, in dem die neuesten Produkte von MyPod\index{MyPod} bis MyPhone\index{MyPhone} angeboten werden. Da Krusty seinen MyPod loswerden möchte, bekommt Lisa diesen geschenkt und stürzt sich in entsetzliche Unkosten, da sie nicht aufhören kann, Songs und Videos von MyTunes\index{MyTunes} herunterzuladen. Indes lernt Bart den aus Jordanien stammenden Bashir kennen und freundet sich mit ihm an. Homer glaubt, dass Bashirs Eltern planen, die Springfield Mall zu zerstören. Also versucht er die Springfielder vor der drohenden Katastrophe zu warnen. 

\notiz{
\begin{itemize}
	\item Steve Mobbs nutzt eine Suchmaschine namens \glqq Oogle\index{Oogle}\grqq , um nach seinen Umsätzen zu suchen. 
	\item Homer erwähnt Bashier gegenüber, dass er 38 Jahre alt sei.
	\item Willie kann nach eigener Aussage weder Schreiben noch Lesen.
\end{itemize}
}

\subsection{Der Tod kommt dreimal}\label{KABF16}
Im Vorspann der Folge ist Homer zu sehen, als er zur Präsidentschaftswahl geht. Er muss feststellen, dass die elektronische Wahlmaschine manipuliert ist und seine Stimmen für Barack Obama John McCain gutgeschrieben werden.
\begin{itemize}
	\item \textbf{Untitled Robot Parody}\\ Bart kauft für Lisa zu Weihnachten ein Malibu Stacy Cabrio. Dieses besitzt jedoch einen Verstand und kann sich in verschiedenste Formen transformieren. Nach und nach aktiviert es alle elektrischen Geräte, bis überall in der Stadt \glqq Transformers\grqq\ herumlaufen, die sich gegenseitig angreifen. Als Marge die Transformers fragt, warum sie sich gegenseitig bekämpfen, kommen die Roboter auf die Idee, die Menschen anzugreifen.
	\item \textbf{How to get ahead in Dead -- Vertising}\\ Homer hat aus Versehen Krusty den Clown getötet. Kurz darauf melden sich einige Leute bei ihm, die ihn bitten, zu Werbezwecken weitere Prominente (George Clooney, Prince und Neil Armstrong) verschwinden zu lassen.
	\item \textbf{It's the Grand Pumpkin, Milhouse}\\ Milhouse glaubt fest daran, dass zu Halloween ein großer, lebender Kürbis erscheinen wird, doch niemand glaubt ihm, da Bart diese Geschichte nur erfunden hat. Aber der große Kürbis erscheint und die Lage verschlechtert sich erheblich, als er bemerkt, was mit Kürbissen an Halloween geschieht. 
\end{itemize}
    
    
\subsection{Ja, diese Biene, die ich meine, die heißt Monty}
Mr. Burns gewinnt im Milliardärsferiencamp beim Pokern das Basketballteam Austin Celtics\index{Austin Celtics}, die er später in Springfield Excitement\index{Springfield Excitement} umbenennt. Nachdem er den schrillen Dallas Mavericks-Besitzer Mark Cuban erlebt hat, möchte auch Burns mit einem spektakulären Rahmenprogramm sein Team zu neuem Ruhm führen. Er beschließt, ein super-modernes Stadion zu bauen. Indes hat Lisa festgestellt, dass in Springfield alle Bienen an Bienen-Masern verstorben sind. Mit der Hilfe von Prof. Frink gelingt es ihr, eine noch gesunde Königin ausfindig zu machen. Der Bau des Stadions gefährdet dabei aber die Bienenkolonie. Nun startet Lisa eine Kampagne, um die Bienen zu retten. Homer versucht Lisa zu helfen, indem er die Bienen mit afrikanischen Killerbienen paart. 


\notiz{
\begin{itemize}
	\item Der reiche Texaner ist ebenfalls im Feriencamp der Milliardäre.
	\item Als Willie die toten Bienen beklagt, sind folgende Grabsteine zu sehen: Buzz Aldrin, Bee Arthur, Gordon Summer (Sting), Bee Bee King, Susan Bee Anthony und Jerry Seinfeld.
	\item Burns Vermögen ist zu Beginn der Folge bei 1.800.037.022 US\$, am Ende bei 996.036.00 US\$.
	\item Es handelt sich um den gleichen weihnachtlichen Vorspann wie in der Episode \glqq Kill Gil: Vol. 1 \& 2\grqq\ (siehe \ref{JABF01}).
\end{itemize}
}

\subsection{Die Chroniken von Equalia}\label{KABF22}
Lisa lernt eine gewisse Juliet Hobbes\index{Hobbes!Juliet} kennen, mit der sie sich auf Anhieb blendend versteht, da sie nicht nur intelligent ist, sondern auch noch die gleichen Interessen wie Lisa hat. Schon bald werden sie beste Freundinnen und spinnen sich ein Traumreich zusammen, das sie \glqq Equalia\index{Equalia}\grqq\ nennen. Marge sorgt sich um ihre Tochter, da sich Lisa immer stärker aus der Realität zu entfernen scheint. Sie beschließt, Lisa den Umgang mit Juliet zu verbieten mit der Folge, dass die Mädchen durchbrennen.

\notiz{
\begin{itemize}
	\item Hausmeister Willie hieß bei Ankunft auf Ellis Island Dr. William MacDougal\index{MacDougal!Dr. William}.
	\item Der Comicbuchverkäufer unterrichtet im Freizeitzentrum Kung Fu.
\end{itemize}
}

\subsection{Quatsch mit Soße}\label{LABF01}
Vance Connor\index{Connor!Vance} wird auf dem Springfielder Wall of Fame geehrt. Homer erzählt, wie er bei der Schülerpräsidentenwahl in der Highschool gegen ihn antrat und verlor und sinniert vor sich hin, was wohl gewesen wäre, wenn er selbst zum Schülersprecher gewählt worden wäre. Von Lenny und Carl muss er erfahren, dass die Wahl damals gefälscht war. Ein Koch in Luigis Restaurant hilft Homer mit seiner magischen Tomatensauce zu sehen, wie sein Leben verlaufen wäre, hätte Homer die Wahl gewonnen. Er wäre erfolgreicher Manager in Mr. Burns Atomkraftwerk, hätte ein großes Haus auf dem Grundstück, auf dem das Haus der Flanders steht und er hätte dennoch Marge geheiratet. Aber beide hätten keine Kinder. 

\notiz{
\begin{itemize}
	\item Die erste Simpsons-Folge, die in HDTV ausgestrahlt wurde.
	\item Neuer Vorspann, der den Betrag 243,26 US\$ anzeigt, als Maggie über die Supermarktkasse gezogen wird. Marge kauft u.\,a. das Meister Glanz Waschmittel aus der Folge \glqq Marge als Seelsorgerin\grqq\ (siehe \ref{4F18}) und Tomacco\index{ToMacco} Saft aus der Episode \glqq Duell bei Sonnenaufgang\grqq\ (siehe \ref{AABF19}).
	\item Lenny und Carl schneiden sich gegenseitig die Haare.
	\item Lenny wohnt im \glqq The Lonely Arms\index{The Lonely Arms}\grqq .
	\item Vance Connor hat sowohl Lenny als auch Carl jeweils eine Niere gespendet.
\end{itemize}
}

\subsection{Beim Testen nichts Neues}\label{LABF02}
Das neue Schuljahr beginnt und die Schüler schreiben den V.P.A.T., einen Test der über die Fördergelder der Schule entscheidet. Deswegen beschließt Rektor Skinner, Bart und ein paar andere schlechte Schüler mit einem Trick von der Schule fernzuhalten, um die Testergebnisse zu verbessern. Oberschulrat Chalmers möchte Skinner aber auch loswerden und kurzerhand findet sich Rektor Skinner mit Bart \& Co. im Schulbus auf dem Weg zu einem Versteck wieder. Auf seiner \glqq Reise\grqq\ lernt Rektor Skinner, was es heißt, Schüler richtig zu unterrichten. Als er und die anderen zurück an der Schule sind, stoppt er den Test, da er ihn für überflüssig hält. Das ist ganz zu Lisas Freude, da sie während des Tests einen völligen Blackout erlitten hatte. 

Unterdessen beauftragt Marge Homer, einen Brief mit der Prämienzahlung an die Versicherung einzuwerfen. Diesen vergisst Homer natürlich und muss feststellen, dass er und seine Familie sowie alle Leute auf seinem Grundstück vorübergehend ohne Versicherungsschutz sind. Nur mit Mühe kann er verhindern, dass ein Unglück geschieht.

\notiz{
\begin{itemize}
	\item Oberschulrat Chalmers hat eine Tochter.
	\item Rektor Skinner streicht in den Sommerferien. Er strich beispielsweise die Garage der Eltern von Dolph.
	\item Die Simpsons haben ihre Unfallversicherung bei der \glqq Blue Umbrella In\-surance\index{Blue Umbrella Insurance}\grqq\ abgeschlossen.
\end{itemize}
}

\subsection{Liebe deinen Nachbarn}
Homer schmeißt eine wilde Mardi-Gras-Party und überschuldet dadurch sein Haus. Daraufhin wird das Haus zwangsversteigert. Flanders ersteigert das Haus aus Mitleid für 101.000 US\$, damit die Simpsons weiter dort wohnen bleiben können. Weil die Simpsons sich äußerst undankbar zeigen, versucht Flanders sie wieder los werden. Schließlich lässt er sie aber wieder im Haus wohnen.

\notiz{
\begin{itemize}
	\item Gil arbeitet als Hypothekenbearbeiter.
	\item Die Hibberts haben ein Au-Pair-Mädchen, das Bernice Hibbert nicht mehr im Haus haben will.
\end{itemize}
}


\subsection{Auf der Jagd nach dem Juwel von Springfield}\label{LABF04}
Homer sorgt aus Versehen dafür, dass Maggie von Nonnen des Klosters Saint Teresa's\index{Saint Teresa's} aufgenommen wird. Um Marge mit ihrer heilenden Augenverletzung zu schonen, will Lisa Maggie heimlich retten, indem sie dem Kloster beitritt. Dabei stößt sie auf einen geheimnisvollen Juwel und begibt sich mit Seymour Skinner und Jeff Albertson auf dessen Suche. An der Suche beteiligen sich außerdem noch Mr. Burns und Mr. Smithers. Es stellt sich schließlich heraus, dass Maggie der Juwel von Saint Teresa's ist. 

\notiz{
\begin{itemize}
	\item Mr. Burns ist Mitglied bei den Freimaurern\index{Freimaurer}. Er gibt an, dass das Auge und die Pyramide auf dem 1-Dollar-Schein im gehören.
	\item Lisa hat das Anagramm \textit{Great Crimes Kill Holy Sage} zu lösen. Lisas Lösung lautet \textit{Regally, The Rock Gem Is Lisa}. Die Lösung lautet aber in Wirklichkeit \textit{It's Really Maggie, Sherlock}.
\end{itemize}
}

\subsection{Im Namen des Großvaters}
Die Simpsons vergessen den Familientag im Altenheim, um den Fauxpas wieder gut zu machen, versprechen sie Grampa, ihm etwas zu ermöglichen, was er vor seinem Tod unbedingt noch erleben möchte. Auf einer langen Liste von Dingen, die er noch vorhat, findet sich der Punkt \glqq noch ein Bier in O'Flanagan's Pub zu trinken\grqq . Homer schlägt vor, Grampa diesen Wunsch zu erfüllen, ohne zu ahnen, dass sich O'Flanagan's Pub in Irland befindet. Also machen sich die Simpsons auf nach Irland. Im Rausch kaufen Abe und Homer den Pub. Da das Geschäft nicht läuft, fliegen sie aus Springfield Moe ein, der ihnen helfen soll. Schließlich machen sie aus dem Pub ein Raucherlokal, bis es von der Polizei geschlossen wird.

\notiz{
\begin{itemize}
	\item Grampas Liste umfasst u.\,a. diese Punkte
	\begin{itemize}
		\item Pitch in the Negro League
		\item See the Chrysler Building
		\item Have one more beer in O'Flanagan's Pub
		\item Get cheated by an Irishman
		\item See NBC go under
	\end{itemize}
	\item In Irland sind u.\,a. diese Technologiefirmen vertreten:
	\begin{itemize}
		\item Hewlett Fitzpackard
		\item Mick-rosoft
		\item Cisc O'Systems
	\end{itemize}
\end{itemize}
}

\subsection{Hochzeit kommt vor dem Fall}\label{LABF05}
Reverend Lovejoy erfährt vom Oberhaupt seiner Kirche, dem Parson\index{Parson}, dass er aufgrund eines Problems mit seiner Kreditkarte vorübergehend keine Zulassung als Priester inne hatte und somit sein Amt unrechtmäßig ausgeübt hat. Deshalb ist jede Amtshandlung, die er während dieser Zeit vollzogen hat, ungültig. Das trifft auch auf die zweite Hochzeit zwischen Homer und Marge zu, die abgehalten werden musste, weil sich Homer zwischendurch \glqq vorsorglich\grqq\ hat scheiden lassen, um einer möglichen Trennung vorzubeugen (siehe \glqq Scheide sich, wer kann\grqq , \ref{4F04}). Doch am Hochzeitstag wird Homer entführt. Bart und Lisa verdächtigen Sideshow Bob, allerdings stellt sich heraus, dass Patty und Selma hinter der Entführung stecken.

\notiz{
\begin{itemize}
	\item Reverend Lovejoy studierte an der Christlichen Universität in Texas. Einer seiner Kommilitonen war der Parson.
	\item Laut neuer Heiratsurkunde wurde Marge in Capitol City geboren.
	\item Jaspers verstorbene Frau hieß Estelle\index{Estelle}.
	\item Reverend Lovejoy hat als PDA einen Psalm Pilot\index{Psalm Pilot}.
\end{itemize}
}

\subsection{Große, kleine Liebe}\label{LABF06}
Moe lernt in einem Internetchat Maya\index{Maya} aus Ogdenville kennen, die nicht nur nett und intelligent, sondern auch sehr attraktiv ist. Auch sie findet an Moe Gefallen und bald kommt es zu einem ersten Treffen. Dabei stellt sich heraus, dass Maya zwergwüchsig ist. Moe verliebt sich dennoch in sie und die beiden verbringen eine wunderbare Zeit, die mit einem Heiratsantrag von Moe endet. Dabei vergreift er sich allerdings im Ton und macht Witze über ihre Größe -- mit fatalen Folgen. 

\notiz{
\begin{itemize}
	\item Moe glaubt, Heinrich Himmler hat \glqq Ein Wintermärchen\grqq\ geschrieben. Tatsächlich stammt dieses Epos von Heinrich Heine.
	\item Moe heißt anscheinend nicht wirklich Moe mit Vornamen. Er hat den Namen nur angenommen, um das Geld für ein neues Schild an seiner Bar zu sparen.
	\item Lenny befolgte einen von Moes Ratschlägen und saß dafür drei Jahre im Gefängnis.
\end{itemize}
}

\subsection{Verliebt und zugedröhnt}
Bart und Milhouse spielen einen Streich an der Schule. Milhouse wird erwischt und wird von der Schule geworfen. Bart verspricht ihm, ihn zu besuchen. Doch Bart lernt ein Mädchen namens Jenny kennen und vergisst Milhouse. Jenny hält Bart für einen netten und ruhigen Jungen. Milhouse zwingt Bart zu gestehen, dass er ein Unruhestifter ist und Jenny verlässt Bart, welcher sich bei Milhouse entschuldigt, woraufhin sie wieder beste Freunde werden. Währenddessen ließt Lisa ein \glqq Worst Case\grqq -Szenario für Springfield in 50 Jahren und ist extrem unglücklich. Sie bekommt Pillen, um zufriedener zu sein und ist so lange weggetreten, bis die Pillen abgesetzt werden. 

\notiz{
\begin{itemize}
	\item Lisa benutzt für ihre Internetrecherche die Suchmaschine Oogle\index{Oogle}.
	\item Willies Freundin ist Inga\index{Inga}, ein schwedisches Bademodenmodel.
\end{itemize}
}

\subsection{Der Papa wird's nicht richten}
Als Homer bemerkt, wie Bart und Lisa schulische und soziale Probleme an der Grundschule haben, behält er als übereifriger Vater seine Kinder unter ständiger Beobachtung, damit sich ihre Leistungen wieder verbessern. Währenddessen entdeckt Marge eine Sauna im Keller des Simpson-Haus, der sie nicht widerstehen kann.  Als sie bemerkt, wie traurig Homer ist, zeigt sie ihm die Sauna und die beiden können sich gemeinsam entspannen. 

\notiz{
\begin{itemize}
	\item Dan Castellaneta erhielt für diese Folge 2009 einen Emmy in der Kategorie Beste Synchronisation.
	\item Otto kauft im Modellbauladen u.\,a. ein Modell der Messerschmitt ME 262\index{ME 262} und des Kriegsschiffes Bismarck.
	\item Beim Modellbauwettbewerb sind folgende Modelle zu sehen:
	\begin{itemize}
		\item CCTV Gebäude Beijing (Milhouse)
		\item Brasilianischer Nationalkongress (Martin)
		\item Tempel von Angkor Wat (Uter)
		\item Taj Mahal (Noah)
		\item Westminster Abbey (Bart)
	\end{itemize}
	\item Filmzitat: Als Homer sagt \glqq Black Hawk down\grqq\ ist das eine Anspielung auf den gleichnamigen Film. Der Film basiert auf Ereignissen in der Mogadischu-Krise, in welcher zwei US-Hubschrauber abgeschossen wurden. 
\end{itemize}
}

\subsection{Auf nach Waverly Hills}
Damit Bart und Lisa auf eine bessere Schule in Waverly Hills gehen können, mieten die Simpsons dort eine Wohnung an, da sie nur so an eine Schulgenehmigung kommen. Weil ein Inspektor kontrollieren möchte, ob Bart und Lisa dort wirklich leben, muss Homer nun für die nächste Zeit die Stellung halten. Nach einiger Zeit gefällt es ihm sogar, vor allem wenn Marge ihn besuchen kommt und gewöhnt sich an den Lebensstil von Junggesellen. Lisa steigt währenddessen an ihrer neuen Schule zum beliebtesten Mädchen auf, da Bart das Gerücht in die Welt gesetzt hat, sie wäre Alaska Nebraskas\index{Nebraska!Alaska} beste Freundin. Allerdings wollen ihre Mitschülerinnen nun Backstage-Pässe haben, die sie nicht besorgen kann. Deshalb möchte Lisa zurück an ihre alte Schule und auch Bart stimmt ihr zu. Unterdessen bekommt Bart Probleme mit der Springfielder Polizei, da er zu einem vereinbarten Bowlingabend mit Ralph nicht erschienen ist.

\notiz{
\begin{itemize}
	\item Der Inspektor ist eine Anspielung Anton Chigurh aus dem Film \glqq No Country For Old Man\grqq\ aus dem Jahre 2007.
	\item Homer wohnt in Waverly Hills im \glqq Royal Waverly\grqq\ in Zimmer 12.
\end{itemize}
}

\subsection{Vier Powerfrauen und eine Maniküre}
Marge und Lisa besuchen einen Nagelsalon, während sie eine Debatte führen, ob eine Frau gleichzeitig intelligent, stark und schön sein kann. Dabei erzählen sie sich vier Geschichten über berühmte Frauen. 
\begin{itemize}
	\item Homer ist ein Prinz im Mittelalter und Marge eine Zofe. Sie verlieben sich, doch das passt der Königin von England (Selma) überhaupt nicht. Sie will die beiden sogar in den Tower werfen lassen, doch ihre Meinung ändert sich, als Homer ihr Königreich vor den Spaniern rettet. 
	\item Die böse Königin will das schöne Schneewittchen (Lisa) vom Jäger (Willie) töten lassen, doch dieser beweist ein gutes Herz und lässt sie laufen. Sie flüchtet zu den sieben Zwergen Stinki (Moe), Saufi (Barney), Fressi (Homer), Geizi (Mr. Burns), Lenny (Lenny), Kearney (Kearney) und Doc Torhibbert (Dr. Hibbert). Als diese zur Arbeit gehen, kommt die böse Königin und vergiftet Schneewittchen mit einem Apfel. Doch durch eine gute Ärztin wird sie geheilt. 
	\item Marge möchte, dass Homer die Hauptrolle im Stück \glqq MacBeth\grqq\ bekommt. Darum befiehlt sie ihm, die anderen Darsteller zu töten. Doch ihre Geister suchen Marge auf, da sie die Drahtzieherin war und bringen sie auch um. Am Ende bleibt also nur noch ihr Geist im Publikum. 
	\item Marge bringt Maggie in eine neue Krabbelgruppe, in der Babys lernen, ihrem Talent freien Lauf zu lassen. Doch der Kindergärtner Mr. Toohey\index{Toohey} zerstört immer wieder Maggies Werke, da sie zu gut sind. Am Ende stehen die beiden vor einem Plüschtier- und Babygericht. 
\end{itemize}


\notiz{
\begin{itemize}
	\item Es ist die erste Episode, in der Bart nicht zu sehen ist. 
	\item Das Gemälde, das Maggie an die Wand malt, ist Vincent van Goghs \glqq Sternenacht\grqq .
	\item In der zweiten Episode ist Spiderschwein\index{Spiderschwein} zu sehen.
\end{itemize}
}


\subsection{Es war einmal in Homerika}
Durch einen vergifteten Burger der Marke Krusty verlieren tausende norwegische Auswanderer, die in Ogdenville\index{Ogdenville} leben, ihren Job als Landwirte. Deshalb müssen sie ihr zu Hause aufgeben und ziehen nach Springfield. Am Anfang gefällt es den Springfieldern von vorne bis hinten bedient zu werden, doch nach einiger Zeit ist ihnen die Überbevölkerung lästig. Da die Polizei mit der Überwachung der Stadtgrenzen überfordert ist, wird eine Bürgerwehr gegründet. Allerdings versagt diese bei der Grenzüberwachung, aus diesem Grund bauen sie einen Zaun, um die Norweger fernzuhalten. Beim Bau der Grenzanlage werden sie von den norwegischen Auswanderern unterstützt. Doch später merken sie, dass sie viele Gemeinsamkeiten mit ihren Ex-Gästen haben und lassen sie kurzum wieder rein. 

\notiz{
\begin{itemize}
	\item Die Episode wurde am 17. Mai ausgestrahlt, am norwegischen Verfassungstag.
	\item Ogdenville liegt südlich von Springfield und wurde vor 100 Jahren von norwegischen Einwanderern gegründet.
	\item Homer hat Angst vor Xylophonen.
\end{itemize}
}

\section{Staffel 21}

\subsection{Everyman begins}\label{LABF13}
Der Comicbuchverkäufer erschafft einen neuen Superhelden namens Everyman, der die Kraft von anderen Superhelden aufnehmen kann, indem er deren Comics berührt. Als Bart und Milhouse ein Comic zufällig lesen, ermutigen sie ihn, seine Comics zu veröffentlichen. Die Comicreihe entwickelt sich zu einem Verkaufsschlager und die Studiobosse in Hollywood zeigen Interesse an einer Verfilmung.  Homer bekommt zufällig die Hauptrolle in dem Film, der über diesen Superhelden gedreht wird. Damit Homer für seine Rolle eine schlanke Figur bekommt, heuert das Filmstudio für ihn einen berühmten Fitness-Trainer an. Homer schafft es auch tatsächlich, eine tolle Figur zu bekommen. Doch dann wird der Fitness-Trainer zu einem anderen Kunden gerufen und Homer beginnt wieder zu essen -- leider noch während der Dreharbeiten, sodass das Filmresultat ein Desaster wird.

\notiz{
\begin{itemize}
	\item Es handelt sich hier um die zweite Folge, die von einem Gaststar geschrieben wurde (Seth Rogen).
	\item Das Wandgemälde, das Nelson an eine Schulwand sprüht, ist eine Anspielung auf Picassos \glqq Guernica\grqq .
\end{itemize}
}

\subsection{Die Antwort}\label{LABF15}
Mrs. Krabappel beschlagnahmt alle Handys der Klasse. Die Schüler beschließen, der Lehrerin eine Lektion zu erteilen. Sie mischen ihr Alkohol in den Kaffee. Rektor Skinner entlässt sie deswegen. Die Schüler werden nun von dem neuen coolen Lehrer Zachary Vaughn betreut. Der neue Lehrer wird wegen seiner Art und seiner Aufgeschlossenheit für neue Medien von den Schülern bewundert. Nebst seiner Sympathie für den neuen Lehrer wird Bart von Schuldgefühlen beklagt und gesteht am Ende Rektor Skinner sein Fehlverhalten. In der Zwischenzeit eröffnet Edna den Muffin-Laden \glqq Edna's Edibles\grqq\ in Springfield.

\notiz{
\begin{itemize}
	\item Der Energy Drink \glqq Blue Bronco\grqq\ ist eine Mischung aus mehreren Energy Drinks (Red Bull, Power Horse und Burn).
	\item Absolut Krusty ist eine Anspielung auf Absolut Vodka.
	\item Edny Krabappel wohnt in 82 Evergreen Terrace, Apt. 13.
	\item Ned Flanders \glqq Viza\grqq -Cards Nummer ist die 4110 0122 3344 5590 und seine Karte lief am 1. Mai 2005 ab.
	\item Neben Edna's Muffin-Laden gibt es noch \glqq 'Muff Said\grqq , \glqq R. Crumb's Keep on Muffin'\grqq , \glqq H. R. Muffin Stuff\grqq\ und \glqq MUFF*MART\grqq . In Kürze wird noch der \glqq Muffin Store\grqq\ eröffnen.
\end{itemize}
}

\subsection{Marge macht mobil}\label{LABF16}
In Springfield findet die Sportart UPKCC (Ultimate Punch, Kick \& Choke Championship) immer mehr Anhänger. Es handelt sich hierbei um \glqq Mixed Martail Arts\grqq . Bei dieser Sportart fügen sich die Akteure teils starke Schermzen zu. Marge und ihren Freundinnen (Luann Van Houten, Bernice Hibbert, Sarah Wiggum und Edna Krabappel) gefällt das überhaupt nicht und protestieren deshalb dagegen. Der Veranstalter der Show fordert Marge zu einem Kampf heraus und verspricht ihr, wenn er verliert, würde er die Kämpfe einstellen.

\notiz{
\begin{itemize}
	\item Krusty hatte über elf Jahre lang eine Affäre mit Sideshow Mels Frau.
	\item Mr. Burns war in Yale im Ringerteam.
	\item Fehler: Barney sitzt zu Beginn des Kampfes im Stadion und ist plötzlich ein paar Minuten später betrunken in Moe's Taverne zu sehen. 
\end{itemize}
}

\subsection{Der Teufel trägt Nada}\label{LABF17}
Marge und ihre Freundinnen, die Wohltätigkeitsbräute, wollen für einen wohl\-tät\-ig\-en Zweck Geld sammeln. Daher beschließen sie, einen freizügigen Kalender zu verkaufen. Der gesamte Kalender besteht nur aus Marges Posen. Homer wird unterdessen zu Carls neuem Assistenten befördert, nachdem Carl zum Supervisor im Sektor 7G aufgestiegen ist. Daraufhin hat Homer nur noch wenig Zeit für Marge und ihre Bedürfnisse. Als Homer mit Carl nach Paris zum einem Kongress fliegen muss, kommen sich Marge und Ned näher. 

\notiz{
\begin{itemize}
	\item Lennys Mutter besaß in Paris ein Restaurant.
	\item Carls Vater starb beim Softball.
	\item Auf Carls Schreibtischschild ist C. Carl Carlson als Name zu lesen.
	\item Carl gibt an, jünger als Homer zu sein.
\end{itemize}
}

\subsection{L wie Looser}
Da Rektor Skinner von Barts neuestem Streichen wenig beeindruckt ist, erzählt er ihm von einem Jungen, der einst noch schlimmere Streiche gespielt hat. Von Hausmeister Willie erfährt Bart den Namen des Jungen: Andy Hamiliton\index{Hamilton!Andy}. Daraufhin macht er sich mit Milhouse auf die Suche nach ihm. Nachdem Bart anfänglich von Andys begeistert ist, macht ihn Lisa darauf aufmerksam dass Andy ein totaler Verlierer ist und in seinem Leben nichts als Streiche auf die Reihe gebracht hat. Um das zu ändern, hilft Bart seinem neuen Freund bei Krusty einen Job zu bekommen. 

\notiz{
\begin{itemize}
	\item Willie hat in der Schule als Schwimmmeister gearbeitet.
	 \item Bart hat für Krusty vor Gericht einen Meineid geleistet.
\end{itemize}
}

\subsection{Die Hexen von Springfield}
Bei einem Unfall auf der Rückfahrt von einem Ski-Urlaub werden die Simpsons von den Spucklers gerettet und in das Hinterwäldler-Leben der Provinz eingeführt. Während Homer zum Chef-Verköstiger des selbstgebrannten Schnaps wird, trifft Lisa auf drei Anhängerinnen des Wicca-Kults, die jedoch verhaftet werden, als die halbe Stadt vorübergehend erblindet. Doch Lisa kann beweisen, dass nicht ein Hexenfluch, sondern zu viel Schnaps im Trinkwasser die Sehstörung verursacht hat.

\notiz{
\begin{itemize}
	\item Am 11. September hat Kent Brockman in einem Hinterzimmer eines Fischrestaurants mit Selma geknutscht.
	\item Auf der Website Wiccapedia sind keine Einträge zu Verabredungen zu finden, dafür über zwei Milliarden Eintrage zu \glqq Buff -- Im Bann der Dämonen\grqq .
\end{itemize}
}

\subsection{Lebe lieber unbebrudert}
Bart muss beim Spielen erkennen, dass Lisa und Maggie etwas verbindet, was er als Junge nie haben wird: Schwesternliebe. Er wird von dem Gedanken besessen, ein Brüderchen zu bekommen und versucht mit allen möglichen Tricks, seine Eltern zum Familienzuwachs zu bringen. Als er ein Waisenhaus besucht, beobachtet ihn der Waisenjunge Charlie. Der Kleine büchst aus und taucht bei Bart auf, um von nun an sein Bruder zu sein. Das geht nicht lange gut. Schließlich wird Charlie von einer Familie mit sechs Töchtern adoptiert.

\notiz{
\begin{itemize}
  \item Bart besitzt die Best Of Itchy \& Scratchy DVD Vol. 63.
	\item Barney ist wieder als Schneepflugköning (\glqq Plow King\grqq) zu sehen, wie in den Episoden \glqq Einmal als Schneekönig\grqq\ (siehe \ref{9F07}) und \glqq Rektor Skinners Gespür für Schnee\grqq\ (siehe \ref{CABF06}).
\end{itemize}
}


\subsection{Mörder, Zombies und Musik}
\begin{itemize}
	\item \textbf{Dial \grqq M\grqq\ for Murder or Press \grqq \#\grqq\ to Return to Main Menu}\\ Lisa wird nicht als Vertreterin ihrer Klasse zum nationalen Lesewettbewerb ausgewählt. Aufgrund ihres Verhaltens muss sie nachsitzen. Im Nachsitzzimmer schlägt Bart ihr vor, sich gegenseitig an ihren Lehrerinnen zu rächen.
	\item \textbf{Don't Have a Cow, Mankind}\\  Bei Krusty Burger wird ein neuer Hamburger angeboten. Der Burger besteht Rindfleisch von Rindern, die Rindfleisch gefressen haben. Krusty bezeichnet dies als $Burger^2$. Durch den Verzehr des Burgers verwandelt sich fast die ganze Bevölkerung von Springfield innerhalb von vier Wochen in Zombies und nur Bart ist dagegen immun und somit der Erlöser.
	\item \textbf{There's No Business Like Moe Business}\\ Homer fällt versehentlich beim Versuch sich am unbewachten Zapfhahn ein Bier zu zapfen durch eine Falltür in Moes Bar und wird auf die Rohre von Moes Braumaschine aufgespießt. Moe schlägt daraus Kapital und serviert Bier mit Homers Blut als Geheimzutat, um sich an Marge ranzumachen. 
\end{itemize}

\notiz{
Moe bezeichnet in der dritten Geschichte Marge in dem gefälschten Brief als Mitch und Lisa als Linda.
}

\subsection{Donnerstags bei Abe}
Der Journalist Marshall Goldman\index{Goldman!Marshall} wird auf die Geschichten von Grampa aufmerksam und veröffentlicht sie regelmäßig in der Zeitung. Die beiden verbringen sehr viel Zeit miteinander, wodurch Homer schnell eifersüchtig wird. Also versucht Homer, selbst Kolumnen über Mr. Burns zu schreiben, doch diese werden vom Springfield Shopper nicht angenommen. Im Hauptquartier der Zeitung entdeckt er allerdings Goldmans Büro und er sieht nach, ob er etwas Interessantes über den Journalisten finden kann. Und tatsächlich entdeckt er, dass Goldman Grampa ermorden will, damit er mit seiner Biographie den Pulitzer-Preis gewinnen kann. Mit Carl und Lennys Hilfe kann er schließlich Grampa retten. Währenddessen muss Bart auf das Klassenstofftier Larry\index{Larry} aufpassen, was jedoch schnell verloren geht.

\notiz{
\begin{itemize}
	\item Marge hat ein Auto mit dem Nummernschild MABF02 fotografiert. Der Produktionscode dieser Episode lautet ebenfalls MABF02.
	\item Marshall Goldmann wurde am 20. Juni 1969 in Ogdenville geboren. Er wohnt zur Zeit in Shelbyville.
	\item Fehler: In der Szene, in der Homer von Lenny und Carl hochgehalten wird, hat Lenny einige Sekunden lang keinen Bart mehr.
\end{itemize}
}

\subsection{Es war einmal in Springfield}\label{LABF20}
Krustys Quoten sind zu schlecht, da er beim weiblichen Publikum nicht ankommt. Deswegen stellt ihm der Sender Prinzessin Penelope\index{Penelope}\index{Prinzessin!Penelope} als weiblichen Co-Star zur Seite. Penelope ist aber bereits seit sie 12 Jahre alt war in Krusty verliebt. Unterdessen streicht Mr. Burns die kostenlosen Donuts für die Belegschaft. Die Verärgerung dieser Maßnahme erfährt Gator McCall\index{McCall!Gator}, ein Head Hunter für Nukleartechniker, der nun versucht Lenny, Carl und Homer für das Capital City Atomkraftwerk abzuwerben.

\notiz{
\begin{itemize}
	\item Homer, Carl und Lenny gehören zur Tauziehmannschaft des Springfielder Atomkraftwerks.
	\item Krusty besitzt Country-Alben und ist süchtig nach Bowlingbahnspray.
	\item Der vollständige Name von Prinzessin Penelope lautet Penelope Mountbatten Habsburg Hohenzollern Mulan Pocahontas.
	\item In der Jury zu \emph{America's Next Krusty} sitzen Kent Brockman, Rainier Wolfcastle und Booberella.
\end{itemize}
}

\subsection{Million Dollar Homie}
Homer und Marge sollten eigentlich auf der Hochzeit einer Cousine singen, doch Homer versetzt Marge. Er schafft es nicht rechtzeitig zum Auftritt, da er im Kwik-E-Mart ein Lotterielos kauft. Als er mit seinem Los den Jackpot in Höhe einer Million Dollar gewinnt, hat er Angst, es Marge zu erzählen. Barney rät ihm, seiner Familie anonym Geschenke zu machen. Bart kommt ihm auf die Schliche und er erpresst ihn.

\notiz{
\begin{itemize}
	\item In dieser Episode tritt zum ersten Mal ein Charakter auf, der von einem Fan erfunden wurde. In einem Wettbewerb sollte man als Fan einen neuen Charakter erfinden. Der beste Charakter wurde als neue Figur in die Serie aufgenommen. Der Charakter nennt sich Ricardo Bomba\index{Bomba!Ricardo} und wird kurz \glqq La Bomba\grqq\ genannt \cite{RicardoBomba}. 
	\item Die Spielekonsole Funtendo Zii\index{Funtendo Zii} ist eine Anspielung auf die Nintendo Wii.
	\item Homers Gewinnzahlen lauten: 1, 6, 17, 22, 24 und 35.
	\item Nach Abzug der Steuern bleiben von der einen Million noch 685443,22 US\$ übrig.
\end{itemize}
}

\subsection{Curling Queen}\label{MABF05}
Homer und Marge beginnen sich für Curling zu interessieren. Die beiden machen das so gut, dass sie Agnes Skinner bittet, ihrem Team beizutreten. Das Team schafft es schließlich, sich für den olympischen Demonstrationswettbewerb im gemischten Curling-Doppel in Vancouver zu qualifizieren. Sensationell gewinnen sie dort sogar gegen Schweden das Finale und holen so die einzige olympische Goldmedaille für die USA, obwohl Marge am rechten Arm verletzt ist und Agnes Homer am liebsten aus dem Team werfen würde. 
Unterdessen beginnt Lisa Anstecker der Olympischen Spiele zu sammeln und wird süchtig danach. Schließlich hilft ihr Bart, von der Sucht wieder loszukommen.

\notiz{
\begin{itemize}
	\item Agnes Skinner hat 1952 an der Sommerolympiade in Helsinki im Stabhochsprung teilgenommen. Sie war damals mit Seymour schwanger.
	\item Marge sagt, dass sie eigentlich Linkshänderin ist.
	\item Homers Führerschein ist zu sehen. Der Führerschein ist gültig bis zum 26.12.2012 und seine Führerscheinnummer lautet S5789034. Er wiegt laut Führerschein 270 Pounds (ca 122 kg) und ist 6 Feet (ca. 1,83 m) groß\footnote{Auf dem Führerschein sind die Einheiten zur Größe und zum Gewicht nicht angegeben. Da es sich um einen amerikanischen Führerschein handelt, kann von der Einheit Pound beim Gewicht und Foot bei der Größe ausgegangen werden.}.
	\item Die Schamanin aus dem Simpsons-Film ist zu sehen, als die Simpsons in Vancouver eintreffen.
\end{itemize}
}

\subsection{Die Farbe Gelb}
Lisa muss als Hausaufgabe die Familiengeschichte erforschen. Dabei stellt sie erschrocken fest, dass die meisten der Vorfahren Verbrecher waren. Auf dem Dachboden findet sie das Tagebuch ihrer Vorfahrin Eliza Simpson. Sie liest darin, dass Eliza einem Sklaven zur Flucht verhelfen will. Homer fordert sie auf, nicht mehr weiterzulesen, um nicht enttäuscht zu werden. Lisa liest dennoch weiter und erfährt, dass der Sklave Virgil an seinen Besitzer, Colonel Burns, zurückgegeben werde sollte. Da Lisa sehr deprimiert ist, erzählt ihr Abe die Wahrheit. Der Sklave gelang mit Mabel, der Mutter von Eliza, nach Kanada in die Freiheit. Und in Wirklichkeit stammen die Simpsons von Virgil and Mabel ab.

\notiz{
\begin{itemize}
	\item Grampa behauptet, Vorfahren der Simpsons wurden aus Australien rausgeworfen. 
	\item Ein weiterer Vorfahr der Simpsons ist der Pirat Bart-Beard\index{Bart-Beard}.
	\item Mr. Burns behauptet, sein Vater war Colonel Burns. Diese Aussage ist widersprüchlich zur Episode \glqq Kampf um Bobo\grqq (siehe \ref{1F01}).
	\item Fehler: Die kanadische Flagge, die zu sehen ist, als Virgil und Mabel 1867 Kanada erreichen, wurde erst am 15. Februar 1965 zum ersten Mal gehisst.
\end{itemize}
}

\subsection{Grand Theft U-Bahn }
Da Bart seinen Hausaufgaben nicht abgibt, schreibt Mrs. Krabappel einen Brief an seine Eltern. Bart versucht zwar den Brief vor seinen Eltern abzufangen, aber Homer bekommt ihn doch zu lesen. Während Homer verärgert reagiert und mit Strafen droht, bevorzugt Marge eine verständnisvollere Herangehensweise. Bart bemerkt schnell, dass er sie gegeneinander ausspielen kann. Doch dann wird Homer klar, dass er Marge braucht und diese will nicht so enden wie ihre Schwestern, also vertragen sich die beiden wieder. Sie beschließen sich nicht mehr von Bart ärgern zu lassen und dieser ist nun selbstständig. Bart freut sich über seine neue Freiheit und spielt Rektor Skinner einen Streich: Er und Milhouse spritzen Skinner mit Zuckerwasser ab. Bart und Milhouse müssen flüchten und Milhouse zeigt seinem Freund den Weg zu der stillgelegten U-Bahn. Die beiden Jungen fahren mit der Bahn und lösen dabei ein leichtes Beben in der Stadt aus. Doch als Homer und Marge sich nicht über Bart ärgern, ist er enttäuscht und stellt fest, dass er nicht böse ist, wenn sich niemand über seine Streiche beschwert. So beschließt er, mit einer weiteren Fahrt mit der U-Bahn, die Grundschule zum Einsturz zu bringen.

\notiz{
\begin{itemize}
	\item Die Itchy \& Scratchy Episode ist eine Parodie auf Dr. House.
	\item Die Grundschule von Springfield befindet sich in der Plympten Street 19.
\end{itemize}
}

\subsection{Der gestohlene Kuss}
Die Abwesenheit von Mrs. Krabappel zwingt Rektor Skinner dazu, die beiden vierten Klassen bis zu ihrer Rückkehr zusammenzulegen. Bart setzt sich neben Nikki\index{Nikki} und verliebt sich Hals über Kopf in sie. Auf Grampas Ratschlag hin küsst er sie, was allerdings auf wenig Begeisterung stößt. Nikkis Eltern -- beide Anwälte -- beschweren sich bei Rektor Skinner und drohen mit Klage. Unterdessen beklagt sich Lisa in einem Chatroom über Blumen über ihre Anfeindungen wegen ihrer Klugheit. Michelle Obama\index{Obama!Michelle} liest ebenfalls in diesem Chatroom mit und fliegt nach Springfield, um die Schüler davon zu überzeugen, dass gute Noten nichts Schlimmes sind. 


\notiz{
\begin{itemize}
	\item Ein Alaska Nebraska Poster ist in der Schule zu sehen.
	\item Maggie ist in dieser Folge nicht zu sehen.
	\item Das Buch, dass Nikki liest, trägt den Titel \glqq Red Moon\grqq . Dies ist eine Anspielung auf den zweiten Teil der Twilight-Saga \glqq New Moon\grqq .
\end{itemize}
}

\subsection{Simpson und Gomorrha}
Ermutigt durch Reverend Timothy Lovejoys Worte will Ned Flanders aus Homer einen guten Menschen machen. Deshalb überredet er die Simpsons, mit der Springfielder Kirchengruppe an der Pilgerreise nach Israel teilzunehmen. Zuerst beachtet Homer die religiösen Sehenswürdig\-kei\-ten nicht, doch dann bekommt er das Jerusalem-Syndrom\footnote{Das Jerusalem-Syndrom bezeichnet eine psychische Störung, von der ca. 100 Besucher und Einwohner der Stadt Jerusalem pro Jahr betroffen sind. Dabei handelt es sich nicht um eine anerkannte Diagnose. Die Symptome fallen im internationalen Diagnoseschlüssel unter \glqq Akute und vorübergehende psychotische Störung\grqq .

Die Erkrankung besitzt den Charakter einer Psychose und äußert sich unter anderem in Wahnvorstellungen: Der oder die Betroffene identifiziert sich vollständig mit einer heiligen Person aus dem Alten oder Neuen Testament und gibt sich als diese aus.

Sehr prominente und wichtige biblische Personen werden dabei besonders häufig zum Objekt einer solchen Identifizierung, so zum Beispiel Mose und König David aus dem Alten Testament oder Jesus und Johannes der Täufer aus dem Neuen Testament. Grundsätzlich wählen Männer männliche Personen aus der Bibel und Frauen weibliche Personen. Auch gibt es konfessionelle Einflüsse bei der Wahl: Juden wählen Personen aus dem Alten Testament, Christen wählen Personen aus dem Neuen Testament \cite{JerusalemSyndrom}.} und glaubt, er sei der Erlöser. 


\notiz{
\begin{itemize}
	\item In der Szene, in der Homer alleine auf einem Kamel durch die Wüste reitet, singt er zur Melodie von \glqq Laurence von Arabien\grqq , in welcher der Hauptdarsteller auch fast in der Wüste verdurstet.
	\item Das heilige Gemüse in der Wüste ist eine Anspielung auf diese Zeichentrick-Serie \glqq Veggie Tales\grqq .
	\item Zu Neds Bibelkreis gehören Jimbo, Agnes Skinner und die Ehepaare Hibbert und Lovejoy.
\end{itemize}
}

\subsection{Jailhouse Blues}\label{MABF08}
Mr. Burns wird aufgrund des Besitzes gestohlener Gemälde verhaftet und muss ins Gefängnis. Während seiner Abwesenheit übernimmt Waylon Smithers die Geschäftsführung. Zunächst ist er ein guter Boss, doch nachdem er bemerkt, wie sich Homer, Carl und Lenny über seine lasche Führung lustig machen, entwickelt er sich mehr und mehr zu einem strengen Boss, sodass sich Homer und seine Freunde Mr. Burns wieder zurück wünschen. Dieser wird indes von seinem Zellengenossen zum Glauben an Jesus Christus aufgerufen. Doch gleich nach seiner Ankunft im Kraftwerk ist er wieder der alte, harte Geschäftsmann. Währenddessen kümmern sich Bart und Lisa um eine Ameise. 

\notiz{
\begin{itemize}
	\item Mr. Burns war Mitglied bei der Schutzstaffel (SS) der NSDAP, die ab 1945 aufgelöst und verboten worden ist.
	\item Mr. Burns hat in der Beatles-Coverband des Gefängnisses das Schlagzeug gespielt.
	\item Mr. Burns Gefangenennummer lautet: 14212019.
	\item Maggie ist, ausgenommen im Vorspann, in dieser Episode nicht zu sehen.
\end{itemize}
}


\subsection{Chief der Herzen}\label{MABF09}
Homer betritt eine Bank mit einem kandierten Apfel in der Jackentasche. Daraufhin glaubt das Sicherheitspersonal, er wolle die Bank überfallen. Deshalb wird Homer zu 100 Sozialstunden verurteilt. Chief Wiggum leitet das Programm. Als sich die beiden Homers Pausenbrot teilen, werden sie plötzlich dickste Freunde und sind von da an nur noch zu zweit unterwegs. Als Wiggum Homer bei einem Polizeieinsatz das Leben rettet und dabei selbst schwer verletzt wird, verlangt er von nun an, dass sich Homer rund um die Uhr um ihn kümmert. Bart findet derweil Gefallen an einem japanischen Spiel namens Battle Ball.

\notiz{
\begin{itemize}
	\item Lisa kommt nur in der Eröffnungsszene vor und hat in dieser Episode keinen Text.
	\item Laut Polizeiakte ist Homer 6 Ft groß und 239 lbs schwer.
	\item Neben Homer müssen außerdem Kearney, Kent Brockman, Herman und Krusty Sozialstunden ableisten.
\end{itemize}
}

\subsection{Walverwandtschaft}
Homer ist verärgert über die hohe Stromrechnung und beschließt auf Lisas Rat hin, ein Windrad im Garten zu installieren. Jedoch gibt es auch mit dem Windrad unerwartete Probleme, da es nur Strom liefert, wenn auch der Wind weht. Ein starker Sturm zerstört nicht nur das Windrad sondern spült auch ein Walweibchen am Strand an. Lisa versucht verzweifelt, es zu retten.

\notiz{
\begin{itemize}
	\item In manchen Regionen war der Tafelgag \glqq South Park, we'd stand beside you if we weren't so scared\grqq\ zu lesen; eine Anspielung auf den Mohammed Skandal von South Park in der 201. South-Park-Folge. Diese Folge wurde nur vier Tage vor der Simpsons-Folge ausgestrahlt.
	\item In Moes Taverne ist der Liebestestautomat aus der Episode \glqq Ihre Lieblings-Fernsehfamilie\grqq\ (siehe \ref{4F20}) zu sehen.
\end{itemize}
}

\subsection{Nedtropolis}
Homer vergisst am Bahnhof eine Reisetasche, worauf die Polizei alarmiert wird, welche die Tasche vorsichtshalber sprengt. Da sich in der Tasche von Mr. Burns entsorgtes Plutonium befand, wird angenommen, dass es sich hierbei um einen Terroranschlag handelte. Wegen der dadurch verursachten Terror-Angst, werden in ganz Springfield Kameras installiert. Ein kleiner Teil im Garten der Simpsons wird allerdings nicht erfasst, weshalb die illegalen Aktivitäten der Springfielder fortan dort stattfinden. Unterdessen muss Lisa gegen das Vorurteil kämpfen, dass blonde Mädchen nicht so klug wie dunkelhaarige Mädchen sind.

\notiz{
\begin{itemize}
	\item Diese Folge ist die eine von sehr wenigen regulären Episoden, bei denen das übliche Intro fehlt. Das gab es bisher nur in der ersten Staffel sowie bei Halloween- und anderen Spezialfolgen. Der Couchgag ist in dieser Folge eine Art Videoclip zu Ke\$has Song \glqq TiK ToK\grqq .
	\item Am Ende der Episode sagt Prinz Charles: \glqq Der Atem meiner Katze riecht nach Katzenfutter.\grqq\ Das Gleiche sagt auch Ralph am Ende der Episode \glqq Lisas Rivalin\grqq\ (siehe \ref{1F17}).
	\item Die Firma \glqq Orwell Security\grqq\ (gegründet 1984) ist eine Anspielung auf den Roman \glqq 1984\grqq\ von George Orwell.
	\item Duffman hat eine Tochter, die das College abgebrochen hat.
\end{itemize}
}


\subsection{Der weinende Dritte}\label{MABF13}
Der Muttertag steht bevor und Marge wünscht sich, diesen alleine verbringen zu können. Sie schickt Homer und die Kinder auf die Insel \glqq Weasel Island\grqq . Außer Homer reisen u.\,a. noch Reverend Lovejoy und Apu auf die Insel.  Auf dem Weg dorthin erhalten sie einen Brief von Moe, in dem er ankündigt, mit einer Ihrer Frauen die Stadt zu verlassen. Daraufhin erinnern sich die drei an die Momente zurück, in denen Moe auf ihre Frauen getroffen ist. Als sie wieder nach Hause fahren, befürchten sie das Schlimmste. Doch dann kommt alles ganz anders. 

\notiz{
\begin{itemize}
	\item In dieser Folge sagt Homer, dass er 39 ist.
	\item Moe sagt im Intro, er sei nach Springfield gekommen \glqq because on a calculator, the ZIP code spells BOOBS\grqq . Demnach ist Springfields Postleitzahl offenbar 80085 und die liegt in dieser Folge mal in Colorado, denn dort beginnen die Postleitzahlen mit 80.
	\item Marges Mutter wird 80 Jahre alt.
	\item Im Abspann ist Prof. Frink zu sehen, wie er den Computer \glqq MOM\grqq\ (Multiplatform Optical Mainframe) umarmt.
\end{itemize}
}

\subsection{Bob von nebenan}
Bart glaubt, dass der neue Nachbar Sideshow Bob in Verkleidung ist. Um ihn vom Gegenteil zu überzeugen, besucht Marge mit ihm das Hochsicherheitsgefängnis von Springfield und Sideshow Bob scheint, sicher in seiner Zelle zu sitzen. Doch in Wirklichkeit ist er geflüchtet und verfolgt sein Ziel, Bart zu töten. Aufgrund der finanziell angespannten Lage musste die Stadt die Kleinkriminellen aus dem Gefängnis entlassen. Bob nutzt dies aus, indem er mit seinem Zellengenossen Walt Warren\index{Warren!Walt} das Gesicht tauschte.

\begin{itemize}
	\item Auf der Zeichnung der Dienstwaffe von Chief Wiggum steht, dass er 43 Jahre alt ist.
	\item Bob versucht Bart am \glqq Fünfländereck\grqq\ umzubringen. In Amerika gibt es keinen Ort, an dem fünf Staaten aneinander grenzen.
\end{itemize}

\subsection{Richte deinen Nächsten}
Als Moe den Wettbewerb des hässlichsten Haustiers in Springfield besucht, will niemand ihn neben sich sitzen haben. Erst als er Juror des Wettbewerbs wird, bewundern ihn alle wegen seiner boshaften Kommentare. Seine Fähigkeit bemerkt auch ein Mitarbeiter von FOX und lädt ihn zu den Studios nach Los Angeles ein, wo er die Jury von American Idol verstärken soll. In der Jury von findet er nun Platz neben Simon Cowell\index{Cowell!Simon}, der jedoch Angst um seinen Ruf hat. Er gibt Moe absichtlich falsche Tipps und lässt ihn vor einem Millionenpublikum auflaufen. Da Homer wegen Moes neuem Job nicht mehr in die Kneipe gehen kann, sucht Marge nach einer Lösung, ihren Mann aus dem Haus zu bringen.

\notiz{
\begin{itemize}
	\item Der Tafelgag spielt darauf an, dass die mit Spannung erwartete letzte Folge der Serie \glqq LOST\grqq\ am selben Abend wie diese Episode ausgestrahlt wurde. 
	\item Obwohl der reiche Texaner an Pogonophobie\index{Pogonophobie} leidet (siehe \glqq Abgeschleppt!\grqq , \ref{JABF21}), ist er beim Bärtewettbewerb zu sehen.
\end{itemize}
}

\section{Staffel 22}

\subsection{Grundschul-Musical}\label{MABF21}
Krusty fliegt in dem Glauben, ihm werde der Friedensnobelpreis verliehen, nach Europa. In Wirklichkeit wird er allerdings nach Den Haag vor den internationalen Gerichtshof gebracht und wird wegen diverser Verbrechen, die er in Europa begangen hat, angeklagt. Er wird aber freigesprochen, da er 1990 einen Auftritt in Südafrika absagte und Nelson Mandela drei Tage später aus dem Gefängnis entlassen wurde. Unterdessen überrascht Marge Lisa damit, dass sie eine Woche lang in ein Lager für darstellende Kunst gehen darf. In diesem Lager wollen die Leiter und einige musikalisch veranlagte Jugendliche ihre kreative Seite wecken.

\notiz{
\begin{itemize}
	\item Krusty gibt an, verheiratet zu sein.
	\item Largo sagt, dass er 53 Jahre alt ist.
	\item Auf Stephen Hawkings Halskette steht $E = mc^2$.
	\item Anklagepunkte gegen Krusty
	\begin{itemize}
		\item Affe vom Eiffelturm geworfen.
		\item Jemanden in Griechenland mit \glqq Hey-Hey\grqq\ beleidigt.
		\item Holländischem Clown die Witze gestohlen.
	\end{itemize}
\end{itemize}
}

\subsection{Auf diese Lisa können sie bauen}\label{MABF17}
Grandpa Simpson gibt jedem Familienmitglied 50 Dollar. Lisa investiert ihren Anteil in das neue Fahrradgeschäft von Nelson. Nach kurzer Zeit erkennt sie, dass sein plötzlicher Erfolg ihn vom Unterricht abhalten wird. Unterdessen kauft Marge aus Versehen eine teure Designerhandtasche und als sie diese am nächsten Tag zurückbringt, kommt Homer auf die Idee, generell teure Sachen kurz zu benutzen und anschließend wieder zurückzugeben.

\notiz{
\begin{itemize}
	\item Nelson gibt an, mit zweitem Vornamen Mandela zu heißen.
	\item Nelson wohnt in Ost-Springfield und seine Firma heißt Snot Wheels\index{Snot Wheels}.
	\item Facebook-Freunde von Mark Zuckerberg sind u.\,a. Moe Szyslak, Dr. Hibbert, Kent Brockman und Lucius Sweet.
	\item Grampa hat auch schon in der Episode \glqq Lisa kontra Malibu Stacy\grqq\ (siehe \ref{1F12}) teuere Münzen als Erbe aufgeteilt, damit er noch miterlebt, wie seine Angehörigen sich davon nützliche Dinge kaufen und daran Freunde haben.
	\item Musiklehrer Mr. Largo nimmt mit seinem Fahrrad am Christopher Street Day teil.
\end{itemize}
}

\subsection{The Lisa Series}\label{MABF18}
Für Lisa ergibt sich die Möglichkeit, das Barts Baseball-Team als Trainerin zu leiten, nachdem Ned Flanders als Trainer zurückgetreten ist. Obwohl sie kaum Erfahrung mit dieser Sportart hat, führt sie Barts kleine Mannschaft zu einer Glückssträhne, indem sie ihr Wissen aus Büchern über Statistik und Wahrscheinlichkeitsrechnung anwendet. Als Bart ihre Fähigkeiten als Coach in Frage stellt und ihr vorwirft, den ganzen Spaß aus dem Spiel zu nehmen, wird er von Lisa aus der Mannschaft geworfen. Während Homer mit Lisa zum Finale geht, besucht Marge mit Bart in einen Vergnügungspark. Dort erhält er einen Anruf von Lisa, die ihn bittet, Ralph zu ersetzen. Doch er will nicht. Erst nach einem Gespräch mit Mike Scioscia\index{Scioscia!Mike} kehrt er ins Team zurück. Nachdem Bart die zweite und dritte Base gestohlen hat, scheitert er bei dem Versuch, auch noch das Home zu stehlen und das Team verliert das Spiel. 

\notiz{
\begin{itemize}
	\item Lisa erhält u.\,a. das Buch \glqq Schröderinger's Bat\grqq . Dies ist eine Anspielung auf das Gedankenexperiment der Schrödinger Katze aus der Quantenphysik.
	\item Fehler: Obwohl Lisa Vegetarierin ist, isst sie trotzdem bei Luigi die Pizza mit Salami.
	\item Der britische Street Artist Banksy\index{Banksy} gestaltete den Vorspann und den Couch Gag.
	\item Wie in der Folge bereits von Marge erwähnt, hatte Trainer Mike Scioscia tatsächlich schon mal einen Auftritt bei den Simpsons, damals als Spieler. In der Folge \glqq Der Wunderschläger\grqq\ (siehe \ref{8F13}) gehört er zu den Profis, die Mr. Burns im Kraftwerk anstellt, um sie in seinem Firmenteam einsetzen zu können und denen dann etwas seltsames passiert. Scioscia erleidet dabei eine Strahlenvergiftung, weil er ungeschützt Atommüll umherschob.
\end{itemize}
}

\subsection{Die Lisa-Studie}
Lisa findet heraus, dass Marge gute Schulnoten hatte, bis sie Homer kennengelernt hat. Da sie befürchtet, genauso wie ihre Mutter zu werden, beschließt sie, sich ausschließlich auf das Lernen zu konzentrieren. Schließlich wird Lisa in der Cloisters Academy aufgenommen. Als Lisa herausfindet, dass sich ihre Eltern diese Eliteschule nur leisten können, weil Marge die Schuluniformen wäscht, beschließt sie, die Schule wieder zu verlassen. Währenddessen erhält Bart von seinen Mitschülern den Titel eines Schlägers, da er Nelson unabsichtlich geschlagen hat.

\notiz{
Moe hat eine Verabredung mit Selma Bouvier.}

\subsection{Das Wunder von Burns}\label{NABF01}
Vertreter der großen Fernsehsender wollen mit einer neuen Seuche die Zuschauer in Angst und Schrecken versetzen und verbreiten die Nachricht über eine Grippe, die durch Katzen übertragen wird. Die Bewohner Springfield wollen sich dagegen impfen lassen, doch Mr. Burns beansprucht einerseits den Impfstoff für sich und zerstört den übrigen Impfstoff. Als er erfährt, dass er nur noch wenige Wochen zu Leben hat, lädt er alle Bewohner auf sein Anwesen ein, um ihnen die Nachricht mitzuteilen. Doch alle Anwesenden hassen ihn, so beschließt er seinen Tod zu beschleunigen und stürzt sich von einer Klippe. Er überlebt es jedoch und wird später von Bart ohne Erinnerungen im Wald gefunden. Nachdem er versucht hat, ihn zu verstecken, entscheiden die Bewohner, dass er ab sofort allen Bewohnern der Stadt für ihre Streiche dienen soll. 

\notiz{
\begin{itemize}
	\item Smithers arbeitet vorübergehend für den Vizepräsidenten Dick Cheney.
	\item Dick Cheney löscht Mr. Smithers Erinnerung mit einem Gedankenneuralisator aus, wie er auch im Film Men in Black vorkommt.
\end{itemize}
}

\subsection{Eine Taube macht noch keinen Sommer}\label{NABF02}
Eine Taube durchschlägt Barts Fenster, während Homer eine Gruselgeschichte erzählt. Da der Besitzer die Taube nicht abholen will, pflegt Bart diese wieder gesund. Nachdem die Taube gesund gepflegt wurde, wird sie von Knecht Ruprecht aufgefressen. Eine Therapeutin rät Homer und Marge, dass sie den Hund weggeben sollen.

\notiz{
\begin{itemize}
  \item Edna Krabappel hatte was mit Willie und Oberschulrat Chalmers.
	\item In der Itchy \& Scratchy Folge wird Scratchy nicht verletzt.
	\item Homer hat den Schriftzug \glqq Cheap Trick\grqq\ oberhalb des Steißbeins tätowiert.
\end{itemize}
}

\subsection{Donnie Fatso}
Zum Jahreswechsel treten in Springfield neue Gesetze in Kraft, die Homer gleich reihenweise bricht. Er will die 1000 Dollar Strafe durch Bestechung umgehen. Daraufhin wird er zu zehn Jahren Gefängnis verurteilt. Anstatt ins Gefängnis zu gehen, bietet das FBI ihm an, Fat Tony auszuspionieren. Er nimmt das Angebot an und steigt schnell in der Gunst von Fat Tony auf. Als die Mafia eine Lieferung belgischer Waffen bekommt, fliegt Homers Tarnung auf. Fat Tony ist schwer enttäuscht von ihm und stirbt kurz darauf. Nun nimmt Fat Tonys Cousin Fit Tony aus San Diego die Geschäfte wahr.

\notiz{
\begin{itemize}
	\item Lenny hat sich vorgenommen, im neuen Jahr Latein zu lernen.
	\item In der Ausnüchterungszelle sind Krusty, Kirk van Houten und Sam zu sehen.
\end{itemize}
}

\subsection{Im Zeichen des Schwertes}\label{NABF03}
Als sich Bart und einer der Fünftklässler prügeln wollen, bemerken sie, dass sie beide die selbe schwertförmige Narbe auf der Hand haben. Daraufhin fragt Bart seine Mutter, woher die Narbe stamme. Sie erzählt ihm, dass sie früher mit drei anderen Müttern befreundet war und auch ihre Söhne gute Freunde waren. Bart meint, dass sie ihre alte Freundschaft erneut aufnehmen können. Während sich Marge gut mit ihren Freundinnen versteht, kommt Homer mit deren Ehemännern gar nicht klar.

\notiz{
\begin{itemize}
	\item Als Homer mit Moe über Skype redet, erwähnt er, dass er Flanders WLAN benutzt. 
	\item Neben Homer und Moe sind auch Carl, Lenny und Barney bei Skype angemeldet.
\end{itemize}
}

\subsection{Moeback Mountain}\label{NABF04}
Weil Mr. Burns seinen Assistenten Smithers nicht in seinem Testament erwähnt, beschließt dieser mit Moe, seine Taverne in eine Schwulenbar umzugestalten. Mit seinem neuen Club Mo's haben die beiden großen Erfolg. Dort lernt Dewey Largo seinen Seelenpartner kennen und verlässt deswegen die Stadt. Als neue Musiklehrerin bekommen die Kinder Ms. Juniper, in die sich Rektor Skinner verliebt. Weil Moe behauptet, er sei homosexuell, wird er als erster Repräsentant der Schwulen und Lesben im Stadtrat vorgeschlagen.

\notiz{
\begin{itemize}
	\item In dieser Episode ist Maggie nicht zu sehen und keine Szene findet im Haus der Simpsons statt.
	\item Burns sagt, sein voller Name lautet Charles Montgomery Plantagenet Schicklgruber Burns.
\end{itemize}
}

\subsection{Blut und Spiele}

\begin{itemize}
	\item \textbf{War and Pieces}\\ Marge ist über die Auswirkungen von gewalttätigen Videospielen besorgt und ermutigt Bart und Milhouse dazu, Brettspiele zu spielen. Schließlich gelangen die Jungen an ein Spiel ähnlich zu Jumanji, in dem sie verschiedene wirklich passierende Herausforderungen bewältigen müssen.
	\item \textbf{Master and Cadaver}\\ Homer und Marge verbringen die zweiten Flitterwochen auf einem Segelschiff. Diese werden allerdings von einem Schiffbrüchigen unterbrochen, der vor einem Giftanschlag auf seinem Schiff fliehen konnte. Überzeugt, dass der Flüchtling Homer und Marges Ausflug sabotieren und die beiden töten will, nehmen die beiden die Sache in ihre eigenen Hände.
	\item \textbf{Tweenlight}\\ Lisa verliebt sich in einen neuen Schüler namens Edmund, welcher sich als ein Vampir herausstellt. Schließlich fliehen die beiden nach Dracula-la-Land, doch Homer versucht, seine Tochter zu retten. 
\end{itemize}

\notiz{ Am Ende des zweiten Plots verkleidet sich Maggie als Alex DeLarge aus A Clockwork Orange. 
}

\subsection{The Fight Before Christmas}
Diese Folge beinhaltet vier geträumte Geschichten zum Thema Weihnachten. In der ersten Teilepisode will sich Bart am Weihnachtsmann rächen, weil er bisher noch kein BMX-Fahrrad bekommen hat. In der zweiten Folge kämpft Marge im Zweiten Weltkrieg. In der dritten Episode dekoriert Martha Stewart das Haus der Simpsons weihnachtlich. In der letzten Episode sind die Simpsons als Muppet-Figuren zu sehen und erhalten Besuch von Katy Perry\index{Perry!Katy}.

\notiz{
\begin{itemize}
	\item In dieser Episode gibt es keinen Vorspann.
	\item In der zweiten Episode zündet Marge ein Kino an, in dem sich Adolf Hitler\index{Hitler!Adolf} befindet -- wie im Film \glqq Inglorious Bastards\grqq .
	\item In der letzten Teilepisode sind auf den Koffern Aufkleber Italiens, Deutschlands und Kanadas zu sehen.
\end{itemize}
}

\subsection{Wenn der Homer mit dem Sohne}
Homer übernimmt aus einer Sitcom der 1980er Jahre die Erziehungsmethoden und weigert sich daher, Bart das gewünschte Pocket Bike, ein Motorrad für Kinder, zu kaufen. Bart will durch gute Noten in der Schule erreichen, dass Homer ihm das Motorrad kauft. Da dieser Plan nicht aufgeht, beschließt Bart daraufhin, Geheimnisse aus dem Kernkraftwerk an chinesische Interessenten weitergeben, damit er im Gegenzug das Motorrad erhält. 

\notiz{
\begin{itemize}
	\item Die Sitcom \glqq UpscAlien in da House\grqq\ ist eine Anspielung auf die US Sitcom \glqq Prince of Belair\grqq\ mit Will Smith.
	\item Barts Minibike hat das Kennzeichen TR80R.
\end{itemize}
}

\subsection{Die Farbe Grau}\label{NABF06}
Nachdem Moe einen weiteren Valentinstag allein verbracht hat, besucht er ein Seminar, um den Umgang mit Frauen zu lernen. Währenddessen färbt sich Marge nicht mehr die Haare und sie entscheidet sich dafür, ihre neue graue Haarfarbe zu behalten. Dies kommt nicht gut an bei ihren Bekannten und das führt zu Selbstzweifeln. Sie befürchtet, dass Homer etwas mit anderen Frauen anfangen könnte und versucht vorbeugend einzugreifen, was jedoch ein peinlicher Auftritt wird. 

\notiz{
\begin{itemize}
	\item Sowohl Lenny als auch Carl haben eine Schwester.
	\item Fehler: Homer sagt in der Episode \glqq Ehegeheimnisse\grqq\ (siehe \ref{1F20}), dass sich Marge die Farbe blau färbt, da ihre Haare seit ihrem 18. Lebensjahr grau sind.
	\item Moe hat mal Satanismus ausprobiert, wurde dann aber aus der Gruppe geworfen.
	\item Laut Ausage des Schulpsychologen, Dr. Loreen Pryor, sind Sherri und Terri eigentlich Drillinge. Es gibt noch ein drittes Kind, dass ihnen Böses will.
\end{itemize}
}

\subsection{Wütender Dad -- Der Film}\label{NABF07}
Bart produziert mit Homers Hilfe den Kurzfilm Angry Dad, basierend auf seinem gleichnamigen Cartoon von früher. Schließlich wird er für einen Golden Globe nominiert. Der Kurzfilm gewinnt den Preis, doch bevor Bart sich dafür bedanken kann, reißt Homer die Dankesrede an sich. Auch bei weiteren Preisverleihungen, z.\,B. dem Children's Choice Awards verhält Homer sich ebenso. Deswegen beschließt Bart, ihm nicht zu sagen, dass der Film für den Oscar nominiert wurde. Doch als Bart allein auf der Bühne steht und sich für den Oscar bedankt, fällt ihm auf, dass keine einzelne Person den Preis verdient hat. So schneiden sie den Oscar in kleine Scheiben und geben jedem Mitarbeiter des Films eine Scheibe. 

\notiz{
\begin{itemize}
	\item In dieser Folge sieht man auf Wänden immer wieder den Namen \glqq Banksy\grqq .
	\item Bart entwarf in der Episode \glqq Der rasende Wüterich\grqq\ (siehe \ref{DABF13}) die Comic Figur \glqq Wütender Dad\grqq .
	\item Luigi ist Mitglied der Auslandspresse von Hollywood.
	\item Barts Kurzfilm gewinnt u.\,a. einen Golden Globe, den Children's Choice Award und einen Oscar.
\end{itemize}
}

\subsection{Skorpione wie wir}
Während eines Ausflugs in die Wüste entdeckt Lisa die beschwichtigende Wirkung einer Wü\-sten\-pflan\-ze. Diese Wirkung will ein Pharmakonzern ausnutzen und stellt diesen Wirkstoff synthetisch her. Abe Simpsons erhält für seine Mithilfe ungetestete Pillen, die Bart in Springfield verkauft. Die Nebenwirkungen dieser Pillen sind verherend, denn den Betroffenen fallen die Augäpfel aus den Augenhöhlen.

\notiz{
\begin{itemize}
	\item Die Katzenlady spricht unter Einfluss der Pillen verständlich.
	\item Werner Herzog spricht Walter Hotenhoffer in der englischen und der deutschen Fassung. 
\end{itemize}
}


\subsection{Ein Sommernachtstrip}
Die Tour von \glqq Cheech \& Chong\grqq\ wird unterbrochen, da Tommy Chong aussteigt. Chong will nicht länger die alten Witze vorführen. Da Homer alle Witze der beiden kennt, springt er als Ersatz ein. Dabei überzeugt er und tritt später unter dem Namen Chunk zusammen mit Cheech auf. Währenddessen befreit Marge die Katzenlady von ihrer Sucht, Gegenstände zu sammeln, dabei wird sie allerdings selbst zu einem Messie.

\notiz{
\begin{itemize}
	\item Mit dem Hinweis \glqq \dots you can watch us tomorrow at hulu.com.\grqq\ wird das erste Mal Werbung für ein existierendes Unternehmen gemacht.
	\item Die Katzenlady spricht deutlich, als ihr Haus vom Müll befreit war.
	\item Im Vorspann spielt Lisa Trompete statt Saxophon.
\end{itemize}
}

\subsection{Denn sie wissen nicht, wen sie würgen}
Weil Homer Bart öffentlich blamiert hat, soll er einen Kurs besuchen, um ein besserer Vater zu werden. Dort erzählt er, dass er Bart würgt. Deswegen beschließt Dr. Zander, Homer solle Barts Rolle übernehmen. Kareem Abdul-Jabbar über\-nimmt Homers Rolle und würgt ihn bei jeder Kleinigkeit. Durch die Therapie kann Homer Bart nicht mehr würgen. Dies führt dazu, dass Bart noch gemeinere Streiche als sonst treibt. Deswegen sucht Marge Dr. Zander auf. Er versucht Bart zu therapieren, doch letztendlich scheitert er daran und würgt Bart. 

\notiz{
\begin{itemize}
	\item In der Eröffnungssequenz fliegt das Planet Express Raumschiff aus Futurama durch das Bild und die Futurama-Titelmusik ist zu hören.
	\item Bart schreibt im Couchgag im Kunststil \glqq ASCII\index{ASCII}\grqq\ das Wort \glqq Fatso\grqq\ auf Homers Brust.
	\item Bart schreibt Moe eine SMS mit dem Inhalt \glqq I. M. A. Wiener\grqq .
\end{itemize}
}


\subsection{Die große Simpsina}
Als Lisa auf den großen Raymondo trifft, beschließt er, ihr seine Tricks zu lehren. Zuletzt weiß Lisa auch noch, wie der Milchkannentrick von Houdini geht, aber sie verrät ihn unabsichtlich an Cregg Demons Sohn. Cregg Demon ist einer der Feinde von Raymondo. Cregg führt nun auch den Trick vor, um Ruhm zu ernten. Doch die eigentlich manipulierte Kanne wurde durch eine Echte ersetzt und Cregg Demon droht zu ertrinken. Schließlich wird er von Raymondo selbst gerettet.

\notiz{
\begin{itemize}
	\item Diese Episode hat keinen Vorspann.
	\item Cregg Demon ist eine Anspielung auf Chris Angel.
\end{itemize}
}

\subsection{Die Mafiosi-Braut}
Fat Tony und Selma kommen sich näher. Er schenkt ihr auch noch eine Fettabsaugung und Selma ist scheinbar ein ganz neuer Mensch. Beide heiraten, nur Selma weiß nicht, dass sie nur Tonys Nebenfrau ist. Währenddessen findet Lisa heraus, dass Bart kostbare Trüffel erschnüffeln kann und beide machen dabei vermeintlich das große Geschäft. 

\notiz{
\begin{itemize}
	\item Laut Marges Aussage, war Disco Stu der einzige von Selmas Ehemännern, den sie leiden konnte.
	\item Im Spieleladen ist das Computerspiel \glqq Triangle Wars\grqq\ zu sehen.
\end{itemize}
}

\subsection{Homer mit den Fingerhänden}\label{NABF13}
Als Homer Patty Farbe aus ihren Haaren schneidet, erkennt er sein Talent als Friseur und eröffnet daraufhin seinen eigenen Salon. Er wird so erfolgreich, dass es ihm allerdings schon wieder zu viel wird. Währenddessen wird Milhouse erneut von Lisa abgelehnt und er freundet sich stattdessen mit der Fünftklässlerin Taffy an. Lisa wird eifersüchtig und spioniert den beiden nach.

\notiz{
\begin{itemize}
	\item Chief Wiggum geht nie wählen, ist Internet abhängig und hat Angst vor Spinnen im Bad.
	\item Gegen Ende der Episode küsst Lisa Milhouse.
\end{itemize}
}

\subsection{500 Schlüssel}
Die Simpsons suchen nach einem Ersatzschlüssel für das Auto, da sich Maggie darin eingeschlossen hat. Sie entdecken eine große Zahl an Schlüsseln. So machen sich die Familienmitglieder auf den Weg und erkunden, wo in Springfield die gesammelten Schlüssel passen.

\notiz{
\begin{itemize}
	\item Lisa spielt im Vorspann Geige.
	\item Homers Autonummernschild trägt die Aufschrift NABF14.
	\item Chalmers ist für 14 Schulen zuständig.
\end{itemize}
}


\subsection{Nedna}
Durch Bart verliert Edna Krabappel ihren Job als Lehrerin und muss zusammen mit anderen in Ungnade gefallenen Lehrern von morgens bis abends in einem Raum sitzen und auf ihre Anhörung warten. Bei der Flucht über die Feuerleiter kommt ihr Ned Flanders zu Hilfe. Die beiden kommen sich näher, was Bart gar nicht passt. 

\notiz{
\begin{itemize}
	\item Lisa ist die Zeugwartin des Schulbasketballteams, der Springfield Lady Pumas.
	\item In Moes Taverne ist ein Foto der Bowlingmannschaft der Pin Pals zu sehen.
\end{itemize}
}

\section{Staffel 23}

\subsection{Homer Impossible}
Homer versucht krampfhaft, sich mit seinem coolen neuen Kollegen, dem Sicherheitsmann Wayne, anzufreunden. Der ist allerdings äußerst zurückhaltend und interessiert sich nicht für neue Bekanntschaften. Wie sich herausstellt, ist er ein ehemaliger CIA-Agent. Wayne rettet Homer schließlich das Leben.


\notiz{
\begin{itemize}
	\item In dieser Episode gibt es keinen Couch- sowie Tafelgag, da der Comic-Book-Guy das Intro moderiert. Er begrüßt die Zuschauer und erzählt, dass am Ende der Folge das Ergebnis der Internet-Abstimmung über Pro Nedna oder Contra Nedna, die am Ende der 22. Staffel angekündigt wurde, verkündet wird. 
	\item Edna und Ned sind in Neds Bett zu sehen.
	\item Wanye Slater verfasste während seiner Gefangenschaft in Nord Korea das Muscial \glqq Being Short Is No Hindernace To Greatness\grqq .
\end{itemize}
}

\subsection{Teddy-Power}\label{NABF17}
Nach einem Streich von Bart ist Rektor Skinner überfordert und Oberschulrat Chalmers über\-nimmt persönlich Barts Unterricht. Er versucht durch die Geschichte von Teddy Roosevelt, Barts Interesse zu erwecken und dies mit großem Erfolg. Allerdings wird Chalmers entlassen. Bart erreicht aber seine Wiedereinstellung als Oberschulrat.

\notiz{
\begin{itemize}
	\item Auf einem Foto ist ein Boxkampf zu sehen, bei dem Theodore Roosevelt gegen Mr. Burns kämpft.
	\item Auf einem Poster in der Schule ist Ralph als Schülersprecherkandidat zu sehen. Sein Motto lautet \glqq Time For A Change\grqq .
	\item Auf dem Schlafsack von Milhouse ist Arielle, die Meerjungfrau von Walt Disney abgebildet. 
\end{itemize}
}

\subsection{Das Ding, das aus Ohio kam}
Für ein Wissenschaftsprojekt erschaffen Bart und Martin Prince einen Seehund-Roboter, der besonders im Altenheim schnell beliebt wird. Währenddessen erhält Homer eine neue Kollegin. Zunächst ist sie seine Assistentin, doch dann schwärzt sie Homer bei Mr. Burns an und übernimmt seine Aufgaben. Durch einen Tipp von Ned gelingt es Homer, dass sie wieder entlassen wird.

\notiz{
\begin{itemize}
	\item Auf der Zeitschrift \glqq Popular Robotics\grqq\ ist ein Foto von Bender\index{Bender} aus Futurama zu sehen.
	\item Alice Glick stirbt.
	\item Frink sagt ein Date mit einer Schönheitskönigin ab, um Bart und Martin zu helfen.
	\item Lisa hat auf einem Globus in der Mitte der USA Springfield eingezeichnet.
\end{itemize}
}

\subsection{FoodFellas}
Nachdem Homer mit den Kindern bei einer Videospiel-Expo eine tolle Zeit hatte, versucht Marge auch etwas mit Bart und Lisa zu unternehmen. Als Marge und die Kinder dann in einem äthiopischen Restaurant Essen aufgetischt bekommen, sind diese davon so begeistert, dass sie einen Lebensmittel-Blog starten. In der Folgezeit fühlt sich Homer unbeachtet.

\notiz{
\begin{itemize}
	\item Der dänische Koch von den Muppets ist unter den anderen Köchen zu sehen.
	\item Die E4-Messe ist natürlich eine Anspielung auf die Computer- und Videospielmesse \glqq E3\grqq\ (Electronic Entertainment Expo), die jedes Jahr in Los Angeles abgehalten wird.
	\item Nachdem der Drogenkoch den Nachtisch von Marge gegessen hat, wird er in seine Kindheit zurückversetzt, genau wie der Restaurantkritiker in Ratatouille.
\end{itemize}
}

\subsection{Homers Sieben}\label{NABF22}
Lisa erfährt, dass ihre Lieblingsbuchreihe \glqq Angelica Button\grqq\ statt einer Autorin von mehreren Schreibern stammt. Dahinter steckt eine seelenlose Vereinigung, die sich nur auf Profit konzentriert. Davon inspiriert will Homer selbst mit einer Gruppe von Springfieldern so ein Buch erschaffen und reich werden.

\notiz{
\begin{itemize}
	\item Dem Team gehören Homer, Bart, Patty, Rektor Skinner, Moe, Prof. Frink und Neil Gaiman an.
	\item Moe hat bereits fünf Kinderbücher veröffentlicht u.\,a. \glqq There's a rainbow in my basement\grqq .
\end{itemize}
}

\subsection{Der Mann im blauen Flanell}
Mr. Burns ernennt Homer zum neuen Kundenbetreuer des Atomkraftwerks. Das neue Leben voller Arbeit und Alkohol wird ihm allerdings schnell zur Last. Inzwischen erkennt Bart den Unterhaltungswert von klassischer Literatur.

\notiz{
\begin{itemize}
	\item In Homers Büro ist auf seinem Schreibtisch eine Jägermeisterflasche\index{Jägermeister} zu sehen.
	\item Im Springfield Atoms Stadion haben u.\,a. das Atomkraftwerk, der Try 'N' Save Markt und Costington's VIP-Lounges.
	\item Auf einem Foto ist Lisa bei ihrer Erstkommunion zu sehen.
\end{itemize}
}

\subsection{Agentin mit Schmerz}
Krusty wird aus seiner eigenen Show gefeuert, jedoch startet er mit seiner alten Managerin Annie Dubinsky\index{Dubinsky!Annie} eine neue Karriere beim Kabelfernsehen, wo er mit dem Nostalgiefaktor punktet.

\notiz{
\begin{itemize}
	\item Nachdem Maggie in einem Itchy \& Scratchy Cartoon den Hitlergruß gesehen hat, macht sie diesen nach.
	\item Im Fernsehmuseum sind die Figuren aus der Serie \glqq King of the Hills\grqq\ zu sehen.
\end{itemize}
}

\subsection{Weihnachten -- Die nächste Generation}
Die Folge spielt an Weihnachten, 30 Jahre in der Zukunft, als Bart, Lisa und Maggie mit ihren eigenen Kindern ihre Eltern besuchen. Bart lebt jetzt in der Grundschule Springfield, die nun ein Appartementkomplex ist. Er hat zwei Kinder mit Jenda, ist aber von ihr geschieden. Lisa hat schließlich doch Milhouse geheiratet. Maggie ist schwanger und bekommt gerade ihr Baby. Man sieht, wie die Kinder jetzt auch Elternprobleme haben. 

\notiz{
\begin{itemize}
	\item Homer hat mit dem Trinken aufgehört.
	\item Am Flughafen wird das Reiseziel New New York angeboten. Die Stadt, in der Futurama\index{Futurama} spielt, heißt auch New New York.
	\item Texas ist eine unabhängige Republik.
	\item Barts Söhne nennen ihren Großvater Homer.
\end{itemize}
}

\subsection{Im Rausch der Macht}\label{PABF03}
Bart dreht ein Video von Homer und stellt es in das Internet, woraufhin dieser schlagartig berühmt wird. Er nutzt den Bekanntheitsgrad für eine politische Talkshow, das zum Sprachrohr für die Bevölkerung wird. Als Homer den nächsten Republikaner für das Präsidentenamt nominieren soll, entscheidet er sich für den Sänger Ted Nugent\index{Nugent!Ted}. Nach Lisas Zureden gibt Homer diese Entscheidung auf. 

\notiz{
\begin{itemize}
	\item Homers Fernsehshow heißt \glqq Gut Check\grqq .
	\item Homers Buch \glqq America: Love it or I'll punch you!\grqq\ umfasst 36 Seiten inklusive Index. Von diesem Buch gibt eine Hörbuchversion, die von Lenny gelesen wurde.
\end{itemize}
}

\subsection{Der Stoff, aus dem die Träume sind}
Moe wird ausgelacht, da sein bester Freund ein Putzlappen ist. Doch dieser hat es in sich und wir erfahren, dass er eine Geschichte von Jahrhunderten hinter sich hat. Der Putzlappen ist ein Überrest eines mittelalterlichen Wandteppichs.  Währenddessen kündigt Milhouse Bart die Freundschaft.

\notiz{
\begin{itemize}
	\item Moe gibt an, auf Facebook mit Alka Seltzer befreundet zu sein.
	\item Als Bürgermeister Quimby mit einem Richterhammer auf die Jukebox in Moes Taverne schlägt, wird das Lied \glqq All Night Long (All Night)\grqq\ von Lionel Richie gespielt. Daraufhin ist auch Lionel Richie in Moes Taverne zu sehen. 
\end{itemize}
}

\subsection{Spider-Killer-Avatar-Man}

\begin{itemize}
	\item \textbf{Flatuleniza}\\ Beim Dekorieren für Halloween wird Homer von einer Spinne gebissen und ist deswegen gelähmt. Als er von einer weiteren Spinne gebissen wird, kann er ähnlich wie Spiderman Spinnenfäden abschießen, er bleibt jedoch weiter gelähmt. 
	\item \textbf{Bei Anruf Ned}\\ Homer bringt Ned dazu, mehrere seiner Feinde zu ermorden. Als dieser das herausfindet, will er Homer umbringen und dieser würde für seine Taten in die Hölle kommen. Doch Homer behauptet, es gäbe keinen Himmel und keine Hölle. Nachdem er versucht hat, eine Bibel anzuzünden, erscheint Gott und würgt Homer zu Tode. 
	\item \textbf{Na'vi Cis}\\ Bart und Milhouse führen auf dem Heimatplaneten von Kang und Kodos eine verdeckte Militäraktion aus, um dort den seltenen Stoff Humorium für Krusty zu finden.
\end{itemize}


\subsection{Freundschaftsanfrage von Lisa}\label{PABF04}
Da Lisa Probleme hat, Freunde zu finden, beschließt sie, einen Online-Treffpunkt mit Namen \glqq SpringFace\index{SpringFace}\grqq\ zu schaffen. Der Ansturm ist riesig, und schon bald tummeln sich neben Jugendlichen auch immer mehr Erwachsene in dem sozialen Netzwerk. Weil die Menschen es exzessiv nutzen, kommt es zu einer Katastrophe, bei der 35 Menschen sterben. Aus diesem Grund wird Lisa angeklagt und erzählt die Geschichte vor Gericht.

\notiz{
\begin{itemize}
  \item Lisa arbeitet in der Schule an dem Mapple\index{Mapple} Computer Lisa. 1983 war der Apple Lisa einer der ersten Personal Computer, der über eine Maus und ein Betriebssystem mit grafischer Benutzeroberfläche verfügte. Lisa war der Vorgänger des Apple Macintosh.
	\item Reverend Lovejoy hat in der Kirche einen WLAN-Zugangspunkt installieren lassen.
	\item Am Ende dieser Episode sieht man die Winklevoss\index{Winklevoss}-Zwillinge, die für viele die eigentlichen Facebook-Erfinder sind.
\end{itemize}
}

\subsection{Unter dem Maulbeerbaum}
Bart und Milhouse sind von der Sendung \glqq Mythcrackers\index{Mythcrackers}\grqq\ so inspiriert, dass sie selber den Wahrheitsgrad von Legenden untersuchen. Inzwischen verliebt sich Lisa in Nick, einem intellektuellen Romantiker, was Marge nicht gefällt.

\notiz{
\begin{itemize}
	\item Im Couchgag will Moe zur 500. Episode gratulieren, erfährt aber dann von Lisa, dass es erst die 499. Folge ist.
	\item Mythcrackers ist eine Anspielung auf die Serie MythBusters.
\end{itemize}
}

\subsection{Fern der Heimat}
Die Simpsons werden aus der Stadt verbannt und beginnen ein neues Leben außerhalb von Springfield in den \glqq Outlands\grqq\ \index{Outlands} in Anarchie. So werden sie Nachbarn von Wikileaks-Gründer\index{Wikileaks} Julian Assange\index{Assange!Julian}. Bald interessieren sich jedoch auch andere Springfieldianer wie Lenny, Carl und Moe an dieser Lebensart. Viele kommen nach und schließlich ist jeder von Springfield in den Outlands, wo nun Springfield neu aufgebaut wird. Nur Rektor Skinner is noch im alten Springfield, woraufhin Bart ihn jedoch holt und in das neue Springfield bringt. 

\notiz{
\begin{itemize}
  \item Im Couchgang werden zuerst alle vorherigen Couchgags im Schnelldurchlauf gezeigt und dann wird aus den Bildern ein Photomosaic erstellt, dass auf die 500. Folge verweist. 
	\item Im Vorspann spielt Lisa im Schulorchester Sousaphon statt Saxophon.
	\item Otto ist Echthaarspender. So trug Selma Hayek\index{Hayek!Selma} Haar von ihm bei der Oscar-Verleihung.
\end{itemize}
}

\subsection{El Barto}
Homer ist mal wieder sauer auf seinen Sohn und steckt Bart in den Hasenkäfig. Bart beschließt, sich zu rächen. Er bastelt eine Bildschablone mit Homers Konterfei und dem Text Vollidiot und besprüht die ganze Stadt damit. Als er immer kreativer wird und die Kunstwerke, die Homer lächerlich machen sollen, immer interessanter werden, ist er schon bald der Star der Straßenkünstler-Szene.

\notiz{
\begin{itemize}
	\item Laut Lisa hat Marge am 19. März Geburtstag, das steht allerdings im Widerspruch zur Marges Aussage, sie habe im Mai Geburtstag (geäußert in der Episode \glqq Kiss, Kiss Bang Bangalore\grqq\ (siehe \ref{HABF10})).
	\item Der englische Originaltitel ist eine Anspielung auf den Film \glqq Exit through the Gift Shop\grqq\ von Banksy\index{Banksy}.
\end{itemize}
}

\subsection{How I Wet Your Mother}
Homer verleitet die anderen Angestellten des Atomkraftwerks dazu, Büro\-ma\-terial\-ien zu stehlen. Er ist der Einzige, welcher nicht von Mr. Burns dabei erwischt wird. Während die Anderen bestraft werden, erhält Homer einen Tag Urlaub. Diesen verbringt er mit Bart beim Angeln. Am darauf folgenden Morgen wacht Homer auf und hat ins Bett gemacht. Nachdem dies kein einmaliger Vorfall bleibt, versucht Homer verschiedene Sachen dagegen. Unter anderem entschuldigt er sich bei seinen Kollegen mit einem Picknick für sein Verhalten. Weil dies allerdings keine Besserung bringt, fragt Marge Prof. Frink, ob dieser Homer helfen kann. Tatsächlich hat er ein Gerät konstruiert, mit dem man sich in Träume einer anderen Person begeben kann. Daher begeben sich die Simpsons gemeinsam in Homers Träume und können letztendlich das Problem lösen. 

\notiz{
\begin{itemize}
	\item Auf den Gelben Seiten Springfields steht: Das Internet für alte Leute.
	\item Bart hat einen Twitter-Account.
	\item Prof. Frink gibt an, bewiesen zu haben, dass es die Hölle gibt.
\end{itemize}
}

\subsection{Mein Freund, der Roboter}
Wegen des Gesundheitszustandes der Mitarbeiter muss Mr. Burns eine Strafe zahlen. Aus diesem Grund beschließt er, die Arbeiter durch Roboter zu ersetzen. Die einzige menschliche Arbeitskraft, die verbleibt, ist Homer. Einen Angestellten aus Fleisch und Blut braucht Mr. Burns, um ihm im Falle eines Störfalles die Schuld in die Schuhe schieben zu können. Sogar Mr. Smithers wird entlassen. Homer versucht, sich mit den Robotern anzufreunden, jedoch ohne Erfolg. Schließlich programmiert Homer die Roboter um -- mit verheerenden Folgen.

\notiz{
\begin{itemize}
	\item Mr. Burns liest das Buch \glqq Bossypants\index{Bossypants}\grqq\ von Tina Fey.
	\item Mr. Burns bedient über eine iPad-App\index{iPad} die Falltür in seinem Büro.
\end{itemize}
}

\subsection{Gestrandet}\label{PABF11}
Bart muss Jimbos Freundin ins Kino begleiten, aber sie entwickelt langsam Gefühle für Bart. Auch er ist von ihr angetan, fürchtet sich aber vor Jimbos Rache. Unterdessen wird Homer von der alten TV-Serie \glqq Stranded\grqq\ \index{Stranded} richtig süchtig und befürchtet überall Spoiler.

\notiz{
\begin{itemize}
	\item Der Couchgag ist ein Kurzfilm von Bill Plympton\index{Plympton!Bill}.
	\item Der pickelige Teenager trägt ein Namensschild, auf dem Steve steht.
	\item In der TV-Serie ist eine außerirdische Inschrift auf einem Stein zu sehen. Es handelt sich um die Sprache, die auch in Futurama verwendet wird. Die Inschrift bedeutet: \glqq WATCH FUTURAMA THURSDAY AT 10\grqq .
\end{itemize}
}

\subsection{Im Zeichen der Kreuzfahrt}
Bart ist von der Monotonie des Alltags genervt und möchte deshalb unbedingt an einer Kreuzfahrt teilnehmen. Die Familie unterstützt ihn und letztendlich erleben die Simpsons ihren besten Urlaub aller Zeiten. Aber auch das wird irgendwann enden und Bart denkt sich was aus, um das zu verhindern.

\notiz{
\begin{itemize}
	\item Alles was man normal beim Couchgag sieht, wurde mit schwarzen Worten auf hellem Grund geschrieben. 
	\item Auf dem Fun-Stundenplan an Bord des Kreuzfahrtschiffes gibt es u.\,a.
	\begin{itemize}
		\item XBox mit Playstation-3-Controllern,
		\item Astronautentraining mit geprüftem Kosmonauten und
		\item Sith-Training mit geprüftem Sith-Lord.
	\end{itemize}
\end{itemize}
}

\subsection{Der Spion, der mich anlernte}
Homer erleidet bei einem Unfall eine Gehirnerschütterung und ist acht Wochen krank geschrieben. Diese Zeit will er nutzen, um ein besserer Ehemann zu werden, denn davor hat er Marge bei einem Kinobesuch stark blamiert. Nelson stiehlt in der Zwischenzeit den anderen Kindern das Essensgeld. Inspiriert von einem Film in der Schule kommt Bart die Idee, Nelson zu mästen.

\notiz{
Der Agent Stradivarius Cain ist eine Anspielung auf James Bond.
}


\subsection{Lisa wird gaga}\label{PABF14}
Lisa wird zur unbeliebtesten Schülerin der Grundschule gewählt. Daraufhin lobt sie sich auf dem Blog der Viertklässler und wird dabei ertappt. Davon erfährt Lady Gaga durch ihre übersinnliche Kraft und fährt nach Springfield, um ihr zu helfen.

\subsection{Schlaflos mit Nedna}\label{PABF15}
Bei der Premiere zu dem Theaterstück \glqq Die Passion Christi\grqq\ kommt es zu einem Unfall und Ned Flanders wird ins Krankenhaus gefahren. Edna begleitet ihn und so erfahren die Einwohner Springfields, dass die beiden heimlich geheiratet haben. Edna versucht die Kinder von Ned zu erziehen und schickt sie auf die Grundschule Springfield. 

\notiz{
\begin{itemize}
	\item Homer hat einen Twitter-Account.
	\item Dr. Hibbert fährt einen Geländewagen von Mercedes.
\end{itemize}
}


\section{Staffel 24}

\subsection{Moonshine River}
Partytime in Springfield und nur Bart hat wieder einmal kein Date. Lisa macht sich über ihn lustig und behauptet, dass kein Mädchen es mit ihm aushalten könne. Bart will seiner Schwester das Gegenteil beweisen und macht sich auf die Suche nach seinen Ex-Freundinnen, um herauszufinden, ob sie ihn noch mögen. Das Ergebnis ist jedoch niederschmetternd. Seine letzte Hoffnung ist Mary Spuckler -- doch die ist nach New York abgehauen. Bart reist ihr kurzerhand nach.  Er kann seine Familie sogar zu einer Reise nach New York überzeugen und findet dort Mary tatsächlich wieder, die inzwischen als Autorin für Saturday Night Life arbeitet. In den folgenden Tagen kommen sich die beiden wieder näher, bis ihr Vater Cletus plötzlich auftaucht, um seine Tochter wieder zurück nach Springfield zu holen. Da Mary allerdings kein Interesse mehr daran hat, bei ihren Eltern zu leben, fährt sie mit dem Zug davon und lässt den am Boden zerstörten Bart zurück.

\begin{itemize}
  \item Es sind Werbeschilder für das MyPad 4 und Canoynero\index{Canoynero} Hybrid zu sehen.
	\item Die Episode ist Andy Williams\index{Williams!Andy} gewidmet.
\end{itemize}

\subsection{Die unheimlich verteufelte Zeitreise durch das schwarze Loch}
Die Halloween-Folge besteht aus drei Geschichten:
\begin{itemize}
	\item \textbf{Das Ungeheuer von Loch Schwarz}\\ Auf Lisas Betreiben wird in Springfield ein Teilchenbeschleuniger gebaut. Der funktioniert zwar nicht besonders gut, schafft es aber, ein kleines schwarzes Loch zu produzieren. Die Bewohner werfen solange Dinge hinein, bis es so groß wird, dass es die gesamte Stadt zerstört.
	\item \textbf{UNnormal Activity}\\ Bei den Simpsons geschehen nachts merkwürdige Dinge: Ein unsichtbarer Dämon treibt sein Unwesen. Am Ende stellt sich heraus, dass Marge vor 30 Jahren einen Pakt mit einem Dämonen geschlossen hat, um ihre Schwestern zu retten.
	\item \textbf{Bart und Homers verrückte Reise durch die Zeit}\\ Bart ärgert sich, dass ein seltener Comic aus dem Jahre 1974 200 Dollar kostet. Da kommt Professor Frink mit einer Zeitmaschine vorbei und Bart reist ins Jahr 1974, um das Heft für 25 Cent zu kaufen und verhindert dort, dass Marge und Homer sich verlieben. Wieder zurück in der Gegenwart stellt er fest, dass Artie Ziff sein neuer Vater ist. 
\end{itemize}

\notiz{
Dies ist die erste Horror-Episode seit fast zehn Jahren (seit Folge \glqq Schickt die Klone rein\grqq , siehe \ref{DABF19}), die ihre Deutschlandpremiere nicht um Halloween herum hatte.
}

\subsection{Homers vergessene Kinder}
Als Marges Auto in einem gewaltigen Erdloch versinkt, muss sie sich einen neuen Wagen kaufen, den sie allerdings hasst. Homer vermutet ein tiefenpsychologisches Problem hinter ihrer Aversion gegen das neue Auto. Bald stellt sich heraus, dass dem tatsächlich so ist: Da der Wagen nur eine kleine Rückbank hat, weiß Marge, dass nie ein viertes Kind auf dieser Rückbank Platz haben könnte und so steht der Autokauf für sie symbolisch für eine Entscheidung gegen ein weiteres Kind. Da Homer jedoch unfruchtbar ist, haben sie keinen Erfolg. Doch Moe schlägt vor, dass Homer sich eine seiner Samenspenden von der Shelbyville Samenbank, die er früher dort gemacht hat, zurück holt. Weil Homer das verhindern möchte, versucht er vergeblich Marge durch einen Umweg über die historische Route abzulenken. Erst in der Samenbank, in der Marge sieht, dass mit Homers Samenspenden zahlreiche Kinder gezeugt worden sind, sieht sie ein, dass es bereits zu viele Kinder von Homer gibt und verzichtet auf ein viertes Kind.

\begin{itemize}
	\item Homer hat die Blutgruppe AB laut Unterlagen in der Samenbank. In der Episode \glqq Der Lebensretter\grqq\ (siehe \ref{7F22}) hatte er die Blutgruppe A+. In dieser Folge hatte Bart die Blutgruppe 0. Diese Blutgruppe ist allerdings nicht möglich, wenn ein Elternteil die Blutgruppe AB hat.
	\item Es gibt ein Motel mit dem Namen \glqq M. Night Shyamalan's Flop House\grqq .
\end{itemize}

\subsection{Grampa auf Abwegen}
Homer und Marge haben wieder einmal den Besuchstag im Altersheim vergessen. Als sie am nächsten Tag hinfahren, ist Grampa verschwunden. Ein Hinweis führt sie in das griechische Restaurant \glqq Spiro's\grqq\index{Spiro}. Dort hat Grampa vor vielen Jahren gearbeitet und war in die dort ebenfalls beschäftigte Sängerin Rita Lafleur verliebt und hat diese sogar geheiratet. Klein Homers zuliebe verzichtete er mit Rita auf eine Europa-Tournee.

\begin{itemize}
	\item Bernice Hibbert hat einen Bruder namens Jester.
	\item Milhouse, Ralph und Martin nehmen am Karatekurs teil.
	\item Abe war mit 35 Jahren Hilfskellner und Homer war damals sechs Jahre alt.
\end{itemize}

\subsection{Goodsimpsons}
Homer, Moe und Apu haben einen neuen Bowlingkumpel namens Dan Gillick\index{Gillick!Dan}, einen etwas unscheinbaren Buchhalter. Was die Freunde nicht ahnen: Dan arbeitet bei der örtlichen Mafia unter Fat Tony. Als Fat Tony seinen Geschworenendienst antreten muss, ernennt er Dan zu seinem Vertreter -- sehr zum Unmut der anderen Mafiosi. Doch die neue Aufgabe verändert Dan sehr, er hat Blut geleckt. In der Zwischenzeit bricht Lisa bei einem Schulkonzert zusammen und Dr. Hibbert diagnostiziert bei ihr einen Eisenmangel. Dieser ist auf ihre vegetarische Ernährung zurück zuführen. Deshalb verschreibt er ihr Eisentabletten, welche sie aber nicht so recht verträgt. Deshalb empfiehlt ihr Schulköchin Doris, Insekten zu essen, was auch kein Verstoß gegen die vegetarische Ernährung sei. So probiert sie es und kommt dadurch fast soweit, dass sie wieder Fleisch isst.

\notiz{
\begin{itemize}
	\item In der Bowling-Mannschaft \glqq Holy Rollers\grqq\ sind Ned Flanders, Reverend Lovejoy, der Parson und Rabbi Krustofski.
	\item Johnny Schmallippe baut in seinem Urlaub freiwillig Brunnen in Darfur.
\end{itemize}
}

\subsection{Mein Freund, der Wunderbaum}\label{PABF22}
Homer gewinnt bei einer Verlosung das neue myPad von Mapple. Schon bald verbringt er 24 Stunden am Tag mit seinem neuen Tablett, was dazu führt, dass er unaufmerksam ist und in ein Loch stürzt, wobei das myPad zerstört wird. Wenig später geschieht ein Wunder: Auf einem Baumstamm sind wie durch göttliche Eingebung Buchstaben erschienen, die das Wort \glqq Hoffnung\grqq\ bilden. Homer lässt daraufhin alle Springfielder zu seinem Baum pilgern, um das göttliche Wunder zu bestaunen.

\notiz{
\begin{itemize}
	\item Moe betrieb früher eine Eisdiele (Moe's Cream Carnival).
	\item Der Hummelmann wechselt von Kanal 8 zu Kanal 6.
\end{itemize}
}

\subsection{Coole Aussichten}\label{PABF20}
Homer macht die Bekanntschaft von Terrence\index{Terrence}, einem Donut-Händler, der unglaublich trendy und cool ist. Da sich Homer selbst für alt und uncool hält, freundet er sich mit Terrence an. Als Homer seinen neuen Freund davon überzeugen kann, in das Nachbarhaus der Simpsons zu ziehen, breitet sich die hippe Coolness des neuen Nachbarn wie ein Virus in Springfield aus und verändert die gesamte Stadt.

\notiz{
\begin{itemize}
	\item Terrence Donut-Geschäft heißt \glqq Devil Donuts\grqq\index{Devil Donuts}.
	\item Lisa wurde neun Monate lang gestillt.
	\item Sideshow Bob fährt Einrad.
	\item Lenny ist Mitglied der Feuerwehr.
\end{itemize}
}

\subsection{Wem der Bongo schlägt}\label{RABF01}
Während Marge und die Kinder Grampa aus dem Altenheim abholen, passt Homer auf Familienhund Knecht Ruprecht auf und sperrt ihn versehentlich im Schrank ein. Seine Familie unterstellt ihm sofort, den Vierbeiner nicht zu mögen. Das kann Grampa so nicht stehenlassen und rückt mit einer traurigen Geschichte aus Homers Kindheit raus. Homer hatte als kleiner Junge einen Hund namens Bongo\index{Bongo}. Da Bongo Mr. Burns gebissen hatte, gab ihn Abe an eine befreundete Familie auf dem Land.

\notiz{
\begin{itemize}
  \item In der Episode \glqq Die erste Liebe\grqq\ (siehe \ref{FABF13}) sind Homer und Clancy Wiggum etwa gleich alt. In dieser Folge ist Clancy älter. Er fährt als Hundefänger den Hundetransporter.
	\item Herman heißt mit Nachnamen Hermann. Ihm wurde von Clancy beim Trampen der rechte Arm abgefahren.
	\item Prof. Frink hat eine Firma namens Frinklabs\index{Frinklabs}.
\end{itemize}
}

\subsection{Homergeddon}
Nach einem Gespräch mit einem Unbekannten in Moes Taverne ist Homer überzeugt, dass der Weltuntergang naht. Er beginnt, Vorräte zu horten und sich über die Gruppe der \glqq Überleber von Springfield\grqq\ zu informieren. Als er im Kernkraftwerk einen Fehler begeht und Springfields Elektrizität lahmlegt, glaubt er, das Ende sei gekommen. Also schnappt er sich seine Familie und die Vorräte, um ins geheime Basislager der \glqq Überleber\grqq\ zu flüchten.

\subsection{Das Bart Ultimatum}\label{RABF03}
Dank Mr. Burns droht der Grundschule von Springfield die Schließung: Mr. Burns treibt die Energiekosten dermaßen in die Höhe, dass die Schule mit den schlechtesten Schülern dichtgemacht werden soll. Nun ist es an Bart, die Katastrophe mit Lisas Hilfe abzuwenden. Homer verdient sich unterdessen mit einer illegal aufgestellten Parkuhr etwas dazu. Logisch, dass die Bürger von Springfield das nicht lustig finden.

\notiz{
\begin{itemize}
	\item Küchenhilfe Doris wurde des Giftmordes an ihrem Mann verdächtigt. Seine Leiche wurde allerdings nie gefunden.
	\item Die Episode ist Huell Howser\index{Howser!Huell} gewidmet. Huell hatte eine Auftritt in der Folge \glqq Drum prüfe, wer sich ewig bindet\grqq\ (siehe \ref{GABF04}).
\end{itemize}
}

\subsection{Stille Wasser sind adoptiv}\label{RABF04}
Ein Tornado tobt über Springfield. Es geht alles gut aus, dennoch bleibt ein Gedanke vorherrschend: Was passiert mit den Kindern, falls Homer und Marge etwas zustößt? Die beiden beschließen, sich nach einem Paar umzusehen, das bereit ist, die Vormundschaft zu übernehmen. Nachdem ihre Suche in der Familie, bei Freunden und Bekannten keinen Erfolg hatte, suchen sie auch außerhalb der Stadt. So begegnen sie dem jungen Paar Mav und Portia, von denen nicht nur Marge und Homer, sondern auch Bart und Lisa begeistert sind. Nachdem die Formalitäten erledigt sind, verbringen die Kinder viel Zeit mit den beiden, während ihre Eltern die kinderfreie Zeit nutzen, um sich zu amüsieren.

\notiz{
\begin{itemize}
	\item Abe Simpsons Kommissionsnummer ist die 1923157 -- eine Primzahl.
	\item Homers Halbbruder Herb ist wieder arm.
	\item Cletus hat 17 Kinder.
\end{itemize}
}

\subsection{Verrückt nach Mary}\label{RABF07}
Bart hat Liebeskummer. Nachdem er es endlich geschafft hat, mit der bezaubernden Mary zusammen zu kommen, benimmt er sich bald wie ein Idiot. Es dauert nicht lange, bis Mary ihm den Laufpass gibt -- der verliebte Bart ist fassungslos. Zur gleichen Zeit schmeißt Marge ihren Homer raus und so kommt es, dass sich Bart und Homer ein Hotelzimmer teilen. Dort schmieden sie die abstrusesten Pläne, die Frauen wieder zurückzugewinnen.

\notiz{
\begin{itemize}
  \item Mary Spuckler ist nach eigener Aussage 13 Jahre alt.
  \item Mary Spuckler hat als Hand-Model für Chanel Nummer 5 gearbeitet.
\end{itemize}
}

\subsection{Eine Glatze macht noch keinen Kirk}
Bart experimentiert mit verheerenden Folgen mit Epoxidharz herum, denn urplötzlich sind die Haare von Milhouse so fürchterlich verklebt, dass er ihm eine Glatze rasieren muss. Mit der Glatze sieht Milhouse aus wie sein Vater Kirk. Die Jungs beschließen kurzerhand, sich das zunutze zu machen und Milhouse fortan als Kirk auftreten zu lassen. Komplikationen sind dabei jedoch programmiert.

\notiz{
\begin{itemize}
	\item Im Buchladen gibt es das Buch \glqq Magic School Bus: Otto's Trip\grqq .
	\item Homer schaut sich den E-Book-Reader Kuddle\index{Kuddle} an. Kuddle ist eine Anspielung auf Amazons Kindle\index{Kindle}.
\end{itemize}
}

\subsection{Der glamouröse Godfrey}
Als Homer in Grampas altem Lagerraum lauter Frauenkleider, Perücken und Magazine mit muskelbepackten Männern findet, folgern er und Marge, dass Grampa schwul ist. Sie konfrontieren ihn mit ihrer Vermutung, doch Grampa streitet alles ab. Er erzählt ihnen, dass er früher Profi-Wrestler war und als Glamouröser Godfrey aufgetreten ist. Zufällig stößt auch Mr. Burns auf Grampas Vergangenheit und will ihn überreden, noch einmal in den Ring zu steigen. Damit inspiriert er Bart, seinem Beispiel zu folgen und Bart provoziert bei allen möglichen Gelegenheiten die Menschen um ihn herum. Erst als Bart und Abe gemeinsam in den Ring steigen wollen, erkennt Abe, was er für ein schlechtes Vorbild war und kämpft ab da an fair.

\begin{itemize}
	\item Die Fernsehsendung Storage Battles ist eine Anspielung auf Storage Wars.
	\item Helen Lovejoy hat einen transsexuellen Cousin.
\end{itemize}

\subsection{Blauauge sei wachsam}
Bei einem gemeinsamen Frühstück provoziert Homer seinen Nachbarn Ned Flanders dermaßen, dass der zuschlägt. Das Resultat: Ein blaues Auge! Ned versucht alles, um seine Tat wiedergutzumachen, doch Homer gefällt sich in der Opferrolle. Währenddessen wird Lisa in der Schule von der neuen Vertretungslehrerin derart gemobbt, dass Homer und Marge eingreifen. Doch ihr Besuch beim Rektor macht alles nur noch schlimmer. Letztendlich findet Homer einen Weg, wie er Ned vergeben kann und Lisas Problem mit ihrer Lehrerin lösen kann. Dazu braucht er nur die Hilfe von Neds Frau Edna. Diese schickt Bart in ihre Klasse und wenig später ist die Vertretungslehrerin freiwillig bereit, die Grundschule zu verlasen. 

\begin{itemize}
	\item Lisa ist die Vertrauensschülerin der zweiten Jahrgangsstufe.
	\item Ned Flanders hat Harfespielen gelernt.
\end{itemize}

\subsection{Burns Begins}
Passend zu Ostern hat sich jemand einen Scherz erlaubt und in den Blasinstrumenten der Schulband Eier versteckt. Schon bei den ersten Tönen fliegen die Eier durch die Luft und verursachen eine riesen Sauerei. Der Verdacht fällt sofort auf Bart und nur mit Mühe kann Lisa die Lynchjustiz abwenden und einen gerechten Prozess für ihren Bruder fordern. Währenddessen wandelt Mr. Burns auf den Spuren von Batman und beschließt, im Fledermauskostüm auf Verbrecherjagd zu gehen. Lisa kann schließlich mit Mr. Burns den wahren Täter stellen: Hausmeister Willie.

\begin{itemize}
  \item Der Comicbuchverkäufer fordert für alle seine Hefte von Mr. Burns den Betrag der Lichtgeschwindigkeit in Dollar. Mr. Burns bietet im stattdessen den Betrag der Faraday-Kon\-stan\-te\index{Faraday-Konstante}\footnote{Die Faraday-Konstante $F$ ist die zur Reduktion einfach geladener Ionen notwendige elektrische Ladung $Q$. Ihr Wert beträgt nach derzeitiger Messgenauigkeit mit einer geschätzten Standardabweichung von $0,0021\, C\, {mol}^{-1}$ (siehe \cite{FaradayKonstante}):
		\[
F = 96 485,3365\;(21)\, \frac{C}{mol}\\
	\]
} an (94485,34 US\$).
	\item Matt Groening\index{Groening!Matt} ist im Gericht als Gerichtszeichner zu sehen. Obwohl er Linkshänder ist, zeichnet er Bart mit der rechten Hand.
\end{itemize}

\subsection{Was animierte Frauen wollen}
Bei Homer und Marge hängt der Haussegen schief, denn Homer hat sich bei einem romantischen Essen mal wieder wie ein Vollidiot benommen. Was für ihn nur eine Lappalie ist, ist für Marge der berühmte Tropfen, der das Fass zum Überlaufen bringt. Um seine Frau zu besänftigen, überrascht Homer sie mit einer kompletten SM-Ausrüstung. Auch Milhouse hat so seine Probleme mit Frauen: Er hat eine neue Taktik entwickelt, um Lisa endlich rumzukriegen.

\notiz{
\begin{itemize}
	\item Im Vorspann sind die beiden Schauspieler Bryan Cranston und Aaron Paul aus der Serie Breaking Bad\index{Breaking Bad} zu sehen, die vor dem Fernseher sitzen und die Simpsons schauen.
	\item Moe liest das Buch\glqq Fifty Shades of Grey\grqq .
\end{itemize}
}

\subsection{Apokalypse Springfield}\label{RABF11}
Ganz Springfield ist von einer Bettwanzenplage befallen, seitdem sich die Simpsons in New York eine neue Couch gekauft haben. Mit der Plage wachsen auch die Ängste der Bewohner und sie suchen Trost in der Kirche. Reverend Lovejoy ist mit der Situation vollkommen überfordert und bekommt Unterstützung von Priester Elijah Hooper\index{Hooper!Elijah}. Mit modernen Predigten und Twitter-Gottesdiensten schafft es Hooper, die Gemeinde zu beruhigen. So verlässt Reverend Lovejoy die Kirche und verkauft Whirlpools. Während der Rest der Stadt von Elijah Hooper begeistert ist, möchte Ned Reverend Lovejoy zurück.

\notiz{
\begin{itemize}
	\item Auf dem Kirchenschild ist zu lesen, dass Jesus auf Twitter über den Hashtag \#IDIEFORYOURSINS zu erreichen ist.
	\item Hans Maulwurf bietet auf der Kirchenanschlagtafel seine Dienste als Kinderbetreuer an.
	\item Lenny trägt Kontaktlinsen.
\end{itemize}
}

\subsection{Whiskey Businesssons}
Um Moe nach seinem Selbstmordversuch wieder aufzubauen, fahren Marge, Carl und Lenny mit ihm nach Capital City. Ein neuer Anzug soll sein Selbstbewusstsein stärken und siehe da: Moe wird prompt von zwei Geschäftsmännern entdeckt. Sie wollen ihn und seinen selbstgebrannten Bourbon groß rausbringen. Zuhause kümmert sich Bart um Grampa, der bei einem von Bart verursachten Unfall verletzt wurde. Doch so sehr sich Bart auch bemüht, Grampa will einfach nicht genesen.

\notiz{
\begin{itemize}
	\item Auf der Fernseherfestplatte der Simpsons sind 1328 Folgen der Serie Itchy \& Scratchy gespeichert.
	\item Auf der Getränkekarte in der Hochhausbar ist das Getränk \glqq The Flaming Moe\grqq\ zu sehen (siehe \glqq Das Erfolgsrezept\grqq , \ref{8F08}).
\end{itemize}
}

\subsection{Der fabelhafte Faker Boy}
Auf Anraten von Rektor Skinner schickt Marge Bart zur Musikschule. Der hat so gar keine Lust auf den Unterricht, doch dann trifft er auf die junge russische Klavierlehrerin Zhenya\index{Zhenya}. Bart ist sofort bis über beide Ohren verliebt und will ihr unbedingt dabei helfen, weitere Klavierschüler zu gewinnen. Doch dazu muss er auf einem Konzert unglaublich gut spielen oder betrügen. Währenddessen befindet sich Homer voll in der Midlife-Crisis: Er hat seine letzten zwei Haare verloren.

\notiz{
Rektor Skinner nimmt Flamenco-Stunden.
}

\subsection{Die Legende von Carl}\label{RABF14}
Homer, Moe, Carl und Lenny gewinnen 200.000 Dollar in der Lotterie. Sie veranstalten eine Feier in Moes Bar und malen sich schon aus, was sie alles mit ihrem Anteil des Gewinns anstellen können. Zu ihrem vollkommenen Glück fehlt nur noch Carl, der das Geld abholen wollte. Doch als der auch nach mehreren Stunden nicht auftaucht, werden die Männer plötzlich nervös. Carl ist nach Island zu seinen Adoptiveltern geflogen, um von dem Geld einen Beweis zu finden, dass vor 1.000 Jahren deren Vorfahren nicht die Invasoren in das Land ließen. Auf diese Art und Weise will er die Familienehre wieder herstellen.

\notiz{
\begin{itemize}
	\item Die vier Gewinnzahlen sind 3 (Moe), 19 (Lenny), 22 (Carl) und 69 (Homer).
	\item Homer, Moe und Lenny fliegen mit der Fluglinie Valhalla Air\index{Valhalla Air} nach Island.
	\item Homer fährt in Island einen Fjord Fjiesta\index{Fjord Fjiesta} als Mietauto.
	\item Carls Adoptiveltern wohnen in der Hjorleitsstr\ae ti\index{Hjorleitsstr\ae ti} 22.
	\item Bereits in Folge \glqq Nacht über Springfield\grqq\ (siehe \ref{EABF11}) gab Carl an, in Island aufgewachsen zu sein.
\end{itemize}
}

\subsection{Glück auf Schienen}
Der zehnte Hochzeitstag von Homer und Marge steht vor der Tür und soll etwas ganz Besonderes werden. Um Marge zu überraschen, will Homer die Kindereisenbahn, in der sie ihren ersten Hochzeitstag gefeiert haben, kaufen und restaurieren. Von Homers bisherigen Geschenken abgeschreckt, legt sich Marge dieses Mal weniger ins Zeug und besorgt ihrem Mann schnell online ein Geschenk. Dabei klickt sie versehentlich auf eine Dating-Seite und lernt den attraktiven Ben kennen.

\notiz{
\begin{itemize}
	\item Schulbusfahrer Otto arbeitet nebenbei als Babysitter.
	\item Die Dating-Seite heißt Sassy Madison (\nolinkurl{http://sassymadison.com}).
	\item Nelsons erste Worte waren \glqq Ha-Ha!\grqq ; daraufhin gibt ihm seine Mutter vom Sekt Moe et Chandon zu trinken.
	\item Fehler: Als Marge mit ihrem Auto an einer Polizeistreife vorbeifährt, fährt sie auf der linken Straßenseite.
\end{itemize}
}


\section{Staffel 25}

\subsection{Homerland}
Homer, Lenny und Carl sind auf einen Nuklear-Arbeiter-Kongress eingeladen, der in erster Linie darin besteht, dass sie sich ordentlich betrinken. Vor der Rückreise verschwindet Homer spurlos. Als er einige Zeit später wieder auftaucht, ist die Erleichterung bei seiner Familie riesig. Doch sein Verhalten ist merkwürdig: Er isst kein Schweinefleisch mehr und verzichtet auf Alkohol. Lisa hat den Verdacht, dass Homer von Islamisten einer Gehirnwäsche unterzogen wurde und einen Anschlag im Atomkraftwerk plane.

\notiz{
\begin{itemize}
	\item Barts \glqq Vitamine\grqq\ sind Focusyn, Blissium, Brozac und Crystal Math.
	\item Weil am 8. Dezember 2013 der deutsche Synchronsprecher von Mr. Burns, Reinhard Brock, verstorben ist, wird dieser nun von Kai Taschner gesprochen.
\end{itemize}
}

\subsection{Nichts bereuen}\label{RABF18}
Nach der Beerdigung von Chip Davis\index{Davis!Chip} beginnen die Einwohner von Springfield, über ihr Leben nachzudenken. Mr. Burns bedauert es, dass er seine Jugendliebe Lilah\index{Lilah} nie für sich gewinnen konnte. Marge glaubt, dass sie Fehler in der Schwangerschaft begangen hat, die Bart zu einem Flegel gemacht haben. Homer bereut, seine Apple-Aktien verkauft und stattdessen sein Geld in eine Bowlingkugel investiert zu haben. Und Kent Brockman ist betrübt, beim Lokalsender geblieben zu sein.

\notiz{
\begin{itemize}
	\item Diese Episode gewann einen Emmy.
	\item Als Marge mit Bart schwanger war, hörte sie die Musikgruppe KISS\index{KISS}.
	\item Chip Davis war u.\,a. der Duffman (1992 -- 1996 und 2008) und er schrieb Hans Maulwurfs Biographie \glqq Magnificant Bastard: The Live and Loves of Hans Moleman\grqq .
	\item Die Eltern des Hummelmanns sind bei einem Paintball-Unfall ums Leben gekommen.
	\item Krusty ist für den Reifenhaufen in Springfield verantwortlich. Er hat ihn auch aus Versehen angezündet.
	\item Es wird nicht mehr von Mapple-Produkten sondern von Apple-Produkten gesprochen; so ist auch auf einem Geschäft deutlich \glqq Apple Store\grqq\ zu lesen.
	\item Diese Folge ist Marcia Wallace (Ms. Krabappel) gewidmet, die am 25. Oktober 2013 verstarb.
\end{itemize}
}

\subsection{YOLO}
Homer wird klar, dass man nur einmal lebt. Plötzlich blickt er sehr kritisch auf sein Leben und ist frustriert über das, was er bisher daraus gemacht hat. Um ihren Mann aufzumuntern, lädt Marge Homers spanischen Brieffreund Eduardo\index{Eduardo} aus Kindertagen ein. Nun versuchen die Männer, ihre damals ausgetauschten Kindheitsträume endlich wahr zu machen. In der Zwischenzeit soll an der Schule ein Ehrenkodex helfen, damit die Schüler bei ihren Arbeiten nicht mehr betrügen.

\notiz{
\begin{itemize}
	\item Das Akronym YOLO steht für \glqq You only live once\grqq .
	\item Llewellyn Sinclair (siehe \glqq Bühne frei für Marge\grqq , \ref{8F18}) hat vor seiner Tätigkeit als Regisseur als Kellner gearbeitet.
	\item Homer hat sich seinen Jugendtraum, mit einem Feuerwehrauto zu fahren, bereits in der Episode \glqq Brand und Beute\grqq\ (siehe \ref{JABF13}) erfüllt.
	\item Homer hat als Kind auf dem 52 1/2 Burns Boulevard in Springfield gewohnt.
	\item Rektor Skinner schaut sich in der Schule auf seinem Laptop die Webseite \url{http://www.japanesegirlswithanimalears.com} an.
\end{itemize}
}

\subsection{Homer Junior}
Homer bleibt mit einer hochschwangeren Frau im Aufzug stecken. Mit seiner Hilfe bekommt sie dort ihr Kind und nennt es aus Dankbarkeit Homer Junior. Homer entwickelt bald eine erstaunliche Zuneigung zu dem Kind. Irgendwann verbringt er mit ihm mehr Zeit als mit seiner eigenen Familie. Als Marge herausfindet, wo sich Homer nach der Arbeit immer herumtreibt, ist sie natürlich nicht begeistert. In der Zwischenzeit verschafft Lisa den Cheerleadern der Springfield Atoms (Atomettes\index{Atomettes}) höhere Gagen.

\notiz{
\begin{itemize}
	\item Carl wohnt in Zimmer 3A in den Plywood Height Apartments in der Bridge Street 189.
	\item Eine Schlagzeile im Springfield Shopper lautet \glqq Apple Buys Facebook On Ebay\grqq .
\end{itemize}
}

\subsection{Silly Simpsony}\label{SABF02}
Lisa hat eine neue Klassenkameradin namens Isabel\index{Isabel}. Die beiden Mädchen verstehen sich auf Anhieb und werden bald beste Freundinnen. Während eines gemeinsamen Referats über Franklin D. Roosevelt wird ihre Freundschaft jedoch auf eine harte Probe gestellt, als sich herausstellt, dass Isabel eine überzeugte Republikanerin ist. Die liberale Lisa ist vollkommen entsetzt. Kurz darauf steht die Wahl zum Klassensprecher an und zwischen den beiden entbrennt ein harter Konkurrenzkampf und Isabel wird von den Republikanern in Springfield um Mr. Burns unterstützt.

\notiz{
\begin{itemize}
	\item Der Vertretungslehrer Mr. Bergstrom\index{Bergstrom} aus der Episode \glqq Der Aushilfslehrer\grqq\ (siehe \ref{7F19}) ist bei den Simpsons zu sehen.
	\item Mr. Smithers hat eine Laktoseintoleranz.
	\item Mr. Burns wurde vom People-Magazin einmal zum Sexiest Man Alive gewählt.
	\item Unter den Zuschauern der Wahlveranstaltung im Jahre 2056 ist die Hypnosekröte aus Futurama\index{Futurama} zu sehen.
\end{itemize}
}

\subsection{Global Clowning}\label{SABF04}
Barts Schule plant einen spektakulären Ausflug an Board eines U-Bootes. Es dürfen aber nur brave Schüler mit und Rektor Skinner entscheidet darüber. Obwohl Bart sich ausnahmsweise vorbildlich benimmt, gibt Rektor Skinner ihm keinen der begehrten Plätze. Daraufhin schmieden Homer und Bart einen fiesen Racheplan. Krusty verkauft unterdessen die Rechte an seiner Show in die ganze Welt und schon bald sind seine Kopien erfolgreicher als er selbst.

\notiz{
\begin{itemize}
	\item Auf der Schülerliste sind u.\,a. folgende Namen zu lesen: John Jacob, Wendell Queasly, Caroline Quimby, Dolph Shapirio, Michael d'Amico, Ernie Hapablap, Sherri Mackleberry, Terry Mackleberry, Cosine Tangent, Victor Sangria und Lewis Clark.
	\item Hausmeister Willie gibt an, die Uhr nicht lesen zu können.
	\item In dem Video zur U-Boot-Fahrt ist der Schriftzug \grqq YVAN EHT NOIJ\grqq\ zu lesen (siehe \glqq Die sensationelle Pop-Gruppe\grqq , \ref{CABF12}).
	\item Krusty besitzt ein Bild des Malers Monet.
\end{itemize}
}

\subsection{Cinema Piratiso}
Bart bringt Homer auf den Geschmack von illegalen Film-Downloads. Als der mit seiner Beute eine öffentliche Filmvorführung macht, bekommt Marge Gewissensbisse und verrät ihren Mann. Ehe Homer sich versieht, wird er als Filmpirat angeklagt. Vor Gericht erreicht er aber einen Freispruch, indem er die anwesenden Hollywood-Größen überzeugt, seine Geschichte zu verfilmen. Was Raubkopien von diesem Film angeht, versteht er allerdings keinen Spaß.

\notiz{
\begin{itemize}
	\item Um Homer aus der schwedischen Botschaft zu treiben, spielen Judas Priest\index{Judas Priest} ihren Klassiker \glqq Breaking the law\grqq\ mit dem Text \glqq Respecting the law\grqq .
	\item Es wird behauptet, Judas Priest sei eine Death-Metal-Band.
	\item Der FBI-Agent schimpft über die Schweden als Friedenspreis verleihende Fischräucherer. Der Nobelpreis wird aber in Norwegen verliehen.
\end{itemize}
}


\subsection{Manga Love Story}\label{SABF03}
Ganz unverhofft taucht eine Frau im Leben des einsamen Comicbuchverkäufer auf: Die kleine Japanerin Kumiko\index{Kumiko}. Dank der Unterstützung von Marge und Homer erobert der alte Junggeselle schnell das Herz der Manga-Freundin und die beiden werden ein Paar. Da taucht plötzlich Kumikos Vater auf, um seine Tochter aus den Fängen des Comicbuchverkäufer zu retten. Um den Japaner zu besänftigen, greift Homer auf ein bewährtes Mittel zurück: Er geht mit ihm einen trinken.

\notiz{
\begin{itemize}
	\item Der Comicbuchverkäufer gibt an, einen Abschluss in Chemie zu besitzen.
	\item Der Twitter-Account des Radioactive Mans lautet: \nolinkurl{@Radioactiveman_01}.
\end{itemize}
}

\subsection{Freaks in der Manege}
Die Halloween-Folge besteht aus drei Geschichten:
\begin{itemize}
	\item \textbf{Der Fette mit Hut}\\ Homer ist \glqq der Fette mit Hut\grqq\ und hinterlässt auf seiner Tour durch Springfield einen Pfad der Zerstörung.
	\item \textbf{Doppelkopf}\\ Bart stellt sich beim Drachen steigen lassen so ungeschickt an, dass er sich mit der Schnur enthauptet. Um ihn zu retten, wird sein Kopf auf Lisas Schulter genäht. Natürlich führt die Tatsache, dass sich die Geschwister nun einen Körper teilen müssen zu Streit, der tödlich endet.
	\item \textbf{So ein Zirkus}\\ Homer als Kraftprotz und Marge als Trapezkünstlerin arbeiten im Wanderzirkus von Mr. Burns. Homer möchte Moe einen Smaragdring abjagen. Dazu soll Marge Moe heiraten und anschließend will er Moe umbringen.
\end{itemize}

\notiz{
Im Vorspann ist die Hypnosekröte\index{Hypnosekröte} aus Futurama zu sehen.
}

\subsection{Auf dänische Steine können Sie bauen}
Die Welt der Simpsons ist zu Legoland geworden: Alles besteht aus kleinen Plastiksteinen, auch die Simpsons selbst. Das hat viele Vorteile, zum Beispiel kann nichts kaputt gehen. Allerdings existiert die tolle neue Simpsons-Welt nur in Homers Kopf. Aus Angst, seine langsam erwachsen werdende Tochter Lisa zu verlieren, hat er sich eine eigene Realität geschaffen, was aber auch keine Lösung für sein Problem ist, wie er bald feststellt.

\notiz{
\begin{itemize}
	\item Diese Episode hat keine Eröffnungssequenz.
	\item Die Episode ist eine Anspielung auf den Kinofilm \glqq The Lego Movie\grqq .
\end{itemize}
}

\subsection{White Christmas Blues}\label{SABF01}
Das Wetter spielt verrückt: Weil es überall anders viel zu warm ist, ist Springfield der einzige Ort in den USA, an dem es weiße Weihnachten gibt. Da lassen natürlich die Touristen nicht lange auf sich warten und jeder in Springfield versucht, irgendwie an den Besuchern zu verdienen. Marge macht aus dem Haus der Simpsons ein Bed-and-Breakfast. Die Gäste verhalten sich jedoch immer rücksichtsloser, sodass Marge ausrastet.

\notiz{
\begin{itemize}
	\item Quimbys Gegenkandidat als Bürgermeister heißt Petrovichnyamilenkossarian\index{Petrovichnyamilenkossarian}.
	\item Es ist unter anderem die DVD \glqq Hitler's Christmas In Hell\grqq\ zu sehen.
	\item Bongo\index{Bongo}-Comics sind laut dem Comicbuchverkäufer nur zum Schlagen von Kindern geeignet.
\end{itemize}
}

\subsection{Enter the Matrix}\label{SABF06}
Mr. Burns verteilt High-Tech-Brillen an seine Mitarbeiter, wodurch er sie ausspionieren kann. Homer findet das neue Gerät toll, jedoch ändert sich seine Meinung darüber, als Marge es ausprobiert und Homer erfahren muss, dass sie regelmäßig einen Eheberater aufsucht. Unterdessen weigert sich Bart, Nelson einen Valentinskarte zu schenken, obwohl Bart jedem anderen Kind in seiner Klasse eine gegeben hat.

\notiz{
\begin{itemize}
	\item Oogle Goggles ist eine Anspielung auf Google Glass.
	\item Ooogle Goggles liefert folgende Erkenntnisse:
	\begin{itemize}
		\item Abraham Simpson hat sich in drei Kriegen jeweils in den Fuß geschossen.
		\item Carl hat einen IQ von 214.
	\end{itemize}
\end{itemize}
}

\subsection{Durch Diggs und dünn}
Bart lernt in der Schule einen Jungen namens Diggs\index{Diggs} kennen. Diggs ist ein wenig seltsam und hat ein außergewöhnliches Hobby, die Falknerei. Bart ist im Gegensatz zu seinen Mitschülern fasziniert von ihm und heftet sich an seine Fersen. Irgendwann glaubt Diggs, fliegen zu können und springt von einem Baum, das verheerende Folgen hat. Er überlebt zwar den Sturz, landet aber in einer psychiatrischen Anstalt. Das tut Barts Bewunderung für seinen Freund allerdings keinen Abbruch.

\notiz{
\begin{itemize}
	\item Der Couchgag ist von Sylvain Chomet\index{Chomet!Sylvain}.
	\item Auf Diggs Gips stehen u.\,a. die Namen Alan Turing und John Swartzwelder.
\end{itemize}
}

\subsection{Der Herr der Gene}
Lisa beschäftigt sich mit gentechnisch veränderten Lebensmitteln und kommt zum Schluss, dass die gar nicht so schlimm sind, wie alle immer behaupten. Zum Dank werden die Simpsons ins Forschungszentrum der Firma Monsamo\index{Monsamo} eingeladen. Dort wartet eine Riesenüberraschung: Der Chef ist Sideshow Bob. Weil er sich scheinbar gebessert hat, darf der Häftling während seiner Freigänge als Forscher arbeiten. Irgendwann erkennt Lisa, was er wirklich im Schilde führt. Währenddessen probiert Marge einer Kirchenjugendgruppe etwas über verantwortungsvolle Sexualität beizubringen, was aber misslingt.

\notiz{
\begin{itemize}
	\item Agnes Skinner hat die E-Mail-Adresse \nolinkurl{agnes@skinnersucks.com}.
	\item Laut Schild im Museum lautet Albert Einsteins richtiger Name Albert Brooks.
\end{itemize}
}

\subsection{Besuch der alten Herren}
Das Altersheim in Springfield hat seine Zulassung verloren, deshalb zieht Grampa mal wieder bei seinem Sohn ein. Marge bietet auch Jasper und dem alten Juden an, bei ihnen zu wohnen, da beide keine lebenden Verwandten mehr haben. Homer merkt bald, dass alt sein seiner Faulheit sehr entgegenkommen würde und er wird den Senioren immer ähnlicher. Erst als Bart von Rowdys gejagt wird, entdeckt Homer doch wieder einen Rest jugendlicher Energie in sich und steht seinem Sohn bei.

\notiz{
\begin{itemize}
	\item Homer sagt, er sei 38 Jahre alt.
	\item Dolph hat in der rechten Augenbraue einen Ring.
	\item Dr. Hibberts Vater wohnt im Altersheim.
	\item Die Stromrechnung beträgt 2467 Dollar. Die Zahl 2467 ist eine Primzahl.
\end{itemize}
}

\subsection{Malen nach Bezahlen}\label{SABF10}
Endlich geht Lisas Herzenswunsch in Erfüllung und sie bekommt ein Meerschweinchen. Sehr zum Ärger von Homer und Marge zerstört das Nagetier das Schiffsgemälde über dem Sofa. Bei einem Garagenflohmarkt der Van Houtens kaufen sie daraufhin ein altes Gemälde für 20 Dollar. Bald stellt sich heraus, dass es sich offensichtlich um das Werk des nicht ganz unbedeutenden Künstlers Johan Oldenveldt \index{Oldenveldt!Johan} handelt und der Schätzwert bei 100.000 Dollar liegt. Homer will das den Van Houtens jedoch verschweigen.

\notiz{
\begin{itemize}
	\item Lisa gibt an, noch nie ein Haustier gehabt zu haben. Die Katzen Snowball waren allerdings ihre Haustiere (siehe \glqq Häuptling Knock-A-Homer\grqq , \ref{FABF04}).
	\item Die Figur des Klaus Zieglers\index{Ziegler!Klaus} dürfte eine Anspielung auf den deutschen Kunstfälscher Wolfgang Beltracchi\index{Beltracchi!Wolfgang}\footnote{Wolfgang Beltracchi ist ein deutscher Maler und Kunstfälscher. Er wurde am 27. Oktober 2011 in einem der größten Kunstfälscherprozesse der Welt seit dem Ende des Zweiten Weltkriegs wegen gewerbsmäßigen Bandenbetrugs zu sechs Jahren Haft verurteilt. Insgesamt gehen Ermittler offenbar von einem Betrugsgewinn von 20 bis 50 Millionen Euro aus (siehe \cite{WikipediaBeltracchi}).} sein.
	\item Homer bietet Kirk kanadisches LeDuff\index{LeDuff}-Bier an.
\end{itemize}
}

\subsection{Homer, die Pfeife}
In der Schule sollen die Kinder etwas über ihre persönliche Helden erzählen. Lisa bereitet eine Rede über die Nobelpreisträgerin Marie Curie vor, muss aber feststellen, dass Martin, der vor ihr dran ist, dasselbe Thema hat und damit abräumt. Kurzerhand entscheidet sie sich -- durch Bart dazu inspiriert -- über Homer zu sprechen und das mit Erfolg, denn ihre Rede wird online gestellt und Homer ein Star. Bald wird auch die FIFA auf ihn aufmerksam und engagiert ihn als Schiedsrichter für die Fußball-WM in Brasilien.

\notiz{
\begin{itemize}
	\item Lisas Video \glqq Lisa's Hero\grqq\ hat 54371 Links. Die Zahl 54371 ist eine Primzahl.
	\item In der Episode findet das Endspiel zwischen Brasilien und Deutschland statt, das Deutschland mit 2:0 gewann. In Wirklichkeit standen sich im Finale Deutschland und Argentinien gegenüber. Deutschland gewann nach Verlängerung mit 1:0 Toren.
	\item Martin Princes Twitter-Account lautet \nolinkurl{@themartinprince}.
\end{itemize}
}

\subsection{Luca\$}\label{SABF12}
Lisa lernt Lucas kennen, einen Jungen, der ein bisschen zu dick, ein bisschen zu naiv und vielleicht auch ein kleines bisschen zu dumm ist. Nichtsdestotrotz hegt Lisa große Gefühle für ihn. Patty und Selma glauben, es läge daran, dass Töchter sich immer in jemanden verlieben, der ihrem Vater ähnlich ist. Marge ist alarmiert und sieht Lisas Zukunft gefährdet. Kurzerhand bittet sie Homer, sich anders zu benehmen, was zur Folge hat, dass er tödlich beleidigt ist.

\notiz{
\begin{itemize}
	\item Auf der Rückseite des Buches \glqq Drinking Games For Advanced Alcoholics\grqq\ ist der Autor mit der Gefangenennummer SABF12 zu sehen.
	\item Die Eröffnungssequenz ist eine Anlehnung an Minecraft\index{Minecraft}.
	\item Snake hat einen Sohn namens Jeremy\index{Jeremy}.
	\item Snake schenkt Bart eine gestohlene Playstatium-4-Spielekonsole.
\end{itemize}
}

\subsection{Vorwärts in die Zukunft}
Homer stirbt an einem Herzinfarkt. Prof. Frink kann den Verstorbenen jedoch klonen und ihm seine Persönlichkeit sowie seine Erinnerung einpflanzen, sodass ein völlig neuer Homer erschaffen wurde, der sich vom alten in nichts unterscheidet. In ferner Zukunft existiert Homer, der virtuell den Haushalt heimsucht, mittlerweile nur noch auf einem USB-Stick. Marge hat die Schnauze voll und wirft ihn raus. Der virtuelle Homer zieht schließlich beim erwachsenen Bart ein. Währenddessen stecken Lisa und Milhouse in einer Ehekrise. Als Milhouse jedoch von einem Zombie gebissen wird und dadurch mutiert, wird er unverwundbar und Lisa findet wieder Gefallen am ihm. Bart unterzieht sich einer Therapie, um über seine Exfreundin Jenda hinwegzukommen.

\notiz{
Auch 30 Jahre in der Zukunft ist der olmekischen Indianerkopf aus \glqq Der Lebensretter\grqq\ (siehe \ref{7F22}) im Keller der Simpsons zu sehen.
}

\subsection{Ihr Kinderlein kommet}\label{SABF14}
Bart hasst seine neue Kunstlehrerin so sehr, dass er beschließt, sie mithilfe eines Voodoo-Zaubers mit Magenschmerzen zu bestrafen. Doch irgendetwas läuft schief und die Lehrerin wird schwanger. Bald gilt Bart als Geheimtipp für Paare, die keine Kinder bekommen können. Auch Fat Tony wünscht sich Nachwuchs, allerdings nicht für sich selbst, sondern für sein geliebtes Rennpferd, das ihm ein Champion-Fohlen gebären soll. Eine schier unlösbare Aufgabe für Bart.

\notiz{
\begin{itemize}
	\item Shauna\index{Shauna} ist die Tochter von Oberschulrat Chalmers.
	\item Moes Vorschläge für Cocktails, die nach Comic-Figuren benannt sind: Nick Fury -- Agent of Schnapps, Sex in the Batmobile und Wolveriskey\index{Wolveriskey}.
	\item Die Bildersequenz im Abspann ist eine Anspielung auf die Serie Modern Family\index{Modern Family}.
\end{itemize}
}

\subsection{Ziemlich beste Freundin}\label{SABF15}
Marge ist deprimiert, weil sie keine echten Freunde hat. Das wiederum soll aber nicht für Lisa gelten. Marge überlegt fieberhaft, wie sie ihrer Tochter eine beste Freundin verschaffen könnte. Das Problem scheint sich von selbst zu lösen: Tumi\index{Tumi}, ein Mädchen aus Lisas Parallelklasse, hat genau dieselben schrägen Vorlieben und Hobbys wie Lisa. Bart kommt das äußerst verdächtig vor und er spioniert ihr nach. Wie sich herausstellt, wurde sie von Marge bezahlt, um Lisas Freundin zu werden.

\notiz{
\begin{itemize}
	\item Shauna arbeitet im Supermarkt Swapper Jack's.
	\item Jimbo (Gitarre), Kearney (Schlagzeug) und Dolph (Gesang und Gitarre) spielen in einer Band namens \glqq Ear Poison\index{Ear Poison}\grqq .
\end{itemize}
}

\subsection{Feigheit kommt vor dem Fall}\label{SABF18}
Die Ferien stehen vor der Tür und wie jedes Jahr steht am letzten Schultag das große Schulwettrennen an. Milhouse hat heimlich trainiert und liegt in Führung. Doch dann beschließen die Rowdys Jimbo, Dolph und Kearney, ihn aus dem Rennen zu holen, da ihr Wettbüro bei einem Außenseitersieg hohe Summen auszahlen müsste. Sie schicken Nelson los, der Milhouse an einer unbeobachteten Stelle abfängt und verprügelt. Bart, derzeit Zweitplatzierter, beobachtet die Szene, hilft aber Milhouse nicht und wird so zum Sieger des Rennens.

\notiz{
\begin{itemize}
	\item Hans Maulwurf arbeitet als Zeitungsreporter beim Springfield Shopper.
	\item Statt der üblichen Eröffnungssequenz sitzen die Simpsons und Matt Groening auf der Bühne einer Comic-Messe. Auf die Frage des Comic-Buchverkäufers, ob es einen weiteren Simpsons-Film geben wird, verschwinden bis auf Maggie alle von der Bühne.
\end{itemize}
}

\section{Staffel 26}

\subsection{Ein trauriger Clown}
Krusty nimmt an einem Roast teil, das stimmt ihn sehr traurig und mündet in einer tiefen Sinnkrise. Ein Gespräch mit seinem Vater soll ihm wieder auf die Beine helfen. Doch es wird alles noch schlimmer, als dieser während der Unterhaltung verstirbt. Krusty weiß nicht mehr ein noch aus. Derweilen ist Lisa um die Sicherheit ihres Vaters besorgt und versucht ihn, so gut wie möglich zu beschützen.

\notiz{
Die Episode ist Louis Castellaneta gewidmet.
}

\subsection{Wir kentern alle in einem Boot}
Bart und Homer liegen sich mal wieder in den Haaren. Am Esstisch entsteht ein so schlimmer Streit über Barts nicht aufgegessenem Brokkoli, dass Marge nur noch einen Ausweg sieht: Sie schickt die beiden auf ein Therapieschiff, auf dem sie zwangsverpflichtet werden. Bart mag das Leben auf See und er wird bald zum Fähnrich zur See befördert. Als ein Sturm aufzieht, liegt es an ihm, die Besatzung sicher zum Hafen zu bringen. Doch dafür braucht er Homers Hilfe, der ihn schließlich unterstützt. Unterdessen kümmert sich Marge um Homers Fantasie-Football-Ligamannschaft.

\notiz{
Moe gewann mit seiner Fantasie-Footballmannschaft \glqq Moeland Raiders\grqq\ von 2002 bis 2013 ununterbrochen die Pigskin Pals League.
}

\subsection{Super Franchise Me}\label{SABF19}
Als Ned herausfindet, dass Homer nicht nur seinen Strom, sondern auch eine Tiefkühltruhe von ihm geklaut hat, will er diese umgehend zurück haben. Mit den Bergen aus Fleisch aus der Truhe macht Marge daraufhin köstliche Sandwiches, die Lisa und Bart in der Schule verkaufen. Ein Franchise-Unternehmen wird auf Marge aufmerksam und prompt wird sie Unternehmerin mit ihrem eigenen Sandwich-Laden. Doch die alltägliche Arbeit wird zur Belastungsprobe für sie und ihre Familie. Die Lage spitzt sich zu, als ein weiterer Laden auf der anderen Straßenseite eröffnet.

\notiz{
\begin{itemize}
	\item Neds Tiefkühltruhe heißt \glqq Freezerino\index{Freezerino}\grqq\ und wurde im Ort Okily Dokahama\index{Okily Dokahama} in Japan hergestellt.
	\item Marge beschäftigt im Sandwich-Laden anfangs Gil, Shauna und den pickeligen Teenager. Auch Professor Frink bewirbt sich, er wird aber von Marge nicht eingestellt.
	\item Das Franchise-Unternehmen, für das Marge arbeitet, heißt Mother Hubbard's Sandwich Cupboard.
\end{itemize}
}

\subsection{Fracking, Freude, Eierkuchen}
Als Patty und Selma vorübergehend bei den Simpsons wohnen und im Badezimmer heimlich rauchen, explodiert dieses plötzlich und das Wasser steht in Flammen. Lisa ist sofort klar, dass der Grund dafür nur Fracking sein kann. Als alle Spuren zu Mr. Burns führen, ruft Lisa die Umweltaktivistin und Politikerin Maxine Lombard\index{Lombard!Maxine} auf den Plan. Homer soll in Mr. Burns Auftrag alle Bewohner der Evergreen Terrace überzeugen, die Mineralienrechte an Mr. Burns abzutreten. Alle sind damit einverstanden nur Marge nicht.

\notiz{
Lisa nutzt auf ihrem Tablett den Netzdienst Webflix\index{Webflix} -- eine Anspielung auf Netflix\index{Netflix}.
}

\subsection{Simpsorama}\label{SABF16}
Die Simpsons bekommen Besuch von ihren Kollegen aus Futurama\index{Futurama}. Bender\index{Bender}, Leela\index{Leela} und Professor Farnsworth\index{Farnsworth} berichten, dass eine Horde wildgewordener Monster Neu New York zerstört. Das Kuriose: Es wurde festgestellt, dass sie Homers und Marges DNS haben.

\notiz{
\begin{itemize}
	\item Bart hat am 23. Februar Geburtstag.
	\item Auf der Innenseite eines Deckels in Benders Kopf steht der kleine Fermatsche Satz geschrieben: $a^{p-1} \equiv 1\pmod p$
	\item In Springfield ist die Pizzeria Panucci's zu sehen. Vor der Pizzeria ist ein Hund zu sehen. Bevor Fry bei Planet Express zu arbeiten begann, arbeitete er als Pizzabote für Panucci's Pizzeria. Er hatte ebenfalls einen Hund.
\end{itemize}
}

\subsection{Fackeln im Sandsturm}
Einmal im Jahr findet der große Lehrertausch in Springfield statt. An diesem Tag werden die schlechtesten oder schlimmsten Lehrer an andere Schulen getauscht. Dieses Jahr muss Bart es mit dem autoritären und knallharten Mr. Jack Lassen\index{Lassen!Jack} aufnehmen. Nach einem von Barts Streichen schert Lassen ihm zur Strafe den halben Kopf. Daraufhin ist Barts Racheinstinkt geweckt und er will ihn beim Blazing-Guy-Festival blamieren.

\notiz{
\begin{itemize}
	\item Blazing Guy ist eine Anspielung auf Burning Man\footnote{Burning Man ist ein jährlich stattfindendes Festival im US-Bundesstaat Nevada in der Black Rock Desert.}
	\item Den kurzen Videofilm, den Chalmers zeigt, wurde von chalmskinn\index{chalmskinn} produziert.
\end{itemize}
}

\subsection{Covercraft}
Homer hat ein neues Hobby: Er spielt Bassgitarre. Marge findet heraus, dass auch die Männer ihrer Freundinnen einen Midlife-Crisis-Musiktick haben und bringt sie zu einer Garagen-Band zusammen. Die Männer covern Songs der Band Sungazer\index{Sungazer} und feiern mit Sänger Apu regionale Erfolge. Die echten Sungazer werden schließlich auf Apu aufmerksam und schon bald steht dieser auf den großen Bühnen der Welt -- sehr zu Homers Leidwesen.

\notiz{
\begin{itemize}
	\item Sideshow Mel feiert mit Covercraft seinen 45. Geburtstag.
	\item Covercraft besteht aus Apu (Gesang), Dr. Hibbert (Schlagzeug), Reverend Lovejoy (Gitarre), Homer (Bass) und Kirk van Houten (Keyboard und Tamburin).
\end{itemize}
}

\subsection{Der Mann, der als Dinner kam}\label{RABF15}
Die Simpsons machen einen Ausflug in einen Vergnügungspark. Nach einiger Zeit entdeckt Bart ein spannendes Fahrgeschäft. Was die Simpsons nicht ahnen, bei der Attraktion handelt es sich um eine echte Rakete, die sie direkt ins Weltall zu den Aliens Kang und Kodos bringt. Dort werden die Simpsons nicht nur in einem Zoo ausgestellt, sondern Homer soll auch für ein blutiges Ritual geopfert werden.

\notiz{
\begin{itemize}
	\item Auf Riegel 7 gilt Fortran\footnote{Fortran ist eine prozedurale und in ihrer neuesten Version zusätzlich eine objektorientierte Programmiersprache, die insbesondere für numerische Berechnungen in Wissenschaft, Technik und Forschung eingesetzt wird.}\index{Fortran} als die beste Programmiersprache.
	\item Im Abspann werden mehrere Star-Trek-Sequenzen gezeigt.
\end{itemize}
}

\subsection{Barts neuer bester Freund}\label{TABF05}
Der zweite Sicherheitschef Don Bookner\index{Bookner!Don} aus Sektor 7G im Atomkraftwerk wird in den Ruhestand versetzt und plötzlich muss Homer Tag und Nacht arbeiten. Marge schenkt ihm zur Entspannung Karten für den Zirkus. Der Hypnotiseur Sven Golly\index{Golly!Seven} verwandelt Homer im Geiste in einen Zehnjährigen. Wenig später wird der Hypnotiseur jedoch verhaftet und Homers geistlicher Zustand bleibt der eines Kindes -- sehr zur Freude von Bart, denn der hat plötzlich einen neuen besten Freund.

\notiz{
\begin{itemize}
	\item Der Regisseur Judd Apatow\index{Apatow!Judd} schrieb das Drehbuch zu dieser Folge bereits Anfang der 90er Jahre.
	\item Cletus erklärt unter Hypnose das Lemma von Zorn.
\end{itemize}
}

\subsection{Hölle, Tod und Geister}
Die Halloween-Folge besteht aus drei Geschichten:
\begin{itemize}
	\item \textbf{Schule ist die Hölle}\\ Bart entdeckt auf einem der Pulte in der Grundschule zufällig ein Portal zur Hölle und stellt fest, dass es dort eine Schule gibt. Ihm gefällt es dort so gut, dass er auf die Schule gehen will. Nachdem Marge und Homer zugestimmt haben. wird Bart dort ein guter Schüler und macht seinen Abschluss.
	\item \textbf{Uhrwerk Gelb}\\ Die Erlebnisse von Moe und seiner Gang erinnern an Stanley Kubricks Klassiker \glqq Uhrwerk Orange\grqq .
	\item \textbf{Die Anderen}\\ Im Haus der Simpsons spukt es. Wie sich herausstellt, handelt es sich bei den Geistern um ihre eigenen Ebenbilder aus der Vergangenheit.
\end{itemize}

\notiz{
Der Name des Produzenten Jeff Westerbrook ist in Form eines Linux-Befehls angegeben: \texttt{cat westerbrook > /dev/null}
}

\subsection{Der Weingeist der Weihnacht}\label{TABF03}
Homer wird auf dem Heimweg vom traurigen und einsamen Moe aufgehalten. Damit Homer ihn nicht gleich wieder stehen lässt, hat er die Uhr verstellt, sodass Homer erst gegen kurz vor Mitternacht zu Hause ist. Marge ist so wütend, dass sie ihn vor die Tür setzt. Homer wandert daraufhin durch die Straßen und hat einige wundersame Weihnachtsbegegnungen. Moe plagt indes das schlechte Gewissen und bringt Marge dazu, Homer zu vergeben.

\notiz{
\begin{itemize}
	\item Homers Führerschein gilt nur für den Weg zur Arbeit und zurück.
	\item Homers Mobilfunktelefonnummer lautet: 555 - 0001
	\item Fehler: Bevor Maggie den Umriss ihres Kopfes zeichnet, hält sich den Stift in der rechten Hand. Die Zeichnung erfolgt allerdings mit der linken Hand und anschließend legt sie den Stift aus der rechten Hand.
\end{itemize}
}

\subsection{Der Musk, der vom Himmel fiel}
Im Garten der Simpsons landet eine Rakete. Ihr entsteigt der geniale Erfinder und Unternehmer Elon Musk\index{Musk!Elon}. Er ist verzweifelt auf der Suche nach neuen Ideen und findet in Homer mit all seinen irren Einfällen den perfekten Quell der Inspiration, dem er nun auf Schritt und Tritt folgt. Im Atomkraftwerk überzeugt der berühmte Visionär Mr. Burns von einem neuen Energiekonzept. Musk kommt auf eine umweltfreundliche Energielösung. Als Mr. Burns erfährt, dass er dadurch Geld verliert, beendet er das Projekt.

\notiz{
Homer behauptet, PayPal\index{PayPal}\footnote{PayPal geht auf den Zusammenschluss von Confinity und X.com im März 2000 zurück. X.com\index{X.com} wurde von Elon Musk im März 1999 gegründet} war seine Idee.
}

\subsection{Fett ist fabelhaft}\label{TABF06}
Homer wird immer dicker. Als er es nicht einmal mehr schafft, von seinem Stuhl aufzustehen, um die von Lisa und Bart komponierte neue Stadthymne zu feiern, schickt Marge ihn zum Abnehmen zu den Anonymen Vielfressern. Stattdessen landet er jedoch bei der Gruppe \glqq Fett ist fabelhaft\grqq , deren Mitglieder stolz auf jedes Kilo am Körper sind. Homer ist hellauf begeistert! Doch dann stirbt Albert, der schwergewichtige Anführer der Gruppe, mit 23 Jahren an einem Herzinfarkt.

\notiz{
\begin{itemize}
	\item Hans Maulwurf war vier mal Bürgermeister von Springfield. Während seiner Amtszeit hatte er achtmal einen ausgeglichenen Haushalt.
	\item Weitere Mitglieder der Gruppe \glqq Fett ist fabelhaft\grqq sind u.\,a. Roy Snyder, Küchenhilfe Doris und der Comic-Buchverkäufer.
	\item Nicht trendige Hash-Tags auf Twitter sind u.\,a. \befehl{\#SpringfieldPride}, \befehl{\#DateNightAtMoe}, \befehl{\#KrustyNudePix}, \befehl{\#BringBackLeno} und \befehl{\#TrustFoxNews}.
\end{itemize}
}

\subsection{Driving Miss Marge}
Weil Homer keine Lust hat, den Chauffeur für seine Kinder zu spielen, flüchtet er sich in Moes Bar und gibt vor, viel zu betrunken zum Fahren zu sein. Daraufhin muss Marge den Job übernehmen. Anschließend beginnt sie, für eine Taxi-App zu arbeiten und so ihre Fahrdienste zu Geld zu machen. Doch die Kundschaft in Springfield geht ihr schnell auf die Nerven und außerdem haben es jetzt die echten Taxifahrer auf sie abgesehen. Unterdessen arbeitet Moe zwischenzeitlich als Supervisor von Sektor 7G im Atomkraftwerk.

\notiz{
\begin{itemize}
	\item Der Vorspann besteht aus einer grob pixeligen Version eines Jump-and-Run-Videospiels im Stile der 1980er Jahre. Dieser wurde von Paul Robertson und Ivan Dixon gestaltet. Die Musik stammt von Jeremy Dower.
	\item Die Simpsons verwenden noch einen Nadeldrucker.
\end{itemize}
}

\subsection{Ein Herz und eine Krone}\label{TABF08}
Während Mr. Burns mit einem nigerianischen König einen lukrativen Uran-Deal aushandelt, soll Homer sich um dessen Tochter, Prinzessin Nemi\index{Nemi}, kümmern. Als die beiden in Moes Bar auftauchen, hat die junge Schönheit Ärger am Hals: Moe glaubt, dass sie mit einem Kerl verwandt sein muss, der ihn als \glqq nigerianischer Prinz\grqq\ mit einer fiesen E-Mail-Abzocke hereingelegt hat. Nun soll Nemi Moe den Verlust ersetzen. Doch dann lässt sie ihren unwiderstehlichen Charme spielen.

\notiz{
\begin{itemize}
	\item Die Folge ist Leonard \glqq Mr. Spock\grqq\ Nimoy\index{Nimoy!Leonard} gewidmet, der kurz vor der US-Ausstrahlung am 27.2.2015 starb.
	\item Mr. Burns neuer Nachbar ist Milliardär Richard Branson\index{Branson!Richard}.
	\item Moes Bar ist von der Gesundheitsbehörde als bedenklich eingestuft worden.
	\item Dieses Folge gewann wegen Hank Azarias Sprechleistung einen Emmy.
\end{itemize}
}

\subsection{Sky-Polizei}\label{TABF09}
Versehentlich wird Chief Wiggum mit einem Raketenrucksack beliefert. Für ihn ein Zeichen: Endlich kann er seiner wahren Berufung folgen und ein heldenhafter Sky-Polizist sein! Doch leider sind seine Flugkünste nicht annähernd so gut, wie er denkt, dadurch kommt es zu einem Unfall, bei dem die Springfielder Kirche in in starke Mitleidenschaft gezogen wird. Um Geld für die Reparatur zu besorgen, schließen sich Helen, Timothy, Sideshow Mel, Ned, Agnes und Marge zusammen und versuchen ihr Glück im Casino mit Kartenzählen beim Black Jack.

\notiz{
\begin{itemize}
  \item Diese Folge enthält keinen Vorspann.
  \item Apu gibt an, vom Massachusetts Institute of Technology (MIT) geflogen zu sein.
\end{itemize}
}

\subsection{Warten auf Duffman}\label{TABF10}
Der Schauspieler, der die berühmte Werbefigur Duffman spielt, ist krank und hängt deshalb seinen Job an den Nagel. Bei einem TV-Casting wird ein würdiger Nachfolger gesucht und Homer gewinnt den Wettbewerb. Doch sein Traumjob verliert für ihn ganz schnell seinen Reiz, als er erfährt, dass er selbst kein Bier trinken darf und sieht, wie viel Leid Bier in der Welt anrichtet. Nun macht Homer als Duffman nur noch Werbung für alkoholfreies Bier, das kommt nicht bei allen gut an.

\notiz{Der Duffman heißt eigentlich Barry Huffman\index{Huffman!Barry}.}

\subsection{Marge will's wissen}
In Springfield gab es einen verheerenden Bulldozer-Unfall, der einen Teil der Stadt dem Erdboden gleich gemacht hat. Chief Wiggum hat als Übeltäter niemand anderen als Bart in Verdacht, doch der Junge beteuert seine Unschuld. Marge sieht sich gezwungen, zu einer besonderen Maßnahme zu greifen: Sie will ihrem Sohn so lange auf Schritt und Tritt folgen, bis er sich zu seiner Schuld bekennt. In der Zwischenzeit legt sich Ned Flanders eine Hündin namens Baz\index{Baz} zu.

\notiz{
\begin{itemize}
	\item Neds Hündin heißt eigentlich Mahershalalhashbaz\index{Mahershalalhashbaz}. Der Name stammt aus dem Buch Jessiah des Alten Testaments (Jessiah 8.1).
	\item Homer war in der nordamerikanischen Sumo-Liga aktiv.
	\item Springfield feiert das 50-jährige Jubiläum des Springfielder Schriftzuges.
\end{itemize}
}

\subsection{Fight Club}
Homer findet in seiner Jackentasche eine sechs Jahre alte Filmrolle, die Moe für ihn entwickelt. Doch die Bilder rufen Erschütterndes in Erinnerung: Bart und Lisa waren als kleine Kinder furchtbar zerstritten -- so sehr, dass Marge und Homer in ihrer Verzweiflung Hilfe bei einer Therapeutin suchten. Doch dann gingen die zwei kleinen Streithähne verloren. Auf einer abenteuerlichen Reise durch Springfield rettete Bart Lisas Leben und plötzlich war alles anders.

\notiz{
\begin{itemize}
	\item Lisa schreibt statt Bart den Tafelgag.
	\item Lisa spielt Harfe statt Saxophon.
	\item Die Telefonnummer der Simpsons lautete: 555 - 0113.
\end{itemize}
}

\subsection{Air Force Grampa}\label{TABF13}
Homer ist der Meinung, Grampa sei alt und nutzlos und bringt damit dessen alte Air-Force-Kameraden gegen sich auf. Die Soldaten beschließen, ihre Waffen auszupacken und Homer eine Lektion zu erteilen. Sie zwingen ihn, mit seinem Vater etwas zu unternehmen und dem alten Mann echte Liebe zu zeigen. Unterdessen ist bei Milhouse seine Cousine Annika\index{van Houten!Annika} aus den Niederlanden zu Besuch. Um ihr zu beeindrucken, beginnt er E-Zigaretten zu rauchen.

\notiz{
\begin{itemize}
	\item Mona Simpson lernte Abe in einer Air-Force-Bar kennen. Sie war dort als Kellnerin beschäftigt.
	\item Milhouse behauptet, seine Mutter sei die Cousine seines Vaters.
\end{itemize}
}

\subsection{Das Schweigen der Rowdys}\label{TABF15}
Nachdem ein paar Rowdys sich Bart beim Schultanz als Opfer auserkoren haben, setzt Marge durch, dass in Springfield ein Anti-Mobbing-Gesetz eingeführt wird. Leider geht der Schuss nach hinten los: Die Söhne von Ned Flanders zeigen prompt Homer an, weil er ihren Vater gemobbt haben soll. Als Strafe muss Homer eine Therapie machen. Diese zeigt eine wunderliche Wirkung.

\notiz{
Der Vater von Oberschulrat Chalmers war Psychologe.
}

\subsection{Eins, zwei oder drei}
Lisa muss mit dem Mathe-Team ihrer Schule eine bittere Niederlage einstecken. Die Gegner sind zwar nicht schlauer, aber technisch besser ausgerüstet. Jetzt gibt es keine Ausrede mehr: Springfield muss komplett digitalisiert werden! Gesagt, getan. Doch bald lässt ein Server-Crash die schöne neue Welt zusammenbrechen. Zum Glück erinnert sich wenigstens Lisa noch daran, wie das Leben vor der digitalen Revolution funktionierte, dabei wird die Schule in eine Waldorfschule umgewandelt. Mit dem Team der Mathlethen unter Trainer Willie treten Lisa und andere Schüler erneut gegen die Waverly Hills Grundschule an.

\notiz{
\begin{itemize}
	\item Nach dem die Simpsons auf der Couch sitzen, kracht das Raumschiff von \glqq Rick \& Morty\grqq\ in das Wohnzimmer der Simpsons.
	\item Norbert Gastell ist das letzte Mal als Synchronsprecher von Homer Simpson zu hören.
\end{itemize}
}

\section{Staffel 27}

\subsection{Traumwelten}
Dr. Hibbert diagnostiziert bei Homer Narkolepsie, eine Schlafkrankheit. Homer und Marge streiten darüber so heftig, dass es zur Trennung kommt. Kurz darauf lernt Homer eine etwas verrückte junge Apothekerin kennen und fängt etwas mit ihr an. Marge hingegen lässt sich auf einen gut situierten älteren Mann ein, der, wie sich herausstellt, der Vater von Homers neuer Flamme ist. Am Ende stellt sich heraus, es war alles nur Homers Traum, in Homers Traum, in einem Traum von Marge.

\notiz{
\begin{itemize}
	\item In dieser Folge ist erstmalig Christoph Jablonka als Homers deutsche Synchronstimme zu hören. Laut Tom Schneider, der bei ProSieben der zuständige Redakteur ist, hat man sich bewusst für eine Stimme entschieden, welche der Stimme von Norbert Gastell nahe kommt (siehe \cite{NeueStimme}).
	\item Ned Flanders hat ebenfalls eine neue Synchronstimme und zwar Claus-Peter Damitz.
	\item Spiegel Online kritisiert, dass der Charakter des Homers nicht mehr zu retten sei. Er wäre ein liebenswerter Verlierer gewesen, der jetzt nur noch ein rücksichtsloser Depp sei (siehe \cite{NeueStimmeNichtsDahinter}).
\end{itemize}
}

\subsection{Grilling Homer}
Die Simpsons merken, dass sie wegen der alten Waschmaschine stinken. Homer soll mit dem letzten Ersparten eine neue Waschmaschine kaufen. Als er jedoch mit einem Smoker, der köstliches Fleisch grillt, nach Hause kommt, werden die Simpsons so beliebt wie noch nie. Sogar das Fernsehen hat Interesse an einem Kochduell. Obwohl ihnen der Smoker gestohlen wird, können Sie das Kochduell gegen den Fernsehkoch Scottie Boom gewinnen.


\notiz{
\begin{itemize}
	\item Im Keller der Simpsons ist Bender aus Futurama zu sehen.
	\item Nelson spielt das Spiel \glqq Clash of Castles\grqq .
	\item Fehler: In der deutschen Fassung nennt Alton Brown den Fernsehkoch einmal Scottie Brown statt Scottie Boom.
\end{itemize}
}

\subsection{Grauer Dunst}\label{TABF19}
Als die Simpsons bei der Geburtstagsfeier von Marges Mutter erfahren, dass Pattys und Selmas Vater an Lungenkrebs gestorben ist, beschließen die beiden leidenschaftlichen Nikotin-Schwestern, dem Rauchen abzuschwören. Doch dann findet Patty heraus, dass Selma heimlich weiter raucht; sie ist schwer enttäuscht und zieht vorübergehend bei Homer und Marge ein. In der Zwischenzeit rettet Maggie mit ihren Tierfreunden eine Beutelratte aus den Fängen von Cletus.

\notiz{
\begin{itemize}
	\item Disco Stu hat ein Profil auf Tinder\index{Tinder}.
	\item Marges Mutter wird 80 Jahre alt.
	\item Duffman hat einen Papagei namens Hoppy\index{Hoppy}.
	\item Spiderschwein ist in der Folge zu sehen.
\end{itemize}
}

\subsection{Freundin mit gewissen Vorzügen}
Lisa lernt ein neues Mädchen namens Harper Jambowski\index{Jambowski!Harper} kennen und freundet sich sogar mit ihr an. Als Harpers Vater Mike\index{Jambowski!Mike} Lisa und Homer zu einem Konzert einlädt, wird schnell deutlich, wie reich Mike ist. Homer möchte ebenfalls von dessen Reichtum profitieren und merkt dabei gar nicht, wie schlecht es um die Freundschaft der beiden Mädchen eigentlich steht.

\notiz{
\begin{itemize}
	\item Lenny hat eine Nichte.
	\item Während des Auftritts der Boy-Band ist der Schriftzug \glqq YVAN EHT NIOJ\grqq\ zu lesen (siehe \glqq Die sensationelle Pop-Gruppe\grqq , \ref{CABF12}).
\end{itemize}
}

\subsection{Lisa on Broadway}
Homer verliert bei einem Pokerabend 5.000 Dollar an die Mitspielerin und alte Broadway-Größe Laney Fontaine. Um sie von seiner Armut zu überzeugen, lädt Homer Laney zum Abendessen ein. Dabei entdeckt Laney Lisas Talent für das Saxofon-Spielen und bietet ihr an, sie mit auf ihre einmonatige Tour zu nehmen. Bei der Show in New York spielt Lisa jedoch unglaublich gut und Marge sieht, dass Lisa für die Bühne geboren ist. Aber Laney wirft Lisa raus: Angeblich weil Lisa mehr Applaus als sie selbst bekommen hat.

\notiz{
\begin{itemize}
\item Ned hat einen Cousin namens Jakob, der den Amischen\index{Amishe} angehört.
	\item In dieser Folge scheint Lenny nicht Poker spielen zu können, obwohl er in der Episode \glqq Ehegeheimnisse\grqq\ (siehe \ref{1F20}) bereits mit Homer Poker spielte.
	\item Obwohl Homer in der Folge \glqq Homer und New York\grqq\ (siehe \ref{4F22}) gesagt, dass er nie mehr nach New York zurückkehren will, fährt er bereitwillig an den Broadway mit.
\end{itemize}
}

\subsection{Wege zum Ruhm}
Lisa verliert bei einem Erfinderwettbewerb in ihrer Schule. Als sie zufällig von einer großen Wissenschaftlerin des vorletzten Jahrhunderts erfährt, begibt sie sich auf deren Spuren. Denn auch diese wurde wie Lisa als Frau und Erfinderin nicht ernst genommen. Währenddessen liest Bart einigen Jungs aus dem Tagebuch eines Mörders vor, das er in der Anstalt gefunden hat. Als einige Seiten des Buchs gefunden werden, glauben Marge und Homer, dass Bart ein Soziopath ist. Der bekommt davon mit und tut so, als wäre er wirklich ein Soziopath. Aber Marge und Homer stecken ihn in eine Anstalt für soziopathische Kinder. Dort muss Bart Drohnen steuern, die in Kriegsgebieten auf Andere schießen; davon entsetzt kommt er nach Hause.

\notiz{
\begin{itemize}
	\item Amelie Vanderbuckle\index{Vanderbuckle!Amelie} hat die erste mechanische Rechenmaschine erfunden. Sie ist eine Anspielung auf de vermutlich erste Programmiererin der Welt: Ada Lovelace\footnote{Ada Lovelace war eine britische Mathematikerin. Für einen nie fertiggestellten mechanischen Computer, die Analytical Engine, veröffentlichte sie als Erste ein komplexes Programm.}.
	\item Nelsons Mutter arbeitet im Knockers\index{Knockers} als Kellnerin.
\end{itemize}
}

\subsection{Barthood}\label{VABF02}
In dieser Folge wird die Lebensgeschichte von Bart erzählt: Seit Lisa auf der Welt ist, fühlt Bart sich vernachlässigt, denn seine Schwester ist in vielen Dingen einfach immer besser gewesen als er. Als sie junge Erwachsene sind, kommt es zur großen Konfrontation zwischen den beiden. Lisa will Bart allerdings aufbauen und gibt ihm deshalb den entscheidenden Tipp für ein erfolgreiches Leben.

\notiz{
\begin{itemize}
	\item Prof. Frink arbeitet als Hauslehrer von Bart.
	\item Lisa malt das Bild, das im Wohnzimmer der Simpsons über der Couch hängt. In der Folge \glqq Die Trillion-Dollar-Note\grqq\ (siehe \ref{5F14}) gibt allerdings Marge an, das Bild gemalt zu haben.
	\item Milhouse will Flugbegleiter werden.
	\item Ralph ist in die Armee eingetreten.
\end{itemize}
}

\subsection{Conrad}\label{VABF03}
Marge postet versehentlich etwas online, was zur Folge hat, dass Homer seinen Job im Kernkraftwerk verliert. Als er daraufhin seinen alten Job als Tellerwäscher in einem griechischen Restaurant wieder annimmt, entdeckt er seine Vorliebe für das Leben als griechischer Mann. Derweil entwickelt Lisa die sehr erfolgreiche neue App Conrad\index{Conrad}, die Online-Nutzer auf die Konsequenzen ihres Handelns hinweist.

\notiz{
\begin{itemize}
  \item Homer bestellt sich mit Siri ein Bier bei Amazon\index{Amazon}, dass mittels einer Drohne geliefert wird.
  \item Die Programmierlehrerin hat auf dem rechten Oberarm \lstinline{C:\>} und auf dem linken Oberarm \lstinline{sudo} tätowiert.
\end{itemize}
}

\subsection{Die Milch macht's}
Apu preist Homer eine genmanipulierte Milch an, die besonders gut für Kinder sein soll. Als bei Lisa plötzlich Pickel sprießen und sich auf Barts Oberlippe ein Flaum entwickelt, stellt sich jedoch heraus, dass das vermeintlich gesunde Produkt den Nachwuchs der Simpsons schnurstracks in die Pubertät katapultiert hat. Während die Hormone Barts Gefühle für seine neue Lehrerin befeuern, leidet bei Lisa das Selbstbewusstsein. Aber dadurch, dass sie Schminke trägt, wird sie beliebt. Bei einer Drittklässlerparty entdeckt Lisa, dass sie keine Pickel mehr hat und wird wieder unbeliebt. Bart fällt der Schnurrbart aus und er hört auf, sich für Carol zu interessieren. Diese trennt sich von Skinner, als sie seine Mutter kennenlernt und geht zurück zur US-Army nach Afghanistan.

\notiz{
In Skinners Büro hängt ein Portrait des ehemaligen US-Präsidenten Richard Nixons.
}

\subsection{Horror-Halloween}
Homer dekoriert wie jedes Jahr zu Halloween das Haus. Lisa freut sich auf die Halloween-Horror-Nacht im Krustyland, an der sie zum ersten Mal teilnehmen darf. Dort gruselt sie sich allerdings so sehr, dass sie ein Trauma davonträgt. Alles, was nur entfernt an Halloween erinnert, jagt ihr einen Heidenschreck ein. Marge beschließt, das Haus halloweenfrei zu gestalten und so muss Homer die komplette Dekoration sehr zu Barts Ärger wieder abräumen. Homer sorgte allerdings einen Tag vorher dafür, dass drei Gelegenheitsarbeiter ihren Job verloren haben. Diese versuchen nun, sich an Homer zu rächen.

\notiz{
\begin{itemize}
	\item Lewis geht als Austin Powers\index{Powers!Austin} verkleidet in die Schule.
	\item Marges Auto hat das Kennzeichen EP7G08.
\end{itemize}
}

\subsection{Killer und Zilla}
\begin{itemize}
	\item \textbf{Gesucht: Tot und dann lebendig}\\ Tingeltangel-Bob tötet Bart. Bald wird ihm klar, dass er mit dem Mord an Bart seinen Lebensinhalt ausgelöscht hat. Doch Bobs einziger Spaß im Leben war, zu versuchen, Bart zu töten. Er baut eine Maschine, die Bart wieder zurück ins Leben bringt und er tötet Bart immer wieder.
	\item \textbf{Homerzilla}\\ In Japan schickt Grampasan jeden Tag einen Donut ins Meer, um ein Meeresmonster zu besänftigen. Alle halten ihn für verrückt. Plötzlich stirbt er und niemand schickt mehr die Donuts zu Homerzilla. Er erwacht und zertrampelt das Dorf. Es handelt sich allerdings nur um einen Film, von dem ein Remake gedreht wird. Es wird aber nur eine Kinokarte verkauft. Als die Filmrollen ins Meer gekippt werden, erwacht der echte Homerzilla.
	\item \textbf{Telepfade zum Erfolg}\\ Bei einer Wanderung stoßen Lisa, Milhouse und Bart auf ein Erdloch, das mit radioaktivem Schleim gefüllt ist. Dieser explodiert und plötzlich verfügen Lisa und Milhouse über telekinetische Fähigkeiten. Als Milhouse die Macht zu Kopf steigt und durchdreht, fällt er plötzlich tot um. Es war Maggie, die durch das Nuckeln an einem radioaktiven Bolzen, ebenfalls Superkräfte erhalten hat.
\end{itemize}

\notiz{
\begin{itemize}
	\item Auf Bobs Fiendbook-Seite\index{Fiendbook} ist zu lesen, dass Russ Cargill (siehe \glqq Simpsons -- Der Film\grqq , \ref{SimpsonsFilm}) 2017 das zehn-jährige Kuppel-Wiedersehenstreffen veranstaltet.
	\item Die Tätowierung \glqq Die Bart Die\grqq\ aus der Folge \glqq Am Kap der Angst\grqq\ (siehe \ref{9F22}) ist auf Bobs Brust zu sehen.
	\item Ein Werbeschild für Meister Glanz (siehe \glqq Marge als Seelsorgerin\grqq, \ref{4F18}) ist zu sehen.
\end{itemize}
}

\subsection{Apucalypse Now}\label{VABF05}
Nachdem der Kwik-E-Mart durch die Unfähigkeit von Chief Wiggum und einen folgenschweren Unfall dem Erdboden gleichgemacht wurde, zieht sich Apus Bruder Sanjay aus dem Geschäft zurück und überlässt seine Anteile seinem Sohn Jamshed, der sich jetzt Jay nennt. Jay baut komplett um und ein hipper Bio-Markt entsteht -- sehr zum Missfallen von Apu, der mittlerweile kein Mitspracherecht mehr hat und von seinem Neffen entlassen wird. Völlig verzweifelt bittet er daraufhin Bart um Hilfe.

\notiz{
\begin{itemize}
	\item Krusty, Rektor Skinner, Sideshow Mel, Moe, Carl und Cletus sind Mitglieder der freiwilligen Feuerwehr in Springfield.
	\item Apu behauptet, im Spielfilm \glqq Der Tempel des Todes\grqq\ von Steven Spielberg mitgespielt zu haben.
\end{itemize}
}

\subsection{Liebe liegt in der N2-O2-Ar-CO2-Ne-He-CH4}\label{VABF07}
Professor Frink leidet darunter alleine zu sein und wird von Homer darauf hingewiesen, dass möglicherweise seine Stimme an seinem Single-Dasein Schuld ist. Wissenschaftlich will er sich diesem Problem stellen und eine Lösung finden. Überwältigt von den Reaktionen erschafft er einen Algorithmus, der perfekt die einsamen Frauen und Männer Springfields zusammenbringen soll. Währenddessen besuchen Marge, Bart und Lisa zum Valentinstag Grampa im Altersheim und versuchen, die Senioren vor einer halluzinogenen Droge zu retten, welche die Rentner nochmal die glücklichsten Momente ihres Lebens durchleben lässt.

\notiz{
\begin{itemize}
	\item Prof. Frink arbeitet manchmal als Berater im Atomkraftwerk.
	\item Cookie Kwan hat eine Cousine namens Nookie Kwan, welche die Nummer Eins auf der Ostseite ist.
	\item Diese Folge ist David Kavner, dem Vater der Synchronsprecherin Julie Kavner (u.\,a. Originalstimme von Marge), gewidmet.
\end{itemize}
}

\subsection{Die Frau im Schrank}
Bart schubst versehentlich den Einkaufswagen einer netten obdachlosen Frau namens Hettie in den Fluss. Als Gegenleistung erlaubt er ihr, für wenig Geld in seinem Kleiderschrank unterzukommen. Als seine Schwester Lisa herausfindet, dass Hettie eine begnadete Folk-Sängerin ist, plant sie ein Konzert mit ihr. Währenddessen mauert Homer, beim Versuch sein handwerkliches Geschick zu beweisen, die Katze in der Wand ein.

\notiz{
\begin{itemize}
  \item Der ehemalige Musiklehrer Largo hasst mittlerweile die Musik.
  \item In der Entzugsklinik sind u.\,a. Krusty, Carl, Lenny, Barney, Disco Stu, Lindsey Naegle, Bernice Hibbert und Captain McCallister.
  \item In der Entzugsklinik arbeitet Moe Szyslak als Krankenpfleger.
\end{itemize}
}

\subsection{Lisa und das liebe Vieh}\label{VABF08}
Durch eine Rettungstat fühlt Lisa sich dazu berufen, hilflosen Tieren zu helfen. Sie kümmert sich um den Klassenhamster und wird Praktikantin beim Tierarzt. Schnell wächst ihr diese Verantwortung allerdings über den Kopf und sie bekommt die Konsequenzen zu spüren. Derweil nimmt Marge einen Job als Tatortreinigerin an und hat schon bald mit psychischen Folgen zu kämpfen, die sie deutlich unterschätzt hat.

\notiz{
\begin{itemize}
  \item Der Couchgag \glqq Roomance\grqq\ stammt vom US-Trickfilmer Bill Plympton\index{Plympton!Bill}.
  \item Otto arbeitet als Bademeister im Erlebnisbad.
\end{itemize}
}

\subsection{Die Marge-Ianer}
Lisa meldet sich für ein Marsbesiedelungsprogramm an, bei der sie in zehn Jahren auf den Mars fliegen soll. Marge ist überhaupt nicht begeistert von dieser Idee und versucht, Lisa davon abzubringen. Homers Idee, umgekehrte Psychologie zu nutzen und auf die Begeisterungswelle aufzuspringen, um sie davon abzubringen, entwickelt sich in eine völlig falsche Richtung. Früher oder später kommt allerdings doch alles anders und vor allem schneller als erwartet.

\notiz{
\begin{itemize}
  \item Rektor Skinner hat hohe Schulden.
  \item Disco Stu hatte einen Hund.
  \item Disco Stu trägt eine Erwachsenenzahnspange.
  \item Marge besuchte einmal alleine ein Rockkonzert der Band Loverboy\index{Loverboy}.
\end{itemize}
}

\subsection{Ein Käfig voller Smithers}
Smithers ist wegen seiner unerwiderten Liebe zu Mister Burns frustriert und völlig am Boden. Gegenüber den Mitarbeitern des Kernkraftwerks ist er unausstehlich, daher macht sich Homer auf die Suche nach einem neuen Lover für ihn. Währenddessen bekommt Lisa die weibliche Hauptrolle im Schultheaterstück \glqq Casablanca\grqq\ und der eifersüchtige Milhouse tut alles, um an ihrer Seite spielen zu können.

\notiz{
\begin{itemize}
  \item Lennys Schwester ist verstorben.
  \item Es ist eine Karte zu sehen, die Springfield in der Mitte der USA zeigt.
  \item Der vorbereitete Scheck von Mr. Burns an Waylon Smithers hat die Nummer 7561, die auch eine Primzahl ist.
  \item Homer sucht in der Smart-Phone-App Grinder\index{Grinder} nach einer Bekanntschaft für Waylon Smithers.
  \item Es ist Werbung er Webseite \href{http://www.fox.com/the-simpsons/article/ashley-madden}{www.ashley-madden.com} zu sehen.
\end{itemize}
}

\subsection{Die Jazz-Krise}\label{VABF11}
Lisa ist von ihrer Mutter tief enttäuscht, als sie nach dem Vorspielen ihrer neuen Komposition hört, wie diese zu Homer sagt, dass Lisas Jazz-Musik eigentlich nicht ihr Ding ist. Mit aller Kraft und einem aufwendig geplanten Ausflug nach Capital City will Marge wieder alles gut machen und sich mit Lisa versöhnen. Diese lässt Marge allerdings ihren Groll deutlich spüren und ist gar nicht so leicht zu besänftigen. In der Zwischenzeit ist Bart alles andere als glücklich, dass niemand mehr auf seine Streiche reinfällt.


\notiz{
\begin{itemize}
  \item Im Vorspann schreibt beim Tafelgag Rektor Skinner statt Bart an der Tafel.
  \item Lisa hat ihre Perlenkette von Marge zum ersten Schultag geschenkt bekommen.
  \item Hans Maulwurf stellt fest, dass er eigentlich keine Brille braucht.
\end{itemize}
}

\subsection{Fland Canyon}
Als Maggie nicht einschlafen kann, erzählt Homer ihr eine Gute-Nacht-Geschichte aus der Zeit, als die Simpsons gemeinsam mit Familie Flanders in den Grand Canyon gefahren sind. Ned Flanders hat den Ausflug von der Kirche gewonnen. Reverend Lovejoy beschließt kurzerhand, dass dies eine gute Möglichkeit sei, das angespannte Familienverhältnis zwischen den Flanders und den Simpsons zu besänftigen. Als der Ausflug anders verläuft als geplant, sind Kreativität und Teamgeist gefragt.

\notiz{
\begin{itemize}
  \item Der Couch-Gag stammt von Eric Goldberg\index{Goldberg!Eric}.
  \item Homer besitzt eine Kreditkarte von Viza, die auf Hangwar Swanson\index{Swanson!Hangwar} ausgestellt ist.
\end{itemize}
}

\subsection{Simprovisation}
Homers jährliche Rede vor seinen Kollegen während der Fortbildung im Kernkraftwerk steht auf dem Programm, bei der er völlig versagt. Mithilfe eines Improvisations-Stand-Up-Comedy-Clubs will er seine extreme Angst bewältigen, vor Publikum zu sprechen. Völlig begeistert von der Improvisationskunst und ihren Möglichkeiten zeigt sich, dass Homer sogar wirkliches Talent in diesem Bereich hat. Lisa unterstützt ihn so gut sie kann. Währenddessen renoviert Marge Barts Baumhaus.

\notiz{
\begin{itemize}
  \item Krusty hat keinen Zutritt mehr zum $22^{nd}$ Comedy Club wegen u.\,a. Trunkenheit, Nacktheit und Taschendiebstahl.
  \item Lenny, Carl, Sideshow Mel und Rektor Skinner sind Mitglieder in Homers Improvisationsgruppe.
  \item Die letzten drei Minuten sitzt Homer in einem vermeintlichen Fox-Studio und gibt verschiedenste Erklärungen ab.
  \item Gegen Ende der Folge ist Bender\index{Bender} aus Futurama\index{Futurama} zu sehen, der ein Schild mit der Aufschrift \glqq Bring Futurama back (again)\grqq\ hält.
\end{itemize}
}

\subsection{Der Kurier, der mich liebte}
Marge ist sehr traurig und beneidet Homer um sein aufregendes, fröhliches und lustiges Leben. Nachdem Homer daraufhin zu Reichtum kommt, verspricht er Marge den Trip ihres Lebens, um sie aufzuheitern. Homer verliert jedoch das ganze Geld wieder. Für einen vergünstigen Familienurlaub nach Paris geht er mit einem obskuren Kurier einen Deal ein. Er soll dabei eine vom Aussterben bedrohte Schlange nach Paris schmuggeln.

\notiz{
\begin{itemize}
  \item Marge und Homer kommen bei ihrem Spaziergang in Paris am Grab von Jim Morrison\index{Morrision!Jim}, dem ehemaligen Doors-Sänger\index{Doors} vorbei.
  \item Im Abspann wird in die Steinzeit zurückgeschwenkt, wobei Matt Groening ein Wandgemälde signiert. Matt Groening signiert dabei mit der rechten Hand, obwohl er Links\-hän\-der ist.
\end{itemize}
}

\subsection{Orange Is the New Yellow}\label{VABF15}
Marge kommt wegen eines neuen Gesetzes 90 Tage ins Gefängnis, da sie Bart unbeaufsichtigt auf den Spielplatz gelassen hat. Martins Mutter zeigt sie wegen Vernachlässigung ihres Sohnes an. Mit aller Kraft versuchen Homer, Bart und Lisa, die ihre Ehefrau und Mutter schon nach kurzer Zeit schrecklich vermissen, Marge wieder nach Hause zu bringen. Diese erkennt allerdings zwischen ihren Zellengenossinnen, dass das Leben im Gefängnis nicht nur schlechte Seiten hat.

\notiz{
\begin{itemize}
  \item Der Couch-Gag stammt von Michal Socha und ist durch IKEA-Anleitungen inspiriert.
  \item Willie leitet in der Schule das Rugby-Training.
  \item Lisa spielt während es Vorspanns Theremin\index{Theremin} statt Saxophon.
\end{itemize}
}

\section{Staffel 28}

\subsection{Springfield aus der Asche}
Springfield wird Opfer eines schrecklichen Feuers, das die Stadt fast völlig verwüstet. Der Bürgermeister verspricht zwar einen schnellen Wiederaufbau, doch nach einem Jahr ist immer noch nichts passiert. Die Simpsons bitten daraufhin Mr. Burns um Geld für den Wiederaufbau. Der willigt ein, aber nur unter der Bedingung, dass er im neuen Freilufttheater eine Show inszenieren darf. Dessen Regie-Assistentin Lisa kommt bald hinter Burns' wahre Motivation.

\notiz{
\begin{itemize}
  \item Laut Homer sind die Simpsons verflucht, weil Vorfahren den Eltern von Jesus keinen Schlafplatz gegeben haben.
  \item Im Wolkengag fliegt Kodos mit einer Untertasse vorbei und hat einen Wahlwerbebanner im Schlepptau, auf dem steht: \glqq Vote Kodos -- Make the universe great again.\grqq 
\end{itemize}
}

\subsection{Die virtuelle Familie}\label{VABF18}
Schon lange träumt Mr. Burns von einer eigenen Familie. Mit technologischer Hilfe scheint sich dieser Wunsch nun zu erfüllen. Eine High-Tech Brille ermöglicht es Burns, eine virtuelle Familie zu gründen. Die Simpsons stellen sich dazu als Schauspieler zur Verfügung. Nur Homer bleibt außen vor, damit Burns das Familienoberhaupt mimen kann. Auf sich selbst gestellt knüpft Homer neue Bande und freundet sich mit seiner Nachbarin Julia an. Diese scheint in vielerlei Hinsicht Homers Ebenbild zu sein.

\notiz{
\begin{itemize}
  \item Maggie spricht während einer Therapiesitzung bei Dr. Nussbaum. Da ihr allerdings niemand zuhört, will sie nie wieder sprechen.
  \item Die VR-Brille von Prof. Frink heißt Froculus\index{Froculus}.
  \item In Moes Taverne sind Flaschen zu sehen, auf denen der deutsche Bundesadler abgebildet ist. 
\end{itemize}
}

\subsection{Stadt ohne Gnade}\label{VABF17}
Bart ist seit kurzem Fan der Boston Americans, der Football-Rivalen von Springfield. Homer gefällt dies gar nicht. Deshalb soll die ganze Familie nach Boston fahren, um zu sehen, wie furchtbar diese Stadt ist. Doch kurz nach ihrer Ankunft passiert Homer ein Unfall. Die Ärzte geben ihr Bestes und die Familie staunt, als sie das Ergebnis sieht. Dies führt sogar dazu, dass die Simpsons kurzzeitig nach Boston ziehen.

\notiz{
\begin{itemize}
  \item Die Springfield Atoms waren bis 2003 das Football-Team aus Portland.
  \item Bart erwähnt, dass Grampa 2.000 km entfernt lebt. Somit müsste Springfield 2.000 km von Boston entfernt sein. 
\end{itemize}  
}

\subsection{Trocken, tot und tödlich}
\begin{itemize}
  \item \textbf{Trockene Zeiten}\\ In Springfield hat es seit Ewigkeiten nicht mehr geregnet. Der Einzige, der noch Wasser besitzt, ist der Herrscher Mr. Burns. Er veranstaltet einen tödlichen Wettkampf zwischen den Kindern der Stadt, um dem Gewinner einen Tag Wasserspaß zu gönnen.
  \item \textbf{Beste Freundin}\\ Es geschehen merkwürdige Todesfälle um Lisa herum. Wie sich herausstellt, ist dafür Lisa imaginäre Freundin Rachel verantwortlich.
  \item \textbf{Moefinger}\\ Bart erfährt, dass unter Moes Taverne ein riesiges Geheimagenten-Zentrum existiert. Homer gehörte dieser Organisation ebenfalls an. Doch er wechselte die Seiten und will nun den gesamten Biervorrat der Welt aufkaufen.
\end{itemize}

\notiz{
\begin{itemize}
  \item Dies ist die 600. Folge der Simpsons.
  \item Homer trägt einen Button mit einer Anspielung auf eine Präsidentschaftskandidatur Ivanka Trump im Jahr 2028 an seinem Bender-Kostüm.
  \item Lisas Freundin Janey heißt mit Familiennamen Powell.  
\end{itemize}}

\subsection{Fake-News}\label{VABF21}
Lisa und Bart sind die neuen Süßigkeiten von Krusty verdächtig. Daraufhin beschließen sie die \glqq Krustaceans\grqq\ auf Herz und Nieren zu prüfen. Inzwischen hofft Homer auf eine Beförderung im Atomkraftwerk. Dazu braucht er Marges Hilfe. Sie soll ihn in Kleidungsfragen beraten. Aber auch der neue Look bringt Homer nicht die gewünschte Beförderung.

\notiz{
\begin{itemize}
  \item Kent Brockman hat auch eine bereits erwachsene Tochter.
  \item Tibor heißt mit Nachnamen Jankovsky\index{Jankovsky!Tibor}.
\end{itemize}}

\subsection{Mein peinlicher Freund}\label{VABF22}
Milhouse Vater Kirk ist auf der Suche nach einem Freund. So kommt es, dass Homer und Kirk Trainer des Kinder-Lacrosseteams\index{Lacrosse} werden. Doch als Kirk all zu anhänglich wird, bezeichnet in Homer als Loser. Daraufhin verschwindet Kirk; ausgerechnet zu jenem Zeitpunkt, als die Meisterschaft beginnt. Homer kann ihn allerdings aufspüren und dafür sorgen, dass er zum entschiedenen Spiel zurückgekehrt.

\notiz{
\begin{itemize}
  \item Kirk war Stürmerstar des Lacrosse-Teams seines Colleges (Gaudgers\index{Gaudgers}).
  \item Kirk zeigt sein sechstes MyTube-Video und am Ende ist eine Erinnerung an Kevin Curran zu sehen.
  \item Das Kinder-Football-Team heißt \glqq Neutronios\grqq\index{Neutrinos}.
\end{itemize}}

\subsection{Fidel Grampa}\label{VABF19}
Die Simpsons stehen vor einem Problem: Weder das Springfielder Altenheim noch das Krankenhaus will für Abe Simpsons Krankenbehandlung aufkommen. Deshalb entscheidet sich die Familie, mit Grandpa nach Kuba zu fliegen. Dort gibt es eine erschwingliche Behandlung. Allerdings hat niemand damit gerechnet, dass es Abe so gut in Kuba gefällt, dass er nicht mehr zurück in die USA möchte.

\notiz{
\begin{itemize}
  \item Der reiche Texaner wurde in New Hampshire geboren.
  \item Bart sprüht in Kuba \glqq The Bart\grqq\ an eine Hauswand.
  \item Im Couchgag sitzen die Simpsons als griechische Götter auf einer Couch im Olymp, dabei entspringt Lisa Homers Kopf -- wie die griechische Göttin Athene, die eine Kopfgeburt des Göttervaters Zeus war.
  \item Abe gibt an, 86 Jahre alt zu sein.
\end{itemize}
}

\subsection{Affenhilfe}\label{WABF01}
Durch Barney erfährt Homer von der App \glqq Chore Monkey\grqq\index{Chore Monkey}, mit der man Menschen mit der Erledigung verschiedener Aufgaben beauftragen kann. Homer kann nun endlich seine lästigen Aufgaben am Haus delegieren. Er engagiert sogar den Football-Spieler Matt Leinart\index{Leinart!Matt} als Barts Spielkameraden. Was dazu führt, dass Homer lieber Zeit mit Milhouse verbringt, als mit seinem Sohn. In der Zwischenzeit ist Abe im Glauben, noch einmal Vater zu werden. Wie sich dann allerdings herausstellt, ist Jasper der Vater.

\notiz{
\begin{itemize}
  \item In der Eröffnungssequenz passiert jedem Simpson in der für ihn typischen Szene ein Unfall. Deshalb steht Bart allein mit dem kaputten Skateboard vor der Couch. Er stellt von jedem ein Bild auf die Couch und setzt sich dazwischen.
  \item Lou hat einen Sohn, der in Baltimore lebt.
  \item Ab dieser Folge erfolgt die Animation durch FOX Television Animation. Von 1992 bis 2016 erfolgte die Animation durch Film Roman.
\end{itemize}
}

\subsection{Mr. und Mrs. Smithers}\label{WABF03}
Mr. Burns will Homer wieder einmal durch seine Falltür verschwinden lassen, doch dieses Mal verletzt er sich dabei. Homer wittert seine Chance auf schnelles Geld und beschließt, seinen Boss zu verklagen. Nun ist es an Mr. Smithers, die Situation zu entschärfen. Er besucht die Simpsons, um Homer von der Klage abzubringen, und freundet sich dabei mit Marge an. Als Waylons Bemühungen erfolglos bleiben, will Burns ihn nach Tschernobyl\index{Tschernobyl} versetzen -- doch damit würde Marge ihren neuen Freund verlieren und deshalb zieht Homer schließlich die Klage zurück.

\begin{itemize}
  \item Grampa behauptet, Notar zu sein.
  \item Mr. Burns gesteht gegenüber Mr. Smithers, dass das Kernkraftwerk in Tschernobyl ihm gehöre.
\end{itemize}

\subsection{Ein Schweinchen namens Propper}\label{WABF06}
Inspiriert von einem japanischen Aufräumratgeber, möchte Marge ihr Haus ausmisten und so von überflüssigen Dingen befreien. Deshalb fordert sie den Rest der Familie dazu auf, alles Unerfreuliche zu entsorgen. Wenn es nach Marge geht, soll auch Schweinchen Propper das Haus so schnell wie möglich verlassen. Homer findet schließlich einen Weg, wie er das Schwein doch noch behalten kann -- wäre da nicht Mr. Burns.

\notiz{
\begin{itemize}
  \item Lenny gibt an, dass er ein Jahr im Gefängnis saß.
  \item Barney findet im Müll der Simpsons Homers Mr. Plow Jacke (siehe \glqq Einmal als Schnee\-könig\grqq , \ref{9F07})
\end{itemize}
}

\subsection{Krustliche Weihnachten}\label{WABF02}
Krusty bekommt an Weihnachten Besuch von seiner Tochter Sophie, die von ihrer Mutter christlich erzogen wird. Als Krusty dies erfährt, behauptet er, ihr zuliebe vom jüdischen zum christlichen Glauben zu konvertieren. Doch er versucht, diesen Schritt so gut wie möglich in einer großen Fernsehshow zu vermarkten, was Sophie ganz und gar nicht gefällt. Krusty muss sich nun zwischen seiner Tochter und seiner TV-Karriere entscheiden.

\notiz{
\begin{itemize}
  \item Das Titellied der Itchy-und-Scratchy-Folge hat einen alternativen Text.
  \item Krusty hat nur für eine Woche im Jahr das Sorgerecht für seine Tochter Sophie.
\end{itemize}
}

\subsection{Der große Phatsby}
Mr. Burns trauert seinen besten Jahren nach. Um zu beweisen, dass er es immer noch drauf hat, will er eine wilde Party veranstalten. Leider wird die Feier aufgrund seines Geizes ein absoluter Reinfall. Wenig später lernt Burns jedoch den Musikproduzenten und Geschäftsmann Jay G. kennen. Jay erklärt sich bereit, Mr. Burns beizubringen, wie man es so richtig krachen lässt. Dazu übergibt er ihm eine schwarze Kreditkarte.

Mr. Burns hat nach Jays Intrige sein gesamtes Vermögen verloren. Selbst von seinem geliebten Kernkraftwerk musste er Abschied nehmen. Er bittet Homer, ihn in dieser schwierigen Stunde zur Seite zu stehen. Gemeinsam entwickeln die beiden einen Racheplan. Ein Rap-Song gegen Jay G. soll diesen vernichten. Mit an Bord ist nicht nur Jays Exfrau Praline, sondern mit Snoop Dogg und RZA auch die Stars der Hip Hop Szene.

\notiz{
\begin{itemize}
  \item Es handelt sich hier um eine Doppelfolge.
  \item Diese Folge hat keinen Vorspann. Ebenso fehlen der Tafel-, Werbe- und Couch-Gag. Die Musik zu Beginn der Folge ist ebenso wenig die typische Simpsons-Melodie.
  \item Mr. Burns schrieb das Buch \glqq The Rungs of Ruthlessness\grqq , welches Jay zu seinem Vorgehen ermutigte.
\end{itemize}}

\subsection{Fettscaraldo}
Homer muss mit Entsetzen feststellen, dass alle örtlichen Fast-Food-Restaurants auf gesunde Ernährung umgestellt haben. Er ist verzweifelt und fährt ziellos in der Gegend herum, bis er zu einem Hot-Dog-Stand kommt, den er zum letzten Mal vor 30 Jahren als kleiner Junge aufgesucht hat. Als Homer nach einiger Zeit den Imbissladen wieder besuchen will, ist dieser geschlossen. Homer ist untröstlich, aber so schnell will er sich damit nicht abfinden. Homer will den kleinen Laden retten.

\notiz{Im Abspann wird dem verstorbenem Animateur Sooan Kim gedacht.}

\subsection{Homersche Eröffnung}\label{WABF08}
Homer überrascht seine Familie mit ungeahnten Schachspielkünsten. Wie sich später herausstellt, ist dieses Talent allerdings mit einem Kindheitstrauma verbunden, was ihn schwer mitnimmt. Weil Homer immerzu gegen ihn gewann, beschloss sein Vater, nie wieder Schach zu spielen. Eine letzte Partie zwischen Vater und Sohn soll die Sache für immer bereinigen. Doch kann Homer sein Trauma so wirklich verarbeiten?

\notiz{Carl Carlson behauptet, dass der norwegische Schachspieler Magnus Carlsen\index{Carlsen!Magnus} sein Cousin ist.}

\subsection{Krise nach Kamp Krusty}
Lisa und Bart kehren traumatisiert aus dem Sommercamp zurück. Marge macht sich deshalb große Sorgen, wodurch ihre Ehe mit Homer auf eine harte Probe gestellt wird. Um mehr Zeit für ihre Kinder zu haben, vernachlässigt sie zunehmend ihren Mann und ihre ehelichen Pflichten. So kommt es, dass sich Homer nun mehr um seine Arbeit als um Marge kümmert. Um ihre Probleme zu lösen, sollen zwei Therapeuten helfen. Doch die schicken Bart und Lisa zurück ins Kamp Krusty.

\notiz{
\begin{itemize}
  \item Diese Folge nimmt Bezug zur Episode \glqq Krise im Kamp Krusty\grqq\ (siehe \ref{8F24}) aus der vierten Staffel.
  \item Das Auge auf der Pyramide der Ein-Dollar-Note, die Homer Bart gibt, befindet sich am falschen Ende.
\end{itemize}
}

\subsection{22 für 30}\label{WABF10}
Bart entwickelt sich zum Basketballsuperstar. Doch der Rummel um seine Person bekommt ihn gar nicht gut, seine Umgangsmanieren verfallen. Er wird hochnäsig und arrogant. Homer wird sein Trainer, nachdem Hausmeister Willie den Trainerjob hinschmeißt. Aber auch er kann nichts gegen die Arroganz seines Sohnes ausrichten. Nebenbei lässt sich Bart noch mit Fat Tony ein. Er manipuliert im Sinne Fat Tonys die Spielergebnisse.

\notiz{
\begin{itemize}
  \item Der Couch-Gag stammt von Bill Plympton\index{Plympton!Bill}.
  \item Johnny Schmallippe heißt eigentlich Giovanni Silencio\index{Silencio!Giovanni}.
  \item Fat Tony heißt eigentlich Marion D'Amico\index{D'Amico!Marion}.
\end{itemize}
}

\subsection{Der Uhr-Großvater}\label{WABF11}
Marge befürchtet, dass Bart im Leben scheitern wird. Sie besucht deshalb den Vortrag einer bekannten Erziehungsexpertin. Diese empfiehlt, das Selbstwertgefühl der Kinder via Trophäen zu steigern. Da es egal ist, wofür diese vergeben werden, wittert Homer das große Geschäft und macht ein Trophäengeschäft auf. Inzwischen wird aber Barts Selbstwertgefühl jedoch mehr durch die alte Uhr seines Großvaters verbessert, bis er diese verliert.

\notiz{
\begin{itemize}
  \item Sideshow Mels Kinder leben bei seiner Mutter in Nebraska.
  \item Lisa hat den ersten Platz bei der Lyrik-Safari belegt.
\end{itemize}
}

\subsection{Homer Academy}
Als es Mr. Burns nicht gelingt, einen Kurs über atomare Energie an der Yale-Universität zu etablieren, beschließt er, seine eigene Privatuniversität zu gründen. Um Geld für Professoren zu sparen, stellt er einfach Arbeiter aus dem Atomkraftwerk als Lehrpersonen ein. Auch Homer wird als Universitätsprofessor angestellt.

\notiz{
\begin{itemize}
  \item Beim Wolkengag steht der Simpsons-Schriftzug auf dem Kopf und Gott dreht ihn um.
  \item Im Couch-Gag sind die Simpsons als X-Men zu sehen: Homer als Professor X, Marge als Mystique, Lisa als Storm, Maggie als Wolverine und Bart als Angel. Stan Lee hat hier seinen Gastauftritt.
\end{itemize}}

\subsection{Auf der Suche nach Mr. Goodbart}\label{WABF13}
Es ist \glqq Großelterntag\grqq\ an Barts Schule und nachdem er sich wieder einmal Ärger eingehandelt hat, wird er von Rektor Skinner gezwungen, sich um dessen Mutter Agnes zu kümmern. Agnes schließt Bart jedoch schon bald in ihr Herz und behandelt ihn wie ihren eigenen Enkel, den sie nach Strich und Faden verwöhnt. So entwickelt Bart eine lukrative Geschäftsidee. Von da an verbringt er Zeit mit fremden Omas und spielt deren eigenes Enkelkind.

\notiz{
\begin{itemize}
  \item Der russische Kabelkanal 2x2 strahlte diese Episode nicht aus, da Homer in der Kirche Peekimon\index{Peekimon} spielt und sich ein ähnlicher Fall tatsächlich in Russland ereignet hat. Der russische Videoblogger Ruslan Sokolovsky ist zu einer Haftstrafe verurteilt worden, weil er in einer Kirche Pokemón Go gespielt hat (\cite{FAZ17}).
  \item Der Couchgag ist an das Intro von The Big Bang Theory angelehnt.
\end{itemize}}

\subsection{Moho House}
Marge und Homer haben Eheprobleme und möchten an ihrer Ehe arbeiten. Was sie nicht ahnen, dass Mr. Burns und sein alter Kuppel Nigel dahinter stecken. Die beiden haben eine Wette laufen, ob Marge und Homer sich trennen oder nicht. Der Gewinner bekommt Smithers als Assistent. Da Nigel unbedingt gewinnen will, heuert er Moe an, dieser soll Marge verführen.

\notiz{
\begin{itemize}
  \item Bart hat nur einen Satz und Lisa spricht überhaupt keinen Satz.
  \item Das Geschenk, welches Smithers gibt, damit Homer Marge zurückgewinnen kann, war ursprünglich für seine Mutter zum 81. Geburtstag gedacht.
\end{itemize}
}

\subsection{Dogtown}\label{WABF15}
Als Homer nicht mehr rechtzeitig mit dem Auto bremsen kann, entschließt er sich, den Hund zu retten und auf den Mann (Gil Gunderson) zu zusteuern. Bei der anschließenden Gerichtsverhandlung wird Homer Recht gegeben und das Leben des Tieres über das des Menschen gestellt. Bürgermeister Quimby ändert daraufhin die Gesetze, um Springfield zur hundefreundlichsten Stadt zu machen.

\notiz{
\begin{itemize}
  \item In der Jury sitzen u.\,a. der Comic-Buchverkäufer, Moe, Dr. Hibbert, Rektor Skinner, der Bienenmann und Agnes Skinner.
  \item Gil Gunderson hat eine Tochter.
\end{itemize}
}


\section{Staffel 29}

\subsection{Die Sklavsons}\label{WABF17}
In der magischen mittelalterlichen Welt des Königreichs Springfieldia\index{Springfieldia} leidet Marges Mutter an einer mysteriösen Krankheit -- sie verwandelt sich in einen \glqq Ice Walker\grqq. Ein Amulett, das sie von dieser Krankheit heilen könnte, ist sehr teuer. Nur gut, dass Lisa zaubern kann und einen Klumpen Blei in Gold verwandeln kann. Da dies verboten ist und der König dies mitbekommt, wird sie vom Oberzauberer des Königs entführt.

\notiz{
\begin{itemize}
  \item Das Intro und der Abspann werden von Billy Boyd\index{Boyd!Billy} gesungen.
  \item In der US-Fassung war der Schauspieler Nikolaj Coster\index{Coster!Nikolaj} zu hören. Er spielt seit 2011 in \glqq Game of Thrones\grqq\ mit.
\end{itemize}
}

\subsection{Sad Girl}
Lisa beschließt, ihr trauriges Gefühlsleben zu Papier zu bringen. Sie möchte alles in einem Comic festhalten. Da sie als Zeichnerin scheitert, übernimmt Marge die Bilder. Kumiko, die Frau des Comicbuchverkäufers, findet die Entwürfe und produziert den Comic. Als der Comic zum Erfolg und Lisa zum Star wird, kommt es zwischen ihr und Marge zum Streit. Der Streit wird vermeintlich beigelegt, als ein Regisseur aus dem Comic eine Broadway-Show machen will.

\notiz{
\begin{itemize}
  \item Im Bücherregal der angehenden Psychiaterin ist das Buch \glqq Did I Die Or Not?\grqq\ von Dr. Marvin Monroe zu sehen.
  \item Im Abspann wird Tom Petty\index{Petty!Tom} gedacht, der am 02.10.2017 verstorben ist.
\end{itemize}
}

\subsection{Talent mit Pfiff}
Obwohl Maggie noch ein Kleinkind ist, kann sie wahnsinnig gut pfeifen. Deshalb meldet Homer sie bei einem Talentwettbewerb für Babys an. Nur Marge erzählt er nichts davon. Die hat inzwischen selbst ein Geheimnis. Nachdem sie in der Schule mit ihrem Talent als Innendekorateurin glänzen konnte, bekommt sie nun einen größeren Auftrag. Sie ahnt allerdings nicht, dass sie für die Mafia arbeitet und die Innenausstatterin eines Bordells ist.

c

\subsection{Grampa ist ganz Ohr}\label{WABF19}
Ohne dass seine Familie etwas davon erfährt, hat Grampa von einem Mitbewohner im Altersheim ein Hörgerät geschenkt bekommen. Nun hört er alles mit, was seine liebe Familie über ihn zu sagen hat. Lisa hat bei einer Arbeit einen Fehler gemacht. Sie und Bart brechen nachts in die Schule ein, damit sie das Malheur beheben kann. Sie treffen im Keller auf Rektor Skinner, der dort wohnt. Dieser erzählt ihnen, warum er aus dem Haus seiner Mutter ausgezogen ist. 

\notiz{
\begin{itemize}
  \item Grampa feiert seinen 87. Geburtstag.
  \item Rektor Skinner wollte eigentlich Mitglied einer Marschband sein und an der Ohio State Universität studieren.
  \item Die Geschichte, dass Rektor Skinner ursprünglich an der Ohio State Universität studieren wollte, steht im Widerspruch zu seiner Jugend, die in der Episode \glqq Alles Schwindel\grqq\ (siehe \ref{4F23}) geschildert wird.
\end{itemize}}

\subsection{Blau im Amt}
Als Marge bei einer Bürgerversammlung vom Bürgermeister abgekanzelt wird, beschließt sie, für das Amt des Bürgermeisters zu kandidieren. Nachdem sie die Wahl gewonnen hat, holt sie sich eher zufällig Homer zur Unterstützung. Dieser bzw. Witze über ihn lockern ihre Reden auf oder sorgen für spaßige Einlagen bei offiziellen Terminen.

\notiz{
\begin{itemize}
  \item Lenny hält ein Krokodil als Haustier.
  \item Springfield baut eine Skypark-Line, die sich auf den ehemaligen Trassen der Monorail befindet (siehe \glqq Homer kommt in Fahrt\grqq, \ref{9F10}).
  \item Mr. Largo ist in der Menschenmenge während der Einweihung des Gullydeckels zu sehen, obwohl er in der Episode \glqq Moeback Mountain\grqq\ (siehe \ref{NABF04}) die Stadt verlassen hat.
\end{itemize}}

\subsection{Die Pin Pals}
Moe fühlt sich vernachlässigt, sodass Homer kurzerhand ein Bowling-Team zusammen mit Lenny, Carl und Barney gründet, für das sich Moe als Coach verantwortlich zeichnet. Das Team feiert einen Erfolg nach dem anderen und schon bald stehen sie im Landesfinale in Capital City. Dort treffen sie auf eine Mannschaft aus überheblichen Hedgefonds-Managern, die Moe eine perfide Wette anbieten: Wenn sein Team verliert, muss er seinen Namen und die Bar aufgeben.

\notiz{
\begin{itemize}
  \item Der Vorspann der Episode zeigt die Simpsons als Meerestiere unter Wasser.
  \item Lenny erzählt, dass sein Stiefvater Libanese ist.
  \item Auf einer Leuchtschrift ist zu lesen \glqq The hateful 8-year old\grqq\ und anschließend \glqq directed by Quentin Tarantino\grqq. In der nächsten Szene bewegen sich Lisa und ihre Nerd-Freunde wie die Gangster am Anfang in Tarantinos Film \glqq Reservoir Dogs\grqq. Diese Anspielung gab es bereits in der Folge \glqq Jazzy and the Pussycats\grqq\ (siehe \ref{HABF18}).
\end{itemize}}

\subsection{Lisa legt los}\label{XABF01}
Die 17-jährige Lisa blickt, während sie ein Bewerbungsschreiben für Harvard verfasst, auf ihr bisheriges Leben zurück und sie stellt fest, dass sie es nicht immer leicht hatte mit ihrer verrückten Familie. Besonders die vielen Geburtstage, die Homer, Marge und ihre Geschwister vergessen haben, waren eine große Enttäuschung. Als Lisa eine Zusage von der Eliteuniversität bekommt, ist sie überglücklich. Allerdings droht sie am Druck, der in Harvard herrscht, zu zerbrechen.

\notiz{
\begin{itemize}
  \item Bart und Leon Kompowsky\index{Kompowsky!Leon} singen Lisa zum 14ten Geburtstag ein Lied wie in der Episode \glqq Die Geburtstagsüberraschung (siehe \ref{7F24}).
  \item Lisa und Luigi Risotto haben am 9. Mai Geburtstag.
  \item Lisas Lehrerin in der ersten Klasse ist Ms. Myles\index{Myles}.
\end{itemize}}

\subsection{Gone Boy}\label{XABF02}
Bart stürzt im Wald in einen geheimnisvollen Schacht, aus dem er sich nicht befreien kann. Er stellt fest, dass es sich um eine geheime Raketenbasis handelt. In Springfield wird indessen eine große Suchaktion eingeleitet, an der unter anderem auch eine Gruppe Häftlinge teilnimmt, zu denen Tingeltangel-Bob gehört. Der ist sehr daran interessiert, Bart zu finden, um ihn endgültig umbringen zu können. Die Suche verläuft jedoch zunächst erfolglos.

\notiz{
\begin{itemize}
  \item Der Vorspann ist an Weihnachten angelehnt und zeigt die Simpsons als Elfen.
  \item Agnes Skinner ist auf dem Playdude-Cover als Miss Kalter Krieg zu sehen.
  \item Sideshow Bob war Artist im Cirque du Soleil\index{Cirque du Soleil}.
  \item Der Titel der Episode ist eine Anspielung auf den Roman \glqq Gone Girl\grqq\ von Gillian Flynn\index{Flynn, Gillian} aus dem Jahre 2012.
\end{itemize}}

\subsection{Ha-Ha Land}
Lisa freut sich riesig auf den Besuch der Wissenschaftskonferenz, die sie gemeinsam mit ihrer Familie besucht. Dort lernt Lisa den begabten jungen Jazz-Pianisten Brendan kennen. Lisa ist vollkommen fasziniert von Brendan. Sie beginnen gemeinsam zu musizieren. Nelson wird eifersüchtig und die beiden Jungs beginnen, um Lisa zu buhlen. Unterdessen entdeckt Bart seine Liebe für Chemie. In seinem Baumhaus richtet er sich einen kleines Labor ein. Homer und Marge sind überzeugt, dass Bart etwas im Schilde führt. 

\notiz{
\begin{itemize}
  \item Die Katzenlady gibt an, einen Abschluss in Jura zu haben. Dies wurde bereits in der Episode \glqq Springfield wird erwachsen\grqq\ (siehe \ref{JABF07}) thematisiert.
  \item Hans Maulwurf ist Dialysepatient.
\end{itemize}}

\subsection{Arche Monty}\label{XABF04}
Nachdem Mr. Burns eine Dokumentation über die Prophezeiungen des Nostradamus\index{Nostradamus} gesehen hat, glaubt er nun, dass das Ende der Welt kurz bevor steht. Um sich zu retten, baut er eine Arche. Er bittet Professor Frink, jeden in Springfield zu testen, ob er oder sie es wert sind, zusammen mit ihm gerettet zu werden.

\notiz{
\begin{itemize}
  \item Prof. Frink hat an der Cornell University studiert.
  \item Mitglieder des Springfielder Mensa-Gruppe sind: Dr. Hibbert, Sideshow Mel, der Comic-Buchverkäufer, Manjula und Lindsey Naegle. In der Episode \glqq Die Stadt der primitiven Langweiler\grqq\ (siehe \ref{AABF18}) war auch Rektor Skinner Mitglied in der Mensa-Gruppe.
\end{itemize}}

\subsection{Manacek}\label{XABF05}
Homers Lieblingsbild -- Joan Mir\'{o}s \glqq Die Dichterin\grqq\ -- aus dem Springfielder Museum der schönen Künste wird versteigert. Er ist darüber sehr traurig. Doch dann wird das Gemälde gestohlen. Der berühmte Detektiv Manacek\index{Manacek} wird beauftragt, das wertvolle Gemälde zu suchen. Dabei gerät neben Mr. Burns auch Homer in den Kreis der Verdächtigen.

\notiz{
\begin{itemize}
  \item Homer hat eine DJ-Ausbildung besucht, aber den Abschlusstest nicht bestanden.
  \item Eddie wurde wegen Sparmaßnahmen aus dem Polizeidienst entlassen.
\end{itemize}}

\subsection{Der Exorzismus von Maggie Simpson}
\begin{itemize}
  \item \textbf{Die Exor-Schwester}\\ Homer bestellt Maggie versehentlich eine verfluchte kleine Skulptur namens Pazuzu\index{Pazuzu}\footnote{Pazuzu ist ursprünglich ein Winddämon der mesopotamischen Mythologie des 1. Jahrtausends v. Chr.}. Marge will den Pazuzu am folgenden Tag zurückschicken, aber sofort ergreift Pazuzus Dämon Besitz von Maggie. Sie braucht einen Exorzisten. Dieser schafft es, den Dämon zwar aus Maggie zu vertreiben, doch dieser schlüpft in Bart. Dieser hat aber die schwärzeste Seele, die Pazuzu je gesehen hat und will heraus, aber Bart benutzt den Dämon für seine eigenen Zwecke. 
  \item \textbf{Coralisa}\\ Lisa folgt Schneeball durch einen Geheimgang in eine Parallelwelt, die absolut perfekt ist und in der auch ihre Familie genauso ist, wie Lisa sie sich wünscht. Zum Bleiben müsste sie sich aber Knöpfe auf die Augen nähen lassen, wie die anderen in der Parallelwelt. Lisa nimmt das Angebot an. Doch auch Bart findet den Weg in die Parallelwelt, ihm folgen Marge, Homer und Maggie. Zunächst bekämpfen sich die Familien, doch schließlich ziehen die Familien zusammen in die \glqq normale\grqq\ Welt. 
  \item \textbf{Hmmm\dots\ Homer}\\ Marge und die Kinder fahren nach Ohio mit Patty und Selma, während Homer alleine Zuhause bleibt. Er genießt die freie Zeit, muss nach einiger Zeit feststellen, dass es im ganzen Haus nur noch ein einziges Würstchen gibt. Als er das Würstchen grillt, passiert ihm allerdings ein furchtbares Missgeschick: Er amputiert sich versehentlich einen Finger, der auf den Grill fällt. Homer kann der Versuchung nicht widerstehen und isst den Finger. Das Unglück nimmt seinen Verlauf: Homer beginnt sich selbst zu essen. Als seine Familie zurückkommt, muss sie feststellen, dass Homer sich komplett verändert hat. Marge erwischt ihn schließlich dabei, wie er sein eigenes Bein isst. Als Homer nicht aufhört, sich selbst zu essen, verlässt sie ihn. Schließlich lässt Homer sich komplett zubereiten und wird ein Food-Trend in mehreren nach ihm benannten Fast-Food-Ketten. 
\end{itemize}

\notiz{
\begin{itemize}
  \item Die Episode Coralisa ist eine Anspielung auf den Film \glqq Coraline\grqq\index{Coraline}. Dieser basiert auf der gleichnamigen Novelle des britischen Schriftstellers Neil Gaiman\index{Gaiman!Neil}. Er spricht in der Originalfolge Snowball.
  \item Die Namen der nach Homer benannten Fast-Food-Restaurants: Homer King, El Pollo Homo, Kentucky Fried Simpson, Der Homerschnitzel und Fatso Bell.
\end{itemize}
}

\subsection{Doppeltes Einkommen, kinderlos}\label{XABF06}
Die Simpsons landen auf dem Nachhauseweg vom Kino zufällig in dem alten Viertel, in dem Homer und Marge gewohnt haben, bevor sie geheiratet und Kinder bekommen haben. Dort treffen sie auf ein junges Hipster-Paar, das in ihrer alten Wohnung wohnt. Schon bald schwelgen Marge und Homer in Erinnerungen an die guten alten Zeiten.

\notiz{
\begin{itemize}
  \item Die Einleitungssequenz ist von Bill Plympton\index{Plympton!Bill} und basiert auf dessen Kurzfilm \glqq Your Face\grqq .
  \item Im Kino sind u.\,a. die Filme \glqq Filleting Dory\grqq\ und \glqq Alexa vs. Siri\grqq\ zu sehen.
  \item Marge arbeitete als Lokalreporterin beim Springfield Shopper. Sie wurde jedoch durch Barbara \glqq Booberella\grqq\ Lelavinsky\index{Lelavinsky!Barbara}\index{Booberella} ersetzt.
\end{itemize}
}

\subsection{Krusty macht ernst}\label{XABF08}
Rektor Skinner spielt Bart einen bösen Streich. Das kann Bart nicht auf sich sitzen lassen und er rächt sich, indem er seinen Mitmenschen im Schlaf heimlich Clownsmasken aufs Gesicht klebt. Das führt schließlich dazu, dass der Ruf der Clowns angekratzt ist und Krusty seine Karriere beenden muss. Lisa schlägt daraufhin vor, dass Krusty sein Glück als ernsthafter Schauspieler versuchen könne. Wenig später ergattert er tatsächlich die Hauptrolle in einer Theaterproduktion.

\notiz{
\begin{itemize}
  \item Llewellyn Sinclair\index{Sinclair!Llewellyn} führte bereits im Theaterstück in der Episode \glqq Bühne frei für Marge\grqq\ (siehe \ref{8F18}) Regie.
  \item Krusty gibt an, dass er das Teammaskottchen der Toronto Raptors\footnote{Die Toronto Raptors sind ein kanadisches Basketballteam der nordamerikanischen Basketball-Profiliga National Basketball Association (NBA) aus Toronto in der Provinz Ontario.} ist.
\end{itemize}
}

\subsection{Politisch unkorrekt}
Marge zwingt ihre fernsehsüchtige Familie, sie in eine Buchhandlung zu begleiten. Dort liest sie Lisa eines ihrer Lieblingskinderbücher, den Roman \glqq Die Prinzessin im Garten\grqq\ vor, bis sie feststellt, dass es eine eher konservative und in einigen Zügen fast rassistische Lektüre ist. Empört beschließt sie, das Buch umzuschreiben, jedoch nicht ohne sich dabei Gedanken über ihre eigene Kindheit zu machen. Bart leiht sich das Buch \glqq Die Kunst des Krieges\grqq\ von Sunzi\index{Sunzi} aus und versucht mit den Ratschlägen im Buch, Homer seinen Willen aufzuzwingen.

\notiz{
\begin{itemize}
  \item Marge sagt in einer Szene, in welcher ein Bild von Apu zu sehen ist, zu Lisa \glqq Mit manchen Dingen muss man sich zu einem späteren Zeitpunkt beschäftigen\grqq\ und Lisa antwortet mit \glqq Wenn überhaupt\grqq . Diese Szene bezieht sich auf die Kritik aus dem Jahr 2017 des aus Indien stammenden amerikanischen Komikers Hari Kondabolu\index{Kondabolu!Hari}. Dieser warf den Simpsons-Machern vor, der Charakter von Apu fördere Stereotype und stelle Südasiaten negativ dar (\cite{ApuRassismus}, \cite{ApuKritik}).
  \item Ned Flanders spielt Banjo.
\end{itemize}
}

\subsection{Der Matratzenkönig}\label{XABF10}
Bart darf sich in der Schule ein Musikinstrument aussuchen. Die Eltern haften für den Wert des Gegenstandes. Bart nützt die Situation schamlos aus, indem er seinen Vater damit erpresst. Als Homer einen Weg gefunden hat, sich aus der beklemmenden Situation zu befreien, kann er endlich wieder seine Lieblingskneipe aufsuchen. Dabei entdeckt er, dass Moe sich mit einem älteren Herrn streitet, der sich als Moes Vater herausstellt. Marge will die zerrütteten Familienmitglieder wieder versöhnen und lädt alle zu einem Abendessen ein.

\notiz{
\begin{itemize}
  \item Moes Vater heißt Morty. Seine jüngeren Geschwister heißen Marv und Minnie.
  \item Moes Vater betreibt die Matratzenkette \glqq Mattress King\index{Mattress King}\grqq.
\end{itemize}
}

\subsection{Lisa hat den Blues}\label{XABF11}
Lisas Musiklehrer hält sie für eine gute Musikerin, doch würde sie es nie zu Ruhm bringen, da es weitaus talentiertere Musiker als sie gäbe. Ab sofort ist Lisas Liebe für das Saxophonspielen zerstört. Als wenn das nicht schon genug wäre, beschließt Marge, zum Geburtstag einer entfernten Verwandten nach Florida zu fliegen. Der Flug wird jedoch umgeleitet, und die Simpsons landen in New Orleans, wo ihnen der Geist von Louis Armstrong erscheint. Währenddessen interessiert sich Bart sehr für die heimische Voodoo-Kultur und Homer für das hervorragende Essen.

\notiz{
\begin{itemize}
  \item Als im Vorspann Maggie über den Scanner an der Supermarktkasse gezogen wird, erscheint der Betrag \$ 486,52 auf dem Display.
  \item Zahnfleischbluter Murphy hieß Oscar mit Vornamen.
  \item Homer erwähnt am Ende der Episode, dass sie mit der nächsten Folge die Serie \glqq Rauchende Colts\grqq\ überholen.
  \item Diese Folge wurde R. Lee Ermey\index{Ermey!R. Lee} gewidmet, der am 15. April 2018 verstarb.
\end{itemize}
}

\subsection{Rezeptfrei}\label{XABF09}
Die Simpsons machen einen Ausflug zum Demolition Derby, einem Rennen, bei dem Autos zu Schrott gefahren werden. Bei der Veranstaltung erleidet Grampa einen Herzinfarkt. Die Ärzte sind sich sicher, ihn nicht mehr retten zu können. Auf dem Sterbebett bittet er Homer um Verzeihung für eine Sache, die viele Jahre zurückliegt. Homer vergibt ihm, auch wenn es ihm schwerfällt. Doch wie durch ein Wunder wird Grampa wieder gesund.

\notiz{
\begin{itemize}
  \item Die Simpsons brachen mit dieser Folge den 43 Jahre alten Rekord der Westernserie \glqq Rauchende Colts\grqq\ als die auf einem Drehbuch basierende Serie mit den meisten Episoden.
  \item Abe behauptet, seine Lieblingsfernsehserie sei \glqq NCIS\index{NCIS}\grqq, eine Anspielung auf die Serien aus dem CSI-Franchise.
\end{itemize}
}

\subsection{Links liegen gelassen}\label{XABF12}
Dank Homers Hilfe bekommt sein Nachbarn Ned Flanders einen Job im Kraftwerk. Aber schon sehr schnell muss Homer feststellen, dass Flanders einige neue Ideen hat, die Homers Bequemlichkeit sabotieren. Unter anderem soll es künftig eine Fahrgemeinschaft geben. Und Homers Arbeitsleistung ist zu gering. Währenddessen beklagt Marge, dass in ihrer Ehe das Feuer erloschen ist.

\notiz{
\begin{itemize}
  \item Ned Flanders gibt an, dass er fließend Aramäisch spricht.
  \item Laut Neds Lebenslauf wohnt er in der 744 Evergreen Terrace. Seine Telefonnummer lautet 555-8904 und seine E-Mail-Adresse lautet \nolinkurl{ned.flanders@springface.com}.
  \item Ned Flanders wird von Mr. Burns gefeuert, nachdem er ihm gegenüber beklagt, dass das Atomkraftwerk noch nie für wohltätige Zwecke gespendet hat.
\end{itemize}
}

\subsection{Dänisches Krankenlager}\label{XABF13}
Als Homer am Morgen die Treppen heruntersteigt, sieht er, dass das ganze Wohnzimmer unter Wasser steht. Doch die Simpsons haben Glück, sie bekommen von ihrer Versicherung den Schaden ersetzt. Grampa muss dringend operiert werden. Lisa erzählt vom kostenlosen Gesundheitssystem in Dänemark und von der Möglichkeit, auch als Ausländer in dessen Genuss zu kommen, wenn man dort einen Unfall hat. Die Simpsons beschließen also, nach Dänemark zu fliegen. Dort angekommen sind alle so begeistert, dass sie für immer dort leben möchten -- bis auf Homer.

\notiz{
\begin{itemize}
  \item Bender\index{Bender} aus Futurama\index{Futurama} ist zu sehen, als er durch das überflutete Haus der Simpsons schwimmt.
  \item Homer trinkt in Dänemark das Bier Düffenbrau\index{Düffenbrau}.
\end{itemize}
}

\subsection{Der Tod steht ihm gut}\label{XABF14}
Bei den Simpsons ist das Internet ausgefallen. Um nicht auf Unterhaltung verzichten zu müssen, werden alte VHS-Kassetten aus dem Keller geholt. Als dies jedoch auch nicht mehr funktioniert, wollen Homer und Bart Flanders Internet anzapfen. Dabei kommt es zu einem gefährlichen Unfall -- Bart wird vom Blitz getroffen und fällt ins Koma. Nun erlebt er in seiner Traumwelt unheimliche Visionen. Unter anderem bitten ihn Poltergeister um einen Gefallen.

\notiz{
\begin{itemize}
  \item In Barts Traum ist unter anderem die Gleichung zu sehen, die Roots of Singh genannt wird:
    \[
      x^5 - 57x^4 + 1249x^3 - 13191x^2 + 67438x - 134064 = 0\\
  	\]
  Diese Gleichung kann in folgender Form geschrieben werden:
    \[
      (x -19)(x - 9)(x - 14)(x - 7)(x - 8) = 0\\
  	\]
  Die Gleichung hat somit die Nullstellen 19, 9, 14, 7 und 8. Werden diese mit der Position im Alphabet in Verbindung gebracht, ergibt sich die Zeichenkette \textit{SINGH}. Hierbei handelt es sich um eine Hommage an den britischen Wissenschaftjournalisten Simon Singh\index{Singh!Simon}, der ein großer Simpsons Fan ist.
  \item Ebenso ist die Schrödingergleichung\index{Schrödingergleichung} in ihrer allgemeinsten Form in Barts Traum zu sehen:
   \[
     i\hbar \frac{\partial}{\partial t} - \Psi = \hat{H} \Psi\\
   \]
  \item Bart sieht auch den Tod einiger Personen voraus:
  \begin{itemize}
    \item Waylon Smithers begeht mit 50 Jahren Selbstmord.
    \item Homer wird mit 59 Jahren von der Polizei erschossen.
    \item Clancy Wiggum stirbt im Alter von 62 Jahren.
    \item Marge stirbt im Alter von 84 Jahren.
    \item Lisa stirbt im Alter von 98 Jahren.
    \item Bart wird von Seymour Skinner im Alter von 80 Jahren mit dessen Rollstuhl tödlich verletzt, als Skinner im Alter von 119 Jahren an einem Herzinfarkt, den wiederum Bart verursacht hat, verstirbt.
    \item Ralph Wiggum wird im Alter von 120 Jahren von seinem Sohn vergiftet.
  \end{itemize}  
  \item Das Lied \glqq Breathe me\grqq, das am Ende der Episode gespielt wird, stammt von der australischen Sängerin Sia\index{Sia}.
\end{itemize}
}

\section{Staffel 30}

\subsection{Bart ist nicht tot}\label{XABF19}
Während Lisa bei einer Schulaufführung Saxophon spielt, wollen Jimbo und die Rowdys Bart dazu zwingen, den Feueralarm auszulösen. Als er sich weigert, verhöhnen sie ihn als Feigling. Für Bart ist klar, er muss dringend eine Mutprobe bestehen. Kurzerhand springt er von einem Damm und landet im Krankenhaus. Als Marge ihren Sohn wütend zur Rede stellt, behauptet Bart, im Himmel gewesen zu sein und Jesus getroffen zu haben. Christliche Filmproduzenten bieten den Simpsons daraufhin einen Filmdeal an. Aber Bart kann nicht lange mit der Lüge leben und vertraut sich nach Fertigstellung des Films seiner Mutter Marge an.

\notiz{
\begin{itemize}
  \item Oberschulrat Chalmers erzählt, dass seine Frau verstorben sei.
  \item Marges Vater hatte rechts ein Holzbein.
  \item Diese Episode gewann den \glqq Writers Guild of America Award for Television: Animation\grqq\ bei den Writers Guild of America Awards 2018. 
\end{itemize}
}

\subsection{Heartbreak Hotel}
Lisa und Bart ermutigen ihre Mutter, bei ihrer Lieblingssendung \glqq The Amazing Place\grqq\ zusammen mit Homer daran teilzunehmen. Doch bereits beim ersten Spiel fliegen die beiden aus dem Rennen um den Hauptgewinn. Um den Verlauf der aufgezeichneten Show geheim zu halten, sind die beiden gezwungen, ein paar Wochen im Flughafenhotel zu verbringen. Die Ehe der beiden wird auf eine harte Probe gestellt, als Marge herausfindet, dass Homer Schuld am Ausscheiden aus der Show hat.

\notiz{Die Schwarz-Weiß-Sequenz ist eine Hommage an das Theaterstück \glqq Wer hat Angst vor Virginia Woolf?\grqq\ von Edward Albee\index{Albee!Edward} aus dem Jahre 1962.}

\subsection{Planet im Nebel}
In der Eröffnungssequenz tritt Homer bei einem Esswettbewerb gegen den mystischen Cthulhu\index{Cthulhu} an. Homer gewinnt den Wettbewerb und am Ende wird Cthulhu von den Simpsons verspeist.
\begin{itemize}
  \item \textbf{Angriff der Pflanzenfresser}\\ In ihrer Unterwasserbasis stellt Mapple\index{Mapple} ihr neues myPhone\index{myPhone} Ultima vor. Die Benutzer bemerken dabei nicht, dass sie von Pflanzen gefressen werden und ihre Seelen auf einen utopischen Planeten ohne jegliche Technologie transportiert werden.
  \item \textbf{Multiplisa-keit}\\ Bart, Milhouse und Nelson trinken bei einer Übernachtungsparty mit Schlafmittel versetzen Orangensaft. Als diese schlafen, werden sie von Lisa im Keller eingesperrt. Lisa will sich dafür rächen, dass Bart dafür gesorgt hat, dass sie in einer Prüfung eine Sechs erhalten hat.
  \item  \textbf{Seniorassic Park}\\ Mr. Burns startet sein spezielles Altersheim \glqq Geriatric Park\grqq\ auf einer geheimen Insel. Um den Heimbewohnern ein besseres Leben zu ermöglichen, wurde mit Dinosaurier-DNA experimentiert. Schon kurze Zeit später ergeben sich daraus ungeahnte Folgen.
\end{itemize}

\notiz{Das Raumschiff aus Futurama\index{Futurama} ist zu sehen, das ein Transparent mit der Aufschrift \glqq Bring back Futurama\grqq\ hinter sich herzieht. Das Raumschiff wird vom Raumschiff aus \glqq The Orville\index{Orville}\footnote{The Orville ist eine US-amerikanische Science-Fiction-Fernsehserie, die auf einer Idee von Seth MacFarlane basiert, der auch der Schöpfer von \glqq Family Guy\grqq\ und \glqq American Dad\grqq\ ist.}\grqq\ zerstört.}

\subsection{Himmlische Geschichten}\label{XABF17}
Gott und Petrus unterhalten sich über eine Erneuerung des Himmels und wie sie mehr Menschen dorthin bekommen. Ned Flanders erzählt anschließend einer Schulklasse wie er religionslos geboren zum Glauben fand. Marge erzählt die Geschichte über ihre atheistische Großmutter Genevieve Bouvier\index{Bouvier!Genevieve}. Lisa bekommt Besuch von Gautama Buddha, der ihr die Geschichte von Prinzessin Siddmartha\index{Siddmartha} erzählt.

\notiz{
\begin{itemize}
  \item Als Homer von Marges Auto im Couchgag angefahren wird, fliegt er durch die Badezimmertür in das Restaurant der Animationsserie \glqq Bob's Burger\grqq.
  \item Ned Flanders arbeitete als Handelsvertreter von Kindertrampolinen.
  \item Ned Flanders hat unter seinem Oberlippenbart eine Narbe.
\end{itemize}
}

\subsection{Passives Fahren}
Homer verliert seinen Job im Atomkraftwerk und wird bei CarGo\index{CarGo}, einer Firma für selbstfahrende Autos, angestellt. Hier soll er die selbstfahrenden Wagen testen. Homer gefällt seine neue Aufgabe, weil sie sehr entspannt ist und er klettert schnell die Karriereleiter nach oben. Als die anderen Mitarbeiter im Kernkraftwerk von diesem leichten Job erfahren, kündigen sie nach und nach. Mr. Burns und Waylon Smithers gehen undercover und wollen herausfinden, warum der neue Arbeitgeber so viel attraktiver ist.

\notiz{Shauna arbeitet bei Krusty Burger. In der Folge \glqq Super Franchise Me\grqq\ (siehe \ref{SABF19}) arbeitete sie in Marges Sandwich-Laden.}

\subsection{Keine Frau für Moe}\label{XABF20}
Bart möchte sich nach einem missglückten Telefonstreich an Moe rächen. Er bestellt dem Barkeeper eine russische Frau im Darknet\index{Darknet}. Als Anastasia\index{Anastasia} ankommt, kann Moe seine Augen nicht von ihr lassen. Außerdem bringt sie seine ganze Bar auf Vordermann und beschert ihm neue Kundschaft. Doch kurz bevor es ernst wird, bekommt Moe kalte Füße. Er trennt sich von Anastasia. Wenig später bereut er diese Entscheidung und möchte sie zurückerobern. Es kommt schließlich zur Hochzeit, die Moe aber absagt, als ihm klar wird, dass sie nur hinter seinem Geld her ist.

\notiz{Auf Moes Hochzeit tritt Kirk van Houten als D. J. Kirk auf.}

\subsection{Genderama}
Um sich ein bisschen Geld dazu zu verdienen, will Marge eine \glqq Tupperparty\grqq\ schmeißen. Allerdings findet sich zunächst niemand, der sie zu sich einladen möchte. Erst Julio, ihr Friseur, erklärt sich schließlich dazu bereit. Um Marges Verkaufschancen zu erhöhen, verschönert Julio sie mit jeder Menge Make-Up. Und tatsächlich: Die Party wird ein voller Erfolg. Doch dann stellt sich heraus, dass die Gäste Marge für eine Drag-Queen halten.

\notiz{Julio arbeitet im Frisuersalon \glqq Curl Up And Dye\grqq.}

\subsection{Krusty, der Clown}
Als die Schulzeitung einen Verlust macht, ernennt Rektor Skinner einen neuen Herausgeber, der Lisa zum Schreiben von TV-Rezensionen verdonnert. Bald schon bemerkt Lisa, dass ihr Vater ein Talent dafür besitzt und überträgt ihm den Job. Homer schlägt sich gut, bis seine Kritik von Krustys TV-Show den beliebten Clown so verärgert, dass dieser versucht, Homer umzubringen. Bart hilft Krusty, bei einem Zirkus unterzutauchen. Wegen Homers neuer Tätigkeit kommt es zu Spannungen in seiner Ehe mit Marge.

\notiz{Krusty fährt einen Tesla.}

\subsection{Weihnachten in Florida}\label{YABF02}
Kaum ist Thanksgiving vorbei, packt die Simpsons die Vorfreude auf das Weihnachtsfest. Bart und Lisa wünschen sich einen Smart-TV-Fernseher. Um ein Black-Friday-Schnäppchen zu ergattern, stellt Marge sich mitten in einer frostigen Nacht vor dem Shopping Center an -- doch vergeblich. Da hat Lisa eine fabelhafte Idee: Warum nicht Weihnachten in Florida verbringen? Prompt wird ein billiges Hotel gebucht und los geht's! Doch das erhoffte Idyll erweist sich als Fiasko.

\notiz{
\begin{itemize}
  \item Gil hat eine Enkelin.
  \item Am 23.12.2020 strahlte der ORF diese Episode erstmals mit einer österreichischen Synchronisation aus.
\end{itemize}}

\subsection{Daddicus Finch}\label{YABF01}
Seit Homer und Lisa sich gemeinsam den Film \glqq Wer die Nachtigall stört\grqq\ angesehen haben, herrscht zwischen Vater und Tochter eine stärkere Bindung. Ganz zum Ärger von Bart, der sich dadurch vernachlässigt fühlt. Marge schaltet sich ein und probiert die Beachtung beider Kinder durch ihren Vater wieder ins Gleichgewicht zu bekommen.

\notiz{
\begin{itemize}
  \item Prof. Farnsworth\index{Farnsworth} aus Futurama ist im Springfielder naturwissenschaftlichem Museum zu sehen.
  \item Lennys Vater war der Herausgeber des Life-Magazins.
  \item Shauna Chalmers feiert ihre Bat Mitzwa.
  \item Kent Brockman fährt einen Tesla.
\end{itemize}}

\subsection{Plastiktrauma}
Homer und Marge haben Hochzeitstag. Während die beiden zu einem romantischen Abend aufbrechen, passt Grampa auf die Kinder auf. Da das Fernsehprogramm wenig Spannendes bietet, wird die Spielkiste nach Brauchbarem durchwühlt. In dieser finden sie schließlich Plastiksoldaten. Diese lösen aber in Grampa eine Erinnerungswelle aus: In seiner Vergangenheit war er nämlich Model für Spielzeugsoldaten gewesen.

\notiz{
\begin{itemize}
  \item Diese Episode gewann einen Emmy.
  \item Die Simpsons nutzen Amazons Alexa\index{Alexa}.
\end{itemize}
}

\subsection{Das Mädchen im Bus}
Während um Lisa herum wieder einmal alles im Chaos versinkt, fällt ihr Blick auf eine Oase stillen Glücks. Vom Schulbus aus beobachtet Lisa ein gleichaltriges Mädchen, das auf der Veranda Klarinette spielt. Wie sich herausstellt, ist Sam nicht nur musikalisch. Ihre ganze Familie entpuppt sich als sehr kultiviert. Lisa wähnt sich im siebenten Himmel, als sie von Sams Eltern zum Abendessen eingeladen wird. Um zu vertuschen, wie primitiv ihre eigenen Eltern sind, verstrickt sich Lisa alsbald in ein Meer aus Lügen.

\subsection{Fett's Dance}
Homer verärgert Marge, als er ihre Lieblingsserie \glqq Odder Stuff\grqq\ ohne ihr Wissen bei Netflix\index{Netflix} weiterschaut. Nun muss er sich was überlegen, um sie wieder zu beruhigen. Um sein Fehlverhalten wiedergutzumachen, beschließt er, gemeinsam mit seiner Frau Tanzen zu gehen. Schließlich liebt Marge Tanzshows und alles, was damit zusammenhängt. Takt- und Rhythmusgefühl waren allerdings noch nie Homers größte Stärke. Kurzerhand bucht er daher heimlich Stunden bei einer Expertin. Doch um den talentlosen Schüler tänzerisch auf Vordermann zu bringen, bräuchte es fast schon ein Wunder. Währenddessen trainiert Bart, um an einem Krusty-Wettbewerb teilnehmen zu können.

\notiz{
\begin{itemize}
  \item Marge hat eine Tante mit dem Namen Eunice\index{Eunice}.
  \item Bei der Netflix-Serie \glqq Odder Stuff\grqq\ handelt es sich um eine Parodie auf \glqq Stranger Things\grqq.
  \item Als Homer Netflix schaut, ist ein Bild von ihm und Marge zu sehen, unter welchem das Wort \glqq Disenchantment\grqq\ steht. Disenchantment\index{Disenchantment} ist auch der Titel einer von Matt Groening geschriebenen Zeichentrickserie, die für Netflix produziert wird.
  \item Homer und Marge gewannen in der Folge \glqq Der Kampf um Marge\grqq\ (siehe \ref{BABF05}) bei einem Tanzwettbewerb eine Harley Davidson.
\end{itemize}
}

\subsection{Projekt Weltraumsand}\label{YABF06}
In Marc Marons Podcast enthüllt Krusty der Clown die bislang unerzählte Geschichte eines Science-Fiction-Filmprojekts mit dem Titel \glqq Der Weltraumsand\grqq\ und für das er vor einiger Zeit vor der Kamera gestanden hatte. Bart und Lisa, die den Podcast hören, erfahren auf diese Weise, dass ihre Eltern in den frühen Tagen ihrer Beziehung in Mexiko als Produktionsassistenten für den Film tätig waren. Dabei wurde Homer entführt.

\notiz{
\begin{itemize}
  \item Vor dem Filmdreh hat Herman noch beide Arme. Am Filmset ist er zu sehen, wobei hier der rechte Arm fehlt. Er muss diesen wohl während des Filmdrehs verloren haben.
  \item Der Seekapitän McCallister verliert während des Filmdrehs den rechten Unterschenkel.
  \item Krusty drehte den Film \glqq Bad Cop, Dog Cop\grqq.
\end{itemize}}

\subsection{Der Prozess}\label{YABF07}
Homer Simpson muss sich vor Gericht verantworten. Ihm wurde nach einem Restaurantbesuch das falsche Auto ausgehändigt und er machte damit eine Spritztour. Der Comicbuchverkäufer sieht das anders und bezichtigt ihn daraufhin des Diebstahls. Da er den Vorsitzenden vor Gericht nicht von seiner Unschuld überzeugen kann, muss er jetzt einen Weg finden, sich mit dem Bestohlenen wieder zu versöhnen. Mithilfe von Lisa versucht er sich an einem Entschuldigungsvideo.

\notiz{
\begin{itemize}
  \item Homer lernte auf Highschool Finnisch.
  \item Gil führt sein Zittern auf einen Vitamin-E-Mangel zurück.
  \item Richter Synder ist geschieden.
  \item Der Comic-Buchladen in Ogdenville, in welchem Homer die Erstausgabe des Radioactive-Man-Comics kaufte, heißt \glqq Zorbot the Geek\grqq.
\end{itemize}}

\subsection{Mission Simpossible}\label{YABF08}
An der Schule in Springfield findet ein Eltern-Informationsabend zum Thema Drogen und Alkohol statt. Obwohl die Teilnahme verpflichtend ist, verlassen Homer und Marge die Veranstaltung vorzeitig und besuchen stattdessen eine Hochzeitsmesse im gleichen Gebäude. Der beinahe perfekte Abend endet schmerzhaft. Als Homer seine Frau die Treppen zum Schlafzimmer hochträgt, verliert er das Gleichgewicht und stürzt mit ihr in die Tiefe. Marge verstaucht sich dabei den Knöchel und Homer erleidet einen Leistenbruch. Marge erholt sich schnell und probiert sich am Kitesurfen. Homer kommt nicht so schnell wieder auf die Beine. Lisa versucht, die Beziehung ihrer Eltern wieder ins Gleichgewicht zu bekommen und sucht dabei Rat bei einer ungewöhnlichen Quelle.

\notiz{
\begin{itemize}
  \item Der Tafelanschrieb ist in dieser Folge eine Zeichnung.
  \item Lenny findet seine leibliche Mutter, doch diese will mit ihm nichts zu tun haben.
\end{itemize}
}

\subsection{Friss meine Sports}
Um seinen Sohn von der Straße zu holen, kauft ihm Homer einen Gaming-Computer. Bald schon ist Bart süchtig nach Computerspielen. Erst hat Homer ein schlechtes Gewissen. Als er allerdings erfährt, dass man in der Gaming-Szene viel Geld verdienen kann, sind seine Bedenken wie weggeblasen und Homer wird Barts Trainer. Marge freut es, dass Vater und Sohn auf einmal so viel Zeit miteinander verbringen. Die neue Harmonie zwischen den beiden hat allerdings auch ihre Schattenseiten. 

\notiz{
\begin{itemize}
	\item Das Gaming-Team \glqq Evergreen Terrors\index{Evergreen Terrors}\grqq\ besteht aus Bart, Nelson, Sophie, Milhouse und Martin. Sie spielen das Computer-Spiel \glqq Conflict of the Enemies\index{Conflict of the Enemies}\grqq.
	\item Homer trinkt aus einer Tasse, auf welcher der Schriftzug \glqq Dad $>$ Mom\grqq\ zu lesen ist.
\end{itemize}
}

\subsection{Bart gegen Itchy \& Scratchy}
Bei seiner alljährlichen Pressekonferenz verkündet Krusty eine wesentliche Änderung bei \glqq Itchy und Scratchy\grqq. Ab sofort spielen nur noch weibliche Charaktere in dem Cartoon mit. Entsetzt schwört Bart, sich die Show nie mehr anzusehen und gründet einen Männerverein. Als Lisa ihren Bruder jedoch dabei filmt, wie er \glqq Itchy und Scratchy\grqq\ immer noch sieht und sich dabei sogar amüsiert, wird er aus seiner eigenen Gruppe verstoßen. Daraufhin schließt er sich einer Gruppe Sechstklässlerinnen an, die Verbrechen gegen die männliche Vorherrschaft verüben: Bossy Riot\index{Bossy Riot}.

\notiz{Das Lied, das am Ende gespielt wird heißt \glqq Extreme Ways\grqq\ und ist von Moby\index{Moby}.}

\subsection{Harmonie vs. Philharmonie}
Ganz zum Unmut ihres Musiklehrers, Dewey Largo, wird Lisa bei einem Auftritt mit der Schulband von einem Talentscout entdeckt. Fortan soll sie im renommierten Capitol City Jugendorchester mitspielen. Für Lisa geht ein großer Traum in Erfüllung! Um die musikalische Förderung ihres Kindes bewerkstelligen zu können, muss der Rest der Familie Simpsons jedoch große Opfer bringen. Dem nicht genug, entpuppt sich der Leiter des Orchesters als gnadenloser Schleifer.

\notiz{
\begin{itemize}
  \item Das Drehbuch dieser Folge stammt von Nancy Cartwright, der Originalstimme von Bart.
  \item Gil arbeitet im Atomkraftwerk.
\end{itemize}
}

\subsection{Oklahoma}
Als im städtischen Theater der Regisseur aufgrund seiner Launen von den Schauspielern kurzerhand gefeuert wird, darf Marge an dessen Stelle treten. Als Newcomerin im Showbusiness fällt es ihr jedoch nicht leicht, sich durchzusetzen. Widrige Umstände erschweren am laufenden Band Marges Inszenierungsarbeit. Zusammen mit der kleinen Maggie wird Homer unterdessen begeisterter Besucher einer Krabbelgruppe, die von der hübschen Betreuerin Chloe geleitet wird.

\notiz{Sideshow Mel gibt an, kinderlos zu sein. In der Episode \glqq Alles über Lisa\grqq\ (siehe \ref{KABF13}) war er verheiratet und seine Frau war schwanger.}

\subsection{Oh, Kanada}
Während eines Familienausflugs der Simpsons zu den berühmten Niagarafällen ereignet sich ein verhängnisvoller Zwischenfall: Lisa stürzt beim Spielen mit Bart ins Wasser und wird in Kanada an Land gespült. Die Kanadier halten Lisa für einen Flüchtling und gewähren ihr bereitwillig Asyl. Marge macht sich auf dem Weg, um Lisa aus Kanada zurückzuholen. Doch zunächst müssen die beiden ihre Liebe für die USA zurückgewinnen.

\notiz{
\begin{itemize}
  \item Laut Ned Flanders Reisepass wurde er im Dezember in Springfield geboren. Der Geburtstag und das Geburtsjahr sind im Reisepass nicht abgedruckt.
  \item Dem Musiker Bruce Moss\index{Moss!Bruce} boten die Produzenten 20.000 US Dollar an, um sein Lied \glqq The Islander\grqq\ verwenden zu dürfen. Er lehnte ab, weil er die Serie für \glqq moralisch bankrott\grqq\ hält (siehe \cite{CBS2019}).
\end{itemize}
}

\subsection{Kriminalakte Springfield}
In seiner Box unter der Spüle bewahrt Marge das College-Geld für Lisa auf. Ihr Entsetzen ist groß, als davon auf einmal jede Spur fehlt. Bei der Ergreifung des Täters erhalten die Simpsons ungewöhnliche Hilfe. Die TV-Show \glqq Kriminalakte Springfield\grqq\ will sich des Falles annehmen und den Diebstahl aufklären, wenn sich die Simpsons im Gegenzug bereiterklären, den Serienmachern Einblick in ihr Familienleben zu gewähren.

\notiz{In Springfield gibt es einen aktiven Vulkan.}

\subsection{Kristallblaue Versuchung}
Ausgerechnet als Bart dringend ein Mittel gegen ADHS braucht, streicht Mr. Burns aus Kostengründen die Krankenversicherung für die Kinder seiner Angestellten. In ihrer Not steigt Marge auf Heilkristalle für ihren Sohn um. Wider Erwarten hat sie damit anscheinend auch großen Erfolg. Beflügelt von deren magischer Wirkung der Steine beschließt Marge kurzerhand, ganz groß in den Esoterik-Kristallhandel einzusteigen.

\notiz{
\begin{itemize}
  \item Marges Laden, in dem sie die Kristalle verkauft, heißt Murmur\index{Murmur}.
  \item Ned Flanders arbeitet als Grundschullehrer.
\end{itemize}}

\section{Staffel 31}

\subsection{Der Winter unseres monetarisierten Vergnügens}
Homer ist peinlich berührt, als er von Sportreporter Anger Watkins\index{Watkins!Anger} wegen einer kleinen Wissenslücke vor laufender Kamera vorgeführt wird. Verärgert befolgt er Marges Rat und entwickelt seine eigene Sportsendung fürs Internet. Als stattdessen jedoch ein Video online geht, auf dem sich Homer und Bart prügeln, treffen Vater und Sohn damit unbewusst den Nerv der Zeit. Binnen kurzer Zeit erlangen die beiden Kultstatus im Netz und träumen schon vom großen Geld. Doch der Ruhm hat seinen Preis. Währenddessen macht Lisa gegen das neue Nachsitzsystem in ihrer Schule mobil.

\notiz{
\begin{itemize}
  \item Beim Werfen der Kroketten ist folgende Gleichung zu sehen: $Y = -aX^2 + bX + c$
  \item Unter den Schildern, welche die Lehrer stanzen, befindet sich ein Schild mit der Aufschrift \glqq Bort\grqq\index{Bort}.
  \item Homers Internet-Show heißt \glqq Walkoff Homer\grqq.
\end{itemize}
}

\subsection{Der Mentor}
Homer bekommt erstmals Lust zu arbeiten. Schuld daran ist der 35-jährige Jungunternehmer Mike Wegman\index{Wegman!Mike}. Seit Jahren eifert er Homer nach. Um endlich sein großes Idol kennenzulernen, hat Mike keine Mühen gescheut und sogar einen unterbezahlten Job als Praktikant in Mr. Burns Atomkraftwerk angenommen. Da trifft es sich gut, dass Homer gerade zum Supervisor der Atomkraftwerkspraktikanten degradiert wurde. Doch überall, wo die beiden auftauchen, hinterlassen sie einen Scherbenhaufen.

\notiz{
Es sind u.\,a. folgende Food Trucks auf dem Autofriedhof zu sehen:
\begin{itemize}
  \item Chicken Pot Arnie's
  \item Cletus' Chicken Thumbs
\end{itemize}
}

\subsection{San Castellaneta}
Während eines Straßenfestes zu Ehren der italienischen Kultur ereignen sich mehrere Diebstähle. Homer soll mithelfen, die Übeltäter zu schnappen. Sein Hintern ist der größte in der Stadt und bietet somit den idealen Köder für Taschendiebe, die es auf Geldbörsen in Gesäßtaschen abgesehen haben. Kurzerhand wird Homers Brieftasche mit einem Peilsender versehen. Das ausgeklügelte Manöver läuft jedoch rasch aus dem Ruder.

\notiz{
\begin{itemize}
  \item Clancy Wiggum ist 38 Jahre alt.
  \item Die Webseite der \glqq Legitimate Businessmen's Club Security Film\grqq\ lautet \href{http://www.lbcsf.com/videos}{\url{http://www.lbcsf.com/videos}}.
\end{itemize}
}

\subsection{Episode 666}\label{YABF18}
In der Eröffnungssequenz zeigt sich Maggie von ihrer dämonischen Seite.
\begin{itemize}
  \item \textbf{Danger Things}\\ Bart, Milhouse und Nelson machen sich in einer \glqq Stranger Things\grqq-Parodie auf die Suche nach ihrem Freund Milhouse. Dieser wird in einer anderen Dimension gefangen gehalten.
  \item \textbf{Der Himmel wischt nach rechts}\\ Homer stirbt, wird aber zurück auf die Erde geschickt und kann fortan in einem anderen Körper weiterleben. Er entscheidet sich zunächst für einen Footballspieler und versucht, Marges Herz erneut zu gewinnen.
  \item  \textbf{Haarig und schleimig}\\ Selma findet endlich die große Liebe: Kodos, den Zerstörer, der von Mr. Burns im Atomkraftwerk gefangen gehalten wird. Dieser scheint der perfekte Partner für sie zu sein.
\end{itemize}

\notiz{
\begin{itemize}
  \item Unter den Kandidaten, die Homer auswählen darf, befindet sich auch Prof. Farnsworth\index{Farnsworth} aus der Serie \glqq Futurama\grqq\index{Futurama}.
  \item Am 28.10.2021 strahlte der ORF diese Episode mit einer österreichischen Synchronisation aus.
\end{itemize}
}

\subsection{Gorilla Ahoi!}\label{YABF20}
Die Simpsons sind zu Besuch im Wasserpark \glqq Aquatraz\index{Aquatraz}\grqq. Homer wünscht sich von ganzem Herzen ein Boot und ein gewiefter Verkäufer schafft es, ihm einen alten Kübel aufzuschwatzen. Die Freude darüber währt nicht lange, denn die Reparaturkosten übersteigen Homers Budget um ein Vielfaches. Lisa hat Mitleid mit einem Orcawal. Sie beschließt, ihn in die Freiheit zu entlassen. Bart hilft ihr dabei und ist so begeistert von seiner guten Tat, dass er weitere wilde Tiere befreien will.

\notiz{
\begin{itemize}
  \item Willie arbeitet im Sommer im Wasserpark Aquatraz.
  \item Hans Maulwurf gibt an, 88 Jahre alt zu sein.
\end{itemize}
}

\subsection{Stammesfehde}
Lisa führt in der Schule ein selbst geschriebenes Stück auf. Es handelt von ihrer Familie, in der Marge als schrecklich langweilig gezeichnet wird. Marge stellt ernüchtert fest, dass sie generell als uninteressant wahrgenommen wird. Sie beschließt, das zu ändern und sucht sich ein neues Hobby, das Holzhacken. So lernt sie Paula kennen, eine Athletin, die Timbersports\index{Timbersports} betreibt. Schon bald präsentiert sich Marge an Paulas Seite als wahrer Champion.

\notiz{
\begin{itemize}
  \item In der Episode ist das Lied \glqq Ace of Spades\grqq\ von Motörhead\index{Motörhead} zu hören.
  \item Ned Flanders, Willie, Nelson und Herman nehmen am Springfielder Timbersports-Wett\-kampf teil.
  \item Jeff Albertson ist oft in Portland, um sich zu erholen.
\end{itemize}
}

\subsection{La Pura Vida}\label{ZABF03}
Marge holt Bart von den Van Houtens ab, wo er übernachtet hat. Luann lädt die komplette Familie Simpson ein, mit ihnen einen Urlaub in Costa Rica zu verbringen. Auf dem Flughafen stellen die Simpsons fest, dass sie nicht die einzige Familie aus Springfield ist, die mit auf die Reise geht. Lisa will herausfinden, wieso sich die Van Houtens eine derart kostspielige Reise leisten können.

\notiz{Lou hat eine Schwester, die ein Windspielgeschäft betreibt.}

\subsection{Das perfekte Dinner}\label{YABF17}
Diese Thanksgiving-Folge besteht aus drei Teilepisoden:
\begin{itemize}
  \item \textbf{Truthahn-Kalypse}\\ Die Simpsons sind eine Truthahnfamilie, die von den Pilgervätern gnadenlos gejagt werden. Bei dem Versuch, Maggie zu retten, wird Truthahn Homer von Wiggum gefangen genommen.
  \item \textbf{Der vierte Donnerstag nach morgen}\\ Statt Marge soll sich eine Küche mit künstlicher Intelligenz um das diesjährige Thanksgiv\-ing-Menü kümmern. Doch die KI entwickelt schnell ein gefährliches Eigenleben.
  \item  \textbf{Das letzte Thanksgiving}\\ Bart und Milhouse gehören zu den letzten Menschen, die auf einem Raumschiff in eine neue Welt fliegen. Als sie aus dem Hyperschlaf erweckt werden, beschließt Bart, ein große Party an Bord zu schmeißen. Doch durch einen Unfall schaffen sie ein unglaubliches Monster.
\end{itemize}

\notiz{
\begin{itemize}
  \item Diese Folge ist meiner Laufzeit von 24 Minuten und 52 Sekunden die bisher längste Einzelepisode der Simpsons.
  \item Am 29.10.2021 strahlte der ORF diese Episode mit einer österreichischen Synchronisation aus.
\end{itemize}
}

\subsection{Mein Todd. mein Todd. warum hast Du mich verlassen?}
Todd Flanders vermisst seine verstorbene Mutter und er ist darüber so traurig, dass er aufhört, an Gott zu glauben. Das stürzt seinen Vater Ned in eine tiefe Verzweiflung. Als er sich nicht weiter zu helfen weiß, schickt er seinen Sprössling zu den Simpsons, damit er dort die Ehrfurcht vor dem Herrn wieder lernt. Todds Anwesenheit geht allen Familienmitgliedern allerdings bald gehörig auf die Nerven.

\notiz{
\begin{itemize}
  \item Im gesamten Vorspann geht es um die Familie Flanders. Daher landet man am Ende des Intros auch nicht wie gewohnt im Wohnzimmer der Simpsons, sondern in jenem der Flanders.
  \item Reverend Lovejoy hat die Telefonnummer 555-6542.
  \item Hans Maulwurf ist Fahrer bei Uber\index{Uber}.
\end{itemize}
}

\subsection{Bitte lächeln!}\label{ZABF05}
Lisa bekommt von ihrer Kieferorthopädin Dr. Kidzrule eine neue Zahnspange eingesetzt. Die Drähte ziehen die Lippen dabei so weit hoch, dass Lisa nun permanent lächeln muss. Erfreut stellt sie fest, dass sie das beliebter macht und so kandidiert sie als Schulsprecherin. Indes hat Artie Ziff, ein alter Bekannter, Homer und Marge zu seiner Hochzeit eingeladen. Sie fahren mit gemischten Gefühlen hin, da Artie einst versucht hat, Homer Marge auszuspannen.

\notiz{
\begin{itemize}
  \item Lisa hatte bereits in der Folge \glqq Prinzessin von Zahnstein\grqq\ (siehe (\ref{9F15}) eine Zahnspange.
  \item Sideshow Mel hat einen unehelichen Sohn, der auf die Grundschule in Springfield zur Schule geht.
\end{itemize}
}

\subsection{Mensch gegen Maschine}\label{ZABF06}
Captain McCallister findet im Meer einen großen Schatz, den die Stadt Springfield für sich beansprucht. In der Stadtversammlung wird entschieden, davon eine neue Schule zu bauen. Schon bald finden sich Lisa, Bart und ihre Freunde in einem hochmodernen Gebäude wieder, in dem ihnen ein Algorithmus zukunftsorientiertes Lernen vermitteln soll, um sie so auf die Jobs von morgen vorzubereiten. Schließlich findet Lisa jedoch heraus, dass ihre Perspektiven alles andere als rosig sind.

\notiz{
\begin{itemize}
  \item Captain McCallister ist verheiratet und kinderlos.
  \item In der neuen Schule empfiehlt der Algorithmus den hochbegabten Schülern \glqq Programmierung in C+++\grqq. Hier ist auch ein Lehrer zu sehen, der seinen Schülern die Riemannsche Vermutung\index{Riemannsche Vermutung}\footnote{Die Riemannsche Vermutung oder Riemannsche Hypothese ist eine Annahme über die Nullstellen der Riemannschen Zetafunktion. Das Institut in Cambridge (Massachusetts) hat ein Preisgeld von einer Million US-Dollar für eine schlüssige Lösung des Problems in Form eines mathematischen Beweises ausgelobt.} erklärt.
\end{itemize}}

\subsection{Frinkcoin}\label{ZABF07}
Lisa beschließt, den Schulaufsatz zum Thema \glqq Die interessanteste Person, die ich kenne\grqq\ über Professor Frink und sein neues Projekt -- eine Kryptowährung namens \glqq Frinkcoin\grqq\ -- zu schreiben. Nach Jahren der Erfolglosigkeit wird der Professor über Nacht zum reichsten Mann der Stadt -- ganz zum Missfallen von Mr. Burns. Doch auch der neue Ruhm kann Frinks Niedergeschlagenheit nicht vertreiben. Lisa schaltet Homer ein, um dem einsamen Wissenschaftler zu helfen. 

\notiz{
\begin{itemize}
  \item Professor Frink behauptet, seine Eltern seien beide Chemiker gewesen.
  \item Waylon Smithers behauptet, allergisch gegen Krustentiere zu sein.
  \item Patty und Selma sagen, sie rauchen nicht mehr seit sie zum Disney-Konzern\index{Disney} gehören und setzen sich dabei Micky-Maus-Ohren auf.
  \item Laut Mr. Burns hat Springfield 32.000 Einwohner.
  \item Gemäß dieser Episode basiere die Sicherheit aller Kryptowährungen auf der Sicherheit des Hash-Verfahrens SHA-256\index{SHA-256}. Die Software-Entwickler, die an Burns Kryptowährung arbeiten, schätzen, dass sie 90.000 Jahre brauchen, um die Unsicherheit von SHA-256 zu beweisen.
\end{itemize}}

\subsection{Spoiler Boy}
Mehr oder weniger durch Zufall hat Bart die noch unveröffentlichte Fortsetzung eines beliebten Superhelden-Films gesehen. Er nutzt nun sein Wissen, um die Fangemeinde auf die Folter zu spannen und droht, den Inhalt des Films zu verraten. Dann rücken ihm jedoch die Produzenten des Streifens auf die Pelle, die verhindern wollen, dass Bart der Öffentlichkeit noch mehr über den Film preisgibt.

\notiz{
\begin{itemize}
  \item Rektor Skinner trägt ein Toupet.
  \item Küchenhilfe Doris ist die Freundin von Moe.
\end{itemize}}

\subsection{Bildschirmlos}\label{ZABF09}
Marge ist der Ansicht, dass die gesamte Familie zu viel Zeit vor dem Bildschirm verbringt. Vertieft in eine Fernsehshow übersehen sie, daas ein Nachbar unter einem umgefallenen Baum in ihrem Garten feststeckt, daher verhängt sie kurzerhand ein Bildschirmzeitlimit. Die neu gewonnene Zeit fördert bei allen Familienmitgliedern bald schon ungewohnte neue Ambitionen und Hobbys zu Tage. Leider muss sie feststellen, dass ausgerechnet sie mit dem Verzicht überhaupt nicht zurechtkommt und besteht darauf, dass sich die Familie einem Entzugsprogramm stellt.

\notiz{Chief Wiggum hat bisher von männlichen Verdächtigen eine Samenprobe genommen, um Material für eine DNA-Analyse zu erhalten.}

\subsection{Die Bart Verschwörung}
Bart steht kurz vor einem Schulverweis, nachdem sein letzter Streich das Fass endgültig zum Überlaufen gebracht hat. Ned Flanders bittet um eine weitere Chance für Bart und bietet an, mit ihm zu arbeiten. Zur Überraschung aller bewirkt er schon bald eine dauerhafte Veränderung in Barts Verhalten. Homer fühlt sich nutzlos und macht sich seinerseits auf die Suche nach jemandem, der Hilfe braucht. Auf der Müllkippe trifft er auf den verzweifelten Nelson und nimmt sich seiner an.

\notiz{
\begin{itemize}
  \item Lisa ist Mitglied im Französisch-Club der Schule.
  \item Die Episode wurde dem Schauspieler Max von Sydow gewidmet.
\end{itemize}
}

\subsection{Highway to Well}
Marge beschließt spontan, am Einstellungstest eines Ladens namens \glqq Well + Good\grqq\ teilzunehmen. Prompt sticht sie die anderen Bewerber aus und bekommt den Job. Als sie erfährt, dass sie Cannabis verkaufen soll, zweifelt sie an ihrer Entscheidung, doch die Familie überzeugt sie, die Arbeit anzutreten. Mit Marges steigendem Erfolg erkennt auch Homer das Potential des Rauschmittelhandels und macht \glqq Well + Good\grqq\ bald mit einem anderen Geschäftsmodell Konkurrenz.

\notiz{Agnes Skinner besitzt ein Apple iPhone.}

\subsection{Maggies erste Liebe}\label{YABF13}
Mr. Burns hat mal wieder einmal einen gemeinen Auftrag für Homer. Er soll Cletus Spuckler seine Helium-Quellen abjagen. Derweil begegnet Maggie ihrer ersten großen Liebe. Sein Name ist Hudson. Sein großer Nachteil ist allerdings seine kontrollsüchtige Mutter. Marge versucht, großzügig darüber hinwegzusehen, doch das gelingt leider nicht sehr lange und sie verlässt verärgert den Schauplatz des Geschehens.

\notiz{
\begin{itemize}
  \item Cletus ist Vater von 31 Kindern. In der Episode \glqq Stille Wasser sind adoptiv\grqq\ (siehe \ref{RABF04}) hatte er 17 Kinder.
  \item Homer ist Mitarbeiter des Monats.
  \item Der Couch-Gag \glqq The Extremesons\grqq\ wurde von Michal Socha animiert und von Ron Diamond produziert.
\end{itemize}}

\subsection{Erbarmungslos}
Lisa wird zur Geburtstagsparty ihrer neuen Freundin Addy\index{Addy} eingeladen, die sich als sehr reich herausstellt. Zunächst erscheint das Anwesen der Familie mit seiner privaten Pferdekoppel Lisas wahrgewordener Traum zu sein, doch Addys arrogante Freundinnen nutzen jede Gelegenheit, um sich über sie lustig zu machen. Als Addys wahre Absichten ans Licht kommen, sieht Lisa nur noch einen Ausweg: Bart muss ihr helfen. Homer geht währenddessen mit Marge auf eine romantische Kreuzfahrt.

\notiz{
\begin{itemize}
  \item Rektor Skinner fährt in seiner Freizeit für den Fahrdienst lift\index{lift}.
  \item Bart nennt sich Juan Starnotip\index{Starnotip!Juan}, als er Fahrgast bei Rektor Skinner ist.
  \item Grampa nutzt das Mobiltelefon Geezerbug\index{Geezerbug}, welches eine Anspielung auf das Seniorentelefon Jitterbug\index{Jitterbug} ist.
\end{itemize}}

\subsection{Krieg der Priester -- Teil 1}
Der junge Prediger Bode Wright kommt von Michigan nach Springfield und macht sich schnell beliebt bei den Mitgliedern der Gemeinde. Es gelingt ihm sogar, neue Gläubige zu gewinnen. So ist auch Lisa Simpson von Bode Wright begeistert. Die Beliebtheit des jungen Predigers stößt allerdings nicht bei allen Gemeindemitgliedern auf Wohlwollen. Schon bald stehen sich zwei Gruppen unversöhnlich gegenüber.

\notiz{Die beiden Episoden wurden vom Gaststar Pete Holmes geschrieben.}

\subsection{Krieg der Priester -- Teil 2}
Bode Wright kann nicht alle Mitglieder der Gemeinde hinter sich versammeln. So tut sich unter anderem auch Ned Flanders schwer mit dem neuen Prediger. Reverend Lovejoy, dem Wright ebenfalls ein Dorn im Auge ist, sucht in Michigan, von wo sein Konkurrent nach Springfield gekommen war, nach möglichen Verfehlungen des Predigers, um ihn zu diskreditieren. Seine Recherchen haben tatsächlich Erfolg. Als die Springfielder von seinem Verbrechen erfahren, müssen sie entscheiden, ob sie ihren neuen Prediger verbannen wollen oder nicht.

\notiz{
\begin{itemize}
  \item Sideshow Mel wünscht sich Bode Wright als Taufpaten für seinen Sohn.
  \item Marge nennt Lisa bei ihrem vollen Namen: Lisa Marie Simpson.
\end{itemize}}

\subsection{Süßer die Glocken nie tingeln}\label{ZABF0}
Kurz vor Weihnachten erschüttert eine Serie von Diebstählen Springfield. Mehreren Bewohnern werden die Geschenke von der Veranda gestohlen. Beim Versuch, den Täter zu überführen, hinterlässt Lenny den Hinweis auf einen gewissen \glqq SB\grqq . In Springfield geht daraufhin eine Hetzjagd auf alle mit diesen beiden Anfangsbuchstaben los. Auch Bart hat einen Verdacht. Er glaubt, dass Sideshow Bob hinter der Diebstahlserie steckt. Dieser hat im Vergnügungspark gerade die Rolle des Weihnachtsmannes übernommen.

\notiz{
\begin{itemize}
  \item Mr. Burns hat von seinen Eltern 100 Milliarden Dollar geerbt.
  \item Waylon Smithers hat an Weihnachten Geburtstag.
  \item Sideshow Bob hat folgende Lebensziele:
  \begin{itemize}
    \item Bis 67 arbeiten
    \item In Rente gehen
    \item Mit altersgemäßen Frauen ausgehen
    \item Chinesisch lernen
    \item Leiblichen Vater finden
  \end{itemize}
  \item Homer trinkt Holi-Daze von Duff.
\end{itemize}}

\subsection{Der Weg des Hundes}\label{ZABF16}
Knecht Ruprecht, der Hund der Simpsons, wird an seine traumatische Vergangenheit erinnert, als Marge ihm eine Weihnachtsmütze aufsetzen will. Nachdem er das halbe Wohnzimmer verwüstet hat, schlägt Lisa vor, die Hundepsychologin Elaine Wolff\index{Wolff!Elaine} um Hilfe zu bitten, die jedoch ablehnt. Entschlossen, ihrem Hund zu helfen, versucht die Familie alles, um ihn vor dem Einschläfern zu bewahren, doch schließlich bleibt Elaine Wolff ihre einzige Hoffnung. Sie findet heraus, dass er unter einer posttraumatischen Belastungsstörung leidet, da er als Welpe seiner Mutter weggenommen wurde.

\notiz{
\begin{itemize}
  \item Marge sagt, die Simpsons besäßen keinen HD-fähigen Fernseher. Der Fernseher aus der Folge \glqq Weihnachten in Florida\grqq\ (siehe \ref{YABF02}) sollte allerdings HD-fähig sein.
  \item In der Folge sind Szenen aus der Episode \glqq Es weihnachtet schwer\grqq\ (siehe \ref{7G08}) zu sehen.
  \item Am 23.12.2021 strahlte der ORF diese Episode mit einer österreichischen Synchronisation aus.
\end{itemize}}


\section{Staffel 32}

\subsection{Undercover Burns}
Homers Chef Mr. Burns merkt, dass ihn seine Angestellten für einen Tyrannen halten. Er mischt sich unter dem Namen Fred Kranepool\index{Kranepool!Fred} unter die Mitarbeiter, zeigt sich von einer ungewohnt sympathischen Seite und schließt sogar Freundschaft mit Homer, Carl und Lenny. Doch sein Verhalten kommt nicht überall gut an. Nachdem seine wahre Identität gelüftet wurde, fällt Mr. Burns in seine alten Gewohnheiten zurück.

\notiz{
\begin{itemize}
  \item Dies ist die erste Folge, in welcher im Original der schwarze Schauspieler Alex Désert Carl Carlson spricht. Vorher wurde dieser von Hank Azaria gesprochen.
  \item Marge erwähnt, dass Lady Gaga einst Springfield besucht hat und dabei Lisa inspiriert hat. Sie bezieht sich dabei auf die Episode \glqq Lisa wird gaga\grqq\ (siehe \ref{PABF14}).
\end{itemize}
}

\subsection{Bartigula}
Die Simpsons besuchen gemeinsam eine Ausstellung über das antike Rom. Als Marge bemerkt, dass sich Homer im Museum furchtbar langweilt, bricht ein Streit zwischen den Eheleuten los. Marge beklagt sich lautstark über Homers Faulheit. Erfolgreich hat er sich bisher vor jeder Beförderungschance gedrückt. Dies nimmt der Kurator der Ausstellung zum Anlass, der Familie die Geschichte vom armen römischen Bauernsohn Adipösius\index{Adipösius} dem Breiten und seiner ehrgeizigen Gattin Marjora zu erzählen.

\notiz{
\begin{itemize}
  \item Bei Barts römischen Namen, Bartigula, handelt es sich um eine Anspielung auf den römischen Kaiser Gaius Caesar Augustus Germanicus, posthum bekannt als Caligula\index{Caligula}\footnote{Gaius Caesar Augustus Germanicus (geb. 31. August 12 in Antium als Gaius Iulius Caesar; gest. 24. Januar 41 in Rom), war von 37 bis 41 römischer Kaiser. Caligulas Jugend war von den Intrigen des ehrgeizigen Prätorianerpräfekten Seianus geprägt. Nach hoffnungsvollem Regierungsbeginn, der durch persönliche Schicksalsschläge getrübt wurde, übte der Kaiser seine Herrschaft zunehmend als autokratischer Monarch aus und ließ in Hochverratsprozessen zahlreiche Senatoren in willkürlicher Ausschöpfung seiner Amtsgewalt zum Tode verurteilen. Seine Gewaltherrschaft endete mit seiner Ermordung durch die Prätorianergarde und Einzelmaßnahmen zur Vernichtung des Andenkens an den Kaiser.}.
  \item In dieser Episode wird der Kurator von Michael Palin, einem Mitglied von Monty Python\index{Monty Python}, gesprochen. Die Simpsons-Macher hatten erfolglos versucht, weitere Monty Python Mitglieder zu Gastauftritten zu überreden \cite{BT202}.
\end{itemize}
}

\subsection{Kunst ist, wenn man trotzdem lacht}
Während Lisa vom Schulunterricht fernbleiben muss, liest sie ein Buch über westliche Kunst. Von ihrer Lektüre zum Träumen verführt, beginnt sie sich ein Leben als Lisanardo da Vinci\index{da Vinci!Lisanardo} in der italienischen Renaissance vorzustellen. Als sie für die Polizei das Bild eines Verdächtigen malen muss, wird ihre Begabung entdeckt und sie steigt zur berühmtesten Künstlerin in Florenz auf. Doch der Ruhm hat auch Schattenseiten. Besorgt macht Lisa sich alsbald Gedanken darüber, vielleicht zu talentiert zu sein. Währenddessen nimmt Bart das Aussehen eines französischen Impressionisten an. Homer und Marge werden zum Künstlerpaar Diego Rivera\index{Rivera!Diego} und Frida Kahlo\index{Kahlo!Frida}. Maggie ist Amor und tötet versehentlich mit einem Pfeil Homer, der anschließend in die Hölle absteigt.

\notiz{
In dieser Folge wird im Original der Bienenmann erstmals von Eric Lopez gesprochen. Von der vierten Staffel an wurde er von Hank Azaria gesprochen.
}

\subsection{Das verflixte siebente Bier}
Marge und die Kinder machen alleine Urlaub am Meer in Neuengland. Homer bleibt allein zu Hause zurück. Weil er sich einsam fühlt, sucht er Trost und Ablenkung in Moes Kneipe. Dort lernt er eine bildschöne Engländerin namens Lily kennen. Die sehr einfühlsame Lily macht dem vereinsamten Familienvater nicht nur schöne Augen, sie scheint sich regelrecht in Homer verliebt zu haben.

\notiz{
\begin{itemize}
  \item In London ist der Pub The Brexiting Swan zu sehen.
  \item Lily singt das deutsche Soldatenlied \glqq Lili Marleen\grqq.
\end{itemize}
}

\subsection{Podcast News}
Marge und ihre Tochter Lisa sind ganz verrückt nach den beliebten Podcasts, welche die Geschichten wahrer Kriminalfälle erzählen. Als sie selbst in einen mysteriösen Fall geraten, denn Grampas neue Freundin Vivienne verschwindet, halten sie sich für Expertinnen. Auch Kent Brockman widmet sich diesem Fall mit einem Podcast und macht ihn in ganz Springfield bekannt. Grampa landet im Gefängnis. Er kann sich an nichts erinnern, doch Dr. Hibbert hat wichtiges Wissen, mit dem er Abe Simpsons Unschuld beweisen will. Abe und seine Partnerin hatten im Rahmen einer medizinischen Untersuchung Mikrosender geschluckt, anhand derer Dr. Hibbert feststellen kann, dass Vivienne sich abgesetzt hat. Am Ende stellt sich heraus, dass Vivienne ihren Tod deswegen vorgetäuscht hat, um mit dem Geld, das die Versicherung in dem Fall bezahlen würde, zusammen mit Abe nach Mexiko zu gehen.

\notiz{
\begin{itemize}
  \item Die Episode wurde Alex Trebek\index{Trebek!Alex} gewidmet.
  \item Laut Kent Brockman ist Agnes Skinner eine in Ungnade gefallene Lehrerin.
  \item Yeardley Smith, Lisas Stimme im Original, tritt hier als sie selbst als Gastgeberin des Podcasts \glqq Small Town Dicks\grqq\ auf.
\end{itemize}
}

\subsection{Drei gescheiterte Träume}
Für drei Springfielder wird ihr großer Traum wahr: Bart bekommt den Job als erfolgreicher Synchronsprecher. Für den Comicbuchverkäufer Jeff geht ein lang gehegter Traum in Erfüllung. Ein lukratives Geschäft spült kräftig Geld in die Kasse, sodass Jeff erstmals in seinem Leben die Comic-Con in Kalifornien besuchen kann. Jeff hofft, bei der Gelegenheit einen Job bei den Marvel Studios zu ergattern. Der Ausflug verläuft aber nicht nach Plan. Unterdessen freut sich Lisa über die Bekanntschaft mit dem neuen Schüler Blake. Er spielt Saxophon wie sie. Schneller als gedacht entpuppt sich Blake jedoch als heimtückischer Rivale!

\notiz{Homer trinkt Duff-Tee.}

\subsection{Der lange Weg nach Cincinnati}
Als Oberschulrat Chalmers dazu auserkoren wird, beim Verwalterkongress in Cincinnati die Hauptrede zu halten, erwacht in Rektor Skinner der Ehrgeiz. Von Bart angestachelt setzt er alles daran, Chalmers nach Cincinnati zu begleiten. Er will eine Nähe zu seinem Vorgesetzten herstellen, um karrieretechnisch voranzukommen. Tatsächlich gelingt es Skinner, seinen aussichtsreichsten Konkurrenten auszutricksen, doch damit ist erst die erste Hürde genommen.

\notiz{
\begin{itemize}
  \item Die Entfernung zwischen Springfield und Cincinnati beträgt 800 Meilen.
  \item Das Lied \glqq Timothy\grqq\ von Rupert Holmes\index{Holmes!Rupert}, welches im Autoradio in einer Endlosschleife läuft, handelt von verschütteten Minenarbeitern, die in ihrer aussichtslosen Lage einen verschütteten Mann töten und aufessen.
\end{itemize}
}

\subsection{Lisa tut es nicht leid}
Lisa und Miss Hoover geraten aneinander, als Lisa für eines ihrer Schulprojekte keine Bestnote erhält. Da sie sich weigert, sich bei ihrer Lehrerin zu entschuldigen, schickt diese die Musterschülerin kurzerhand zum Nachsitzen. Erst ein überraschender Einblick in das Leben von Miss Hoover ändert Lisas Einstellung.

\notiz{Die Episode wurde von Nell Scovell\index{Scovell!Nell} geschrieben. Über 30 Jahre zuvor schrieb sie ihre letzte Simpsons-Folge \glqq Die 24-Stunden-Frist\grqq\ (siehe \ref{7F11})}

\subsection{Sommer-Weihnacht in Springfield}
Ein Filmteam kommt in die Stadt. Die Filmemacherin Mary Tannenbaum\index{Tannenbaum!Mary} soll im Auftrag des Heartmark Channels in Springfield einen Weihnachtsfilm drehen und bezieht ein Zimmer bei den Simpsons. Marge ist völlig aus dem Häuschen, sie liebt Weihnachten, während Homer ein profitables Geschäft wittert. Bald wohnt der größte Teil des Produktionsteams für 500 US-Dollar pro Nacht bei den Simpsons. Homer verkauft dem Filmteam außerdem noch selbst hergestellten künstlichen Schnee.

\notiz{
\begin{itemize}
  \item Moe ist hier als Taxifahrer zu sehen.
  \item Sideshow Bob besitzt einen Monstertruck.
\end{itemize}
}

\subsection{Vorwärts in die Vergangenheit}
Homer und Marge sind im Stress. Anstatt auszuspannen, chauffieren sie den Nachwuchs sonntags von Party zu Party. Bald steht für die zwei fest: Sie brauchen dringend eine Auszeit vom Elterndasein. Da trifft es sich gut, dass Moe in seinem Lokal gerade einen Quiz-Abend veranstaltet. Begeistert nehmen auch Marge und Homer daran teil und kommen dort mit dem Comicbuchverkäufer und seiner Frau Kumiko ins Gespräch. Ohne es zu wollen, weckt Marge in Kumiko die tiefe Sehnsucht nach eigenen Kindern.

\notiz{Der Comicbuchverkäufer ist ein Einzelkind.}

\subsection{Das Tagebuch der Mrs. K}
Ned Flanders veranstaltet einen Flohmarkt. Unter seinen ausrangierten Sachen entdecken Bart und Milhouse zwei Kartons mit alten Büchern, die sie für einen Stunt verbrennen wollen. Dabei fällt ihnen das Tagebuch ihrer verstorbenen Lehrerin Edna Krabappel in die Hände. Bart beginnt, darin zu lesen. Wegen eines Missverständnisses glaubt er, dass Edna ihn für sehr talentiert gehalten hatte, die eigentlich über ihren Kater geschrieben hatte. Er krempelt deshalb sein Leben um. Als er kurz darauf gute Noten nach Hause bringt, wird Lisa misstrauisch. Auf der Suche nach Barts Geheimnis, findet auch sie das alte Tagebuch. Sie erzählt Bart aber zunächst nichts von dem Missverständnis und bekommt einen Ausschlag. Der Ausschlag geht erst weg, nachdem sie es doch Bart erzählt hat. Dieser ist geknickt, aber Ned kann ihm sagen, dass Edna als einzige der Familie Flanders nicht von Springfield wegziehen wollte und das wegen Kinder wie Bart, die noch ihre Hilfe brauchen.

\notiz{Im amerikanischen Original spricht Harry Shearer zuletzt Dr. Hibbert. Mit der nächsten Folge übernimmt Kevin Michael Richardson die Rolle.}

\subsection{C.R.E.A.M.}
Über Ralph wird Bart auf einen Golfplatz mitten in Springfield aufmerksam. Dort sucht man immer wieder neue Caddies. Als ihm sein erster Assistenzeinsatz üppig bezahlt wird, glaubt Bart, auf eine Goldader gestoßen zu sein. In Aussicht auf ein höheres Trinkgeld fängt er an, sich bei den Golfspielern einzuschleimen. Doch Bart hat die Rechnung ohne seine Mutter gemacht. Aus Sorge, dass das viele Geld ihrem Sohn zu Kopf steigt, ruft Marge eine Online-Petition gegen den Golfclub ins Leben.

\notiz{
\begin{itemize}
  \item Diese Episode wurde dem ehemaligen Produzenten Marc Wilmore gewidmet, welcher am 30. Januar 2021 verstorben ist.
  \item Fehler: Nachdem Ralph von Nelson in einen Spind gesteckt wurde, kommt er aus einem anderen Spind zurück.
\end{itemize}
}

\subsection{Cletus 4 Ever}
Als Homer eine Nacht mit dem verarmten Farmer Cletus in der Ausnüchterungszelle verbringt, wird er gezwungenermaßen Zeuge von dessen durchdringendem Gesang. Reumütig schwört Homer, ab sofort ein besserer Ehemann und Vater zu werden. Zugleich will er Cletus dabei helfen, sein Gesangstalent zu fördern. Leider wird Homer diese großzügige Geste nicht gedankt. Kaum feiert Cletus als Country-Sänger erste, bescheidene Erfolge, strebt er eine Profi-Karriere ohne seinen Mentor an und er entfremdet sich von seiner Frau Brandine. Nach gutem Zureden von Homer und Marge kehrt er aber wieder zu seiner Frau und zu seinen Kindern zurück.

\notiz{
\begin{itemize}
  \item Lenny sagt, es habe Vorteile, alleine zu leben. Die deutet darauf hin, dass er nicht liiert oder verheiratet ist.
  \item Cletus gibt an, am MITT\index{MITT} (Mississippi Institute of Trailor Trash) studiert zu haben.
  \item Homer managte bereit die Country-Sängerin Lurleen Lumpkin\index{Lumpkin!Lurleen} in der Episode \glqq Homer auf Abwegen\grqq\ (siehe \ref{8F19}).
\end{itemize}
}

\subsection{Die Rückkehr der Pizza-Bots}\label{QABF08}
Als Homer vor dem früheren \glqq Razzle Dazzle's Pizza-Tainment Palace\grqq\ steht, überkommen ihn plötzliche wehmütige Erinnerungen. Als 16-Jähriger hätte er dort beinahe eine steile Karriere als DJ begonnen. Doch sein erster Einsatz wurde von einer Polizeirazzia unterbrochen. Alte Wunden reißen wieder auf. Ehe sich die Familie versieht, fällt Homer in ein tiefes, schwarzes Loch. Entschlossen setzen Lisa und Bart sogleich alles daran, ihren Vater wieder glücklich zu machen, indem sie die Roboter aus seiner Jugend auftreiben. Nur einer ist schwierig zu bekommen: Der Militärantiquitätenhändler Herman hat seinen nämlich an den Filmemacher J.J. Abrams verkauft. Daher besuchen Bart und Lisa ihn in seinem Springfielder Büro.

\notiz{
\begin{itemize}
  \item Gil Gunderson war Homers Vorgesetzter im Razzle Dazzle's.
  \item Ein Kind im Razzle Dazzle's trägt ein Bart-Simpson-T-Shirt.
  \item Die Befragung von Sideshow Mel durch Moe ist eine Anspielung auf die Columbo-Krimi\-rei\-he\index{Columbo}.
  \item Diese Episode wurde dem ehemaligen Autor und Produzenten David Richardson gewidmet, welcher am 18. Januar 2021 verstorben ist. 
\end{itemize}
}

\subsection{Die weiblichen Verdächtigen}\label{QABF10}
Da Homer im letzten Moment abgesagt hat, muss Marge die Kinder zu einem Wochenende auf einem ehemaligen Kriegsschiff begleiten. Dabei landet sie ausgerechnet in der Gruppe der nicht gerade mitteilsamen Sarah Wiggum, mit der zunächst kaum Konversation möglich ist. Aber dann, als die beiden Mütter gemeinsam Nachtwache halten müssen, kommt Sarah langsam aus der Reserve und die beiden Frauen werden Freundinnen, wodurch Marge von Sarahs schockierenden Geheimnis erfährt. 

\notiz{
\begin{itemize}
  \item Sarah Wiggum war Mitglied einer Diebesbande, die von Lindsey Naegle angeführt wurde.
  \item In der Folge \glqq Ein Stern wird neu geboren\grqq\ (siehe \ref{EABF08}) behauptet Clancy, dass er Sarah kennengelernt hat, indem er ihr Drogen zugesteckt hat.
\end{itemize}
}

\subsection{Saures oder Zeitschleife}
In der Eröffnungssequenz wird ein neuer Präsident gewählt. Bis auf Homer gehen auch alle zur Wahl. Er ist in der Hängematte eingeschlafen, was schwerwiegende Folgen für den Wahlausgang hat.
\begin{itemize}
  \item \textbf{Toy Horror}\\ Bart geht mit seiner neuen Radioactive Man Figur nicht behutsam um. Als Strafe erwachen alle seine Spielzeuge zum Leben und wollen Bart töten.
  \item \textbf{Geheimnisse des Homerversums}\\ Homer zerstört das Raum-Zeit-Kontinuum, dadurch sieht sich Springfield plötzlich mit vielen Homers aus verschiedenen Paralleluniversen bedroht.
  \item \textbf{Neun werden und zurückspulen}\\ Lisa stirbt durch ein Unglück an ihrem neunten Geburtstag, wird aber direkt wieder lebendig und hat die Chance den Lauf der Dinge zu verändern. Allerdings bringt sie damit ihre Familie und Freunde in Gefahr.
\end{itemize}

\notiz{In der Eröffnungssequenz tragen alle Charaktere bis auf Homer Covid19\index{Covid19}-Schutzmasken.}

\subsection{Burger Kings}\label{QABF11}
Als Mr. Burns eine scharfkantige, einzelne Rosine zum Frühstück serviert bekommt, wird er ungemütlich. Im Atomkraftwerk begegnet ihm Homer. Hungrig schnappt er ihm das Essen weg. Es ist der erste Burger in Burns' Leben. Prompt kann er nicht genug davon kriegen. Da zu viel Fleischkonsum aber ungesund ist, wird extra für Burns ein veganer Burger von Prof. Frink entwickelt, der sogar Lisa als Testesserin überzeugen kann. Burns gründet daraufhin die vegane Burger-Kette \glqq EX-CELL-ENT burger\grqq\ -- mit Homer als Star der Werbekampagne.

\notiz{Die beiden schweizerischen Illustratorinnen Katrin von Niederhäusern und Janine Wiget gestalteten den Eröffnungs-Gag. Die beiden Zürcherinnen hatten 2019 einen viralen Hit kreiert. Sie stellten eine Szene aus der Folge \glqq Lisa hat den Blues\grqq\ (siehe \ref{XABF11}) in großartiger Detailverliebtheit nach \cite{TB21}.}

\subsection{Panik auf den Straßen von Springfield}
Um nach einem deprimierenden Gesundheits-Check sein ramponiertes Ego aufzupolieren, kauft sich Homer einen PS-starken Pick-up-Truck. Zur High-End-Ausstattung gehört auch ein Gratis-Abbo eines Online-Musik-Portals. Lisa ist sofort Feuer und Flamme. Erst recht, als sie über den Streamingdienst den Brit-Pop-Künstler Quilloughby\index{Quilloughby} kennenlernt. Quilloughby war schon in den 1980er ein überzeugter Veganer und hat sich für Tierrechte stark gemacht. Seine militante Haltung färbt schnell auf Lisa ab.

\notiz{
\begin{itemize}
  \item Dr. Hibbert praktiziert auch im Springfield Medical Center.
  \item Mr. Largo ist nach eigener Aussage 1,77 m groß.
  \item Der Streamingdienst Slapify\index{Slapify} ist eine Anspielung auf Spotify\index{Spotify}.
  \item Der Frontman Quilloughby parodiert Morrissey\index{Morrissey} und The Smiths. 
  \item Diese Episode ist dem ehemaligen Animator Edwin E. Aguilar gewidmet, der am 11. April 2021 verstorben ist.
\end{itemize}
}

\subsection{Die Königin der Staaten}
Die Simpsons wollen für Lisa ein Geburtstagsgeschenk im Zauberladen besorgen. Der Ladenbesitzer will für die Simpsons die Tarotkarten legen und mehr über ihre Zukunft erzählen. Lisa wird, nachdem sie sich gegen den Willen ihrer Mutter geweigert hatte, das College zu besuchen, US-Präsidentin, Homer hört auf zu trinken und Bart führt ein zivilisiertes Leben. Wegen der Weigerung stimmt die Chemie zwischen Marge und Lisa lange Zeit nicht mehr. Letztlich ist Marge dennoch stolz auf ihre Tochter.

\notiz{
\begin{itemize}
  \item In der Episode \glqq Barts Blick in die Zukunft\grqq\ (siehe \ref{BABF13}) wird ebenfalls in einer Zukunftsvision davon erzählt, dass Lisa die erste Präsidentin der Vereinigten Staaten wird.
  \item Diese Folge ist der Schauspielerin Olympia Dukakis\index{Dukakis!Olympia} gewidmet, welche in der Folge \glqq Abraham und Zelda\grqq\ (siehe \ref{DABF09}) Zelda gesprochen hat.
\end{itemize}
}

\subsection{Codename G.R.A.M.P.A.}
Homer lernt den stark gealterten, britischen Geheimagenten Terrance kennen. Seit fünfzig Jahren jagt Terrance einem Amerikaner nach, der zu den Russen übergelaufen sein soll, bei den Fliegenden Höllenfischen war und den Tarnnamen \glqq Grey Fox\grqq\ trägt. Dabei ist Terrance nicht zufällig in Springfield gelandet. Nach einem leutseligen Abend im Moe's konfrontiert er Homer mit irritierenden Erkenntnissen. Er glaubt, \glqq Grey Fox\grqq\ sei in Wirklichkeit Abe Simpson.

\notiz{Die Fliegenden Höllenfische wurden erstmals in der Episode \glqq Simpson und sein Enkel in \glq Die Schatzsuche\grq\grqq\ (siehe \ref{3F19}) erwähnt.}

\subsection{Moe Szyslak und das Königreich des Kristallschädels}\label{QABF15}
In einer TV-Show gewinnt Bart einen mit Tequila gefüllten Kristalltotenschädel. Homer klaut den Gewinn und trinkt mit seinen Freunden und Moe mehr als nur eine Runde Edel-Tequila. Im Laufe der durchzechten Nacht stellt sich heraus, dass Moe Mitglied der geheimen Barkeeper-Gemeinschaft \glqq Confidential\index{Confidential}\grqq\ ist. Dort erzählen sich die Barkeeper die Geheimnisse ihrer Gäste. Als Moe betrunken aus dem Nähkästchen plaudert, hat das üble Folgen für alle. Er wird aus dem Geheimbund ausgeschlossen und seine Stammgäste sollen \glqq trockengelegt\grqq\ werden, also zu Nichttrinkern werden.

\notiz{
\begin{itemize}
  \item Milhouse spricht Spanisch.
  \item Dr. Hibbert verdiente sich während des Studiums Geld, indem er als Barkeeper arbeitete.
  \item Diese Episode ist eine Parodie auf die John-Wick-Filme. Die Stimme des Artemis ist Ian McShane, der in der John-Wick-Reihe Winston verkörpert.
\end{itemize}
}

\subsection{Es ist ein Todd entsprungen}\label{QABF09}
Es ist Weihnachten in Springfield. Während Marge mit ihrer Familie den Baum dekoriert, spielt Maggie mit dem Weihnachtsschmuck und beginnt, davon zu essen. Bart fällt beim Schmücken des Christbaums eine Dekoration mit der Aufschrift \glqq Todds erstes Weihnachten\grqq\ ins Auge. Sie ist der Anstoß für eine Geschichte, die sechs Jahre zurückliegt. Maggie und Todd waren zu diesem Zeitpunkt noch gar nicht auf der Welt. Homer betrinkt sich dermaßen auf einer Firmenfeier, sodass Marge ihn nicht mehr ins Haus lässt. Glücklicherweise kommt er bei seinem Nachbarn Ned Flanders unter. Maude ist nicht sehr glücklich über die Situation und schon bald beginnt Homer im Haus der Flanders Probleme zu bekommen. Maude Flanders sorgt dafür, dass Homer wieder geht. Er begibt sich zu Moe, der mit ihm einen Spaziergang unternimmt und ihm rät, im Raum unter dem Dach seiner Garage zu wohnen. Seine Familie geht ohne ihn in den Weihnachtsgottesdienst. Homer will währenddessen bei seiner Familie einen guten Eindruck hinterlassen. Er richtet gerade beim Backen Chaos in der Küche an, als er hört, wie bei Maude Flanders die Wehen einsetzen. Homer hilft ihr, ihren zweiten Sohn Todd, auf die Welt zu bringen. Homer ist nun versöhnt mit Marge, die von Homers Hilfsbereitschaft begeistert ist.

\notiz{
\begin{itemize}
  \item Es handelt sich hierbei um die 700. Folge.
  \item Der Couchgag wurde von Bill Plympton\index{Plympton!Bill} gestaltet.
  \item Maggies zweiter Vorname ist Lenny. Todds zweiter Vorname lautet Homer.
  \item Fehler: Vor sechs Jahren ist in Moes Taverne ist eine Plakette mit der Aufschrift \glqq Dr. Marvin Monroe Memorial Booth\grqq\ zu sehen. Zu dieser Zeit lebte allerdings Dr. Marvin Monroe noch.
\end{itemize}
}

\section{Staffel 33}

\subsection{Millennium-Bug -- Das Musical}
Bei der Beerdigung ihres Ex-Lehrers Franklin Chase\index{Chase!Franklin} schwelgt Marge in Erinnerungen an ein Musical, an dem sie in ihrer Schulzeit als Bühnenmanagerin beteiligt war und beschließt, das Stück \grqq Y2K: The Millenium Bug\grqq\ wieder auf die Bühne zu bringen. Voller Elan macht sich Marge ans Werk, doch als ihre Mitschülerin Sasha Reed\index{Reed!Sasha} auftaucht, die vermeintlich ein Broadway-Star geworden ist, gerät die Produktion aus den Fugen.

\notiz{
\begin{itemize}
  \item Der ursprüngliche Originaltitel lautete \grqq No Day but Yesterday\grqq.
  \item Jack Dolgen\index{Dolgen!Jack} schrieb die Musik dieser Folge.
\end{itemize}
}

\subsection{Bart ist im Knast}
Trickbetrüger locken Homers Vater Abe Simpson Geld aus der Tasche, indem sie behaupten, dass Bart im Gefängnis sitzt und eine hohe Haftstrafe zu erwarten hat. Es sei denn, Abe überweist ihnen 10.000 Dollar. Alle in der Familie haben Mitleid mit dem geprellten alten Mann, nur Homer ist sauer. Er fühlt sich um sein Erbe gebracht. Als sich die Gelegenheit bietet, die Telefonbetrüger zu überführen, zögern die Simpsons nicht lange.

\notiz{Abe Simpsons arbeitete früher als Fabrikarbeiter bei einem Fleischklopsfabrikant.}

\subsection{A Nightmare on Elm Tree}
Diese Halloween-Folge besteht aus fünf Episoden.
\begin{itemize}
  \item \textbf{Barti}\\ In einer Parodie auf den Film \glqq Bambi\grqq, werden Bart und seine Mutter von dem Jäger Mr. Burns verfolgt. Milhouse stirbt dabei und Bart schwört blutige Rache.
  \item \textbf{Diesseits von Parasite von Bong Joon-Ho}\\ Angelehnt an den südkoreanischen Film \glqq Parasite\grqq, wird Bart Simpson der Nachhilfelehrer von Rainer Wolfcastles Tochter Greta und zieht in die Villa des Filmstars ein. Auch die übrigen Mitglieder der Familie übernehmen Jobs im Hause Wolfcastle. Dort leben im Keller ganz viele weitere Bewohner Springfields. Die fangen an, sich zu prügeln und bis auf die Simpsons sterben dabei alle.
  \item \textbf{Nightmare on Elm Tree}\\  Ähnlich den \glqq Nightmare on Elm Street\grqq -Geschichten, probiert Bart seine Geschwister Lisa und Maggie mit Gruselgeschichten im Baumhaus zu erschrecken. Nach einer Weile geht dies Homer auf die Nerven und er will den Baum im Garten fällen. Doch plötzlich erwacht dieser zum Leben!
  \item \textbf{Der verräterische Bart}\\ Ein Erzähler liest Maggie das Buch \glqq The Telltale Bart\grqq\ vor. Darin wird beschrieben, wie Barts Unfug von Monat zu Monat schlimmer wird.
  \item \textbf{Bei Anruf Tod}\\ Eine Parodie auf den Horrorfilm \glqq The Ring\grqq : Ein TikTok-Video tötet jeden, der es gesehen hat, sieben Tage danach. Lisa und Bart wollen wissen, was im Video vorkommt und bitten ihren Großvater, es für sie anzusehen, da der sowieso nicht mehr lange leben würde. Sie finden dabei heraus, dass im Brunnen der Schule ein Mädchen ums Leben gekommen ist. Sie kehrt wieder, springt aber erneut in den Brunnen, als Lisa Saxophon spielt. 
\end{itemize}

\notiz{Als Professor Frink von einem Baum weg geschleudert wird, erwähnt er den Doppler-Effekt. Der Doppler-Effekt ist die zeitliche Stauchung bzw. Dehnung einer Welle durch die Veränderungen des Abstands zwischen Sender und Empfänger.}

\subsection{Moe-Zart}
Die Straße, in der die Simpsons wohnen, wird von einer Verkehrslawine überrollt. Als sich die Familie mit den Nachbarn beraten will, wie dagegen vorgegangen werden soll, stellt sich heraus, dass die meisten Nachbarn nicht vom Verkehr, sondern vielmehr von den Simpsons genervt sind. Homer gibt daraufhin sein Bestes, um die Herzen der Nachbarschaft zurückzugewinnen, indem er Professor Frink um Hilfe bittet. Indessen trifft Moe im Stau seine alte Liebe Maya wieder. Sie gesteht ihm, dass es ein Fehler von ihr war, ihn zu verlassen und möchte wieder mit ihm zusammen sein. Moe hat jedoch Angst davor, wieder verletzt zu werden. Er versteckt sich in Barts Baumhaus. Homer möchte ihn zum Gehen bewegen und schlägt ihm vor, sie zu heiraten. Er macht ihr den Antrag und sie sagt zu.

\notiz{Maya war erstmals in der Episode \glqq Große, kleine Liebe\grqq\ (siehe \ref{LABF06}) zu sehen.}

\subsection{Bauchgefühl}
Ein Besuch in einem stillgelegten Wasserpark hat für Lisa und Bart ein übles Nachspiel. Nachdem sie in verseuchtem Wasser geplantscht haben, erkranken die beiden an einer Infektion. Zum Glück gibt es Medikamente dagegen. Allerdings haben diese einen unangenehmen Nebeneffekt. Sie sorgen vorübergehend für eine Gewichtszunahme. Während Lisa aus Angst, deshalb zum Gespött der Leute zu werden, eine schwere Neurose entwickelt, wird Bart bei den Schulrüpeln plötzlich sehr beliebt. Patty und Selma zeigen Lisa aber, dass man auch damit glücklich sein kann.

\notiz{Als sich Homer an Wasserpark in seiner Jugendzeit erinnert, ist das Lied \glqq I Wanna Rock\grqq\ der Band Twisted Sister zu hören.}

\subsection{A Serious Flanders (1)}
Beim Müllsammeln im Park entdeckt Flanders eine in einem Baum versteckte Tasche voller Geld. Anstatt den Fund selbst zu behalten, will er sich ein Beispiel an seinem Großvaters Sheriff Flanders nehmen und das kleine Vermögen an Bedürftige spenden. Dummerweise berichtet das Fernsehen davon und feiert den großzügigen Spender als Helden. Ehe sich Flanders versieht, sind ihm skrupellose Schuldeneintreiber auf den Fersen. Dabei kommt es jedoch zu einer fatalen Verwechslung.

\notiz{
\begin{itemize}
  \item Folgende Titel sind u.\,a. bei Simpflix\index{Simpflix} zu sehen:
  \begin{itemize}
    \item Bee-Jack Horseman
    \item Wig Mouth
    \item House of Carls
  \end{itemize}
  \item Sideshow Mel ist mit Barb verheiratet, welche ein Waisenhaus leitet.
  \item Auf einem Weihnachtsmarktstand ist die Aufschrift \glqq Knoblauchbrautwurstenschnitzel\grqq\ zu lesen.
\end{itemize}
}

\subsection{A Serious Flanders (2)}
Noch immer befindet sich Homer in der Gewalt skrupelloser Geiselnehmer. Ziel der Gauner ist es, von Homers vermögendem Nachbarn Flanders Geld zu erpressen. Ihr rücksichtsloses Vorgehen lässt Flanders nachdenklich werden. Irritiert beginnt er sich zu fragen, woher das im Baum versteckte Bargeld eigentlich stammt. Er klaut schließlich das von ihm dem Waisenhaus gespendete Geld. Homer wird allerdings in der Zwischenzeit von Marge gerettet. Ned steht als Dieb da und zieht sich im Wald in eine einsame Hütte zurück, in der er heimlich von Homer mit Lebensmitteln versorgt wird. Der Killer an ihn aber aufspüren. Ned kann ihm entkommen.

\notiz{
\begin{itemize}
  \item Es handelt sich um die vierte zweiteilige Folge nach \glqq Wer erschoss Mr. Burns?\grqq, \glqq Der große Phatsby\grqq\ und \glqq Krieg der Priester\grqq.
  \item Homer bringt Ned das Buch \glqq The Dirt\grqq\ der Möetley Crüe\index{Möetley Crüe} in die Waldhütte mit.
\end{itemize}
}

\subsection{Porträt eines jungen Lakaien in Flammen}
Smithers ist wegen seiner nicht erwiderten Liebe zu Mr. Burns frustriert. Der verkauft einige Dobermann-Wachhund-Welpen an seine vermögenden Freunde. Homer lernt dadurch den Multimilliardär Michael de Graaf kennen, der im Modegeschäft tätig ist. Da Michael ebenso wie Smithers homosexuell ist, kommt Homer auf die Idee, ihn mit Smithers bekannt zu machen, um die beiden zu verkuppeln.
In der Firma von de Graaf stellen Homer, Bart und Lisa fest, dass der Betrieb die Umwelt mit der hergestellten Fast Fashion stark verschmutzt und damit schädigt. Homer konfrontiert Smithers damit. So ist das Glück nicht von langer Dauer, da Smithers die Beziehung beendet.

\notiz{
\begin{itemize}
  \item Mr. Burns züchtet Dobermannhunde und Waylon ist der Hundezuchtbeauftragte.
  \item Barney, Gil und Kirk arbeiten in der Firma MDGE\footnote{Michael de Graaf Express} von Michael de Graaf.
  \item Dies ist die erste Folge, in der sich die Haupthandlung um die Verliebtheit von Waylon Smithers dreht \cite{AVClub21}.
\end{itemize}
}

\subsection{Muttertag}
Eigentlich sollte Marge am Muttertag im Zentrum der Aufmerksamkeit stehen. Doch als Homer plötzlich von schmerzhaften Erinnerungen an seine eigene Mutter heimgesucht wird und er nicht mehr aufhören kann zu weinen, stiehlt er seiner Frau ungewollt die Show. Zum Glück hat Lisa eine gute Idee, wie Homer wieder aus seinem Tief herausgeholt werden kann. Sie rät ihrem Vater, eine Online-Therapie zu machen. Das führt Homer 30 Jahre in die Vergangenheit zurück: Den Tag als Mona ihren Mann Abe Simpson verließ, weil sie vom FBI gesucht wurde.

\notiz{
\begin{itemize}
  \item Homer war neun Jahre alt, als seine Mutter ihn und seinen Vater verließ.
  \item Als Homer ein Teenager war, lebte er mit seinem Vater in der Rural Route 9 in Springfield.
\end{itemize}}

\subsection{Der Pate}
Während die Simpsons einen Vergnügungspark besuchen, entwischt Maggie aus Grandpas Obhut. Zum Glück erweist sich Nachbar Flanders als rettender Engel und bringt die Kleine sicher zurück. Für die Simpsons steht danach fest, dass für Maggie eine neue Betreuung gefunden werden muss, doch das Interesse an dieser Aufgabe hält sich in Grenzen. Nach einer Reihe von Absagen fällt Homer auf, dass Mafiaboss Fat Tony sehr religiös ist. So wird er der Pate von Maggie und nimmt die Aufgabe ernst: Er beschenkt das Kind und dessen Familie, die er mit in den Gottesdienst nimmt. Da er Maggie aber Mafiamethoden lehrt, wird das alles Marge zu viel.

\notiz{
\begin{itemize}
  \item Fat Tony Name lautet: Anthony Joseph D'Amico.
  \item Ned Flanders wirkt verwundert, als er erfährt, dass Maggie nicht getauft ist. Als er kurzzeitig in der Folge \glqq Bei Simpsons stimmt was nicht!\grqq (siehe \ref{3F01}) für die Simpsons-Kinder die Vormundschaft inne hat, will er selbst Maggie taufen, was schließlich durch Homer verhindert wird.
\end{itemize}
}

\subsection{Football-Mom}
Marge und Mr. Burns haben beide ein Auge auf den talentierten Football-Spieler Grayson Mathers\index{Mathers!Grayson} geworfen. Nach einer verlustreichen Saison soll er die \glqq Springfield Atoms\grqq\ aus ihrem Tief herausholen. Während Marge dem jungen Athleten ein wenig familiäre Wärme abseits des Profisports ermöglichen will, indem sie ihn in ihr Haus aufnimmt, hat Mr. Burns nur den eigenen Vorteil im Sinn. Der beliebte Football-Spieler soll das neue Werbegesicht seiner Brandy-Marke werden.

\notiz{
\begin{itemize}
  \item Mr. Burns besitzt eine Spirituosenfabrik, welche den \glqq Mr. Gentleman Brandy\grqq\ herstellt.
  \item Diese Episode ist John Madden\index{Madden!John} gewidmet, der vor der Erstausstrahlung dieser Folge verstorben ist.
\end{itemize}
}

\subsection{Survivor}
Lisa und Bart sind in Sorge: Homer und Marge hocken den ganzen Tag nur noch auf der Couch und schauen Trash-TV. Schließlich können die Kinder ihre Eltern dazu überreden, einen Gutschein, der seit einem Jahr am Kühlschrank klebt, einzulösen -- in einem Hotel, in dem gelangweilte Paare wieder zueinanderfinden sollen, und zwar ganz ohne Fernseher und Handys. Homer und Marge machen sich auf den Weg, verirren sich jedoch in der Wildnis und sind plötzlich ganz auf sich allein gestellt.

\notiz{Diese Episode war für einen Emmy nominiert.}

\subsection{Boyz N Highland}\label{UABF06}
Bart, Nelson und Dolph haben sich einiger Vergehen schuldig gemacht und müssen nun eine Wanderung durch die Wildnis antreten. Dort sind die Jungs auf sich allein gestellt. Gesellschaft bekommen sie von Streber Martin, der sich angeblich freiwillig gemeldet hat. Ob die vier Knaben tatsächlich Zusammenhalt und vielleicht auch ein bisschen Anstand lernen? Mitnichten, denn schnell kommt es zu unüberwindbaren Konflikten. Während sich Homer und Marge über ihr freies Wochenende freuen, gibt Lisa Maggie zu ihren Tanten Patty und Selma, um für ein paar Tage den Luxus eines Einzelkindes genießen zu können.

\notiz{
\begin{itemize}
  \item Dolph heißt Shapiro mit Nachnamen.
  \item Martins Eltern heißen mit Vornamen Gloria und Gareth.
\end{itemize}
}

\subsection{Das Institut}
Nachdem Homer den Hund im brütend heißen Auto eingesperrt und Reverend Lovejoy versehentlich aus dem Fenster geschubst hat, ist sein Ruf in Springfield vollends ruiniert. Fortan muss er sein Dasein als Außenseiter fristen. Ein Fremder eilt ihm jedoch zu Hilfe. Er nimmt Homer in ein Institut auf, das ihm verspricht, seine Ehre wiederherzustellen, indem die Aufnahmen aus dem Netz gelöscht werden. Homer und andere, ähnlich Leidtragende dringen gemeinsam in ein Rechenzentrum ein, um die entsprechenden Videos zu löschen; doch sie stellen fest, dass auch die Aufnahmen von Prominenten Verbrechern entfernt werden sollen, da macht Homer aber nicht mit. 

\notiz{
\begin{itemize}
  \item Lenny hält einen Hund.
  \item Milhouse ist Torwart in einer Fußballmannschaft.
  \item Die Eisdiele \glqq Dairy Girls Ice Cream\grqq\ ist eine Hommage an die britische Fernsehserie \glqq Derry Girls\grqq\ (siehe \cite{BBC22}).
\end{itemize}
}

\subsection{Bartman One}
Bart wünscht sich sehnlichst ein Paar der angesagten Sneakers der Marke Slipremes\index{Slipremes}. Homer verspricht ihm, welche zu besorgen. Da die Schlange vor dem Schuhladen unendlich lang ist, kauft er kurzerhand bei einem Straßenhändler eine billige Fälschung. Als Bart in der Schule seine neuen Schuhe stolz präsentiert, fallen sie prompt auseinander. Bart ist fassungslos, dass sein Vater ihn damit zum Gespött der Gesellschaft gemacht hat. Durch einen glücklichen Zufall trifft er den Jungen Orion, dem die Sneakers-Marke gehört und freundet sich mit ihm an. Zusammen wollen sie an einem noch besseren Turnschuh arbeiten. Homer überlegt währenddessen, wie er ein coolerer Dad werden kann.

\notiz{Kirk von Houten trägt einen Poochie-Hut\index{Poochie}}.

\subsection{Ich bin Smartacus!}
Springfield ist in Aufruhr, als bekannt wird, dass die Hinterwäldlerin Brandine Spuckler gar nicht so dumm ist, wie alle glauben. Sie hat sich ihr Wissen heimlich angeeignet und bislang geschickt verborgen. Ihr Ehemann Cletus ist so baff, dass er sich kurzerhand von ihr trennt. Lisa tritt indes einem Geheimclub bei. Die Mitglieder sind hochbegabte Kinder, die beschlossen haben, ihre Intelligenz zu verbergen, damit sie so ohne Neid und Mobbing durchs Leben kommen.

\notiz{Lisa hält ein Referat über den Matilda-Effekt\index{Matilda-Effekt}\footnote{Der Matilda-Effekt beschreibt die systematische Verdrängung und Leugnung des Beitrags von Frauen in der Wissenschaft, deren Arbeit häufig ihren männlichen Kollegen zugerechnet wird. Der Effekt wurde 1993 von der Wissenschaftshistorikerin Margaret W. Rossiter postuliert.}.}

\subsection{Ausgeblutet}\label{UABF10}
Monk Murphy\index{Murphy!Monk} ist der Sohn des bekannten Jazz-Musikers Zahnfleischbluter Murphy. Lisa sucht ihn auf, weil ein Song seines verstorbenen Vaters mit abgewandeltem Text für einen TV-Spot der Springfielder Lotterie benutzt wird und sie das unter allen Umständen verbieten lassen will. Im Gespräch mit Monk findet sie heraus, dass er gehörlos ist. Da eine Operation sehr kostspielig ist, beschließt Lisa, ihm bei der Finanzierung zu helfen.

\notiz{
\begin{itemize}
  \item Die Lottozahlen lauten: 6, 27, 32, 14, 47, 55 und 62.
  \item Der Vorname von Zahnfleischbluter Murphy lautet Oscar.
\end{itemize}
}

\subsection{Ein kurzer Film über die Liebe}
Lisa Simpson nimmt an einem Naturschutz-Camp teil und rettet einen Oktopus vor einem Hai. Schnell entwickelt sich eine Freundschaft zwischen Lisa und dem Oktopus Molly. Doch schnell wird klar, dass das Haus der Simpsons kein guter Ort für ihn zum Leben ist. Währenddessen bekommt Bart Simpson eine neue Lehrerin mit Namen Rayshelle Peyton\index{Peyton!Rayshelle}. Er kann seine Gefühle für sie nicht kontrollieren und wird zum Musterschüler, um sie zu beeindrucken.

\notiz{Der Sportlehrer Mr. Krupt war im Gefängnis.}

\subsection{Es braut sich was zusammen}
Lisa platzt vor Stolz, als sie in der Marschband der Highschool, in der ihre Babysitterin Shauna an den Drums sitzt, als Ersatz einspringen darf. Derweil freunden sich Homer und Shaunas Vater, Oberschulrat Chalmers, an und brauen gemeinsam Bier. Der Gerstensaft wird jedoch versehentlich an Minderjährige ausgeschenkt, sodass Homer und Chalmers von der Polizei verhaftet werden.

\notiz{
\begin{itemize}
  \item In Gary Chalmers Garage ist ein Schild mit der Aufschrift \glqq Dubbel Bock\grqq\ zu sehen.
  \item Martin Prince hat einen älteren Bruder, der auf die High School geht.
\end{itemize}
}

\subsection{Marge, das Monster}
Bei einem Shuffleboard-Turnier der Senioren, bei dem auch Grampa teilnimmt, wird Marge von einer alten Dame als Monster beschimpft. Es stellt sich heraus, dass es sich bei der Frau um die ehemalige Rektorin von Marge handelt. Und schon bald wird klar, dass die junge Marge für einige böse Streiche in der Schule verantwortlich war. Das wiederum ruft Bart auf den Plan, der so begeistert von seiner Mutter ist wie nie zuvor. Währenddessen versucht Lisa Simpson ihren Vater Homer für vegetarisches Essen zu begeistern.

\notiz{Diese Folge ist dem Animateur Ian Wilcox gewidmet, der acht Tage vor der US-Erstausstrahlung gestorben ist.}

\subsection{Fleisch ist Mord}
Krusty der Clown ist außer sich vor Freude: Krusty Burger feiert sein fünfzigjähriges Bestehen. Just an diesem Tag taucht plötzlich ein Mann namens Gus in Springfield auf, der vor fünfzig Jahren von Krusty böse übers Ohr gehauen wurde. Gus ist mittlerweile einer der reichsten Männer der Welt und will sich nun endlich rächen. Gus ist ein früherer Geschäftspartner von Abe Simpson. Dadurch ist auch Homers Familie von der Sache betroffen. Krusty verliert nun alle seine Rechte auf seinen Namen und seine Produkte. Doch Gus wollte nur seine Kinder hintergehen. Dank Abe scheitert der Plan aber doch und Krusty wird aus Israel wieder zurückgeholt, um seinen alten Job wieder zu übernehmen.

\notiz{Angela Merkel hat einen Sitz im Aufsichtsrat bei Redstar\index{Redstar}, der Firma von Gus.}

\subsection{Poorhouse Rock}
Homer erfährt, dass Bart ihn für einen Verlierer hält. Um ihm das Gegenteil zu beweisen, nimmt er seinen Sohn mit in das Atomkraftwerk. Und der Plan geht auf, denn der Junge ist begeistert davon, wie Homer mehr oder weniger mit Nichtstun sein Geld verdient. Nun wünscht sich Bart nichts sehnlicher, als selbst eines Tages als nuklearer Sicherheitsinspektor arbeiten zu können. Doch er erfährt, dass seine Chancen eher schlecht stehen und lässt diesen Plan fallen. Als er in seinem Baumhaus von einem Brand bedroht wird, entdeckt er den Job des Feuerwehrmanns und stellt fest, das könnte sein Ziel sein.

\notiz{
\begin{itemize}
  \item Helen Lovejoys Netflix-Kennwort lautet: \befehl{NO LOVE NO JOY}
  \item Während Helen und Timothy Lovejoy eine Eheauszeit hatten, wohnte Helen bei den Wiggums.
  \item Der Couch Gag wurde von Spiker Monster, einem venezolanischen Fankünstler gestaltet. Er ist auch der Schöpfer des Fan-Webcomics \grqq Those Springfield Kids\grqq. 
\end{itemize}
}

\section{Staffel 34}

\subsection{Die Schildkrötenverschwörung}\label{UABF16}
Homer stellt fest, dass die beliebte Riesenschildkröte langsamer Leonard aus dem Zoo ausgebüxt ist. Er beschließt, dem Geheimnis des verschwundenen Reptils auf die Spur zu kommen. Dazu schließt er sich einer dubiosen Facelook-Gruppe an, die schon bald ein Sammelsurium von Verschwörungstheorien aufstellt, wo das Kriechtier sein könnte. Homer fühlt sich in der Gruppe wohl und endlich respektiert. Schließlich stellt sich jedoch heraus, dass Leonard überhaupt nicht verschwunden ist. Homer entdeckte, dass sich Leonard im Zoo in einem Kaninchenbau versteckt hatte und gar nicht verschwunden war.

\notiz{
\begin{itemize}
	\item Oberschulrat Chalmers wohnt gegenüber dem Eingag des Zoos.
	\item Gil Gunderson macht Elizabeth Hoover einen Heiratsantrag und sie sagt ja.
	\item Der Couch-Gag \glqq Dinosaur Game\grqq\ stammt von den schweizerischen Illustratorinnen Katrin von Niederhäusern und Janine Wiget. Die beiden hatten bereits in der Folge \glqq Burger Kings\grqq\ (siehe \ref{QABF11}) den Eröffnungs-Gag gestaltet.
\end{itemize}
}

\subsection{Eine sportliche Affäre}\label{UABF19}
Marges Geburtstag steht an und sie träumt von einem Fitness-Fahrrad. Homer setzt alles in Bewegung, um seiner Frau diesen Wunsch zu erfüllen. Nach einigen Anfangsschwierigkeiten ist Marge Feuer und Flamme. Sie trainiert rund um die Uhr -- was nicht zuletzt an ihrem attraktiven Online-Coach Jesse liegt. In Homer wächst die Eifersucht. Er muss sich etwas einfallen lassen, um Marge wieder für sich zu gewinnen. Lisa wird derweil in den Geschworenendienst berufen.

\notiz{
\begin{itemize}
  \item Der Couchgag stammt vom US-Trickfilmer Bill Plympton\index{Plympton!Bill}.
  \item Im Hintergrund von Jessie sind beim Fahrradfahren kurz Leela und Fry aus Futurama zu sehen.
  \item Gil Gunderson ist als Rechtsanwalt tätig.
  \item Diese Episode wurde in Hongkong von der Disney+-Streaming-Plattform wegen folgenden Satzes aus der Originalfolge genommen: \grqq Behold the wonders of China. Bitcoin mines, forced labor camps where children make smartphones.\footnote{Deutsche Übersetzung: Entdecken Sie die Wunder Chinas. Bitcoin-Minen, Zwangsarbeitslager, in denen Kinder Smartphones herstellen.}\grqq
\end{itemize}
}

\subsection{Alles Lüge}
Bart und Lisa Simpson sind mit den Pfadfindern unterwegs und streiten mal wieder. Als sie von ihren Eltern abgeholt werden, schlagen sie vor, nach Hause zu fahren, TV zu schauen und nie mehr nach draußen zu gehen. Plötzlich wird die Sendung von Hackern namens \glqq Pseudo-nonymous\index{Pseudo-nonymous}\grqq\ übernommen, die ein Lösegeld in Höhe von 20 Millionen Dollar fordern und solange unveröffentlichte Simpsons-Szenen zeigen werden, bis ihre Forderungen erfüllt werden. Um zu beweisen, dass sie es ernst meinen, leaken sie die Wahrheit über Lenny, den zukünftigen Bart und Martin -- und das ist nur der Anfang.

\notiz{
\begin{itemize}
  \item Der ausführende Produzent Matt Selman twitterte während der Episode ebenfalls live, als wäre die Episode tatsächlich gehackt worden.
  \item Die Adresse, an welche die 20 Millionen Dollar überwiesen werden sollte: \href{https://lets-go.Raiders@superfans.darkweb/money-stuff}{\url{https://lets-go.Raiders@superfans.darkweb/money-stuff}}
\end{itemize}
}

\subsection{Daytime-Marge}
Krusty will an seinem Image arbeiten und eine neue Daytime-Show auf die Bildschirme bringen. Bei einer Marktforschungsstudie zum aktuellen Fernsehprogramm überzeugt Marge durch ihre Kreativität und wird als Content-Managerin für Krustys Show angeworben. Anfangs läuft alles bestens und das Format kommt beim Publikum gut an. Doch schon bald lernt Marge die Schattenseiten des Fernsehgeschäfts kennen. Homer sagt ihr, sie werde noch zerstört. Auch Marge stellt das an sich selbst fest. So gibt sie den Job wieder auf.

\notiz{
\begin{itemize}
  \item Der Fokusgruppe gehören Marge, Helen, Bernice, Agnes und Julio an.
  \item Krustys Daytime-Show wird von Viacalm produziert. Viacalm ist eine Anspielung auf den US-amerikanischen Medienkonzern Viacom Inc, zu dem u.\,a. MTV und Comedy Central gehören.
\end{itemize}
}

\subsection{Nicht Es}
Stephen Kings \glqq Es\grqq\ meets \glqq Die Simpsons\grqq : Alle 27 Jahre kommt es in Kingfield\index{Kingfield} zu dramatischen Vorfällen. Zahlreiche Kinder verschwinden spurlos, so auch Homers Freund Barney. Als Homer eines Tages von einer Gruppe Teenager gejagt wird, sieht er zufällig einen Horrorclown im Gebüsch. Schnell ist klar, dass er der Täter sein muss. Gemeinsam mit Marge, Moe, Jeff und Carl will Homer dem Clown das Handwerk legen, doch der Kampf gegen Krusto\index{Krusto} verläuft anders als geplant.

\notiz{
\begin{itemize}
  \item Zu den verschwundenen Kindern gehören u.\,a. Apu, Otto, Waylon, Horatio, Cletus und Frank an.
  \item Metal Moe singt das Lied \glqq Rock you like a hurricance\grqq\ der Scorpions\index{Scorpions}.
\end{itemize}
}

\subsection{Duff-Dad}
Der Duffman bekommt Konkurrenz, denn die Geschäftsleitung sucht ein frisches Werbegesicht. In einer Fan-Wahl soll das neue Maskottchen gekürt werden. Gemeinsam mit seinem langjährigen Unterstützer Homer versucht der Duffman, wieder zum Botschafter der Biermarke gewählt zu werden. Als negative Schlagzeilen seinen Erfolg gefährden, will er die Öffentlichkeit von sich überzeugen und greift dabei zu unlauteren Mitteln, denn er gibt Lisa als seine Tochter aus. Als Homer und Lisa Duffman zur Rede stellen, verrät er, dass er tatsächlich eine entfremdete Tochter namens Amber hat. Als er bemerkt, wie schlau Lisa geworden ist, betrachtet er Homer als den ultimativen Vater und bittet ihn um Hilfe, um wieder Kontakt zu Amber aufzunehmen und den Wettbewerb zu gewinnen.

\notiz{Diese Episode wurde zum Gedenken an Julie Kavners Mann, David Davis\index{Davis!David}, gewidmet.}

\subsection{Der Stiefbruder}
Abe ist verliebt und zieht zu seiner neuen Freundin und deren Adoptivsohn Calvin\index{Calvin}. Als Homer sieht, wie herzlich sein sonst so rabiater Vater mit dem Elfjährigen umgeht, ist er eifersüchtig. Weil Marge eine neue Möbelpolitur verwendet, auf die ihr Mann allergisch reagiert, muss Homer ein paar Tage bei Abes neuer Familie übernachten. Calvin freut sich auf seinen großen Bruder, doch der reagiert derart ablehnend, dass zwischen den beiden ein heftiger Geschwisterkrieg ausbricht. Lisa und Bart wollen währenddessen die perfekte Übernachtungsparty in Abes Zimmer im Altersheim veranstalten.

\notiz{Abe gibt an, dass er sich im Krieg in den Fuß geschossen hat, um sich vom Küchendienst zu drücken.}

\subsection{Nelson und Lisa}
Viele Jahre nach ihrer gemeinsamen Schulzeit trifft Lisa auf Nelson. Schnell fühlt sie sich wieder zu ihrem Jugendfreund hingezogen, doch Nelson zeigt wenig Interesse. Weitere Jahre gehen ins Land. Lisa ist mittlerweile mit einem Geschäftsmann und Nelson mit der Kopfgeldjägerin Rott zusammen. Aber auch nach der Trennung von ihren neuen Partnern will es mit der Beziehung zwischen Lisa und Nelson einfach nicht klappen -- bis sie die Hochzeit von Sophie und Jimbo erneut zusammenführt.

\notiz{
\begin{itemize}
  \item Lisa schließt die Universität als Jahrgangsbeste ab.
  \item Das Motto \glqq Veritas et mendicam\grqq\ auf dem Rednerpult bedeutet \glqq Wahrheit und Betteln\grqq .
\end{itemize}
}

\subsection{Simpsonsworld}
\begin{itemize}
	\item \textbf{Der Pookadook}\\ In einer Parodie auf den Film \glqq Der Babadook\grqq\ liest Marge Maggie eine Gute-Nacht-Geschichte über eine mörderische Macht vor. Sie wird von dieser befallen und will ihre Tochter daraufhin töten.
	\item \textbf{Todesbuch}\\ Im Anime-Stil spielt diese Geschichte auf den erfolgreichen Manga \glqq Death Note\grqq\ an. Lisa Simpson findet ein Buch mit dem Titel \glqq Death Tome\grqq . Sobald man einen Namen in das Buch schreibt, wird diese Person sterben. Nach anfänglichen Bedenken verliert Lisa jegliche Skrupel.
	\item \textbf{SimpsonsWorld}\\ Diese Parodie auf \grqq Westworld\grqq\ beginnt in der Mitte der Episode \glqq Homer kommt in Fahrt\grqq\ (siehe \ref{9F10}). Homer wird von zwei Touristen schwer verletzt. Wie sich herausstellt, ist er nur ein Roboter in einem Vergnügungspark.
\end{itemize}

\notiz{
\begin{itemize}
  \item Die Mitarbeiter in dem Simpsons-Vergnügungspark haben fünf Finger.
  \item Am Ende der letzten Teilepisode sind folgende Vergnüngsparks zu sehen: \glqq Bob's Burgers Land\grqq, \glqq South Park Park\grqq, \glqq Family Guy Town\grqq , \glqq Futurama-Rama\grqq, \glqq Rick and Morty Universe\grqq, \glqq SpongeBob Sea\grqq\ und \glqq Big Mouth Mountain\grqq.
\end{itemize}
}

\subsection{Das Skinner-Bart-System}
Marge verbietet ihrem Sohn gewaltverherrlichende Videospiele. Also muss sich Bart mit einem kinderfreundlichen Spiel Boblox\index{Boblox} zufriedengeben. Zusammen mit Milhouse entdeckt er eine Fehlfunktion im Spiel, mit der sie sich bereichern können. Doch Rektor Skinner kommt hinter ihr Geheimnis und will in ihr Geschäft einsteigen. Bart und Milhouse sind jedoch nicht die einzigen, die den Glitch bemerken und so kommt es zu einem erbitterten Kampf mit den Kindern der benachbarten Montessori-Schule. Währenddessen entdeckt Marge, dass sie in Boblox mit Maggie über Emojis kommunizieren kann.

\notiz{Bei Boblox handelt es sich um eine Parodie auf Roblox\index{Roblox}.}

\subsection{Top Goon}
Moe ist es leid, dass sein Nachbar King Toot in allem besser zu sein scheint. Nachdem der Angeber auch noch mit seinem Eishockey-Team die Meisterschaft gewinnt, reicht es Moe endgültig. Er coacht seine eigene Mannschaft -- mit Bart als Starspieler und Erfolgsgarant. Das bemerkt die Konkurrenz auch schnell und Bart wird zu deren Zielscheibe. Moe braucht also einen Mann fürs Grobe, einen \glqq Goon\grqq, der Bart auf dem Spielfeld den Rücken freihält. Dafür ist Nelson genau der Richtige.

\notiz{
Folgende Jugend-Eishockey-Teams gibt es:
\begin{itemize}
\item Bar Flyers Springfield
\item Li'l Tooters
\item Ice Lubbers
\item The Holy Moleys
\item Bumblebee Boys
\item Burns' Cheapskaters
\item Stick-E Marts
\item Toronto Maple Leafs
\end{itemize}
}

\subsection{Mein Leben als Vlog}
Homer hilft Maggie bei einem Tanzauftritt über ihr Lampenfieber hinweg. Ein Video davon geht viral und macht die Simpsons zu Internetstars. Jeder bekommt seinen eigenen YouTube-Kanal und der Familienalltag besteht nur noch aus dem Produzieren von Content. Mit dem Erfolg kommt allerdings auch der Neid. Und so dauert es nicht lange, bis sich die ersten Negativschlagzeilen verbreiten. Doch bevor die Simpsons dazu Stellung nehmen können, verschwinden sie plötzlich von der Bildfläche.

\notiz{
\begin{itemize}
  \item In dieser Folge wird erstmals Youtube genannt. In früheren Episoden wie \glqq Mein peinlicher Freund\grqq\ (siehe \ref{VABF22}) wurde stattdessen Mytube verwendet.
  \item Bart auf seiner Ablagebank beim Scrabble das Wort Kwyjibo\index{Kwyjibo} liegen -- wie bereits in Folge \glqq Bart wird ein Genie\grqq\ (siehe \ref{7G02}).
  \item Diese Episode wurde Chris Ledesma, einem langjährigen Musikredakteur, gewidmet.
\end{itemize}
}

\subsection{The Many Saints of Springfield}
Erschrocken beobachten die Simpsons, dass bei ihrem Nachbarn in letzter Zeit alles schiefläuft: Ned verliert seinen Job, sein Mülleimer explodiert und das ist nur der Anfang. Er sucht Zuflucht in der Kirche, stößt dort aber nur auf Fat Tony, der ihm seine Hilfe anbietet. Ohne es zu ahnen, lässt sich Ned auf die Mafia ein. Als er Einblick in die kriminellen Machenschaften von Fat Tony erhält, will er aussteigen. Fat Tony will ihn ermorden lassen. Doch Ned stellt sich als verdeckter Ermittler für die Polizei heraus, die das Mafia-Versteck stürmt und Ned rettet.

\notiz{
\begin{itemize}
  \item Als Ned Flanders als Lehrer tätig ist, muss er zur Strafe folgenden Text an die Tafel schreiben: I will not attempt the salvation of a bureaucrat\footnote{Deutsche Übersetzung: I werde nicht die Erlösung eines Bürokraten anstreben}.
  \item Diese Folge wurde dem Musiker David Crosby\index{Crosby!David} gewidmet.
\end{itemize}
}

\subsection{Carl Carlson reitet wieder}\label{OABF08}
Im Bowlingcenter trifft Carl eine hinreißende Frau, mit der er sich zum Essen verabredet. Die Begegnung mit Naima stürzt ihn jedoch in eine Identitätskrise. Um ihr zu gefallen, ändert er alles an sich. Naima will jedoch den echten Carl kennenlernen und rät ihm, sich erst einmal selbst zu finden. Daraufhin schaut er zurück auf seine eigene Geschichte und Wurzeln, in denen eine mysteriöse Rodeo-Schnalle eine wichtige Rolle spielt.

\notiz{Carls leiblicher Vater hieß Wyatt und war Rodeo-Reiter.}

\subsection{Bartlos}
Am Vorlesetag in der Schule schießt Bart etwas übers Ziel hinaus: Um die Kleinen zu unterhalten, schreibt er die Geschichten um und malt neue Bilder in die Bücher. Homer und Marge ärgern sich über diesen Streich und denken sich bereits eine Reihe von Bestrafungen aus, doch die Lehrerin nimmt Bart in Schutz. Schließlich bringt er die anderen Kinder dazu, mit Freude zu lesen. Das weckt das schlechte Gewissen in seinen Eltern. Sie wollen Bart endlich so sehen, wie andere es tun. Sie stellen sich die Frage, ob sie Bart überhaupt lieben. Am nächsten Morgen wachen sie in einer neuen Welt auf, in der Bart nie als ihr Sohn existiert hat.

\subsection{Wut im Bauch}
Als die Grundschule von Springfield für einige Wochen geschlossen werden muss, müssen die Kinder Homeschooling machen. Kirk Van Houten stellt dabei fest, dass der Geschichtsunterricht einen seiner Vorfahren in ein schlechtes Licht rückt. Er beginnt, sich aktiv gegen Zensur und Kontrolle des Lehrplans zur Wehr zu setzen. Unterstützung erfährt er auch von Homer Simpson.

\notiz{Kirks Ururgroßvater war Bürgermeister von Springfield.}

\subsection{Queenpin}
Homer ist zu Tode betrübt, als er erfährt, dass die Bowl-A-Rama schließen soll. Er beschwört den neuen Besitzer Terrence, die Anlage nicht abzureißen und verspricht, ihm neue Kunden zu besorgen. Er kann sogar Marge davon überzeugen, mit ihm dorthin zu gehen. Es stellt sich heraus, dass Marge eine sehr gute Bowlerin ist und so beschließt Terrence, dass ein Bowling-Match über die Zukunft der Bowlinganlage entscheiden soll. Sie muss dann an ihre vergangene Zeit mit ihrem Verehrer Jacques denken, mit dem sie viel Zeit auf der Bowlingbahn verbracht hat. Urplötzlich taucht Jacques dann wieder in Springfield auf.

\notiz{
\begin{itemize}
  \item Der Zigarettenautomat in Moes Taverne, der seit Beginn der Serie aufgetaucht war, wurde endgültig entfernt (siehe \cite{Metro23}).
  \item An der Wall of Fame im Bowl-A-Rama hängen u.\,a. Fotos von Homer, Barney und Jacques.
  \item Jacques wohnt in der Wohnung Nummer 17 in der Fiesta Terrace. 
\end{itemize}
}

\subsection{Fan-ilien-Fehde}
Homer beleidigt die beliebte Popsängerin Ashlee Starling\index{Starling!Ashlee} vor laufender Kamera -- mit Folgen, denn er wird von ihren Fans, der Murmur Nation, sofort massiv unter Druck gesetzt. Lisa, selbst ein großer Ashlee-Starling-Fan, ist gemeinsam mit ihrem Bruder Bart ganz vorn mit dabei. Um sich zu wehren, macht Homer mit Ashlees Konkurrentin Echo\index{Echo}, die auch über eine große Menge treuer Anhänger verfügt, gemeinsame Sache. Schon bald eskaliert der Krieg zwischen den beiden Fanlagern.

\subsection{Blue-Washing}
Marge entwickelt ein Verfahren, das es ihr ermöglicht, lediglich mit einem Kopfkissenbezug und einigen wenigen Zutaten die Dreckwäsche der Familie ohne Maschine zu reinigen. Das bringt Lisa auf die Idee, vor allem ärmeren Menschen damit zu helfen. Kurzerhand gründen Marge und Lisa die Lisa M. Simpson Stiftung. Schon bald tritt der ursprüngliche Zweck der Stiftung jedoch völlig in den Hintergrund -- sehr zum Ärger von Lisa.

\notiz{Auf der Webseite, auf welcher ungewöhnliche Wetten getätigt werden können, kann auf das Ereignis \glqq Armageddon tritt ein\grqq\ mit einer Quote von 666 zu 1 gewettet werden.}

\subsection{Die kleinen Raupen Nimmersatt}\label{OABF14}
Springfield wird von einer sehr seltenen, aber höchst gefährlichen Raupenart befallen und alle Bürger werden gezwungen, ihr Zuhause nicht mehr zu verlassen. Der Lockdown sorgt allerdings für Unmut. Maggie weigert sich, jegliches Nahrungsmittel ohne ihr geliebtes Ranch-Dressing zu essen. Als die Menge der Salatsauce knapp wird, müssen Marge und Homer sich überlegen, wie sie schnellstmöglich an Nachschub kommen. Unterdessen denkt Lisa, dass ihre Malibu Stacy-Puppen lebendig werden und Bart fängt an, Rektor Skinner hinterher zu spionieren.

\notiz{
\begin{itemize}
  \item Prof. Dr. John Frink ist Direktor des Zentrums für ekelhafte Raupen.
  \item Jimbos Zimmer ist voll von Harry Styles Postern.
\end{itemize}
}

\subsection{Pädagogische Ansichten eines Clowns}\label{OABF15}
Krusty hat Geldsorgen und gründet deshalb eine Clown-Schule für Kinder. Die neue Einrichtung ist schon bald ein großer Erfolg, selbst Bart ist motiviert und will etwas lernen. Und als seine Schüler gegen die Grundschulkinder von Springfield bei einem Wissenswettbewerb gewinnen, platzt Krusty fast vor Stolz. Der Clown ist überglücklich -- bis Fat Tony auftaucht und Teilhaber der Schule werden will.

\notiz{
\begin{itemize}
  \item Rainier Wolfcastle hat einen Sohn namens Dieter.
  \item Krusty betrieb bereits in der Folge \glqq Homie der Clown\grqq\ (siehe \ref{2F12}) ein Clown-College für Erwachsene.
\end{itemize}
}

\subsection{Homers Abenteuer durch die Windschutzscheibe}\label{OABF13}
Homer ist in einen schweren Verkehrsunfall verwickelt. Während er durch die Windschutzscheibe fliegt, steht die Zeit praktisch still und ihm erscheint eine Elfe in Gestalt der kleinen Puppe Goobie-Woo\index{Goobie-Woo}, die Maggie gehört. Goobie-Woo begibt sich in der letzten halben Sekunde von Homers Leben gemeinsam mit ihm auf eine unglaubliche Reise, um den Ursprung seiner unbändigen Wut zu verstehen.

\notiz{
\begin{itemize}
  \item Dies ist 750. Folge, deshalb ist in der Eröffnungssequenz auf der Kasse der Betrag 750,00 zu sehen und es erscheinen 750 Figuren (siehe \cite{Gamespot2023}).
  \item In der Hölle kann der Streaming-Dienst Hellflix\index{Hellflix} empfangen werden.
  \item Maggie spricht den Namen \glqq Goobie-Woo\grqq .
\end{itemize}
}

\section{Staffel 35}

\subsection{Schülerlotse Homer}
Als Otto, der Schulbusfahrer, plötzlich spurlos verschwindet, steht die Grundschule in Springfield ohne Fahrer da. Die Eltern bringen daraufhin ihre Kinder selbst zur Schule, was zu einem Verkehrschaos führt, das die Schülerlotsen nicht bewältigen können. Das Team sucht nach Verstärkung und findet die in Homer, der sich versehentlich meldet. Aber seine Einheit, die eigentlich für die Sicherheit der Kinder im Straßenverkehr einstehen sollte, bekommt soviel finanzielle Unterstützung, dass sie sich bald schon zur gefürchteten Gesetzeshüter-Gang entwickelt.

\notiz{Neben Homer sind Hans Maulwurf, Gil Gunderson und Jeremy Peterson Schülerlotsen.}

\subsection{Der Traum vom Erwachsenwerden}
Marge leidet unter einem verdorbenen Magen und wird von schrecklichen Albträumen gequält. Sie fürchtet, dass sie bald keine Mutter mehr sein wird, da ihre Kinder allmählich erwachsen werden. Der Auslöser dafür ist ein Gespräch mit Barts Lehrerin.

\notiz{Die Autorin Carolyn Omine wurde für diese Folge bei den 76. Writers Guild of America Awards für den Writers Guild of America Award for Television: Animation nominiert.}

\subsection{Lambo-Man}\label{OABF20}
Die neuen Nachbarn der Simpsons, Anne und Thayer Blackburn\index{Blackburn!Anne}\index{Blackburn:Thayer}, sind äußerst freundlich und warmherzig. Bald wird jedoch klar, warum sie so liebenswürdig sind: Thayer bringt Homer nämlich dazu, eine Zustimmung zu unterschreiben, dass das Haus nebenan komplett renoviert werden darf. Die Simpsons werden daraufhin Tag und Nacht durch Baulärm belästigt. Nun ist guter Rat teuer. Währenddessen probiert Lisa, den Schläger Nelson in seine Schranken zu verweisen.

\notiz{
\begin{itemize}
  \item Gil Gunderson ist als Autoverkäufer zu sehen.
  \item Diese Folge wurde der Schauspielerin Suzanne Somers\index{Somers!Suzanne} gewidmet. Sie hatte einen Auftritt in der Folge \glqq Wer erfand Itchy und Scratchy?\grqq\ (siehe \ref{3F16}).
\end{itemize}
}

\subsection{Entsalzt}\label{OABF21}
Eines Tages taucht die CEO der Firma LifeBoat\index{LifeBoat} Persephone Odair\index{Odair!Persephone} in Springfield auf. Ihre Mission: die Welt retten. Ihr Plan besteht darin, Meerwasser mithilfe eines einfachen Geräts zu entsalzen und so der Menschheit unbegrenzt Trinkwasser zur Verfügung zu stellen. Lisa ist begeistert von der Idee und vermittelt ihr als Finanzier Mr. Burns. Der verliebt sich Hals über Kopf in Persephone, und schon bald läuten die Hochzeitsglocken. Recht schnell ziehen jedoch dunkle Wolken am Horizont auf.

\notiz{
\begin{itemize}
  \item Mr. Burns gibt an, dass er Twitter gekauft hat.
  \item Persephone Odair ist eine Anspielung auf Elizabeth Anne Holmes, die eine verurteilte Anlagebetrügerin und ehemalige US-amerikanische Biotechnologie-Unternehmerin ist. Sie war Geschäftsführerin des inzwischen insolventen und liquidierten Laborunternehmens Theranos\index{Theranos}.
  \item Der Autor Rob LaZebnik wurde für diese Folge bei den 76. Writers Guild of America Awards für den Writers Guild of America Award for Television: Animation nominiert.
\end{itemize}
}

\subsection{Die Homer-Mutation}
\begin{itemize}
  \item \textbf{Wild Barts Can't Be Token}\\
  Als Bart zum ersten digitalen menschlichen NFT\index{NFT}\footnote{Ein Non-Fungible Token ist ein Kryptowert, der im Gegensatz zu Kryptowährungen einmalig und nicht teilbar (non-fungible, deutsch: \glqq nicht austauschbar\grqq) ist.} wird, liegt es an seiner Mutter Marge Simpson ihn von dem Verderben zu Bewahren.
  \item \textbf{Ei8ht}\\
  Lisa sieht als Kind, wie Tingel-Tangel-Bob Bart ermordet. Um ihr Trauma zu verarbeiten, wird sie als Erwachsene Kriminalpsychologin und Serienkillerin.
  \item \textbf{Lout Break}\\
  Homer isst einen nuklear verseuchten Donut. Die Folge: Er rülpst ein atomares Virus, durch das andere befallen werden und zu faulen, bierliebenden Einfaltspinseln werden.
\end{itemize}

\notiz{
\begin{itemize}
  \item Die Episode \glqq Ei8ht\grqq\ bezieht sich auf den Film \glqq Das Schweigen der Lämmer\grqq\ aus dem Jahr 1991 und auf die Ähnlichkeit mit Filmen von David Fincher, darunter \glqq Sieben\grqq\ (1995), stilisiert als Se7en.
  \item Die Episode \glqq Lout Break\grqq\ bezieht sich hauptsächlich auf den Film \glqq Outbreak\grqq\ von Wolfgang Petersen aus dem Jahr 1995.
\end{itemize}
}

\subsection{Ein Freund aus Townsburg}
Marge ist tieftraurig, als sie an ihrem Geburtstag von ihren Kindern billige Werbegeschenke bekommt. Lisa und Bart beschließen daraufhin, ihre Mutter zu überraschen. Sie finden heraus, dass es in Marges Vergangenheit einen Papagei gab, der ihr bester Freund war. Die Geschwister schalten ihren detektivischen Spürsinn ein und finden den Vogel, um ihn ihrer Mutter stolz als Geburtstagsgeschenk zu präsentieren. Marge reagiert auf das Präsent allerdings sehr verhalten. Währenddessen wird Homer Simpson zum Panikmacher der Nachbarschaft.

\notiz{
Die Top-3 Nutzer der Alarmy Army:
\begin{enumerate}
  \item Herman's High Alert
  \item Dr. Panick
  \item Disco Sus
\end{enumerate}
}

\subsection{Unser kleines Farmhaus}
Um Geld zu sparen, beabsichtigt Mr. Burns, Arbeiter einzustellen, die nicht gewerkschaftlich organisiert sind. Doch bevor er dies tun kann, muss er alle loswerden, die Mitglieder der Gewerkschaft sind. Er plant einen großen Stromausfall, um allen Mitarbeitern zu kündigen, und schiebt Homer die Schuld dafür in die Schuhe. Homer versucht verzweifelt, sich zu verteidigen, doch niemand in der Stadt glaubt ihm. Sogar Marge fällt es schwer, ihm Vertrauen zu schenken.

\notiz{
Der Couch-Gag stammt von den schweizerischen Illustratorinnen Katrin von Niederhäusern und Janine Wiget. Beide hatten bereits mehrere Couch-Gags erschaffen.
}

\subsection{Liebe unterm Schottenrock}
Bart hat sich beim Nachsitzen mit Hausmeister Willie angefreundet. Eines Tages wird Willie nach Schottland entführt, und Bart überzeugt seine Familie, ihn zu retten. In Edinburgh entdecken sie jedoch, dass Willie nicht gekidnappt wurde, sondern seine Jugendliebe Maisie getroffen hat, die er heiraten möchte. Während Homer, der Hochzeiten im Ausland verabscheut, in einen Streit mit Marge gerät, merkt Bart, dass die Familie der Braut finstere Absichten hat.

\notiz{
\begin{itemize}
  \item Die Simpsons fliegen mit der Fluggesellschaft Planespotting nach Schottland. Der Name der Fluggesellschaft ist eine Anspielung auf den schottischen Film \glqq Trainspotting\grqq.
  \item Die schottische Band Belle and Sebastian trat als sie selbst auf und spielte während der Hochzeitsmontage ihren Song \glqq If You Find Yourself Caught in Love\grqq. Im Abspann spielten sie außerdem den Originalsong \glqq Willie And The Dream Of Peat Bogs\grqq.
  \item Als die Folge im März 2024 erneut ausgestrahlt wurde, war sie dem Komiker Richard Lewis\index{Lewis!Richard} gewidmet, der dem männlichen Golem in der Folge \glqq Krieg der Welten\grqq\ (siehe \ref{HABF17}) der achtzehnten Staffel seine Stimme lieh.
\end{itemize}
}

\subsection{Nerd ist ihr Hobby}
Die Simpsons machen eine Kreuzfahrt. Auf dem Kreuzfahrtschiff \glqq Pacific Princess Leia\grqq\ tummeln sich jede Menge Nerds, da es sich um eine Reise für Fans von Science-Fiction, Fantasy, Comics, Anime und Videospielen handelt. Während Lisa den Regisseur Taika Waititi\index{Waititi!Taika} kennenlernt, präsentiert der Comicverkäufer eine Actionfigur, die es nur zweimal auf der Welt gibt. Nach einem Stromausfall fehlt jedoch der Kopf der Figur. Bart gerät daraufhin unter Verdacht und wird verhaftet.

\notiz{
\begin{itemize}
  \item Maurice LaMarche tritt als Cosplayer auf und ist als Hedonismbot verkleidet, eine Figur aus der Fernsehserie Futurama\index{Futurama}, die von Matt Groening geschaffen und von LaMarche gesprochen wurde.
  \item In der Folge \glqq Der rasende Wüterich\grqq\ (siehe \ref{DABF13}) zeichnet Bart das Comic \glqq Angry Dad\grqq.
\end{itemize}
}

\subsection{Do The Wrong Thing}
Grampa, der seit zwanzig Jahren der Angel-Champion von Springfield ist, wünscht sich, dass Homer sein Erbe antritt. Da Homer jedoch nicht angeln kann, ist er erstaunt, als er den Angelwettbewerb von Springfield trotzdem gewinnt. Es stellt sich heraus, dass Bart ihm geholfen hat, indem er Murmeln in den gefangenen Fisch geschmuggelt hat, damit dieser auf der Waage das benötigte Gewicht erreicht. Ab sofort mogeln sich Vater und Sohn durch mehrere Wettbewerbe und gewinnen sie alle. Währenddessen bewirbt sich Lisa für ein Sommer Camp der Universität.

\notiz{
\begin{itemize}
  \item Nachdem Ape zum 20. Mal die Angelmeisterschaft gewonnen hat, setzt er sich als amtierender Champion zur Ruhe.
  \item Der Dekan der Universität, Belichick, ist eine Anspielung auf den Football-Trainer Bill Belichick, der als Cheftrainer der New England Patriots in die Skandale Spygate und Deflategate verwickelt war. Belichick wurde mit einer Geldstrafe von 500.000 Dollar belegt, weil er im ersteren Skandal die Signale gegnerischer Trainer aufgezeichnet hatte. Im letzteren Skandal wurde er nicht dafür bestraft, dass er die Bälle entleert hatte.
\end{itemize}
}

\subsection{Frinkensteins Monster}
Homer plant, sich für eine Stelle im neuen Kernkraftwerk in Shelbyville zu bewerben. Professor Frink unterstützt ihn dabei, den Job zu bekommen, indem er ihm beim Bewerbungsgespräch, das über Zoom\index{Zoom} stattfindet, Hinweise gibt. Homer erhält die Stelle, doch um die Arbeit auszuführen, benötigt er weiterhin Frinks Fachwissen. Daher entwickelt Frink eine Brille mit einem integrierten Kopfhörer, über den er Homer alle wichtigen Antworten zu wissenschaftlichen Fragen übermittelt.

\subsection{Lisa bleibt auf der Strecke}
Lisa leidet unter schrecklichen Angstzuständen. Die Psychiaterin findet heraus, dass der Auslöser Panikattacken sind, die Lisa dann befallen, wenn sie mit Homer im Auto sitzt. Die Therapeutin hat eine Idee: Sie meldet Lisa zu einem Go-Kart-Rennen an, damit sie, wenn sie am Steuer sitzt, lernt, ihre Ängste zu bewältigen. Bald darauf wird Lisa zu einer der erfolgreichsten Fahrerinnen in der Kinder-Formel-1.

\subsection{Cave Mom}
Nachdem Marge mit den Kindern einen Film über die Steinzeit angeschaut hat, ist sie wie verwandelt. Als Milhouse Karten für das ausverkaufte Konzert von Violencia Gigante\index{ Violencia Gigante} geschenkt bekommt, aber dessen Mutter Luann ihm verbietet, seinen besten Freund Bart mitzunehmen, ruft das Marges Urinstinkte wach. Zwischen Marge und Luann kommt es schließlich zu einem Kampf auf Leben und Tod.